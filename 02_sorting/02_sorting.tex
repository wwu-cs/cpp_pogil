\documentclass{exam}
%\documentclass[answers]{exam}
\setlength{\textheight}{9.5in}
\setlength{\textwidth}{6.5in}
\setlength{\topmargin}{-0.75in}
\setlength{\oddsidemargin}{0in}
\setlength{\evensidemargin}{0in}

\usepackage{amsmath}
\usepackage{amssymb}
\usepackage{enumerate}
\usepackage[table]{xcolor}
\usepackage{hhline}
\usepackage{graphicx}
\usepackage{tikz}
\usepackage{pgfplots}
\usepackage{multicol}
\usepackage{fancyvrb}

% for syntax highlighting
\usepackage{minted}
\usemintedstyle[cpp]{xcode}

% for overlay of output
\usepackage[overlay,showboxes]{textpos}

\pagestyle{plain}

\setlength\columnsep{50pt}
\newcommand{\key}{\hfill
      \raisebox{-.3\height}{\includegraphics[width=0.6in]{figures/key.png}}}

\begin{document}
\thispagestyle{empty}
\setlength{\parindent}{0pt}

\begin{center}
      \Large Activity \#2: Sorting \\[5pt]
      \large Recorder's Report\\[20pt]
      \normalsize
      \begin{tabular}{lrp{0.1in}lr}
            Manager:  & \fillin[][2.25in] &  & Reader: & \fillin[][2.25in]            \\[15pt]
            Recorder: & \fillin[][2.25in] &  & Driver: & \fillin[][2.25in]            \\[15pt]
            Date:     & \fillin[][2.25in] &  & Score:  & Satisfactory \hspace{10pt} /
            \hspace{10pt} Not Satisfactory
      \end{tabular}
\end{center}
\par\vskip 15pt

Record your team's answers to the key questions (marked with
\raisebox{-.3\height}{\includegraphics[width=0.5in]{figures/key.png}})
below.
\begin{enumerate}[(a)]
      \itemsep 1.35in
      \item Model 1, Question \#5
      \item Model 1, Question \#7
      \item Model 2, Question \#14
      \item Model 2, Question \#19
\end{enumerate}

\clearpage\pagenumbering{arabic}

\begin{center}
      \Large Activity \#2: Sorting \\[5pt]
      \large Activity Guide\\[20pt]
\end{center}
\vskip -30pt\null

\begin{center}
      \fbox{
            \begin{minipage}{5.5in}
                  {\bf Learning Objectives:} Students will be able to:
                  \begin{itemize}
                        \item Content:\\[-20pt]
                              \begin{itemize}
                                    \itemsep 0pt
                                    \item Explain the structure of linked list
                                    \item Explain difference between a single, circular, and double linked list
                              \end{itemize}
                        \item Process\\[-20pt]
                              \begin{itemize}
                                    \itemsep 0pt
                                    \item Write code that adds, removes, and accesses a linked list\\[-5pt]
                              \end{itemize}
                  \end{itemize}
            \end{minipage}
      }
\end{center}
\par\vskip 10pt

      {\bf\large Model 1: Comparing} \\

In computing, we often need to \textbf{sort} a set of items. As computer scientists,
we study ways to sort very large sets, with thousands or millions of values,
since searching and other operations are often easier when the set of values is sorted.

For example, the Harvard University Library has roughly 16,000,000 volumes, and the
US Library of Congress has roughly 22,000,000 books, and over 100,000,000 total items.
In 2010, a team at UC San Diego sorted one trillion ($10^{12}$) data records in 172 minutes.
Simple $O(N^2)$ sorting algorithms work well for small lists, but are too slow for larger lists.
Most software libraries (APIs) include excellent sorting algorithms,  but exploring better
sorting algorithms also demonstrates more general concepts in algorithm design and analysis.


\begin{minted}[
          bgcolor=gray!15,
          baselinestretch=1
        ]{cpp}
        public    int[] sort(   int[] d) { ... } 
        public String[] sort(String[] d) { ... } 
        public   City[] sort(  City[] d) { ... } 
         
        public void demo() { 
          int   [] iu = { 6,3,8,2,9 }; 
          int   [] is = sort(iu); 
          String[] su = { "banana", "grape", "apple", "mango" }; 
          String[] ss = sort(su); 
          City  [] cu = { City.NYC, City.LAX, City.PHL, City.CHI }; 
          City  [] cs = sort(cu); 
        }
      \end{minted}

\begin{enumerate}

      \item (1 min) Each \textbf{sort()} method above takes an array input and returns a sorted array.
            What is \textbf{different} about the first 3 methods?

            \begin{solution}[0.6in]
                  The 1st works on ints,  the 2nd on Strings; the 3rd on Cities.
            \end{solution}


      \item (2 min) What sequence of values should appear in each variable below:

            \begin{center}
                  \renewcommand{\arraystretch}{2}
                  \begin{tabular}{ |c|m{3cm}|m{11cm}| }
                        \hline
                        a. & is & \ifprintanswers is = { 2,3,6,8,9 } \fi                           \\
                        \hline
                        b. & ss & \ifprintanswers ss = { "apple", "banana", "grape", "mango" } \fi \\
                        \hline
                  \end{tabular}
            \end{center}

            \newpage

      \item   (1 min) In the table below, specify tests for \textbf{sort()},

            \begin{center}
                  \renewcommand{\arraystretch}{2}
                  \begin{tabular}{ |c|l|m{11cm}| }
                        \hline
                        \rowcolor{lightgray} & \textbf{Input: array to sort} & \textbf{Expected Result}                \\
                        \hline
                        a.                   & [ ] (empty list)              & empty list                              \\
                        \hline
                        b.                   & [ 5 ]                         & 5                                       \\
                        \hline
                        c.                   & [ 2, 4, 8 ]                   & \ifprintanswers { 2,4,8 } \fi           \\
                        \hline
                        d.                   & [ 9, 7, 3, 5 ]                & \ifprintanswers { 3,5,7,9 } \fi         \\
                        \hline
                        e.                   & [ 27, 42, 35 ]                & \ifprintanswers { 27,35,42 } \fi        \\
                        \hline
                        f.                   & [ "a" ]                       & \ifprintanswers { a } \fi               \\
                        \hline
                        g.                   & [ "a", "b", "c" ]             & \ifprintanswers { a, b, c } \fi         \\
                        \hline
                        h.                   & [ "c", "b", "a" ]             & \ifprintanswers { a, b, c } \fi         \\
                        \hline
                        i.                   & [ "a3", "a1", "a2" ]          & \ifprintanswers { a1, a2, a3 } \fi      \\
                        \hline
                        j.                   & [ "bar", "ball", "back" ]     & \ifprintanswers { back, ball, bar } \fi \\
                        \hline
                  \end{tabular}
            \end{center}

      \item  (2 min) It is easier to compare integers than Strings. Explain why.

            \begin{solution}[0.6in]
                  Integers have a single value, but Strings have sets of character values.
                  Integers have one clear ordering, but Strings have upper/lower case, unusual characters, etc.
            \end{solution}

            Suppose we had a list (e.g. a database or spreadsheet) of detailed information about cities:

            \begin{center}
                  \begin{tabular}{ |c|c|c|c|c|c| }
                        \hline
                        \rowcolor{lightgray} \textbf{Name} & \textbf{Area (sq mi)} & \textbf{Population} & \textbf{Altitude} & \textbf{Latitude} & \textbf{Longitude} \\
                        \hline
                        Chicago                            & 227.6                 & 2,700,000           & ...               & ...               & ...                \\
                        \hline
                        Los Angeles                        & 468.7                 & 3,800,000           & ...               & ...               & ...                \\
                        \hline
                        New York                           & 302.6                 & 8,200,000           & ...               & ...               & ...                \\
                        \hline
                        ...                                & ...                   & ...                 & ...               & ...               & ...                \\
                        \hline
                  \end{tabular}
            \end{center}


      \item (2 min) Explain why it could be harder to compare Cities than Strings.


            \begin{solution}[0.6in]
                  Cities could be compared in many different ways - population, area, location, etc.
            \end{solution}

      \item (3 min) It can be useful to \textbf{abstract} the comparisons used in searching & sorting.

            \begin{enumerate}
                  \item In the \textbf{Sample Code} handout, what \textbf{method} is declared in interface Comparable?

                        \begin{solution}[0.6in]
                              \texttt{int compareTo(T that);}
                        \end{solution}


                  \item The Java API method \texttt{Math.signum()} returns 0 if its input is 0,
                        +1 if its input is positive, and -1 if its input is negative.
                        In the Sample Code handout, how does class City compare two cities?


                        \begin{solution}[0.6in]
                              It subtracts their areas, and takes the signum of the result (-1,0,+1).
                        \end{solution}



                  \item The Java API class String implements interface Comparable.
                        Given this, what Java expression would compare two Strings: s1 and s2?

                        \begin{solution}[0.6in]
                              \texttt{s1.compareTo(s2)}
                        \end{solution}


            \end{enumerate}
      \item (4 min) In class City, modify the return statement to compare Cities using:
            \begin{center}
                  \renewcommand{\arraystretch}{2}
                  \begin{tabular}{ |c|m{2.3cm}|m{11cm}| }
                        \hline
                        a. & population                  & \ifprintanswers return (int) Math.signum(this.getPop () - that.getPop () ); \fi \\
                        \hline
                        b. & name (alphabetical)         & \ifprintanswers return this.getName().compareTo( that.getName() ); \fi          \\
                        \hline
                        c. & name (reverse alphabetical) & \ifprintanswers return that.getName().compareTo( this.getName() ); \fi          \\
                        \hline
                        d. & population density          & \ifprintanswers return (int) Math.signum(
                              this.getPop () / this.getArea()
                              - that.getPop () / that.getArea() );
                              return (int) Math.signum(
                              this.getDens() - that.getDens() ); \fi                                                                       \\
                        \hline
                  \end{tabular}
            \end{center}

      \item (1 min) Comparison can be abstracted in more than one way.
            In the Sample Code handout, what methods are declared in interface Comparator?

            \begin{solution}[0.6in]
                  \texttt{int     compare(T o1, T o2);}
                  \texttt{boolean equals (Object obj);}
            \end{solution}


      \item (2 min) For each class listed below, describe how it compares 2 objects.
            \begin{center}
                  \renewcommand{\arraystretch}{2}
                  \begin{tabular}{ |c|m{2.3cm}|m{11cm}| }
                        \hline
                        a. & \texttt{class StringCompI} & \ifprintanswers compares 2 Strings alphabetically, ignoring uppercase/lowercase differences. \fi \\
                        \hline
                        b. & \texttt{class StringCompL} & \ifprintanswers compares 2 Strings using their length (not their contents) \fi                   \\
                        \hline
                        c. & \texttt{class CityCompA}   & \ifprintanswers compares 2 Cities using their area \fi                                           \\
                        \hline
                        d. & \texttt{class CityCompP}   & \ifprintanswers compares 2 Cities using their population \fi                                     \\
                        \hline
                  \end{tabular}
            \end{center}


      \item (3 min) Change the return statement in \texttt{CityCompA.compare()} to compare:
            \begin{center}
                  \renewcommand{\arraystretch}{2}
                  \begin{tabular}{ |c|m{2.3cm}|m{11cm}| }
                        \hline
                        a. & name (alphabetical)         & \ifprintanswers return c1.getName().compareTo( c2.getName() ); \fi \\
                        \hline
                        b. & name (reverse alphabetical) & \ifprintanswers return c2.getName().compareTo( c1.getName() ); \fi \\
                        \hline
                        c. & population density          & \ifprintanswers return (int)Math.signum(
                              c1.getPop () / c1.getArea()
                              - c2.getPop () / c2.getArea() );
                              return (int)Math.signum(
                              c1.getDens() - c2.getDens() );  \fi                                                             \\
                        \hline
                  \end{tabular}
            \end{center}


            \newpage


            {\bf\large Model 2: Merge} \\

            Some sort algorithms (e.g. bubble sort and selection sort) are simple but inefficient.
            To understand a better sort algorithm, we'll start with something easier - \textbf{merge}.

      \item Recall that \textbf{$\theta()$}, \textbf{$O()$}, and \textbf{$\omega()$ notation} describes how input size
            affects operation count and run time. If the input size \textbf{doubles},
            what happens to the run time if an algorithm is O(1), O(N), etc?
            Which line at right shows this?


            \begin{center}
                  \begin{tabular}{ |l|l|l|l| }
                        \hline
                        a. O(1) & b. O(N) & c. $O(N^2))$ & O(log N) \\
                        \hline
                                &         &              &          \\
                                &         &              &          \\
                        \hline
                  \end{tabular}
            \end{center}




      \item  (3 min) Use the table below to specify \textbf{unit tests} for \textbf{merge(arrA,arrB)},
            which \textbf{merges} two sorted arrays into one sorted array.

            \begin{center}
                  \begin{tabular}{ |l|l|l| }
                        \hline
                        \rowcolor{lightgray} \textbf{arrA} & \textbf{arrB} & \textbf{Expected Result} (return value, exception, etc) \\
                        \hline
                        (empty)                            & (empty)       & (empty)                                                 \\
                        \hline
                        "B"                                & (empty)       & \ifprintanswers "B" \fi                                 \\
                        \hline
                        (empty)                            & "A"           & \ifprintanswers "A" \fi                                 \\
                        \hline
                        "B"                                & "A"           & \ifprintanswers "A", "B" \fi                            \\
                        \hline
                        "B", "D"                           & "A", "C"      & \ifprintanswers "A", "B", "C", "D" \fi                  \\
                        \hline
                        "B", "C", "D"                      & "A", "C"      & \ifprintanswers "A", "B", "C", "C", "D" \fi             \\
                        \hline
                  \end{tabular}
            \end{center}


      \item (5 min) Given 2 sorted arrays arrA \& arrB that will be merged into one sorted array arrC.

            \begin{minipage}[c]{0.25\textwidth}
                  \begin{tabular}{ |c|c|c|c| }
                        \hline
                        \multicolumn{4}{|c|}{\textbf{arrA}} \\
                        \hline
                        $a^0$ & $a^1$ & $a^2$ & $a^3$       \\
                        \hline
                  \end{tabular}
            \end{minipage}
            \begin{minipage}[c]{0.25\textwidth}
                  \begin{tabular}{ |c|c|c|c| }
                        \hline
                        \multicolumn{4}{|c|}{\textbf{arrB}} \\
                        \hline
                        $b^0$ & $b^1$ & $b^2$ & $b^3$       \\
                        \hline
                  \end{tabular}
            \end{minipage}
            \begin{minipage}[c]{0.5\textwidth}
                  \begin{tabular}{ |c|c|c|c|c|c|c|c| }
                        \hline
                        \multicolumn{8}{|c|}{\textbf{$\Rightarrow$ arrC}}             \\
                        \hline
                        $c^0$ & $c^1$ & $c^2$ & $c^3$ & $c^4$ & $c^5$ & $c^6$ & $c^7$
                        \\
                        \hline
                  \end{tabular}
            \end{minipage}


            a. What is the length of \texttt{arrC}, in terms of \texttt{arrA.length} and \texttt{arrB.length}?

            \begin{solution}[0.6in]
                  \texttt{arrC.length = arrA.length + arrB.length}
            \end{solution}

            b. There are only 2 values ($a_0$ and $b_0$) that could go into $c_0$. Explain why.

            \begin{solution}[0.6in]
                  Since A and B are sorted, a0 is the smallest value in A and b0 is the smallest value in B,
                  and so one of these two must be the smallest value in C.
            \end{solution}

            c. Once $c_0$ is chosen, there are only 2 values that could go into $c_1$. Explain why.


            \begin{solution}[0.6in]
                  If a0 goes into c0, then compare a1 and b0. If b0 goes into c0, then compare a0 and b1.
            \end{solution}

            d. Similarly, once $c_0$ to $c_i$ are chosen, there are only 2 values that could go into $c_{i+1}$. Explain.

            \begin{solution}[0.6in]
                  At each index we compare the smallest unused values from A \& B.
            \end{solution}

            \newpage
      \item (3 min) Given your answers, above, how could we \textbf{merge()} 2 arrays efficiently?
            In complete sentences or pseudocode, describe a general approach for \textbf{merge()}.

            \texttt{String[] merge(String[] arrA, String[] arrB)}

            \begin{solution}[0.6in]
                  Start at the beginning of both arrays. Step through both arrays at the same time,
                  keeping track of current position in each. Choose the smaller of the two values,
                  copy it to the merged list, and shift the position in that list. Continue until
                  the end of one array, and then copy any values left in the other array to the merged list.

                  // pseudocode - see instructor sample code for more details
                  String[] merge(String[] arrA, String[] arrB) {
                              String[] result = new String[arrA.length + arrB.length];
                              int ia = 0, ib = 0;
                              for (int i=0; i<result.length; i++) {
                                          if (ib == arrB.length || arrA[ia].compareTo(arrB[ib]) < 0) {
                                                      result[i] = arrA[ia]; ia++; }
                                          else {
                                                      result[i] = arrB[ib]; ib++; }
                                    }
                              return result;
                        }
            \end{solution}


            \newpage
      \item (3 min) The table below shows the contents of sorted arrays arrayA  and arrayB.
            Complete the table to show how arrayC and the 3 index variables (ai,bi,ci) change over time.


            \begin{center}
                  \begin{tabular}{ |r|r|r|r|r|r|r|r|r|r|r|r|r|r|r|r|r|r|r|r|r|r|r|r|r| }
                        \hline
                        \rowcolor{lightgray} \multicolumn{4}{|c|}{\textbf{arrayA}} &            & \multicolumn{4}{|c|}{\textbf{arrayB}} &            & \multicolumn{8}{|c|}{\textbf{arrayC}} &            & \multicolumn{3}{|c|}{\textbf{vars}}                                                                                                                                                                                                                                                                                       \\
                        \hline
                        \textbf{0}                                                 & \textbf{1} & \textbf{2}                            & \textbf{3} &                                       & \textbf{0} & \textbf{1}                          & \textbf{2} & \textbf{3} &  & \textbf{0} & \textbf{1} & \textbf{2}            & \textbf{3}            & \textbf{4}             & \textbf{5}             & \textbf{6}             & \textbf{7}             &  & \textbf{ai}           & \textbf{bi}           & \textbf{ci}           \\
                        \hline
                        2                                                          & 6          & 12                                    & 14         &                                       & 3          & 5                                   & 15         & 17         &  &            &            &                       &                       &                        &                        &                        &                        &  & 0                     & 0                     & 0                     \\
                        \hline
                        2                                                          & 6          & 12                                    & 14         &                                       & 3          & 5                                   & 15         & 17         &  & 2          &            &                       &                       &                        &                        &                        &                        &  & 1                     & 0                     & 1                     \\
                        \hline
                        2                                                          & 6          & 12                                    & 14         &                                       & 3          & 5                                   & 15         & 17         &  & 2          & 3          &                       &                       &                        &                        &                        &                        &  & 1                     & 1                     & 2                     \\
                        \hline
                        2                                                          & 6          & 12                                    & 14         &                                       & 3          & 5                                   & 15         & 17         &  & 2          & 3          & \ifprintanswers 5 \fi &                       &                        &                        &                        &                        &  & \ifprintanswers 1 \fi & \ifprintanswers 2 \fi & \ifprintanswers 3 \fi \\
                        \hline
                        2                                                          & 6          & 12                                    & 14         &                                       & 3          & 5                                   & 15         & 17         &  & 2          & 3          & \ifprintanswers 5 \fi & \ifprintanswers 6 \fi &                        &                        &                        &                        &  & \ifprintanswers 2 \fi & \ifprintanswers 2 \fi & \ifprintanswers 4 \fi \\
                        \hline
                        2                                                          & 6          & 12                                    & 14         &                                       & 3          & 5                                   & 15         & 17         &  & 2          & 3          & \ifprintanswers 5 \fi & \ifprintanswers 6 \fi & \ifprintanswers 12 \fi &                        &                        &                        &  & \ifprintanswers 3 \fi & \ifprintanswers 2 \fi & \ifprintanswers 5 \fi \\
                        \hline
                        2                                                          & 6          & 12                                    & 14         &                                       & 3          & 5                                   & 15         & 17         &  & 2          & 3          & \ifprintanswers 5 \fi & \ifprintanswers 6 \fi & \ifprintanswers 12 \fi & \ifprintanswers 14 \fi &                        &                        &  & \ifprintanswers 4 \fi & \ifprintanswers 2 \fi & \ifprintanswers 6 \fi \\
                        \hline
                        2                                                          & 6          & 12                                    & 14         &                                       & 3          & 5                                   & 15         & 17         &  & 2          & 3          & \ifprintanswers 5 \fi & \ifprintanswers 6 \fi & \ifprintanswers 12 \fi & \ifprintanswers 14 \fi & \ifprintanswers 15 \fi &                        &  & \ifprintanswers 4 \fi & \ifprintanswers 3 \fi & \ifprintanswers 7 \fi \\
                        \hline
                        2                                                          & 6          & 12                                    & 14         &                                       & 3          & 5                                   & 15         & 17         &  & 2          & 3          & \ifprintanswers 5 \fi & \ifprintanswers 6 \fi & \ifprintanswers 12 \fi & \ifprintanswers 14 \fi & \ifprintanswers 15 \fi & \ifprintanswers 17 \fi &  & \ifprintanswers 4 \fi & \ifprintanswers 4 \fi & \ifprintanswers 8 \fi \\
                        \hline
                  \end{tabular}
            \end{center}

      \item (3 min) During merge(), if there are N/2 values in each sorted array,
            and N values in the merged array (also sorted), explain the \textbf{O()} effort to find:


            \begin{center}
                  \begin{tabular}{ |l|m{3.5cm}|l|m{9cm}| }
                        \hline
                        \rowcolor{lightgray} &                                                     & \textbf{O()}             & \textbf{Explanation}                                                            \\
                        \hline
                        a.                   & The \textbf{first value} in the merged array.       & \ifprintanswers O(1) \fi & \ifprintanswers Compare 1st 2 values \fi                                        \\
                        \hline
                        b.                   & The \textbf{second value} in the merged array.      & \ifprintanswers O(1) \fi & \ifprintanswers Compare 1st value in one array and 2nd value in other array \fi \\
                        \hline
                        c.                   & Each \textbf{successive} value in the merged array. & \ifprintanswers O(1) \fi & \ifprintanswers Compare one value in each array \fi                             \\
                        \hline
                        d.                   & \textbf{All N} values in the merged array.          & \ifprintanswers O(N) \fi & \ifprintanswers Repeat O(1) comparison for N values in merged array \fi         \\
                        \hline
                  \end{tabular}
            \end{center}


      \item (2 min) In the following questions, a 2 item list is called a 2-list, a N item list
            is called an N-list, etc. Decide if each of the following is always sorted (Y or N):


            \begin{center}
                  \renewcommand{\arraystretch}{1.5}
                  \begin{tabular}{ |l|m{9cm}|m{3cm}| }
                        \hline
                        a. & a 16-list of random numbers                       & \ifprintanswers N=unsorted \fi \\
                        \hline
                        b. & the first half of a 32-list of random numbers     & \ifprintanswers N=unsorted \fi \\
                        \hline
                        c. & a 1-list                                          & \ifprintanswers Y=sorted \fi   \\
                        \hline
                        d. & a 2-list made by merging 2 (sorted) 1-lists       & \ifprintanswers Y=sorted \fi   \\
                        \hline
                        3. & a a 16-list made by merging 2 (sorted) 8-lists    & \ifprintanswers Y=sorted \fi   \\
                        \hline
                        f. & a a N-list made by merging 2 (sorted) (N/2)-lists & \ifprintanswers Y=sorted \fi   \\
                        \hline
                  \end{tabular}
            \end{center}


      \item (2 min) Decide how many new lists are made when pairs of lists are merged from:

            \begin{center}
                  \renewcommand{\arraystretch}{1.5}
                  \begin{tabular}{ |l|m{9cm}|m{3cm}| }
                        \hline
                        a. & 1024 1-lists into 2-lists          & \ifprintanswers 512 2-lists \fi \\
                        \hline
                        b. & all of those 2-lists into 4-lists  & \ifprintanswers 256 4-lists \fi \\
                        \hline
                        c. & N 1-lists (for any N) into 2-lists & \ifprintanswers N/2 2-lists \fi \\
                        \hline
                        d. & all of those 2-lists into 4-lists  & \ifprintanswers N/4 4-lists \fi \\
                        \hline
                  \end{tabular}
            \end{center}
            \newpage
      \item (3 min) How could we use merge to sort an unsorted list? (Hint: think recursively.)
            In complete English sentences, describe a general approach for mergesort.


            \begin{solution}[0.6in]
                  Treat N-list as N 1-lists. Merge pairs of 1-lists into sorted 2-lists. Merge pairs of 2-lists
                  into sorted 4-lists. Repeat until 2 N/2 lists are merged into 1 sorted N-list.
            \end{solution}

\end{enumerate}

\end{document}
