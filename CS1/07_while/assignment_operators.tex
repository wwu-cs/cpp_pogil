 \model{Assignment Operators}
  \begin{center}
    \renewcommand{\arraystretch}{1.3}
    \begin{tabular}{|c|c|c|}
      \hline
      \rowcolor{orange!20} Initial {\tt x} Value & Assignment Operator & Final {\tt x} Value \\
      \hline
      2 & \cpp{x += 1;} & 3 \\
      \hline
      2 & \cpp{x -= 1;} & 1 \\
      \hline
      4 & \cpp{x *= 2;} & 8 \\
      \hline
      4 & \cpp{x /= 2;} & 2 \\
      \hline
      7 & \cpp{x \%= 4;} & 3 \\
      \hline
    \end{tabular}
  \end{center}
  
  {\it\large Refer to Model 3 above as your team develops consensus answers
    to the questions below.}

  \quest{15 min}

  \Q The code \cpp{x += 5;} is equivalent to which of the following lines of code?
    \begin{enumerate}[label=]
      \begin{minipage}{3in}
        \item \ans[0.3in] \cpp{ x = 5;}
        \item \ans[0.3in] \cpp{ x = y + 5;}
      \end{minipage}
      \begin{minipage}{3in}
        \item \ans[0.3in]{\checkmark} \cpp{ x = x + 5;}
        \item \ans[0.3in] \cpp{ y = x + 5;}
      \end{minipage}
    \end{enumerate}
    
  \Q An {\it assignment operator} provides a concise way of creating assignment statements when the variable on the
    left-hand side (LHS) will also appear on the right-hand side (RHS).  In your own words, describe what each of the 
    following assignment operators does.
    \begin{enumerate}
      \itemsep 10pt
      \item \cpp{+=} \hspace{20pt} 
        \ans[5.25in]{Adds the amount on the RHS to the variable on the LHS}

      \item \cpp{-=} \hspace{20pt}
        \ans[5.25in]{Subtracts the amount on the RHS from the variable on the LHS}

      \item \cpp{*=} \hspace{20pt} 
        \ans[5.25in]{Multiplies the variable on the LHS by the amount on the RHS}
        
      \item \cpp{/=} \hspace{20pt} 
        \ans[5.25in]{Divides the variable on the LHS by the amount on the RHS}
        
      \item \cpp{\%=} \hspace{20pt} 
        \ans[5in]{Computes the remainder of the amount on the RHS to the variable on the LHS}
    \end{enumerate}

    \newpage
    
  \Q The table below is similar to that seen in Model 3.  Fill in the missing pieces with appropriate values or
    assignment operator statements.  Assume that {\tt x} is an integer variable.
    \begin{center}
      \renewcommand{\arraystretch}{1.5}
      \begin{tabular}{|c|c|c|c|}
        \hline
        \rowcolor{orange!20} Operator & Initial {\tt x} Value & Statement & Final {\tt x} Value \\
        \hline
        \cpp{+=} & 6 & \ans[0.5in]{\cpp{x += 2;}} & 8 \\
        \hline
        \cpp{+=} & 5 & \cpp{x += 1;} & \ans[0.5in]{6} \\
        \hline
        \cpp{-=} & \ans[0.5in]{9}  & \cpp{x -= 3;} & 6 \\
        \hline
        \cpp{*=} & 4 & \ans[0.5in]{\cpp{x *= 1;}} & 4 \\
        \hline
        \cpp{/=} & 23 & \ans[0.5in]{\cpp{x /= 10;}} & 2 \\
        \hline
      \end{tabular}
    \end{center}
    
  \Q Is the assignment operator \cpp{23 += total}
  valid?  Why or why not?\key
    \begin{answer}[0.5in]
      No, it is not valid.  There must be only a variable name on the left-hand side.
    \end{answer}
    
  \Q The following code snippet should print the numbers beginning with 100 and counting down to 1.  However, it is missing a
    line of code.  Add the missing code using an assignment operator.  Indicate the line number at which the code should be inserted.
    \par\vskip 10pt
    \begin{tabular}{p{3in}p{3.5in}}
      \begin{minipage}{3in}
        \begin{cpplst}
int countdown = 100;
while (countdown > 0) {
  cout << countdown << endl;
}
        \end{cpplst}
      \end{minipage}
      &
      \begin{minipage}{3.5in}
        \begin{answer}[0.5in]
          \par
          Insert the command \cpp{x -= 1;} between lines 3 and 4.
        \end{answer}            
      \end{minipage}
    \end{tabular}
      
  \Q Use assignment operators to write loops to do each of the following.  
    You can test your code in the {\tt activity07c.cpp} file.
    \begin{enumerate}
      \item A loop that prints out all multiples of three between 0 and 100. (i.e. 0, 3, 6, 9, {\ldots} )
        \begin{answer}[1.5in]
          \begin{center}
            \begin{minipage}{3in}
              \begin{cpplst}
int num = 0;
while (num <= 100) {
  cout << num << endl;
  num += 3;
}
              \end{cpplst}
            \end{minipage}
          \end{center}
        \end{answer}

      \item A loop that prints all powers of 2 (i.e. $2^n$) from 1 up to 100.
        \begin{answer}[1.5in]
          \begin{center}
            \begin{minipage}{3in}
              \begin{cpplst}
int num = 1;
while (num <= 100) {
  cout << num << endl;
  num *= 2;
}
              \end{cpplst} 
            \end{minipage}
          \end{center}              
        \end{answer}
      \end{enumerate}