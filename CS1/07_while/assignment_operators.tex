 {\bf\large Model 3: Assignment Operators} \\[-10pt]
  \begin{center}
    \renewcommand{\arraystretch}{1.3}
    \begin{tabular}{|c|c|c|}
      \hline
      \rowcolor{orange!20} Initial {\tt x} Value & Assignment Operator & Final {\tt x} Value \\
      \hline
      2 & \mintinline{cpp}|x += 1;| & 3 \\
      \hline
      2 & \mintinline{cpp}|x -= 1;| & 1 \\
      \hline
      4 & \mintinline{cpp}|x *= 2;| & 8 \\
      \hline
      4 & \mintinline{cpp}|x /= 2;| & 2 \\
      \hline
      7 & \mintinline{cpp}|x %= 4;| & 3 \\
      \hline
    \end{tabular}
  \end{center}
  
  {\it\large Refer to Model 3 above as your team develops consensus answers
    to the questions below.}

    \item The code \mintinline{cpp}|x += 5;| is equivalent to which of the following lines of code?\par\vskip 10pt
      \hspace{20pt}
      \begin{oneparcheckboxes}
        \choice \mintinline{cpp}|x = 5;|
        \choice \mintinline{cpp}|x = y + 5;|
        \correctchoice \mintinline{cpp}|x = x + 5;|
        \choice \mintinline{cpp}|y = x + 5;|
      \end{oneparcheckboxes}
      
    \item An {\it assignment operator} provides a concise way of creating assignment statements when the variable on the
      left-hand side (LHS) will also appear on the right-hand side (RHS).  In your own words, describe what each of the 
      following assignment operators does.
      \par\vskip 20pt
      
      \begin{enumerate}[(a)]
        \itemsep 15pt
        \item \mintinline{cpp}|+=| \hspace{20pt} 
          \fillin[Adds the amount on the RHS to the variable on the LHS][5.25in]
        \item \mintinline{cpp}|-=| \hspace{20pt}
          \fillin[Subtracts the amount on the RHS from the variable on the LHS][5.25in]
        \item \mintinline{cpp}|*=| \hspace{20pt} 
          \fillin[Multiplies the variable on the LHS by the amount on the RHS][5.25in]
        \item \mintinline{cpp}|/=| \hspace{20pt} 
          \fillin[Divides the variable on the LHS by the amount on the RHS][5.25in]
        \item \mintinline{cpp}|%=| \hspace{20pt} 
          \fillin[Adds the amount on the RHS to the variable on the LHS][5in]
      \end{enumerate}
      
\newpage

    \item The table below is similar to that seen in Model 3.  Fill in the missing pieces with appropriate values or
      assignment operator statements.  Assume that {\tt x} is an integer variable.
      
      \begin{center}
        \renewcommand{\arraystretch}{1.5}
        \begin{tabular}{|c|c|c|c|}
          \hline
          \rowcolor{orange!20} Operator & Initial {\tt x} Value & Statement & Final {\tt x} Value \\
          \hline
          \mintinline{cpp}|+=| & 6 & \ifprintanswers\mintinline{cpp}|x += 1;|\fi & 8 \\
          \hline
          \mintinline{cpp}|+=| & 5 & \mintinline{cpp}|x -= x+1;| & \ifprintanswers 11\fi \\
          \hline
          \mintinline{cpp}|-=| & \ifprintanswers 9\fi  & \mintinline{cpp}|x -= 3;| & 6 \\
          \hline
          \mintinline{cpp}|*=| & 4 & \ifprintanswers\mintinline{cpp}|x *= 4;|\fi & 4 \\
          \hline
          \mintinline{cpp}|/=| & 23 & \ifprintanswers\mintinline{cpp}|x /= 10;|\fi & 2 \\
          \hline
        \end{tabular}
      \end{center}
      \par\vskip -40pt\null
      
    \item Is the assignment operator \mintinline{cpp}|23 += total|
    valid?  Why or why not?\key
      \begin{solution}[0.5in]
        No, it is not valid.  There must be only a variable name on the left-hand side.
      \end{solution}
      
    \item The following code snippet should print the numbers beginning with 100 and counting down to 1.  However, it is missing a
      line of code.  Add the missing code using an assignment operator.  Indicate the line number at which the code should be inserted.
      \par\vskip 10pt
      \begin{tabular}{p{2.4in}p{3.4in}}
        \begin{minipage}{2.4in}
          \begin{minted}[
            frame=lines,
            framesep=2mm,
            bgcolor=gray!15,
            baselinestretch=1.2,
            linenos
          ]{cpp}
  int countdown = 100;
  while (countdown > 0) {
    cout << countdown << endl;
  }
          \end{minted}
          \par\vskip 10pt\null
        \end{minipage}
        &
        \begin{minipage}{3.4in}
          \begin{solution}[0.5in]
            \par
            Insert the command \mintinline{cpp}|x -= 1;| between lines 3 and 4.
          \end{solution}            
        \end{minipage}
      \end{tabular}
      \par\vskip -40pt\null
      
    \item Use assignment operators to write loops to do each of the following.  
      You can test your code in the {\tt activity07c.cpp} file.
      
      \begin{enumerate}[(a)]
        \item A loop that prints out all multiples of three between 0 and 100. (i.e. 0, 3, 6, 9, {\ldots} )
          \begin{solution}[1.5in]
            \begin{center}
              \scriptsize
              \begin{minipage}{3in}
                \begin{minted}[
                  frame=lines,
                  framesep=2mm,
                  bgcolor=gray!15,
                  baselinestretch=1.2,
                  linenos
                ]{cpp}
  int num = 0;
  while (num <= 100) {
    cout << num << endl;
    num += 3;
  }
                \end{minted}
              \end{minipage}
            \end{center}
          \end{solution}
        \item A loop that prints all powers of 2 (i.e. $2^n$) from 1 up to 100.
          \begin{solution}[1.5in]
            \begin{center}
              \scriptsize
              \begin{minipage}{3in}
                \begin{minted}[
                  frame=lines,
                  framesep=2mm,
                  bgcolor=gray!15,
                  baselinestretch=1.2,
                  linenos
                ]{cpp}
  int num = 1;
  while (num <= 100) {
    cout << num << endl;
    num *= 2;
  }
                \end{minted} 
              \end{minipage}
           \end{center}              
          \end{solution}
      \end{enumerate}

  \end{enumerate}  