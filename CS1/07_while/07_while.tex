%\documentclass{exam}
\documentclass[answers]{exam}
\hbadness=99999
\setlength{\textheight}{9.5in}
\setlength{\textwidth}{6.5in}
\setlength{\topmargin}{-0.75in}
\setlength{\oddsidemargin}{0in}
\setlength{\evensidemargin}{0in}

\usepackage{amsmath}
%\usepackage{amsfonts}
\usepackage{amssymb}
\usepackage{enumerate}
\usepackage[table]{xcolor}
\usepackage{graphicx}
\usepackage{tikz}
%\usepackage{pgfplots}
\usepackage{multicol}

% for syntax highlighting
\usepackage{minted}
\usemintedstyle[cpp]{xcode}

% for overlay of output
\usepackage[overlay,showboxes]{textpos}

\pagestyle{plain}

\setlength\columnsep{50pt}
\newcommand{\key}{\hfill
      \raisebox{-.3\height}{\includegraphics[width=0.6in]{figures/key.png}}}

\begin{document}
  \thispagestyle{empty}
  \setlength{\parindent}{0pt}

  \begin{center}
    \Large Activity \#7: While Loops \\[5pt]
    \large Recorder's Report\\[20pt]
    \normalsize
    \begin{tabular}{lrp{0.1in}lr}
      Manager:  & \fillin[][2.0in] & & Presenter: & \fillin[][2.0in]\\[15pt]
      Recorder: & \fillin[][2.0in] & & Driver:    & \fillin[][2.0in]\\[15pt]
      Date:     & \fillin[][2.0in] & & Score:     & Satisfactory \hspace{10pt} /
      \hspace{10pt} Not Satisfactory
    \end{tabular}
  \end{center}
  \par\vskip 15pt
  
  Record your team's answers to the key questions (marked with
  \raisebox{-.3\height}{\includegraphics[width=0.5in]{figures/key.png}})
  below.
  \begin{enumerate}[(a)]
    \itemsep 1.75in
    \item Model 1, Question \#4
    \item Model 2, Question \#11.f
    \item Model 3, Question \#15
  \end{enumerate}

  \clearpage\pagenumbering{arabic} 
  
  \begin{center}
    \Large Activity \#7: While Loops \\[5pt]
    \large Activity Guide\\[20pt]
  \end{center}

  \begin{center}
    \fbox{
      \begin{minipage}{5.5in}
        {\bf Learning Objectives:} Students will be able to:
        \begin{itemize}
          \item Content:\\[-20pt]
            \begin{itemize}
              \itemsep 0pt
              \item Explain the three parts of a loop.
              \item Explain the syntax of a {\tt while} loop in C++.
              \item Explain {\bf sentinel-controlled} and {\bf counter-controlled} loops.
              \item Explain {\bf assignment operators}.
            \end{itemize}
          \item Process\\[-20pt]
            \begin{itemize}
              \itemsep 0pt
              \item Write code that includes {\bf sentinel-controlled} and {\bf counter-controlled} loops.
              \item Write code that uses {\bf assignment operators}. \\[-5pt]
            \end{itemize}
        \end{itemize}
      \end{minipage}
      }
  \end{center}
  \par\vskip 10pt
  

  {\bf\large Model 1: A Diagram and A C++ Code Snippet} \\[-15pt]
  \begin{center}
    \small
    \begin{tabular}{p{2in}p{3in}}
      \begin{minipage}{2in}
        \centering\par\vskip 10pt
        \begin{tikzpicture}
          % two rectangles
          \draw (0.25,1.5) -- (1.75,1.5) -- (1.75,2.5) -- (0.25,2.5) -- (0.25,1.5);
          \draw (0.25,0) -- (1.75,0) -- (1.75,1) -- (0.25,1) -- (0.25,0);
          % one triangle
          \draw (0,3.5) -- (1,3) -- (2,3.5) -- (1,4) -- (0,3.5);
          % five arrows
          \draw[-stealth] (1,4.25) -- (1,4);
          \draw[-stealth] (1,3) -- node[right] {\scriptsize True} (1,2.5);
          \draw[-stealth] (0.25,2) -- (-0.5,2) -- (-0.5,3.5) -- (0,3.5);
          \draw[-stealth] (2,3.5) -- (2.5,3.5) -- node[right] {\scriptsize False} (2.5,0.5) -- (1.75,0.5);
          % text
          \node[] at (1,2) {\scriptsize Block \#1};
          \node[] at (1,0.5) {\scriptsize Block \#2};
        \end{tikzpicture}
        \par\vskip 4pt\ \      
      \end{minipage}
      &
      \begin{minipage}{3in}
        \begin{minted}[
          frame=lines,
          framesep=2mm,
          bgcolor=gray!15,
          baselinestretch=1.2,
          linenos,
	  firstnumber=6
        ]{cpp}
  // print a person's name 10 times        
  string name;
  cout << "Enter your name: ";
  cin >> name;
  int x = 0;
  while (x < 10) {
    cout << name << endl;
    x = x + 1;
  }
  cout << "Nice to meet you!" << endl;
        \end{minted}
      \end{minipage}
    \end{tabular}
  \end{center}
  \TPMargin{5pt}
  
  
  {\it\large Refer to Model 1 above as your team develops consensus answers
    to the questions below.}
    \par\vskip 10pt
    
  \begin{enumerate}
    \itemsep 20pt
    
    \item Which lines of code go with the following shapes on the provided diagram?
      \par\vskip 20pt
      \begin{enumerate}[(a)]
        \itemsep 15pt
        \item The diamond \hfill 
          \fillin[Line 11 contains the condition in the diamond][4in]
        \item The ``Block \#1'' rectangle \hfill 
          \fillin[Lines 12 and 13 make up the statements in block \#1][4in]
        \item The ``Block \#2'' rectangle \hfill 
          \fillin[Line 15 contains the statement in block \#2][4in]
      \end{enumerate}
      
    \item Every loop structure requires three different actions.  Identify the line in the C++
      code snippet above that corresponds to each of these actions.
      \par\vskip 20pt
      \begin{enumerate}[(a)]
        \itemsep 15pt
        \item {\bf Initialize} a variable to control the number of loop iterations. \hfill\fillin[Line 10][1in]
        \item {\bf Test} a condition to determine if we should keep looping. \hfill\fillin[Line 11][1in]
        \item {\bf Update} the variable involved in the test condition. \hfill\fillin[Line 13][1in]
      \end{enumerate}     

\newpage

    \item The complete program is in {\tt activity07a.cpp}.  Run the program and
      answer the following questions.
      \begin{enumerate}[(a)]
        \item What would the model program do differently if line 10 was: \mintinline{cpp}|int x = 1;|?
          \begin{solution}[0.4in]
            It would only print out 9 copies of the name.
          \end{solution}
        \item What would the model program do differently if line 11 was: \mintinline{cpp}|while (x <= 10) {|?
          \begin{solution}[0.4in]          
            It would print out 11 copies of the name.
          \end{solution}
        \item What would the model program do differently if line 13 was: \mintinline{cpp}|x=x+2;|?
          \begin{solution}[0.4in]
            It would only print out 5 copies of the name.
          \end{solution}
      \end{enumerate}
      \par\vskip -40pt\null
      
    \item Change one or more of lines 10, 11, and 13 to make the program print the name 23 times.\key\\[-2.5mm]
      
      \begin{solution}[1.25in]
        \par
        Answers will vary, but one solution is to change line 6 to read:
        \begin{center}
          \mintinline{cpp}| while ( x < 23 ) {|
        \end{center}        
      \end{solution}
      
  
  {\bf\large Model 2: Another C++ Code Snippet} \\[-10pt]
  \ifprintanswers\vskip -30pt\null\fi
  \begin{center}
    \small
    \begin{minipage}{5.5in}
      \begin{minted}[
        frame=lines,
        framesep=2mm,
        bgcolor=gray!15,
        baselinestretch=1.2,
        linenos,
        firstnumber=7
      ]{cpp}
  cout << "Enter a positive integer: ";
  cin >> number;
  int x = 1;
  while (x <= number) {
    if (x % 10 == 0 ) {
      cout << setw(2) << x << endl;
    } else {
      cout << setw(2) << x << " ";
    }
    x = x + 1;
  }
      \end{minted}
    \end{minipage}
  \end{center}
  \TPMargin{5pt}
  \ifprintanswers\vskip -20pt\null\fi
  

  {\it\large Refer to Model 2 above as your team develops consensus answers
    to the questions below.}
    \par\vskip 5pt

    \item Explain what each line or range of lines in this code snippet does.
      \par\vskip 10pt
      \begin{enumerate}[(a)]
        \itemsep 15pt
        \item Lines 7-8: \hfill \fillin[Prompts the user for a number and saves it in {\tt number}][5.15in]
        \item Line 9:    \hfill \fillin[Initializes the counter {\tt x} to 1][5.15in]
        \item Line 10:    \hfill \fillin[Repeats lines 5-10 as long as the counter {\tt x} is not bigger than {\tt number}][5.15in]
        \item Lines 11-15: \hfill \fillin[Prints the counter, with a new line when it is divisible by 10][5.15in]
        \item Line 16:   \hfill \fillin[Increments the counter by one before going back to line 4][5.15in]
      \end{enumerate}
      
\newpage

    \item The complete program can be found in {\tt activity07b.cpp}.  Run it and answer the following questions.
      \begin{enumerate}[(a)]
        \item What is the output when you enter the number 5?  How many times did the loop execute?
          \begin{solution}[0.5in]
            The loop executed five times.  The output is:\par
            \mintinline{html}| 1  2  3  4  5|
          \end{solution}
        \item What is the output when you enter the number 25? How many times did the loop execute?
          \begin{solution}[0.5in]
            The loop executed 25 times.  The output is:\par
            \mintinline{html}| 1  2  3  4  5  6  7  8  9 10|\\
            \mintinline{html}|11 12 13 14 15 16 17 18 19 20|\\
            \mintinline{html}|21 22 23 24 25|
          \end{solution}
        \item Unlike the loop in model 1, this loop prints the counter.  How might this make testing easier?
          \begin{solution}[0.5in]
            Debugging is easier as you can see the counter values each time through the loop.
          \end{solution}
      \end{enumerate}

    \item The following code snippet should print the numbers from 1 to 10, but it doesn't print anything.  Correct the
      problem.  Replace the contents of {\tt activity07b.cpp} to help in your debugging.
      \ifprintanswers\vskip -15pt\null\else\par\vskip 10pt\fi
      \begin{tabular}{p{2.4in}p{3.4in}}
        \begin{minipage}{2.4in}
          \begin{minted}[
            frame=lines,
            framesep=2mm,
            bgcolor=gray!15,
            baselinestretch=1.2
          ]{cpp}
  int number = 12;
  while (number <= 10) {
    cout << number << endl;
    number = number + 1;
  }
          \end{minted}
        \end{minipage}
        &
        \begin{minipage}{3.4in}
          \begin{solution}[1.5in]
            \par
            The counter should be initialized to 1 instead of 12 on
            line 1 of the code snippet.
          \end{solution}
        \end{minipage}
      \end{tabular}
      \ifprintanswers\vskip -25pt\null\fi
      
    \item Enter and execute the code below, then answer the questions that follow.
      \ifprintanswers\vskip -15pt\null\else\par\vskip 10pt\fi
      \begin{tabular}{p{2.4in}p{3.4in}}
        \begin{minipage}{2.4in}
          \begin{minted}[
            frame=lines,
            framesep=2mm,
            bgcolor=gray!15,
            baselinestretch=1.2
          ]{cpp}
  int number = 0;
  while (number <= 10) {
    cout << number << endl;
    number = number - 1;
  }
          \end{minted}
          \par\vskip 10pt\null
        \end{minipage}
        &
        \begin{minipage}{3.4in}
          \begin{enumerate}[(a)]
            \item Describe the output.
              \ifprintanswers\vskip -25pt\null\fi
              \begin{solution}[0.5in]
                Prints decreasing numbers.
              \end{solution}
              \ifprintanswers\vskip -25pt\null\fi
            \item Does the program end?  Why or why not?
              \ifprintanswers\vskip -25pt\null\fi
              \begin{solution}[0.5in]
                Yes.  The \mintinline{cpp}|int| can only get so small.
              \end{solution}            
              \ifprintanswers\vskip -25pt\null\fi
          \end{enumerate}
        \end{minipage}
      \end{tabular}
      \ifprintanswers\vskip -30pt\null\fi
      
    \item The following step-by-step instructions will assist you in creating a program that prompts the user for a number
      between 1 and 10 (inclusively).  As long as the number is out of range, it re-prompts the user for a valid number.
      
      \begin{enumerate}[(a)]
        \item Write code to prompt the user for a number between 1 and 10 and store it in a variable.
          \begin{solution}[0.5in]
            \mintinline{cpp}|int number; cout "Enter a number between 1 and 10: "; cin >> number;|
          \end{solution}
          
        \item Write a {\bf Boolean expression} that is true exactly when the variable is {\bf not} between 1 and 10.
          \begin{solution}[0.5in]
            \mintinline{cpp}{(number < 1) || (number > 10)}
          \end{solution}
          
\newpage

        \item Use the Boolean expression you just created to write a {\bf while loop} that executes when the user input is out
          of range.  Leave the body of the while loop empty for now.
          \begin{solution}[1in]
            \mintinline{cpp}!while ((number < 1) || (number > 10)) { }!
          \end{solution}
          
        \item Write code to prompt the user to re-enter a valid number and store it in the same variable.
          \begin{solution}[1in]
            \mintinline{cpp}|cout << "Invalid! Enter a number between 1 and 10: "; cin >> number;|
          \end{solution}
          
        \item Finally, write code that prints out a message telling the user they entered a valid number.
          \begin{solution}[0.5in]
            \mintinline{cpp}|cout << "Congrats! You (finally) entered a valid number." << endl;|
          \end{solution}
        
        \item Now put all of the pieces together and replace the code in {\tt activity07b.cpp}.  Does your code work properly?
          If not, correct it and test again.        
      
      \end{enumerate}
      
    \item A {\it counter-controlled loop} is one for which the number of times
      the loop will execute is known ahead of time. Which of the loops in this
      model (questions 5, 7, 8, and 9) are counter-controlled?
      \begin{solution}[0.4in]
        The loops in 5, 7, and 8 are counter-controlled.  The loop in 9 is not.
      \end{solution}
      
    \item Enter and execute the following code in the file {\tt activity07b.cpp}.  Remember to 
      \mintinline{cpp}|#include <string>| at the top of your file.
      \par\vskip -20pt\null
      \begin{center}
        \small
        \begin{minipage}{5in}
          \begin{minted}[
            frame=lines,
            framesep=2mm,
            bgcolor=gray!15,
            baselinestretch=1.2,
            linenos,
            firstnumber=7
          ]{cpp}
  string word;
  char doAgain = 'y';
  while (doAgain == 'y') {
    cout << "Enter a word: ";
    cin >> word;
    cout << "The first letter is " << word.at(0) << endl;
    cout << "Type 'y' to enter another word, anything else to quit. ";
    cin >> doAgain;
  }
  cout << "Done!" << endl;  
          \end{minted}
          \par\vskip 10pt\null
        \end{minipage}
      \end{center}
      \par\vskip -30pt\null
      
      \begin{enumerate}[(a)]
        \item What does this program do?
          \begin{solution}[0.5in]
            It prints out the first letter of any word entered until the user enters `y'.
          \end{solution}
        \item What is the variable name used to store the user's input?
          \begin{solution}[0.5in]
            The word entered is stored in {\tt word} and the request to do again in {\tt doAgain}.
          \end{solution}
\newpage          
        \item What does \mintinline{cpp}|word.at(0)| represent?
          \begin{solution}[0.5in]
            The first character of the word entered.
          \end{solution}
        \item What happens if you change 0 to 1 in \mintinline{cpp}|word.at(0)|?
          \begin{solution}[0.5in]
            You would get the second character of the word entered.
          \end{solution}
        \item When will the program end?
          \begin{solution}[0.5in]
            When the user types anything other than `y' when prompted to continue.
          \end{solution}
          \par\vskip -30pt\null          
        \item A {\it sentinel-controlled loop} is one in which the loop body repeats
          until the user enters a\key\\[-2.5mm] particular value or values.  Why is 
          this an example of a {\it sentinel-controlled loop}?
          \begin{solution}[0.5in]
            Because it will keep going until the user decides not to.
          \end{solution}
      \end{enumerate}


  {\bf\large Model 3: Assignment Operators} \\[-10pt]
  \begin{center}
    \renewcommand{\arraystretch}{1.3}
    \begin{tabular}{|c|c|c|}
      \hline
      \rowcolor{orange!20} Initial {\tt x} Value & Assignment Operator & Final {\tt x} Value \\
      \hline
      2 & \mintinline{cpp}|x += 1;| & 3 \\
      \hline
      2 & \mintinline{cpp}|x -= 1;| & 1 \\
      \hline
      4 & \mintinline{cpp}|x *= 2;| & 8 \\
      \hline
      4 & \mintinline{cpp}|x /= 2;| & 2 \\
      \hline
      7 & \mintinline{cpp}|x %= 4;| & 3 \\
      \hline
    \end{tabular}
  \end{center}
  
  {\it\large Refer to Model 3 above as your team develops consensus answers
    to the questions below.}

    \item The code \mintinline{cpp}|x += 5;| is equivalent to which of the following lines of code?\par\vskip 10pt
      \hspace{20pt}
      \begin{oneparcheckboxes}
        \choice \mintinline{cpp}|x = 5;|
        \choice \mintinline{cpp}|x = y + 5;|
        \correctchoice \mintinline{cpp}|x = x + 5;|
        \choice \mintinline{cpp}|y = x + 5;|
      \end{oneparcheckboxes}
      
    \item An {\it assignment operator} provides a concise way of creating assignment statements when the variable on the
      left-hand side (LHS) will also appear on the right-hand side (RHS).  In your own words, describe what each of the 
      following assignment operators does.
      \par\vskip 20pt
      
      \begin{enumerate}[(a)]
        \itemsep 15pt
        \item \mintinline{cpp}|+=| \hspace{20pt} 
          \fillin[Adds the amount on the RHS to the variable on the LHS][5.25in]
        \item \mintinline{cpp}|-=| \hspace{20pt}
          \fillin[Subtracts the amount on the RHS from the variable on the LHS][5.25in]
        \item \mintinline{cpp}|*=| \hspace{20pt} 
          \fillin[Multiplies the variable on the LHS by the amount on the RHS][5.25in]
        \item \mintinline{cpp}|/=| \hspace{20pt} 
          \fillin[Divides the variable on the LHS by the amount on the RHS][5.25in]
        \item \mintinline{cpp}|%=| \hspace{20pt} 
          \fillin[Adds the amount on the RHS to the variable on the LHS][5in]
      \end{enumerate}
      
\newpage

    \item The table below is similar to that seen in Model 3.  Fill in the missing pieces with appropriate values or
      assignment operator statements.  Assume that {\tt x} is an integer variable.
      
      \begin{center}
        \renewcommand{\arraystretch}{1.5}
        \begin{tabular}{|c|c|c|c|}
          \hline
          \rowcolor{orange!20} Operator & Initial {\tt x} Value & Statement & Final {\tt x} Value \\
          \hline
          \mintinline{cpp}|+=| & 6 & \ifprintanswers\mintinline{cpp}|x += 1;|\fi & 8 \\
          \hline
          \mintinline{cpp}|+=| & 5 & \mintinline{cpp}|x -= x+1;| & \ifprintanswers 11\fi \\
          \hline
          \mintinline{cpp}|-=| & \ifprintanswers 9\fi  & \mintinline{cpp}|x -= 3;| & 6 \\
          \hline
          \mintinline{cpp}|*=| & 4 & \ifprintanswers\mintinline{cpp}|x *= 4;|\fi & 4 \\
          \hline
          \mintinline{cpp}|/=| & 23 & \ifprintanswers\mintinline{cpp}|x /= 10;|\fi & 2 \\
          \hline
        \end{tabular}
      \end{center}
      \par\vskip -40pt\null
      
    \item Is the assignment operator \mintinline{cpp}|23 += total|
    valid?  Why or why not?\key
      \begin{solution}[0.5in]
        No, it is not valid.  There must be only a variable name on the left-hand side.
      \end{solution}
      
    \item The following code snippet should print the numbers beginning with 100 and counting down to 1.  However, it is missing a
      line of code.  Add the missing code using an assignment operator.  Indicate the line number at which the code should be inserted.
      \par\vskip 10pt
      \begin{tabular}{p{2.4in}p{3.4in}}
        \begin{minipage}{2.4in}
          \begin{minted}[
            frame=lines,
            framesep=2mm,
            bgcolor=gray!15,
            baselinestretch=1.2,
            linenos
          ]{cpp}
  int countdown = 100;
  while (countdown > 0) {
    cout << countdown << endl;
  }
          \end{minted}
          \par\vskip 10pt\null
        \end{minipage}
        &
        \begin{minipage}{3.4in}
          \begin{solution}[0.5in]
            \par
            Insert the command \mintinline{cpp}|x -= 1;| between lines 3 and 4.
          \end{solution}            
        \end{minipage}
      \end{tabular}
      \par\vskip -40pt\null
      
    \item Use assignment operators to write loops to do each of the following.  
      You can test your code in the {\tt activity07c.cpp} file.
      
      \begin{enumerate}[(a)]
        \item A loop that prints out all multiples of three between 0 and 100. (i.e. 0, 3, 6, 9, {\ldots} )
          \begin{solution}[1.5in]
            \begin{center}
              \scriptsize
              \begin{minipage}{3in}
                \begin{minted}[
                  frame=lines,
                  framesep=2mm,
                  bgcolor=gray!15,
                  baselinestretch=1.2,
                  linenos
                ]{cpp}
  int num = 0;
  while (num <= 100) {
    cout << num << endl;
    num += 3;
  }
                \end{minted}
              \end{minipage}
            \end{center}
          \end{solution}
        \item A loop that prints all powers of 2 (i.e. $2^n$) from 1 up to 100.
          \begin{solution}[1.5in]
            \begin{center}
              \scriptsize
              \begin{minipage}{3in}
                \begin{minted}[
                  frame=lines,
                  framesep=2mm,
                  bgcolor=gray!15,
                  baselinestretch=1.2,
                  linenos
                ]{cpp}
  int num = 1;
  while (num <= 100) {
    cout << num << endl;
    num *= 2;
  }
                \end{minted} 
              \end{minipage}
           \end{center}              
          \end{solution}
      \end{enumerate}

  \end{enumerate}  
    
\end{document}
