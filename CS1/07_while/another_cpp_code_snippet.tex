  {\bf\large Model 2: Another C++ Code Snippet} \\[-10pt]
  \ifprintanswers\vskip -30pt\null\fi
  \begin{center}
    \small
    \begin{minipage}{5.5in}
      \begin{minted}[
        frame=lines,
        framesep=2mm,
        bgcolor=gray!15,
        baselinestretch=1.2,
        linenos,
        firstnumber=7
      ]{cpp}
  cout << "Enter a positive integer: ";
  cin >> number;
  int x = 1;
  while (x <= number) {
    if (x % 10 == 0 ) {
      cout << setw(2) << x << endl;
    } else {
      cout << setw(2) << x << " ";
    }
    x = x + 1;
  }
      \end{minted}
    \end{minipage}
  \end{center}
  \TPMargin{5pt}
  \ifprintanswers\vskip -20pt\null\fi
  

  {\it\large Refer to Model 2 above as your team develops consensus answers
    to the questions below.}
    \par\vskip 5pt

    \item Explain what each line or range of lines in this code snippet does.
      \par\vskip 10pt
      \begin{enumerate}[(a)]
        \itemsep 15pt
        \item Lines 7-8: \hfill \fillin[Prompts the user for a number and saves it in {\tt number}][5.15in]
        \item Line 9:    \hfill \fillin[Initializes the counter {\tt x} to 1][5.15in]
        \item Line 10:    \hfill \fillin[Repeats lines 5-10 as long as the counter {\tt x} is not bigger than {\tt number}][5.15in]
        \item Lines 11-15: \hfill \fillin[Prints the counter, with a new line when it is divisible by 10][5.15in]
        \item Line 16:   \hfill \fillin[Increments the counter by one before going back to line 4][5.15in]
      \end{enumerate}
      
\newpage

    \item The complete program can be found in {\tt activity07b.cpp}.  Run it and answer the following questions.
      \begin{enumerate}[(a)]
        \item What is the output when you enter the number 5?  How many times did the loop execute?
          \begin{solution}[0.5in]
            The loop executed five times.  The output is:\par
            \mintinline{html}| 1  2  3  4  5|
          \end{solution}
        \item What is the output when you enter the number 25? How many times did the loop execute?
          \begin{solution}[0.5in]
            The loop executed 25 times.  The output is:\par
            \mintinline{html}| 1  2  3  4  5  6  7  8  9 10|\\
            \mintinline{html}|11 12 13 14 15 16 17 18 19 20|\\
            \mintinline{html}|21 22 23 24 25|
          \end{solution}
        \item Unlike the loop in model 1, this loop prints the counter.  How might this make testing easier?
          \begin{solution}[0.5in]
            Debugging is easier as you can see the counter values each time through the loop.
          \end{solution}
      \end{enumerate}

    \item The following code snippet should print the numbers from 1 to 10, but it doesn't print anything.  Correct the
      problem.  Replace the contents of {\tt activity07b.cpp} to help in your debugging.
      \ifprintanswers\vskip -15pt\null\else\par\vskip 10pt\fi
      \begin{tabular}{p{2.4in}p{3.4in}}
        \begin{minipage}{2.4in}
          \begin{minted}[
            frame=lines,
            framesep=2mm,
            bgcolor=gray!15,
            baselinestretch=1.2
          ]{cpp}
  int number = 12;
  while (number <= 10) {
    cout << number << endl;
    number = number + 1;
  }
          \end{minted}
        \end{minipage}
        &
        \begin{minipage}{3.4in}
          \begin{solution}[1.5in]
            \par
            The counter should be initialized to 1 instead of 12 on
            line 1 of the code snippet.
          \end{solution}
        \end{minipage}
      \end{tabular}
      \ifprintanswers\vskip -25pt\null\fi
      
    \item Enter and execute the code below, then answer the questions that follow.
      \ifprintanswers\vskip -15pt\null\else\par\vskip 10pt\fi
      \begin{tabular}{p{2.4in}p{3.4in}}
        \begin{minipage}{2.4in}
          \begin{minted}[
            frame=lines,
            framesep=2mm,
            bgcolor=gray!15,
            baselinestretch=1.2
          ]{cpp}
  int number = 0;
  while (number <= 10) {
    cout << number << endl;
    number = number - 1;
  }
          \end{minted}
          \par\vskip 10pt\null
        \end{minipage}
        &
        \begin{minipage}{3.4in}
          \begin{enumerate}[(a)]
            \item Describe the output.
              \ifprintanswers\vskip -25pt\null\fi
              \begin{solution}[0.5in]
                Prints decreasing numbers.
              \end{solution}
              \ifprintanswers\vskip -25pt\null\fi
            \item Does the program end?  Why or why not?
              \ifprintanswers\vskip -25pt\null\fi
              \begin{solution}[0.5in]
                Yes.  The \mintinline{cpp}|int| can only get so small.
              \end{solution}            
              \ifprintanswers\vskip -25pt\null\fi
          \end{enumerate}
        \end{minipage}
      \end{tabular}
      \ifprintanswers\vskip -30pt\null\fi
      
    \item The following step-by-step instructions will assist you in creating a program that prompts the user for a number
      between 1 and 10 (inclusively).  As long as the number is out of range, it re-prompts the user for a valid number.
      
      \begin{enumerate}[(a)]
        \item Write code to prompt the user for a number between 1 and 10 and store it in a variable.
          \begin{solution}[0.5in]
            \mintinline{cpp}|int number; cout "Enter a number between 1 and 10: "; cin >> number;|
          \end{solution}
          
        \item Write a {\bf Boolean expression} that is true exactly when the variable is {\bf not} between 1 and 10.
          \begin{solution}[0.5in]
            \mintinline{cpp}{(number < 1) || (number > 10)}
          \end{solution}
          
\newpage

        \item Use the Boolean expression you just created to write a {\bf while loop} that executes when the user input is out
          of range.  Leave the body of the while loop empty for now.
          \begin{solution}[1in]
            \mintinline{cpp}!while ((number < 1) || (number > 10)) { }!
          \end{solution}
          
        \item Write code to prompt the user to re-enter a valid number and store it in the same variable.
          \begin{solution}[1in]
            \mintinline{cpp}|cout << "Invalid! Enter a number between 1 and 10: "; cin >> number;|
          \end{solution}
          
        \item Finally, write code that prints out a message telling the user they entered a valid number.
          \begin{solution}[0.5in]
            \mintinline{cpp}|cout << "Congrats! You (finally) entered a valid number." << endl;|
          \end{solution}
        
        \item Now put all of the pieces together and replace the code in {\tt activity07b.cpp}.  Does your code work properly?
          If not, correct it and test again.        
      
      \end{enumerate}
      
    \item A {\it counter-controlled loop} is one for which the number of times
      the loop will execute is known ahead of time. Which of the loops in this
      model (questions 5, 7, 8, and 9) are counter-controlled?
      \begin{solution}[0.4in]
        The loops in 5, 7, and 8 are counter-controlled.  The loop in 9 is not.
      \end{solution}
      
    \item Enter and execute the following code in the file {\tt activity07b.cpp}.  Remember to 
      \mintinline{cpp}|#include <string>| at the top of your file.
      \par\vskip -20pt\null
      \begin{center}
        \small
        \begin{minipage}{5in}
          \begin{minted}[
            frame=lines,
            framesep=2mm,
            bgcolor=gray!15,
            baselinestretch=1.2,
            linenos,
            firstnumber=7
          ]{cpp}
  string word;
  char doAgain = 'y';
  while (doAgain == 'y') {
    cout << "Enter a word: ";
    cin >> word;
    cout << "The first letter is " << word.at(0) << endl;
    cout << "Type 'y' to enter another word, anything else to quit. ";
    cin >> doAgain;
  }
  cout << "Done!" << endl;  
          \end{minted}
          \par\vskip 10pt\null
        \end{minipage}
      \end{center}
      \par\vskip -30pt\null
      
      \begin{enumerate}[(a)]
        \item What does this program do?
          \begin{solution}[0.5in]
            It prints out the first letter of any word entered until the user enters `y'.
          \end{solution}
        \item What is the variable name used to store the user's input?
          \begin{solution}[0.5in]
            The word entered is stored in {\tt word} and the request to do again in {\tt doAgain}.
          \end{solution}
\newpage          
        \item What does \mintinline{cpp}|word.at(0)| represent?
          \begin{solution}[0.5in]
            The first character of the word entered.
          \end{solution}
        \item What happens if you change 0 to 1 in \mintinline{cpp}|word.at(0)|?
          \begin{solution}[0.5in]
            You would get the second character of the word entered.
          \end{solution}
        \item When will the program end?
          \begin{solution}[0.5in]
            When the user types anything other than `y' when prompted to continue.
          \end{solution}
          \par\vskip -30pt\null          
        \item A {\it sentinel-controlled loop} is one in which the loop body repeats
          until the user enters a\key\\[-2.5mm] particular value or values.  Why is 
          this an example of a {\it sentinel-controlled loop}?
          \begin{solution}[0.5in]
            Because it will keep going until the user decides not to.
          \end{solution}
      \end{enumerate}
