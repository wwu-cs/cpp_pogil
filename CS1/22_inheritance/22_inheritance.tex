\documentclass{exam}
%\documentclass[answers]{exam}
\hbadness=99999
\usepackage[total={6.5in,9in}]{geometry}

\usepackage{enumerate}
\usepackage{amsmath}
\usepackage[table]{xcolor}
\usepackage{graphicx}
\usepackage{tikz}
%\usepackage{pgfplots}
\usepackage{multicol}

% for syntax highlighting
\usepackage{minted}
\usemintedstyle[cpp]{xcode}

% for overlay of output
\usepackage[overlay,showboxes]{textpos}

\pagestyle{plain}

\setlength\columnsep{50pt}
\newcommand{\key}{\hfill
      \raisebox{-.3\height}{\includegraphics[width=0.6in]{figures/key.png}}}

\begin{document}
  \thispagestyle{empty}
  \setlength{\parindent}{0pt}

  \begin{center}
    \Large Activity \#3: Inheritance \\[5pt]
    \large Recorder's Report\\[20pt]
    \normalsize
    \begin{tabular}{lrp{0.1in}lr}
      Manager:  & \fillin[][2.0in] & & Presenter: & \fillin[][2.0in]\\[15pt]
      Recorder: & \fillin[][2.0in] & & Driver:    & \fillin[][2.0in]\\[15pt]
      Date:     & \fillin[][2.0in] & & Score:     & Satisfactory \hspace{10pt} /
      \hspace{10pt} Not Satisfactory
    \end{tabular}
  \end{center}
  \par\vskip 15pt
  
  Record your group's answers to the key questions (marked with
  \raisebox{-.3\height}{\includegraphics[width=0.5in]{figures/key.png}})
  below.
  \begin{enumerate}[(a)]
    \itemsep 1.75in
    \item Model 1, Question \#4
    \item Model 2, Question \#7.d
    \item Model 2, Question \#10.a
  \end{enumerate}

  \clearpage\pagenumbering{arabic} 
  
  \begin{center}
    \Large Activity \#3: Inheritance \\[5pt]
    \large Activity Guide\\[20pt]
  \end{center}

  \begin{center}
    \fbox{
      \begin{minipage}{5.5in}
        {\bf Learning Objectives:} Students will be able to:
        \begin{itemize}
          \item Content:\\[-20pt]
            \begin{itemize}           
              \itemsep 0pt
              \item Explain the difference between ``Is-a'' and
                ``Has-a'' relationships
              \item Explain how derived classes and base classes and
                their members are related
            \end{itemize}
          \item Process\\[-20pt]
            \begin{itemize}
              \itemsep 0pt
              \item Write a member function which overrides the base class's function\\[-5pt]
            \end{itemize}
        \end{itemize}
      \end{minipage}
      }
  \end{center}
  \par\vskip 10pt
  
  
  {\bf\large Model 1: Several Classes from an University Information System}\\[-10pt]
  \begin{center}
    \footnotesize
    \renewcommand{\arraystretch}{1.2}
    \begin{tabular}{p{1.8in}p{1.8in}p{1.8in}}
      \begin{minipage}{1.7in}
        \begin{tabular}{|l|}
          \hline
          \rowcolor{orange!20} Address \\
          \hline
          \rowcolor{white}
          \begin{minipage}{\textwidth}
            \vskip 2pt\null\tt
            -number:string\\
            -street:string\\
            -city:string\\
            -state:string\\
            -zip:int\\
          \end{minipage} \\
          \hline
          \begin{minipage}{\textwidth}
            \vskip 2pt\null\tt
            +getNumber():string\\
            +setNumber(num:string):void\\
          \end{minipage} \\[-6pt]
          \hfill $\vdots$\\
          \hline
        \end{tabular}
      \end{minipage}
      & 
      \begin{minipage}{1.7in}
        \begin{tabular}{|l|}
          \hline
          \rowcolor{orange!20}Person \\
          \hline
          \rowcolor{white}
          \begin{minipage}{\textwidth}
            \vskip 2pt\null\tt
            -name:string\\
            -addr:Address\\
            -phone:string\\
          \end{minipage} \\
          \hline
          \begin{minipage}{\textwidth}
            \vskip 2pt\null\tt
            +setName(name:string):void\\
            +getName():string\\
            +setAddress(a:Address):void\\
            +getAddress():Address\\
          \end{minipage} \\[-6pt]
          \hfill $\vdots$\\
          \hline
        \end{tabular}
      \end{minipage}
      &
      \begin{minipage}{1.7in}
        \begin{tabular}{|l|}
          \hline
          \rowcolor{orange!20}Student (extends Person) \\
          \hline
          \rowcolor{white}
          \begin{minipage}{\textwidth}
            \vskip 2pt\null\tt
            -classStanding:string\\
            -studentID:string\\
            -GPA:float\\
          \end{minipage} \\
          \hline
          \begin{minipage}{\textwidth}
            \vskip 2pt\null\tt
            +setClass(class:string):void\\
            +getClass():string\\
            +setStudentID(id:string):void\\
            +getStudentID():string\\
          \end{minipage} \\[-6pt]
          \hfill $\vdots$\\
          \hline
        \end{tabular}
      \end{minipage}
    \end{tabular}
  \end{center}
  \par\vskip 5pt
  
  {\it\large Refer to Model 1 above as your group develops consensus answers
    to the questions below.}
    \par\vskip 10pt
    
  \begin{enumerate}
    \itemsep 20pt
    
    \item Large programs often contain many different classes.  These classes will
      frequently be related to each other through ``has-a''
      relationship, where one class has a data member that is itself
      an object of another class.  Another less-common relationship is
      an ``is a'' relationship, where one class is a more general
      class, while the second is a more specialized version of the
      original class.
      \par\vskip 10pt
      
      \begin{enumerate}[(a)]
        \item Which class is used as a data member type inside another class?
          \begin{solution}[0.5in]
            The {\tt Address} class is used as a variable type inside
            the {\tt Person} class.
          \end{solution}
        \item Is this an example of a ``has-a'' or an ``is-a''
          relationship?  Explain.
          \begin{solution}[0.75in]
            It is a ``has-a'' relationship because the Person class
            has a variable that is an Address object.
          \end{solution}
        \item Which two classes have an ``is-a'' relationship?  Explain.
          \begin{solution}[0.75in]
            The {\tt Person} and {\tt Student} classes have an
            ``is-a'' relationship because every Student is a Person.
          \end{solution}
        \item Of the classes indicated above, which is the more
          general and which is the more specialized?
          \begin{solution}[0.5in]
            The {\tt Person} class is more general and the {\tt
            Student} class is more specialized.
          \end{solution}
        \item Would the member functions shown for the more general class be
          appropriate in the specialized class as well?  Explain.
          \begin{solution}[0.75in]
            Yes -- every student still has a name, address, and phone
            number that can be manipulated.            
          \end{solution}
        \item Give an example not shown in the model of a member function that
          might appear in the specialized class but not in the more
          general class.
          \begin{solution}[0.75in]
            Answers vary, but examples might include things like
            {\tt getAdvisor()} or {\tt isOnHonorRoll()}. 
          \end{solution}
      \end{enumerate}
      
    \item Explain why the relationship between {\tt Person} and {\tt
      Address} is not an ``is-a'' relationship.
      \begin{solution}[0.75in]
        A Person is not an address, but a person has an address.
      \end{solution}

    \item As a group, come up with another example of a ``has-a''
      relationship.  Fill in the information below about your example.
      \par\vskip 15pt
      
      \begin{enumerate}[(a)]
        \itemsep 15pt
        \item What class is used as a variable inside the other?
          \hfill \fillin[Color, Name, Circle][3in]
        \item What class is the variable inside of?
          \hfill \fillin[Pen, Animal, Drawing][3in]
        \item Fill in the blanks below to describe the relationship
          between your two classes.\par\vskip 10pt
          \begin{center}
            \it Every \fillin[Pen][2in] has a(n) \fillin[Color][2in].
          \end{center}
      \end{enumerate}
      
    \item As a group, come up with another example of an ``is-a''
      relationship.  Fill in the\key\\[-2.5mm] information below about your example.      
      \par\vskip 15pt
      
      \begin{enumerate}[(a)]
        \itemsep 15pt
        \item What class is the more general class?
          \hfill \fillin[Cat, Pen][3in]
        \item What class is the specialized class?
          \hfill \fillin[Animal, Writing Implement][3in]
        \item Fill in the blanks below to describe the relationship
          between your two classes.\par\vskip 10pt
          \begin{center}
            \it Every \fillin[Cat][2in] is a(n) \fillin[Animal][2in].
          \end{center}
      \end{enumerate}
      
\newpage

  {\bf\large Model 2: Another ``Is-A'' Relationship}\\[-30pt]
  \begin{center}
    \small
    \begin{tabular}{p{2.8in}p{0.1in}p{2.8in}}
      \begin{minipage}{2.8in}
        \begin{minted}[
          frame=lines,
          framesep=2mm,
          bgcolor=gray!15,
          baselinestretch=1.2,
          linenos,
          firstnumber=5
        ]{cpp}
// class for a deposit account
//   at a bank
class Account {
  public:
    Account(double balance);
    void deposit(double amount);
    virtual void withdraw(double amount);
    double getBalance() const {
      return balance;
    };
  private:
    double balance = 0;
};
        \end{minted}      
      \end{minipage}
      & &
      \begin{minipage}{2.8in}
        \begin{minted}[
          frame=lines,
          framesep=2mm,
          bgcolor=gray!15,
          baselinestretch=1.2,
          linenos,
          firstnumber=19
        ]{cpp}       
// class for a savings account        
class Savings : public Account {
  public:
    Savings(double balance, double rate);
    void setRate(double rate);
    void newMonth();
    int getWithdrawals() const {
      return numWithdrawals;
    };   
  private:
    double rate;
    int numWithdrawals;
};
        \end{minted}      
      \end{minipage}
    \end{tabular}
  \end{center}
  
  {\it\large Refer to Model 2 above as your group develops consensus answers
    to the questions below.}
  \par\vskip 10pt
    
  \item In C++ the more general class in an ``is-a'' relationship
    is called the {\it base} class and the specific class is called
    the {\it derived} class.  Answer the following questions about
    the model above.
    \par\vskip 15pt
    
    \begin{enumerate}
      \itemsep 15pt
      \item What is the name of the base class? 
        \hfill \fillin[\tt Account][3in]
      \item What is the name of the derived class?
        \hfill \fillin[\tt Account][3in]
      \item In C++ class definitions, how is the ``is-a'' relationship indicated?
        \begin{solution}[0.5in]
          By a colon, the keyword public (for now), and the name
          of the base class. The example from the model is:
          \begin{center}
            \mintinline{cpp}|class Savings : public Account|
          \end{center}
        \end{solution}
    \end{enumerate}
    
  \item An important property of derived classes is that they 
    {\it inherit} the properties of the base class.  Suppose 
    that a new savings account was declared using the code:
    \mintinline{cpp}|Savings myAccount(50,0.01)|.
    \par\vskip 10pt
    
    \begin{enumerate}
      \item What data members does the object {\tt myAccount} have?
        \begin{solution}[0.5in]
          The members are: {\tt rate}, {\tt numWithdrawals}, and
          {\tt balance}.
        \end{solution}
      \item In which class is each of these data members declared?
        \begin{solution}[0.5in]
          {\tt rate} and {\tt numWithdrawals} are declared in the
          {\tt Savings} class and {\tt balance} is declared in the
          {\tt Account} class.
        \end{solution}
      \item What member functions does the object {\tt myAccount} have?
        \begin{solution}[0.5in]
          The members functions are: {\tt deposit()}, {\tt
          withdraw()}, {\tt getBalance()}, {\tt setRate()}, {\tt
          newMonth()}, and {\tt getWithdrawals()}.
        \end{solution}
      \item Which of these are inherited from the base class?
        \begin{solution}[0.5in]
          The members functions {\tt deposit()}, {\tt
          withdraw()}, and {\tt getBalance()} are inherited from
          the {\tt Account} class.
        \end{solution}
      \end{enumerate}
      
    \item The file {\tt activity03.cpp} contains a preliminary
      implementation of these two classes along with a sample {\tt main}
      function.  Use it to answer the following questions.
      \par\vskip 10pt
      
      \begin{enumerate}
        \itemsep 10pt
        \item Without attempting to compile the code, describe what
          the member function {\tt Savings::newMonth()} on line 94 is
          supposed to do.
          \begin{solution}[0.75in]
            It is supposed to set the number of withdrawals to 0 and
            deposit the account balance times the interest rate into
            the account (i.e. pay monthly interest).
          \end{solution}
        \item Now compile the code and describe what happens.
          \begin{solution}[0.5in]
            We get an error that says:
            \begin{center}
              \tt
              `double Account::balance' is private within this context
            \end{center}
          \end{solution}
        \item Replace {\tt this->balance} on line 96 with {\tt this->getBalance()}
          Does this fix the problem? \fillin[Yes][0.5in]
          \vskip -25pt\null
        \item Based on what you've observed in this question, does the
          derived class {\tt Savings} have\key\\[-2.5mm] direct access to all
          of the members that it inherits from the base {\tt Account} class?
          Explain.
          \begin{solution}[0.5in]
            No, it only has access to the public members that it inherits.
          \end{solution}
      \end{enumerate}
        
    \item The function {\tt testAccount()} defined in {\tt activity03.cpp}
      tests our implementation of these classes by depositing \$50 and
      then attempting to make ten \$10 withdrawals.  
      \begin{enumerate}
        \itemsep 15pt
        \item What are the results of running on this function on the
          checking account with an initial balance of \$100, as
          defined in the {\tt main} program?
          \begin{solution}[0.75in]
            It successfully deposits \$50, then makes the ten
            withdrawals, leaving a final balance of \$50.
          \end{solution}
        \item What are the results of running this function twice in
          successive months on the savings account with an initial 
          balance of \$80, an interest rate of 1\%?
          \begin{solution}[0.75in]
            It successfully deposits \$50, then makes the ten
            withdrawals, leaving a final balance of \$30 after the first
            month.  Then \$0.53 interest is added, another \$50 is
            deposited, and eight \$10 withdrawals are successfully
            made.  The last two \$10 withdrawals fail because of
            insufficient funds, leaving us with \$0.30 as a final
            balance.
          \end{solution}
        \item What type is the {\tt testAccount} function's single parameter?
          \hfill \fillin[\tt Account\&][2in]
        \item What type is the argument passed to this function on line 55?
          \hfill \fillin[\tt Account][2in]
        \item What type is the argument passed to this function on line 58?
          \hfill \fillin[\tt Savings][2in]
        \item Why is this worth noting?
          \begin{solution}[0.5in]
            We used the same function for objects of the base and
            derived class types.
          \end{solution}          
      \end{enumerate}
        
    \item In the United States, you are not allowed to make more than
      six withdrawals from a savings account in one month. Based on
      this new information, are any of the member functions that
      {\tt Savings} inherits from {\tt Account} not quite appropriate?
      Explain what is missing from these functions.
      
      \begin{solution}[0.75in]
        Yes.  The {\tt withdraw} function should impose this limit and
        refuse to withdraw money if there have already been 6 withdraws
        from the account.
      \end{solution}
      
    \item We can {\it override} a member function from the base class
      by defining a member function the same name, 
      parameters, and return type in the derived class.
      
      \begin{enumerate}
        \item How is this different from {\it overloading} a function? \key
          \begin{solution}[0.75in]
            Overloading a function means that we define functions with
            the same name, but different signatures.  When we override
            a function, we use the same signature (and return type).
          \end{solution}
        \item Define a \mintinline{cpp}|bool Savings::withdraw(double balance)|
          function that overrides the {\tt withdraw()} function from
          the {\tt Account} class and does not allow more than 6 withdrawals.
          Hint: You can use {\tt Account::withdraw()} to call the base class
          {\tt withdraw()} from within your new definition.
          \begin{solution}[1.5in]
            \begin{minipage}{3in}
              \begin{minted}[
                frame=lines,
                framesep=2mm,
                bgcolor=gray!15,
                baselinestretch=1.2,
                linenos,
              ]{cpp}
bool Savings::withdraw(double amount) {
  if(this->numWithdrawals < 6 && Account::withdraw(amount)) {
    this->numWithdrawals++;
    return true;
  }
  return false;
}
                      
              \end{minted}
            \end{minipage}
          \end{solution}
        \item What did you have to add to the declaration of the {\tt
          Savings} class?
          \begin{solution}[0.5in]
            A declaration of the \mintinline{cpp}{bool withdraw(double)} function.
          \end{solution}
        \item Now run the program again.  How did the results of the
          savings account test change?
          \begin{solution}[1in]
            Now we can only withdraw the first six \$10 amounts in the
            first month leaving us with a balance of \$70.00 at the end
            of that month.  The same thing happens in the second month,
            leaving us with a final balance of \$60.70.
          \end{solution}
        \item What happens if you remove the keyword {\tt virtual}
          from line 11?
          \begin{solution}[0.5in]
            The program goes back to functioning as before.  This is
            because the function {\tt testAccount} calls the base
            version of {\tt withdraw} since its parameter is of type
            {\tt Account}.  We'll learn more about this notion of {\it
            Polymorphism} in later class periods.
          \end{solution}
      \end{enumerate}

  \end{enumerate}
  
  
    
\end{document}
