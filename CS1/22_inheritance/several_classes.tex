\model{Several Classes from an University Information System}
  \begin{center}
    \scriptsize
    \renewcommand{\arraystretch}{1.2}
    \begin{tabular}{p{1.8in}p{1.8in}p{1.8in}}
      \begin{minipage}{1.7in}
        \begin{tabular}{|l|}
          \hline
          \rowcolor{orange!20} Address \\
          \hline
          \rowcolor{white}
          \begin{minipage}{\textwidth}
            \vskip 2pt\null\tt
            -number:string\\
            -street:string\\
            -city:string\\
            -state:string\\
            -zip:int\\
          \end{minipage} \\
          \hline
          \begin{minipage}{\textwidth}
            \vskip 2pt\null\tt
            +getNumber():string\\
            +setNumber(num:string):void\\
          \end{minipage} \\[-6pt]
          \hfill $\vdots$\\
          \hline
        \end{tabular}
      \end{minipage}
      & 
      \begin{minipage}{1.7in}
        \begin{tabular}{|l|}
          \hline
          \rowcolor{orange!20}Person \\
          \hline
          \rowcolor{white}
          \begin{minipage}{\textwidth}
            \vskip 2pt\null\tt
            -name:string\\
            -addr:Address\\
            -phone:string\\
          \end{minipage} \\
          \hline
          \begin{minipage}{\textwidth}
            \vskip 2pt\null\tt
            +setName(name:string):void\\
            +getName():string\\
            +setAddress(a:Address):void\\
            +getAddress():Address\\
          \end{minipage} \\[-6pt]
          \hfill $\vdots$\\
          \hline
        \end{tabular}
      \end{minipage}
      &
      \begin{minipage}{1.7in}
        \begin{tabular}{|l|}
          \hline
          \rowcolor{orange!20}Student (extends Person) \\
          \hline
          \rowcolor{white}
          \begin{minipage}{\textwidth}
            \vskip 2pt\null\tt
            -classStanding:string\\
            -studentID:string\\
            -GPA:float\\
          \end{minipage} \\
          \hline
          \begin{minipage}{\textwidth}
            \vskip 2pt\null\tt
            +setClass(class:string):void\\
            +getClass():string\\
            +setStudentID(id:string):void\\
            +getStudentID():string\\
          \end{minipage} \\[-6pt]
          \hfill $\vdots$\\
          \hline
        \end{tabular}
      \end{minipage}
    \end{tabular}
  \end{center}
  
  {\it\large Refer to Model 1 above as your team develops consensus answers
    to the questions below.}

  \quest{25 min}
    
  \Q Large programs often contain many different classes.  These classes will
    frequently be related to each other through ``has-a''
    relationship, where one class has a data member that is itself
    an object of another class.  Another less-common relationship is
    an ``is a'' relationship, where one class is a more general
    class, while the second is a more specialized version of the
    original class.
    \par\vskip 10pt
    
    \begin{enumerate}
      \item Which class is used as a data member type inside another class?
        \begin{answer}[0.5in]
          The {\tt Address} class is used as a variable type inside
          the {\tt Person} class.
        \end{answer}

      \item Is this an example of a ``has-a'' or an ``is-a''
        relationship?  Explain.
        \begin{answer}[0.75in]
          It is a ``has-a'' relationship because the Person class
          has a variable that is an Address object.
        \end{answer}

      \item Which two classes have an ``is-a'' relationship?  Explain.
        \begin{answer}[0.75in]
          The {\tt Person} and {\tt Student} classes have an
          ``is-a'' relationship because every Student is a Person.
        \end{answer}

      \item Of the classes indicated above, which is the more
        general and which is the more specialized?
        \begin{answer}[0.5in]
          The {\tt Person} class is more general and the {\tt
          Student} class is more specialized.
        \end{answer}

      \item Would the member functions shown for the more general class be
        appropriate in the specialized class as well?  Explain.
        \begin{answer}[0.75in]
          Yes -- every student still has a name, address, and phone
          number that can be manipulated.            
        \end{answer}

      \item Give an example not shown in the model of a member function that
        might appear in the specialized class but not in the more
        general class.
        \begin{answer}[0.75in]
          Answers vary, but examples might include things like
          {\tt getAdvisor()} or {\tt isOnHonorRoll()}. 
        \end{answer}
    \end{enumerate}

  \vskip -20pt
    
  \Q Explain why the relationship between {\tt Person} and {\tt
    Address} is not an ``is-a'' relationship.
    \begin{answer}[0.75in]
      A Person is not an address, but a person has an address.
    \end{answer}

  \Q As a group, come up with another example of a ``has-a''
    relationship.  Fill in the information below about your example.
    \begin{enumerate}
      \itemsep 10pt
      \item What class is used as a variable inside the other?
        \hfill \ans[3in]{Color, Name, Circle}

      \item What class is the variable inside of?
        \hfill \ans[3in]{Pen, Animal, Drawing}

      \item Fill in the blanks below to describe the relationship
        between your two classes.\par\vskip 10pt
        \begin{center}
          \it Every \ans[2in]{Pen} has a(n) \ans[2in]{Color}
        \end{center}
    \end{enumerate}
  
  \vskip -30pt
  \Q As a group, come up with another example of an ``is-a''
    relationship.  Fill in the\key\\[-2.5mm] information below about your example.      
    \par\vskip 10pt
    
    \begin{enumerate}
      \itemsep 10pt
      \item What class is the more general class?
        \hfill \ans[3in]{Cat, Pen}

      \item What class is the specialized class?
        \hfill \ans[3in]{Animal, Writing Implement}
        
      \item Fill in the blanks below to describe the relationship
        between your two classes.\par\vskip 10pt
        \begin{center}
          \it Every \ans[2in]{Cat} is a(n) \ans[2in]{Animal}
        \end{center}
    \end{enumerate}