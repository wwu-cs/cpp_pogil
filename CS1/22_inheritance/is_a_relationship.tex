\model{Another ``Is-A'' Relationship}
  \begin{center}
    \small
    \begin{tabular}{p{2.8in}p{0.1in}p{3.2in}}
      \begin{minipage}{2.8in}
        \begin{cpplst}
// class for a deposit account
//   at a bank
class Account {
  public:
    Account(double balance);
    void deposit(double amount);
    virtual void withdraw(double amount);
    double getBalance() const {
      return balance;
    };
  private:
    double balance = 0;
};
        \end{cpplst}      
      \end{minipage}
      & &
      \begin{minipage}{3.2in}
        \begin{cpplst}       
// class for a savings account        
class Savings : public Account {
  public:
    Savings(double balance, double rate);
    void setRate(double rate);
    void newMonth();
    int getWithdrawals() const {
      return numWithdrawals;
    };   
  private:
    double rate;
    int numWithdrawals;
};
        \end{cpplst}      
      \end{minipage}
    \end{tabular}
  \end{center}
  
  {\it\large Refer to Model 2 above as your group develops consensus answers
    to the questions below.}

  \quest{25 min}
    
  \Q In C++ the more general class in an ``is-a'' relationship
    is called the {\it base} class and the specific class is called
    the {\it derived} class.  Answer the following questions about
    the model above.
    \begin{enumerate}
      \itemsep 10pt
      \item What is the name of the base class? 
        \hfill \ans[3in]{\tt Account}

      \item What is the name of the derived class?
        \hfill \ans[3in]{\tt Savings}

      \item In C++ class definitions, how is the ``is-a'' relationship indicated?
        \begin{answer}[0.5in]
          By a colon, the keyword public (for now), and the name
          \cpp{class Savings : public Account}
        \end{answer}
    \end{enumerate}

  \vskip -20pt
    
  \Q An important property of derived classes is that they 
    {\it inherit} the properties of the base class.  Suppose 
    that a new savings account was declared using the code:
    \cpp{Savings myAccount(50,0.01)}.    
    \begin{enumerate}
      \item What data members does the object {\tt myAccount} have?
        \begin{answer}[0.5in]
          The members are: {\tt rate}, {\tt numWithdrawals}, and
          {\tt balance}.
        \end{answer}

      \item In which class is each of these data members declared?
        \begin{answer}[0.5in]
          {\tt rate} and {\tt numWithdrawals} are declared in the
          {\tt Savings} class and {\tt balance} is declared in the
          {\tt Account} class.
        \end{answer}

      \item What member functions does the object {\tt myAccount} have?
        \begin{answer}[0.5in]
          The members functions are: {\tt deposit()}, {\tt
          withdraw()}, {\tt getBalance()}, {\tt setRate()}, {\tt
          newMonth()}, and {\tt getWithdrawals()}.
        \end{answer}

      \item Which of these are inherited from the base class?
        \begin{answer}[0.5in]
          The members functions {\tt deposit()}, {\tt
          withdraw()}, and {\tt getBalance()} are inherited from
          the {\tt Account} class.
        \end{answer}
      \end{enumerate}

    \vskip -20pt
      
  \Q The file {\tt activity22.cpp} contains a preliminary
    implementation of these two classes along with a sample {\tt main}
    function.  Use it to answer the following questions.
    \begin{enumerate}
      \itemsep 10pt
      \item Without attempting to compile the code, describe what
        the member function {\tt Savings::newMonth()} on line 94 is
        supposed to do.
        \begin{answer}[0.75in]
          It is supposed to set the number of withdrawals to 0 and
          deposit the account balance times the interest rate into
          the account (i.e. pay monthly interest).
        \end{answer}

      \item Now compile the code and describe what happens.
        \begin{answer}[0.5in]
          We get an error that says:
          \begin{center}
            \tt
            `double Account::balance' is private within this context
          \end{center}
        \end{answer}

      \item Replace {\tt this->balance} on line 85 with {\tt this->getBalance()}
        Does this fix the problem? \ans[0.5in]{Yes}
        \vskip -25pt\null
        
      \item Based on what you've observed in this question, does the
        derived class {\tt Savings}\key\\[-2.5mm] have direct access to all
        of the members that it inherits from the base {\tt Account} class?
        Explain.
        \begin{answer}[0.5in]
          No, it only has access to the public members that it inherits.
        \end{answer}
    \end{enumerate}
      
  \newpage

  \Q The function {\tt testAccount()} defined in {\tt activity22.cpp}
    tests our implementation of these classes by depositing \$50 and
    then attempting to make ten \$10 withdrawals.  
    \begin{enumerate}
      \itemsep 10pt
      \item What are the results of running on this function on the
        checking account with an initial balance of \$100, as
        defined in the {\tt main} program?
        \begin{answer}[0.75in]
          It successfully deposits \$50, then makes the ten
          withdrawals, leaving a final balance of \$50.
        \end{answer}

      \item What are the results of running this function twice in
        successive months on the savings account with an initial 
        balance of \$80, an interest rate of 1\%?
        \begin{answer}[0.75in]
          It successfully deposits \$50, then makes the ten
          withdrawals, leaving a final balance of \$30 after the first
          month.  Then \$0.53 interest is added, another \$50 is
          deposited, and eight \$10 withdrawals are successfully
          made.  The last two \$10 withdrawals fail because of
          insufficient funds, leaving us with \$0.30 as a final
          balance.
        \end{answer}

      \item What type is the {\tt testAccount} function's single parameter?
        \hfill \ans[2in]{\tt Account\&}

      \item What type is the argument passed to this function on line 52?
        \hfill \ans[2in]{\tt Account}

      \item What type is the argument passed to this function on line 55?
        \hfill \ans[2in]{\tt Savings}

      \item Why is this worth noting?
        \begin{answer}[0.5in]
          We used the same function for objects of the base and
          derived class types.
        \end{answer}          
    \end{enumerate}

  \vskip -20pt
      
  \Q In the United States, you are not allowed to make more than
    six withdrawals from a savings account in one month. Based on
    this new information, are any of the member functions that
    {\tt Savings} inherits from {\tt Account} not quite appropriate?
    Explain what is missing from these functions.
    \begin{answer}[0.75in]
      Yes.  The {\tt withdraw} function should impose this limit and
      refuse to withdraw money if there have already been 6 withdraws
      from the account.
    \end{answer}

  \newpage
    
  \Q We can {\it override} a member function from the base class
    by defining a member function the same name, 
    parameters, and return type in the derived class.
    \begin{enumerate}
      \item How is this different from {\it overloading} a function? \key
        \begin{answer}[0.75in]
          Overloading a function means that we define functions with
          the same name, but different signatures.  When we override
          a function, we use the same signature (and return type).
        \end{answer}

      \item Define a \cpp{bool Savings::withdraw(double balance)}
        function that overrides the {\tt withdraw()} function from
        the {\tt Account} class and does not allow more than 6 withdrawals.
        Hint: You can use {\tt Account::withdraw()} to call the base class
        {\tt withdraw()} from within your new definition.
        \begin{answer}[1.5in]
          \begin{minipage}{3in}
            \begin{cpplst}
bool Savings::withdraw(double amount) {
  if(this->numWithdrawals < 6 && Account::withdraw(amount)) {
    this->numWithdrawals++;
    return true;
  }
  return false;
}
                      
            \end{cpplst}
          \end{minipage}
        \end{answer}

      \item What did you have to add to the declaration of the {\tt
        Savings} class?
        \begin{answer}[0.5in]
          A declaration of the \mintinline{cpp}{bool withdraw(double)} function.
        \end{answer}

      \item Now run the program again.  How did the results of the
        savings account test change?
        \begin{answer}[1in]
          Now we can only withdraw the first six \$10 amounts in the
          first month leaving us with a balance of \$70.00 at the end
          of that month.  The same thing happens in the second month,
          leaving us with a final balance of \$60.70.
        \end{answer}

      \item What happens if you remove the keyword {\tt virtual}
        from line 12?
        \begin{answer}[0.5in]
          The program goes back to functioning as before.  This is
          because the function {\tt testAccount} calls the base
          version of {\tt withdraw} since its parameter is of type
          {\tt Account}.  We'll learn more about this notion of {\it
          Polymorphism} in later class periods.
        \end{answer}
      \end{enumerate}
