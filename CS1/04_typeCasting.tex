\documentclass{exam}
%\documentclass[answers]{exam}
\hbadness=99999
\setlength{\textheight}{9.5in}
\setlength{\textwidth}{6.5in}
\setlength{\topmargin}{-0.75in}
\setlength{\oddsidemargin}{0in}
\setlength{\evensidemargin}{0in}

\usepackage{amsmath}
%\usepackage{amsfonts}
\usepackage{amssymb}
\usepackage{enumerate}
\usepackage[table]{xcolor}
\usepackage{graphicx}
\usepackage{tikz}
%\usepackage{pgfplots}
\usepackage{multicol}

% for syntax highlighting
\usepackage{minted}
\usemintedstyle[cpp]{xcode}

% for overlay of output
\usepackage[overlay,showboxes]{textpos}

\pagestyle{plain}

\setlength\columnsep{50pt}
\newcommand{\key}{\hfill
      \raisebox{-.3\height}{\includegraphics[width=0.6in]{figures/key.png}}}

\begin{document}
  \thispagestyle{empty}
  \setlength{\parindent}{0pt}

  \begin{center}
    \Large Activity \#4: C++ Types and Typecasting \\[5pt]
    \large Recorder's Report\\[20pt]
    \normalsize
    \begin{tabular}{lrp{0.1in}lr}
      Manager:  & \fillin[][2.0in] & & Presenter: & \fillin[][2.0in]\\[15pt]
      Recorder: & \fillin[][2.0in] & & Driver:    & \fillin[][2.0in]\\[15pt]
      Date:     & \fillin[][2.0in] & & Score:     & Satisfactory \hspace{10pt} /
      \hspace{10pt} Not Satisfactory
    \end{tabular}
  \end{center}
  \par\vskip 15pt
  
  Record your team's answers to the key questions (marked with
  \raisebox{-.3\height}{\includegraphics[width=0.5in]{figures/key.png}})
  below.
  \begin{enumerate}[(a)]
    \itemsep 1.75in
    \item Model 1, Question \#5
    \item Model 2, Question \#8.b
    \item Model 3, Question \#15
  \end{enumerate}

  \clearpage\pagenumbering{arabic} 
  
  \begin{center}
    \Large Activity \#4: C++ Types and Typecasting \\[5pt]
    \large Activity Guide\\[20pt]
  \end{center}

  \begin{center}
    \fbox{
      \begin{minipage}{5.5in}
        {\bf Learning Objectives:} Students will be able to:
        \begin{itemize}
          \item Content:\\[-20pt]
            \begin{itemize}
              \itemsep 0pt
              \item Explain how the size of a variable type affects the range of values it can store.
              \item Explain how a {\tt char} variable encodes non-numeric values as numbers.
              \item Recognize an implicit type cast in a C++ program
              \item Predict the side effects of an explicit type cast.
            \end{itemize}
          \item Process\\[-20pt]
            \begin{itemize}
              \itemsep 0pt
              \item Write C++ code for an explicit type cast.              
              \item Evaluate C++ code as a computer would execute it.\\[-5pt]
            \end{itemize}
        \end{itemize}
      \end{minipage}
      }
  \end{center}
  \par\vskip 10pt
  
  
  {\bf\large Model 1: Type Sizes} \\[5pt]
  Primitive data types in C++ have predefined sizes, which determines 
  the range of values that can be stored in each data type.  Several examples
  are given below.
  
  \begin{center}
    \renewcommand{\arraystretch}{1.8}
    \begin{tabular}{|l|c|c|c|c|}
      \hline
      \cellcolor{orange!20} type & {\tt bool} & {\tt char} & {\tt int} & {\tt double} \\
      \hline
      \cellcolor{orange!20} size & 1 byte & 1 byte & 4 bytes & 8 bytes \\
      \hline
    \end{tabular}
  \end{center}
  
  {\it\large Refer to Model 1 above as your team develops consensus answers
    to the questions below.}
    \par\vskip 10pt
    
  \begin{enumerate}
    \itemsep 20pt

    \item A single {\bf bit} has only two values (0 or 1), whereas two bits can store four values (00, 01, 10, and 11).
      \par\vskip 20pt
      \begin{enumerate}[(a)]
        \item How many different values can be stored in three bits, and what are they?
          \begin{solution}[0.75in]
            8 different values: 000, 001, 010, 011, 100, 101, 110, and 111
          \end{solution}
        \item What is the smallest binary number in your list in part (a) and what is its base 10 value?
          \begin{solution}[0.75in]
            The smallest number is 000 and its base 10 value is 0.
          \end{solution}
        \item What is the largest binary number in your list in part (a) and what is its base 10 value?
          \begin{solution}[0.75in]
            The smallest number is 111 and its base 10 value is 7.
          \end{solution}
      \end{enumerate}
      
    \item A {\bf byte} is equivalent to 8 bits.  How many values can be stored in a single byte, and what is the largest
      possible value?  You may wish to use Google as a calculator to solve this problem.
      \begin{solution}[0.75in]
        A byte can hold $2^8 = 256$ values and the largest is 255.
      \end{solution}
      
\newpage

    \item Complete the following table to determine how many values can be stored in a different number of bytes.  The first
      one is done for you.
      
      \begin{center}
        \renewcommand{\arraystretch}{2}
        \begin{tabular}{|c|c|c|c|}
          \hline
          \rowcolor{orange!20} \# bytes & \# bits & number of values & largest unsigned number possible \\
          \hline
          1 byte & 8 & $2^8 = 256$ & 255 \\
          \hline
          2 bytes & \ifprintanswers 16\fi & \ifprintanswers $2^{16}=65,536$\fi & \ifprintanswers 65,535 \fi\\
          \hline
          4 bytes & \ifprintanswers 32\fi & \ifprintanswers $2^{32}=4,294,967,296$\fi & \ifprintanswers 4,294,967,295\fi\\
          \hline
          8 bytes & \ifprintanswers 64\fi & \ifprintanswers $2^{64}$\fi & \ifprintanswers $2^{64}-1$\fi\\
          \hline
          $x$ bytes & \ifprintanswers $8x$\fi & \ifprintanswers $2^{8x}$\fi & \ifprintanswers $2^{8x}-1$\fi\\
          \hline
        \end{tabular}        
      \end{center}
      
    \item What is the largest value that a {\tt unsigned short int} variable (of size 2 bytes) can hold?
      \begin{solution}[0.5in]
        An {\tt unsigned short int} can hold $2^16 = 65,536$ values, the largest is 65,535.
      \end{solution}
    
    \item What happens when you write a C++ assignment statement to set a {\tt unsigned short int}\key\\[-2.5mm] variable equal to one more than this value?
      Try it out using the file `activity04a.cpp`.  Why?
      \begin{solution}[1in]
        \par
        The value wraps around to 0.  That is, the number 65535, represented as 1111111111111111 becomes 0000000000000000
        when we add one to it.
      \end{solution}
      
  
  {\bf\large Model 2: ASCII Character Codes} \\
  The American Standard Code for Information Interchange is a system that assigns a numeric value to various characters.  A
  selection of assignments from that system is shown below.
  
    \begin{center}
    \footnotesize\renewcommand{\arraystretch}{1.2}
    \begin{tabular}{|c|c||c|c||c|c||c|c||c|c||c|c|}
      \hline
      \rowcolor{orange!20} Code & Symbol & Code & Symbol & Code & Symbol & Code & Symbol & Code & Symbol & Code & Symbol \\
      \hline
        32 & (space) & 48 & 0 & 64 & @ & 80 & P & 96  & ` & 112 & p \\
        33 & !       & 49 & 1 & 65 & A & 81 & Q & 97  & a & 113 & q \\
        34 & "       & 50 & 2 & 66 & B & 82 & R & 98  & b & 114 & r \\
        35 & \#      & 51 & 3 & 67 & C & 83 & S & 99  & c & 115 & s \\
        36 & \$      & 52 & 4 & 68 & D & 84 & T & 100 & d & 116 & t \\
        37 & \%      & 53 & 5 & 69 & E & 85 & U & 101 & e & 117 & u \\
        38 & \&      & 54 & 6 & 70 & F & 86 & V & 102 & f & 118 & v \\
        39 & '       & 55 & 7 & 71 & G & 87 & W & 103 & g & 119 & w \\
        40 & (       & 56 & 8 & 72 & H & 88 & X & 104 & h & 120 & x \\
        41 & )       & 57 & 9 & 73 & I & 89 & Y & 105 & i & 121 & y \\
        42 & *       & 58 & : & 74 & J & 90 & Z & 106 & j & 122 & z \\
        43 & +       & 59 & ; & 75 & K & 91 & [ & 107 & k & 123 & \{ \\
        44 & ,       & 60 & \textless & 76 & L & 92 & $\backslash$ & 108 & l & 124 & $\vert$ \\
        45 & -       & 61 & = & 77 & M & 93 & ] & 109 & m & 125 & \} \\
        46 & .       & 62 & \textgreater & 78 & N & 94 & \^{} & 110 & b & 126 & \textasciitilde \\
        47 & /       & 63 & ? & 79 & O & 95 & \_ & 110 & o & 127 & \\
        \hline
    \end{tabular}
    \end{center}

  {\it\large Refer to Model 2 above as your team develops consensus answers
    to the questions below.}
    \par\vskip 10pt

\newpage

    \item Determine the numeric value associated with each character below in the ASCII system.
      \begin{enumerate}[(a)]
        \itemsep 10pt
        \begin{multicols}{2}
          \item capital A   \hspace{7pt}  \fillin[65][1.5in]
          \item lowercase f \hspace{1pt}  \fillin[102][1.5in]
          \item capital Z   \hspace{8pt}  \fillin[90][1.5in]
          \item number 0    \hspace{34pt} \fillin[48][1.5in]
          \item open square bracket       \fillin[91][1.35in]
          \item plus sign   \hspace{38pt} \fillin[43][1.5in]
        \end{multicols}
      \end{enumerate}
      
    \item Based on your work in model 1, how many distinct characters can be stored in a C++ {\tt char} variable?
      \begin{solution}[0.5in]
        A {\tt char} variable is 1 byte, so it can hold 256 different characters.
      \end{solution}

    \item In the file {\tt activity04b.cpp} you will find two character variables and a statement that outputs
      the ``sum'' of the two variables.
      \par\vskip 10pt
      
      \begin{enumerate}[(a)]
        \item What is the output for each pair of variable values?
          \begin{enumerate}[i.]
            \begin{multicols}{2}
              \item \mintinline{cpp}|char charOne = '(';|\\\mintinline{cpp}|char charTwo = ')';|\\[10pt]
                \fillin[ sum = Q ][2in]\par\vskip 20pt
              \item \mintinline{cpp}|char charOne = '!';|\\\mintinline{cpp}|char charTwo = ':';|\\[10pt]
                \fillin[ sum = [ ][2in]\par\vskip 20pt
              \item \mintinline{cpp}|char charOne = '1';|\\\mintinline{cpp}|char charTwo = '4';|\\[10pt]
                \fillin[ sum = e ][2in]\par\vskip 20pt
              \item \mintinline{cpp}|char charOne = '%';|\\\mintinline{cpp}|char charTwo = '=';|\\[10pt]
                \fillin[ sum = b ][2in]\par\vskip 20pt
            \end{multicols}
          \end{enumerate}
          \par\vskip 20pt
          
        \item Explain how C++ calculated these answers.  Refer to the ASCII table given in the model. \key\\[-2.5mm]
          \begin{solution}[1in]
            \par
            C++ computes the answers by adding the ASCII code numbers together.  The character that is output is the character
            that has an ASCII code that matches the sum of the two initial characters' ASCII codes.
          \end{solution}
      \end{enumerate}
      
    \item An {\it implicit type cast} happens when C++ treats a variable of one type as if it were a variable of another type.
      This happened in the program above.
      \begin{enumerate}
        \item How can you tell that an implicit type cast happen in the program above?
          \begin{solution}[0.5in]
            \par
            Because the ``characters'' were added together as if they were integer values.
          \end{solution}
        \item Do you think the type of the variables {\tt charOne} and {\tt charTwo} actually change?
          \begin{solution}[0.5in]
            \par
            No, types in C++ can not be changed.  You can verify that these variables are still characters 
            outputting their value, which will be printed as a character, not a number.
          \end{solution}
      \end{enumerate}
      
      
\newpage

  {\bf\large Model 3: A C++ Program} \\[-10pt]
  \begin{center}
    \begin{minipage}{5.5in}
      \begin{minted}[
        frame=lines,
        framesep=2mm,
        bgcolor=gray!15,
        baselinestretch=1.2,
        linenos
      ]{cpp}
#include <iostream>
using namespace std;

int main() {
  // ----------------------------------------- declare variables
  double floatOne,floatTwo;
  int integer = 35;
  // ----------------------------------------- part I
  floatOne = integer;
  cout << "original number: " << integer << endl;
  cout << "after conversion: " << floatOne << endl << endl;
  // ----------------------------------------- part II
  floatOne = 3.9;
  floatTwo = static_cast<int>(floatOne);
  cout << "second number: " << floatOne << endl;
  cout << "after conversion: " << floatTwo << endl;
}      
      \end{minted}
    \end{minipage}
  \end{center}
  \TPMargin{5pt}
  
  {\it\large Refer to Model 3 above as your group develops consensus answers
    to the questions below.}
    \par\vskip 10pt
    
      \item This program can be found in {\tt activity02c.cpp}.  Run it and then answer the following questions.
        \par\vskip 20pt
        \begin{enumerate}[(a)]
          \itemsep 15pt
          \item What is the printed value of the variable {\tt integer} in part I?  \hfill \fillin[35][1.5in]
          \item What is the printed value of the variable {\tt floatOne} in part I?  \hfill \fillin[35][1.5in]
          \item What is the printed value of the variable {\tt floatOne} in part II? \hfill \fillin[3.9][1.5in]
          \item What is the printed value of the variable {\tt floatTwo} in part II? \hfill \fillin[3][1.5in]
        \end{enumerate}
                
      \item Lines 9 and 14 are both examples of {\it type casting}.  What is a type cast?
        \begin{solution}[0.5in]
          It is when a variable of one type is treated as if it were a variable of another type.
        \end{solution}
        
      \item In the previous model we saw the term {\it implicit type casting}.  Another type of type casting is {\it explicit
        type casting}.  Which statement (line 9 or 14) is an example of an explicit type cast? 
        \begin{solution}[0.5in]
          Line 14 is an explicit type cast because we specifically asked the computer to treat {\tt floatOne} as an integer.
        \end{solution}
        
\newpage

    \item What is the difference between an implicit type cast and an explicit type cast?
      \begin{solution}[1in]
        \par
        An implicit type cast happens automatically, while an explicit type cast is asked for.
      \end{solution}
      
    \item What is the C++ syntax for an explicit type cast?  Give an example other than the one in the model.
      \begin{solution}[1in]
        \par
        The syntax is \mintinline{cpp}|static_cast<newType>(variable)|.  Examples will vary.
      \end{solution}
      
    \item Write a snippet of code that will output the string WWU without using the ``W'' or\key\\[-2.5mm] ``U'' characters.
      Hint: use explicit type casting and the ASCII table seen in model 2.\\
      Feel free to test your code in the file {\tt activity04c.cpp}.
      \begin{solution}[1in]
        \par
  \begin{center}
    \begin{minipage}{5.5in}
      \begin{minted}[
        frame=lines,
        framesep=2mm,
        bgcolor=gray!15,
        baselinestretch=1.2,
        linenos
      ]{cpp}
  cout << static_cast<char>(87);
  cout << static_cast<char>(87);
  cout << static_cast<char>(85);
      \end{minted}
    \end{minipage}
  \end{center}
      \end{solution}
        
  \end{enumerate}  
    
\end{document}
