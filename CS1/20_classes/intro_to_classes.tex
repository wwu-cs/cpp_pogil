\model{C++ Code Snippets}
  \begin{center}
    \small
    \begin{tabular}{p{2.8in}p{0.1in}p{3.5in}}
      \begin{minipage}{2.8in}
        \begin{cpplst}
/* Data for multiple rectangles */        
const int MAX_RECTANGLES = 100;
int rectX[MAX_RECTANGLES];
int rectY[MAX_RECTANGLES];
int rectWidth[MAX_RECTANGLES];
int rectHeight[MAX_RECTANGLES];
        \end{cpplst}      
      \end{minipage}
      & &
      \begin{minipage}{3.5in}
        \begin{cpplst}       
/* Functions to use with rectangles */        
int getArea(int length, int width);
int getPerimeter(int length, int width);
void move(int &x, int &y, int dx, int dy);
void draw(int x, int y, int wd, int ht);
        \end{cpplst}      
      \end{minipage}
    \end{tabular}
  \end{center}
  
  {\it\large Refer to Model 1 above as your group develops consensus answers
    to the questions below.}    

  \quest{10 min}

  \Q The C++ code snippets above come from a program designed to
  define and manipulate rectangles.  Answer the following questions
  related to this code.
  \begin{enumerate}
    \itemsep 10pt
    \item Give C++ code to define a rectangle at the point
      $(1,1)$ with a width of 3 and a height of 2 stored at index
      0 in the arrays above.
      \begin{answer}[1.5in]
        \begin{minipage}{3in}
          \begin{cpplst}    
rectX[0] = 1;
rectY[0] = 1;
rectWidth[0] = 3;
rectHeight[0] = 2;
              \end{cpplst}
            \end{minipage}
          \end{answer}

    \item Write an appropriate function call for each task
      described below.
      \begin{enumerate}
        \itemsep 10pt
        \item Move your rectangle up 2 and over 1. \hfill
          \ans[3in]{\cpp{move(rectX[0], rectY[0], 1, 2)}}

        \item Compute the area of your rectangle. \hfill
          \ans[3in]{\cpp{getArea(rectWidth[0], rectHeight[0])}}

        \item Draw your rectangle \hfill
          \ans[3in]{\cpp{draw(rectX[0], rectY[0], rectWidth[0], rectHeight[0])}}
      \end{enumerate}

    \item What would you have to change in order to make the
      function calls above reference another rectangle stored at
      index 1?
      \begin{answer}[0.5in]
        You would have to change all the array indexes to 1 in the
        arguments.
      \end{answer}
    \end{enumerate}

  \newpage

  \Q A complete version of the program can be found in the file 
    {\tt activity20a.cpp}. Add\key\\[-2.5mm] to this code to complete the following
    tasks.
    \begin{itemize}
      \item Create a second rectangle at point $(2,5)$ with width 10
        and height 8.
      \item Move your second rectangle so that its point is at $(0,0)$.
      \item Determine which of the two rectangles has a larger
        perimeter (using C++ code).
      \item Draw your second rectangle.
    \end{itemize}
    
    Describe the most frustrating part of completing the tasks above.
    
    \begin{answer}[0.5in]
      Answers will vary, but likely keeping track of all the
      different arrays and their indexes will add frustration to
      this process.
    \end{answer}
    
  \Q In programming, the term {\it abstraction} is used to
    describe representing complex things simply.  For example, we
    all know how to turn on a light switch even if we don't
    understand how it actually works. Decide if each of the
    following statements regarding this model is true or false.
    \begin{enumerate}
      \itemsep 10pt
      \item \ans[0.5in]{False} We can define new rectangles without
        knowing the details of how rectangles are stored.
      \item \ans[0.5in]{False} We can compute the area or
        perimeter of a rectangle without understanding the details.
      \item \ans[0.5in]{False} We can draw or move a rectangle
        without understanding the details.
      \item \ans[0.5in]{False} If we needed to expand our program to 
        define and manipulate circles, we could do so without 
        renaming functions or confusing circle and rectangle code.
    \end{enumerate}