\documentclass{exam}
%\documentclass[answers]{exam}
\hbadness=99999
\usepackage[total={6.5in,9in}]{geometry}

\usepackage{enumerate}
\usepackage[table]{xcolor}
\usepackage{graphicx}
\usepackage{tikz}
%\usepackage{pgfplots}
\usepackage{multicol}

% for syntax highlighting
\usepackage{minted}
\usemintedstyle[cpp]{xcode}

% for overlay of output
\usepackage[overlay,showboxes]{textpos}

\pagestyle{plain}

\setlength\columnsep{50pt}
\newcommand{\key}{\hfill
      \raisebox{-.3\height}{\includegraphics[width=0.6in]{figures/key.png}}}

\begin{document}
  \thispagestyle{empty}
  \setlength{\parindent}{0pt}

  \begin{center}
    \Large Activity \#1: Introduction to Classes \\[5pt]
    \large Recorder's Report\\[20pt]
    \normalsize
    \begin{tabular}{lrp{0.1in}lr}
      Manager:  & \fillin[][2.0in] & & Presenter: & \fillin[][2.0in]\\[15pt]
      Recorder: & \fillin[][2.0in] & & Driver:    & \fillin[][2.0in]\\[15pt]
      Date:     & \fillin[][2.0in] & & Score:     & Satisfactory \hspace{10pt} /
      \hspace{10pt} Not Satisfactory
    \end{tabular}
  \end{center}
  \par\vskip 15pt
  
  Record your group's answers to the key questions (marked with
  \raisebox{-.3\height}{\includegraphics[width=0.5in]{figures/key.png}})
  below.
  \begin{enumerate}[(a)]
    \itemsep 1.75in
    \item Model 1, Question \#2
    \item Model 2, Question \#6
    \item Model 3, Question \#12 (a)
  \end{enumerate}

  \clearpage\pagenumbering{arabic} 
  
  \begin{center}
    \Large Activity \#1: Introduction to Classes \\[5pt]
    \large Activity Guide\\[20pt]
  \end{center}

  \begin{center}
    \fbox{
      \begin{minipage}{5.5in}
        {\bf Learning Objectives:} Students will be able to:
        \begin{itemize}
          \item Content:\\[-20pt]
            \begin{itemize}
              \itemsep 0pt
              \item Explain the benefits of abstraction in programming
              \item Explain the syntax for defining classes  in C++
              \item Explain the benefits of encapsulation in programming
              \item Explain the use of mutator and accessor methods
            \end{itemize}
          \item Process\\[-20pt]
            \begin{itemize}
              \itemsep 0pt
              \item Construct a simple class with data fields and methods\\[-5pt]
            \end{itemize}
        \end{itemize}
      \end{minipage}
      }
  \end{center}
  \par\vskip 10pt
  
  
  {\bf\large Model 1: C++ Code Snippets}\\[-10pt]
  \begin{center}
    \small
    \begin{tabular}{p{2.8in}p{0.1in}p{2.8in}}
      \begin{minipage}{2.8in}
        \begin{minted}[
          frame=lines,
          framesep=2mm,
          bgcolor=gray!15,
          baselinestretch=1.2,
          linenos,
          firstnumber=12
        ]{cpp}
/* Data for multiple rectangles */        
const int MAX_RECTANGLES = 100;
int rectX[MAX_RECTANGLES];
int rectY[MAX_RECTANGLES];
int rectWidth[MAX_RECTANGLES];
int rectHeight[MAX_RECTANGLES];
        \end{minted}      
      \end{minipage}
      & &
      \begin{minipage}{2.8in}
        \begin{minted}[
          frame=lines,
          framesep=2mm,
          bgcolor=gray!15,
          baselinestretch=1.2,
          linenos,
          firstnumber=4
        ]{cpp}        
/* Functions to use with rectangles */        
int getArea(int length, int width);
int getPerimeter(int length, int width);
void move(int &x, int &y, int dx, int dy);
void draw(int x, int y, int wd, int ht);
        \end{minted}      
      \end{minipage}
    \end{tabular}
  \end{center}
  \par\vskip 5pt
  
  {\it\large Refer to Model 1 above as your group develops consensus answers
    to the questions below.}
    \par\vskip 10pt
    
  \begin{enumerate}
    \itemsep 20pt
    
    \item The C++ code snippets above come from a program designed to
      define and manipulate rectangles.  Answer the following questions
      related to this code.
      \begin{enumerate}[(a)]
        \itemsep 10pt
        \item Give C++ code to define a rectangle at the point
          $(1,1)$ with a width of 3 and a height of 2 stored at index
          0 in the arrays above.
          \begin{solution}[1.5in]
            \begin{minipage}{3in}
              \begin{minted}[
                frame=lines,
                framesep=2mm,
                bgcolor=gray!15,
                baselinestretch=1.2,
              ]{cpp}      
rectX[0] = 1;
rectY[0] = 1;
rectWidth[0] = 3;
rectHeight[0] = 2;
              \end{minted}
            \end{minipage}
          \end{solution}
        \item Write an appropriate function call for each task
          described below. \par\vskip 20pt
          \begin{enumerate}[i.]
            \itemsep 15pt
            \item Move your rectangle up 2 and over 1. \hfill
              \fillin[{\mintinline{cpp}|move(rectX[0],rectY[0],1,2)|}][3in]
            \item Compute the area of your rectangle. \hfill
              \fillin[{\mintinline{cpp}|getArea(rectWidth[0],rectHeight[0])|}][3in]
            \item Draw your rectangle \hfill
              \fillin[{\mintinline{cpp}|draw(rectX[0],rectY[0],rectWidth[0],rectHeight[0])|}][3in]
          \end{enumerate}
        \item What would you have to change in order to make the
          function calls above reference another rectangle stored at
          index 1?
          \begin{solution}[0.5in]
            You would have to change all the array indexes to 1 in the
            arguments.
          \end{solution}
      \end{enumerate}
      \par\vskip -30pt\null

    \item A complete version of the program can be found in the file 
      {\tt activity01a.cpp}. Add to\key\\[-2.5mm] this code to complete the following
      tasks.
      
      \begin{itemize}
        \item Create a second rectangle at point $(2,5)$ with width 10
          and height 8.
        \item Move your second rectangle so that its point is at $(0,0)$.
        \item Determine which of the two rectangles has a larger
          perimeter (using C++ code).
        \item Draw your second rectangle.
      \end{itemize}
      
      Describe the most frustrating part of completing the tasks above.
      
      \begin{solution}[0.5in]
        Answers will vary, but likely keeping track of all the
        different arrays and their indexes will add frustration to
        this process.
      \end{solution}
      
    \item In programming, the term {\it abstraction} is used to
      describe representing complex things simply.  For example, we
      all know how to turn on a light switch even if we don't
      understand how it actually works. Decide if each of the
      following statements regarding this model is true or false.
      \par\vskip 20pt
      \begin{enumerate}[(a)]
        \itemsep 8pt
        \item \fillin[F][0.5in] We can define new rectangles without
          knowing the details of how rectangles are stored.
        \item \fillin[F][0.5in] We can compute the area or
          perimeter of a rectangle without understanding the details.
        \item \fillin[F][0.5in] We can draw or move a rectangle
          without understanding the details.
        \item \fillin[F][0.5in] If we needed to expand our program to 
          define and manipulate circles, we could do so without 
          renaming functions or confusing circle and rectangle code.
      \end{enumerate}              
      \par\vskip 10pt


  {\bf\large Model 2: A Different Approach}\\[-30pt]
  \begin{center}
    \small
    \begin{tabular}{p{2.8in}p{0.1in}p{2.7in}}
      \begin{minipage}{2.8in}
        \begin{minted}[
          frame=lines,
          framesep=2mm,
          bgcolor=gray!15,
          baselinestretch=1.2,
          linenos,
          firstnumber=4
        ]{cpp}
class Rectangle {
  public:
    int x;         // (x,y) coords 
    int y;
    int width;     // width (dx)
    int height;    // height (dy)
    int getArea();
    int getPerimeter();
    void move(int dx, int dy);
    void draw();
};
        \end{minted}
      \end{minipage}
      & &
      \begin{minipage}{2.7in}
        \begin{minted}[
          frame=lines,
          framesep=2mm,
          bgcolor=gray!15,
          baselinestretch=1.2,
          linenos,
          firstnumber=18
        ]{cpp}    
/* Data for multiple rectangles */
const int MAX_RECTANGLES = 100;
Rectangle myRects[MAX_RECTANGLES];

/* create rectangle at point (1,1) */
myRects[0].x = 1;
myRects[0].y = 1;
myRects[0].width = 2;
myRects[0].height = 3;

/* move this rectangle up 2 and over 1 */
myRects[0].move(1,2);
        \end{minted}
      \end{minipage}
    \end{tabular}
  \end{center}
  
  {\it\large Refer to Model 2 above as your group develops consensus answers
    to the questions below.}
    \par\vskip 10pt
    
      \item The code for this model can be found in {\tt
        activity01b.cpp}.  Use it to help you complete the 
        following table, filling in the equivalent code 
        in each model.
        \vskip -30pt\ \
        \begin{center}
          \renewcommand{\arraystretch}{2}
          \begin{tabular}{|p{2.9in}|p{2.9in}|}
            \hline
            \rowcolor{orange!20}\multicolumn{1}{|c|}{\bf Model 1} & \multicolumn{1}{|c|}{\bf Model 2}\\
            \hline
            \begin{minipage}{2.9in}
              \ifprintanswers
                \small
                \begin{minted}[
                  frame=lines,
                  framesep=2mm,
                  bgcolor=gray!15,
                  baselinestretch=1,
                ]{cpp}
rectX[0] = 1;
rectY[0] = 1;
rectWidth[0] = 2;
rectHeight[0] = 3;
                \end{minted}
              \fi
            \end{minipage}
            &
            \begin{minipage}{2.9in}
              \small
              \begin{minted}[
                frame=lines,
                framesep=2mm,
                bgcolor=gray!15,
                baselinestretch=1,
              ]{cpp}
// define a new rectangle              
myRects[0].x = 1;
myRects[0].y = 1;
myRects[0].width = 2;
myRects[0].height = 3;
              \end{minted}
            \end{minipage}
            \\
            \hline
            \begin{minipage}{2.9in}
              \small
              \begin{minted}[
                frame=lines,
                framesep=2mm,
                bgcolor=gray!15,
                baselinestretch=1,
              ]{cpp}
// draw the 3rd rectangle
draw(rectX[2],rectY[2],
     rectWidth[2],rectHeight[2]);
              \end{minted}
            \end{minipage}
            &
            \begin{minipage}{2.9in}
              \ifprintanswers
                \small
                \begin{minted}[
                  frame=lines,
                  framesep=2mm,
                  bgcolor=gray!15,
                  baselinestretch=1,
                ]{cpp}
myRects[2].draw();                
                \end{minted}
              \fi
            \end{minipage}
            \\
            \hline
            \begin{minipage}{3in}
              \ifprintanswers
              \small
              \begin{minted}[
                frame=lines,
                framesep=2mm,
                bgcolor=gray!15,
                baselinestretch=1,
              ]{cpp}
if (getArea(rectWidth[0],rectHeight[0]) >
    getArea(rectWidth[1],rectHeight[1]) {
  cout << "First rectangle has more area";
}
              \end{minted}
              \fi
            \end{minipage}
            &
            \begin{minipage}{2.9in}
              \small
              \begin{minted}[
                frame=lines,
                framesep=2mm,
                bgcolor=gray!15,
                baselinestretch=1,
              ]{cpp}
// compare rectangle areas
if (
  myRect[0].getArea() > myRect[1].getArea()
) {
  cout << "First rectangle has more area";
}
              \end{minted}            
            \end{minipage}
            \\
            \hline
            \begin{minipage}{2.9in}
              \small
              \begin{minted}[
                frame=lines,
                framesep=2mm,
                bgcolor=gray!15,
                baselinestretch=1,
              ]{cpp}
/* error moving x from rectangle 0
  and y form rectangle 1           */
move( rectX[0], rectY[1], 3, 5);
              \end{minted}            
            \end{minipage}
            &
            \begin{minipage}{3in}
              \ifprintanswers
              Not possible in this model
              \fi
            \end{minipage}
            \\
            \hline
          \end{tabular}
        \end{center}
        \par\vskip 10pt
      
        
      \item Now decide if each of the following statements is true or
        false for model 2.
        \par\vskip 10pt
        \begin{enumerate}[(a)]
          \itemsep 8pt
          \item \fillin[F][0.5in] We can define new rectangles without
            knowing the details of how rectangles are stored.
          \item \fillin[T][0.5in] We can compute the area or
            perimeter of a rectangle without understanding the details.
          \item \fillin[T][0.5in] We can draw or move a rectangle
            without understanding the details.
          \item \fillin[T][0.5in] If we needed to expand our program to 
            define and manipulate circles, we could do so without
            renaming functions or confusing circle and rectangle code.
        \end{enumerate}
        \par\vskip -30pt\null
        
      \item Describe at least opne advantage of the approach in
        model 2 over that seen in model 1.\key
        \begin{solution}[0.5in]
           Answers will vary.
        \end{solution}
        
      \item Suppose we wish to ensure that other programmers never set
        rectangle widths and heights to be negative. Can this be done in
        either model?  Explain.
        \begin{solution}[0.5in]
          No, any part of the program can access the width and height
          directly.
        \end{solution}
 
\newpage 

  {\bf\large Model 3: Revising Our Approach} \\[-20pt]
  \begin{center}
    \small
    \begin{minipage}{4in}
      \begin{minted}[
        frame=lines,
        framesep=2mm,
        bgcolor=gray!15,
        baselinestretch=1.2,
        linenos,
        firstnumber=4
      ]{cpp}
class Rectangle {
  public:
    bool init(int xVal, int yVal, int wVal, int hVal);
    int getArea();
    int getPerimeter();
    void move(int dx, int dy);
    void draw();
  private:
    int x;       // (x,y) coords of bottom left corner
    int y;
    int width;   // width (dx) and height (dy)
    int height;
};
      \end{minted}
    \end{minipage}
  \end{center}
  
  {\it\large Refer to Model 3 above as your group develops consensus answers
    to the questions below.}
    \par\vskip 10pt

    \item Formulate a hypothesis about what each of the lines from the model mentioned below is for.
      \par\vskip 20pt
      
      \begin{enumerate}[(a)]
        \itemsep 15pt
        \item Line 5:  \hfill \fillin[Says anything on lines 3-7 can be used anywhere][5.25in]
        \item Line 11: \hfill \fillin[Says anything on lines 9-12 can only be used by member functions][5.25in]
        \item Line 6:  \hfill \fillin[Allows us to initialize the private variables][5.25in]
      \end{enumerate}
      
    \item The full conde for this model can be found in {\tt activity01c.cpp}.  Make the
      following changes to this code (resetting after each change) and describe what 
      happens.
      
      \begin{enumerate}[(a)]
        \item On line 24, add the code \mintinline{cpp}|myRects[0].x = 5;| and compile.
          \begin{solution}[0.75in]
            A compile error is generated because the variable Rectangle::x is private (so it
            can not be changed directly).
          \end{solution}
        \item Change line 26 to use the function call \mintinline{cpp}|myRects[0].init(1,1,-5,3)|.
          \begin{solution}[0.75in]
            The program compiles, but says that an invalid rectangle was given.
          \end{solution}
      \end{enumerate}
      
    \item We've previously seen that global variables should be avoided.  Similarly, when
      defininig classes, we should protect the variables in the class by making them private.
      This practice is called {\it encapsulation}.  Give at least one reason why
      encapsulation is a desirable feature in a program.
      \begin{solution}[0.5in]
        It allows us to check the integrety of our data.
      \end{solution}

    \item Functions that belong to a class are called {\it member functions}. How 
      are member functions defined?
      \begin{solution}[0.5in]
        Member functions are defined using \mintinline{cpp}|className::functionName(...)|.
      \end{solution}

    \item We've previously asked about expanding our program to define and manipulate
      circles.  Assuming that a circle is defined by a center $(x,y)$ point and a non-negative 
      radius, complete the following.
      \begin{enumerate}[(a)]
        \item Give a definition of a class for circles.  Try to use abstraction and
          encapsulation as\key\\[-2.5mm] much as possible.
          \begin{solution}[1.75in]
            \begin{center}
              \small
              \begin{minipage}{4in}
                \begin{minted}[
                  frame=lines,
                  framesep=2mm,
                  bgcolor=gray!15,
                  baselinestretch=1,
                ]{cpp}
class Circle {
  public:
    bool init(int xVal, int yVal, int rVal);
    int getArea();
    int getCircumference();
    void move(int dx, int dy);
    void draw();
  private:
    int x;       // (x,y) coords of bottom left corner
    int y;
    int radius; 
};
                \end{minted}
              \end{minipage}
            \end{center}
          \end{solution}
        \item Add code to the {\tt main} program to define a circle at the point $(-2,1)$
          with radius 3.
          \begin{solution}[1.25in]
            \begin{center}
              \small
              \begin{minipage}{4in}
                \begin{minted}[
                  frame=lines,
                  framesep=2mm,
                  bgcolor=gray!15,
                  baselinestretch=1,
                ]{cpp}
Circle myCirc;
myCirc.init(-2,1,3);
                \end{minted}
              \end{minipage}
            \end{center}
          \end{solution}
        \item Add member functions to compute the area ($\pi r^2$) and circumference
          ($2\pi r$) of a circle.
          \begin{solution}[1.25in]
            \begin{center}
              \small
              \begin{minipage}{4in}
                \begin{minted}[
                  frame=lines,
                  framesep=2mm,
                  bgcolor=gray!15,
                  baselinestretch=1,
                ]{cpp}
Circle::getArea() { return 3.14159*radius*radius; }
Circle::getCircumference() { return 2*3.14159*radius; }
                \end{minted}
              \end{minipage}
            \end{center}
          \end{solution}
        \item Write code to compare the areas of the circle you defined and the rectangle
          defined in the program and indicate which is bigger.
          \begin{solution}[1.25in]
            \begin{center}
              \small
              \begin{minipage}{4in}
                \begin{minted}[
                  frame=lines,
                  framesep=2mm,
                  bgcolor=gray!15,
                  baselinestretch=1,
                ]{cpp}
if( myCirc.getArea() > myRects[0].getArea() ) {
  cout << "The circle is bigger" << endl;
} else if( myCirc.getArea() < myRects[0].getArea() ) {
  cout << "The rectangle is bigger" << endl;
} else {
  cout << "They are equal" << endl;
}
                \end{minted}
              \end{minipage}
            \end{center}
          \end{solution}
      \end{enumerate}

  \end{enumerate}
  
  
    
\end{document}
