\model{Revising Our Approach}
  \begin{center}
    \small
    \begin{minipage}{4in}
      \begin{cpplst}
class Rectangle {
  public:
    bool init(int xVal, int yVal, int wVal, int hVal);
    int getArea();
    int getPerimeter();
    void move(int dx, int dy);
    void draw();
  private:
    int x;       // (x,y) coords of bottom left corner
    int y;
    int width;   // width (dx) and height (dy)
    int height;
};
      \end{cpplst}
    \end{minipage}
  \end{center}
  
  {\it\large Refer to Model \M above as your group develops consensus answers
    to the questions below.}

  \Q Formulate a hypothesis about what each of the lines from the model mentioned below is for.    
    \begin{enumerate}
      \itemsep 10pt
      \item Line 2:  \hfill \ans[5.25in]{Says anything on lines 3-7 can be used anywhere}
      \item Line 8: \hfill \ans[5.25in]{Says anything on lines 9-12 can only be used by member functions}
      \item Line 3:  \hfill \ans[5.25in]{Allows us to initialize the private variables}
    \end{enumerate}
    
  \Q The full code for this model can be found in {\tt activity01c.cpp}.  Make the
    following changes to this code (resetting after each change) and describe what 
    happens.
    \begin{enumerate}
      \item On line 24, add the code \cpp{myRects[0].x = 5;} and compile.
        \begin{answer}[0.75in]
          A compile error is generated because the variable Rectangle::x is private (so it
          can not be changed directly).
        \end{answer}

      \item Change line 26 to use the function call \cpp{myRects[0].init(1,1,-5,3)}
        \begin{answer}[0.75in]
          The program compiles, but says that an invalid rectangle was given.
        \end{answer}
    \end{enumerate}
    
  \Q We've previously seen that global variables should be avoided.  Similarly, when
    defininig classes, we should protect the variables in the class by making them private.
    This practice is called {\it encapsulation}.  Give at least one reason why
    encapsulation is a desirable feature in a program.
    \begin{answer}[0.5in]
      It allows us to check the integrity of our data.
    \end{answer}

  \Q Functions that belong to a class are called {\it member functions}. How 
    are member functions defined?
    \begin{answer}[0.5in]
      Member functions are defined using \cpp{className::functionName(...)}
    \end{answer}

  \Q We've previously asked about expanding our program to define and manipulate
    circles.  Assuming that a circle is defined by a center $(x,y)$ point and a non-negative 
    radius, complete the following.
    \begin{enumerate}
      \item Give a definition of a class for circles.  Try to use abstraction and
        encapsulation as\key\\[-2.5mm] much as possible.
        \begin{answer}[1.75in]
          \begin{center}
            \small
            \begin{minipage}{4in}
              \begin{cpplst}
class Circle {
  public:
    bool init(int xVal, int yVal, int rVal);
    int getArea();
    int getCircumference();
    void move(int dx, int dy);
    void draw();
  private:
    int x;       // (x,y) coords of bottom left corner
    int y;
    int radius; 
};
              \end{cpplst}
            \end{minipage}
          \end{center}
        \end{answer}

  \Q Add code to the {\tt main} program to define a circle at the point $(-2,1)$
    with radius 3.
    \begin{answer}[1.25in]
      \begin{center}
        \small
        \begin{minipage}{4in}
          \begin{cpplst}
Circle myCirc;
myCirc.init(-2,1,3);
          \end{cpplst}
        \end{minipage}
      \end{center}
    \end{answer}

      \item Add member functions to compute the area ($\pi r^2$) and circumference
        ($2\pi r$) of a circle.
        \begin{answer}[1.25in]
          \begin{center}
            \small
            \begin{minipage}{4in}
              \begin{cpplst}
Circle::getArea() { return 3.14159*radius*radius; }
Circle::getCircumference() { return 2*3.14159*radius; }
              \end{cpplst}
            \end{minipage}
          \end{center}
        \end{answer}

      \item Write code to compare the areas of the circle you defined and the rectangle
        defined in the program and indicate which is bigger.
        \begin{answer}[1.25in]
          \begin{center}
            \small
            \begin{minipage}{4in}
              \begin{cpplst}
if( myCirc.getArea() > myRects[0].getArea() ) {
  cout << "The circle is bigger" << endl;
} else if( myCirc.getArea() < myRects[0].getArea() ) {
  cout << "The rectangle is bigger" << endl;
} else {
  cout << "They are equal" << endl;
}
              \end{cpplst}
            \end{minipage}
          \end{center}
        \end{answer}
      \end{enumerate}