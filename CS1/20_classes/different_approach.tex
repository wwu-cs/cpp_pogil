\model{A Different Approach}
  \begin{center}
    \small
    \begin{tabular}{p{2.8in}p{0.1in}p{3.2in}}
      \begin{minipage}{2.8in}
        \fs
        \begin{cpplst}
class Rectangle {
  public:
    int x;         // (x,y) coords 
    int y;
    int width;     // width (dx)
    int height;    // height (dy)
    int getArea();
    int getPerimeter();
    void move(int dx, int dy);
    void draw();
};
        \end{cpplst}
      \end{minipage}
      & &
      \begin{minipage}{3.2in}
        \fs
        \begin{cpplst}  
/* Data for multiple rectangles */
const int MAX_RECTANGLES = 100;
Rectangle myRects[MAX_RECTANGLES];

/* create rectangle at point (1,1) */
myRects[0].x = 1;
myRects[0].y = 1;
myRects[0].width = 2;
myRects[0].height = 3;

/* move this rectangle up 2 and over 1 */
myRects[0].move(1,2);
        \end{cpplst}
      \end{minipage}
    \end{tabular}
  \end{center}
  
  {\it\large Refer to Model 2 above as your group develops consensus answers
    to the questions below.}

  \quest{10 min}
    
  \Q The code for this model can be found in {\tt
    activity01b.cpp}.  Use it to help you complete the 
    following table, filling in the equivalent code 
    in each model.
    \begin{center}
      \renewcommand{\arraystretch}{2}
      \begin{tabular}{|p{2.9in}|p{2.9in}|}
        \hline
        \rowcolor{orange!20}\multicolumn{1}{|c|}{\bf Model 1} & \multicolumn{1}{|c|}{\bf Model 2}\\
        \hline
        \begin{minipage}{2.9in}
          \begin{answer}[1.2in]
            \small
            \begin{cpplst}
rectX[0] = 1;
rectY[0] = 1;
rectWidth[0] = 2;
rectHeight[0] = 3;
            \end{cpplst}
          \end{answer}
        \end{minipage}
        &
        \begin{minipage}{2.9in}
          \small
          \begin{cpplst}
// define a new rectangle              
myRects[0].x = 1;
myRects[0].y = 1;
myRects[0].width = 2;
myRects[0].height = 3;
          \end{cpplst}
        \end{minipage}
        \\
        \hline
        \begin{minipage}{2.9in}
          \small
          \begin{cpplst}
// draw the 3rd rectangle
draw(rectX[2],rectY[2],
     rectWidth[2],rectHeight[2]);
          \end{cpplst}
        \end{minipage}
        &
        \begin{minipage}{2.9in}
          \begin{answer}[1in]
            \small
            \begin{cpplst}
myRects[2].draw();                
            \end{cpplst}
          \end{answer}
        \end{minipage}
        \\
        \hline
        \begin{minipage}{3in}
          \begin{answer}[1.2in]
            \small
            \begin{cpplst}
if (getArea(rectWidth[0],rectHeight[0]) >
    getArea(rectWidth[1],rectHeight[1]) {
  cout << "First rectangle has more area";
}
            \end{cpplst}
          \end{answer}
        \end{minipage}
        &
        \begin{minipage}{2.9in}
          \scriptsize
          \begin{cpplst}
// compare rectangle areas
if (
  myRect[0].getArea() > myRect[1].getArea()
) {
  cout << "First rectangle has more area";
}
          \end{cpplst}            
        \end{minipage}
        \\
        \hline
        \begin{minipage}{2.9in}
          \fs
          \begin{cpplst}
/* error moving x from rectangle 0
  and y form rectangle 1           */
move( rectX[0], rectY[1], 3, 5);
          \end{cpplst}            
        \end{minipage}
        &
        \begin{minipage}{3in}
          \begin{answer}[0.5in]
          Not possible in this model
          \end{answer}
        \end{minipage}
        \\
        \hline
      \end{tabular}
    \end{center}

  \Q Now decide if each of the following statements is true or
    false for model 2.
    \begin{enumerate}
      \itemsep 10pt
      \item \ans[0.5in]{False} We can define new rectangles without
        knowing the details of how rectangles are stored.

      \item \ans[0.5in]{True} We can compute the area or
        perimeter of a rectangle without understanding the details.

      \item \ans[0.5in]{True} We can draw or move a rectangle
        without understanding the details.

      \item \ans[0.5in]{True} If we needed to expand our program to 
        define and manipulate circles, we could do so without
        renaming functions or confusing circle and rectangle code.
    \end{enumerate}
    \par\vskip -30pt\null
    
  \Q Describe at least open advantage of the approach in
    model 2 over that seen in\key\\[-2.5mm] model 1.
    \begin{answer}[0.5in]
       Answers will vary.
    \end{answer}
    
  \Q Suppose we wish to ensure that other programmers never set
    rectangle widths and heights to be negative. Can this be done in
    either model?  Explain.
    \begin{answer}[0.5in]
      No, any part of the program can access the width and height
      directly.
    \end{answer}
