
  {\bf\large Model 2: A Truth Table} \\
  
    \begin{center}
      \renewcommand{\arraystretch}{1.4}
      \begin{tabular}{|c|c|c|c|c|}
        \hline
        \rowcolor{orange!20} Condition \#1 & Condition \#2 & Negation (NOT) & Conjunction (AND) & Disjunction (OR) \\
        \rowcolor{orange!20} \mintinline{cpp}|p| & \mintinline{cpp}|q| & \mintinline{cpp}|! p| & \mintinline{cpp}|p && q| & \mintinline{cpp}{p || q} \\
        \hline
        True & True   & False & True  & True \\
        True & False  & False & False & True \\
        False & True  & True  & False & True \\
        False & False & True  & False & False \\
        \hline
      \end{tabular}
    \end{center}

  {\it\large Refer to Model 2 above as your team develops consensus answers
    to the questions below.}
    \par\vskip 10pt

    \item The symbols \mintinline{cpp}{&&}, \mintinline{cpp}{||}, and \mintinline{cpp}{!} are called
      {\it logical operators} because they combine conditions that are either true or false to create
      new compound conditions. Given the variable definitions below, fill in the
      appropriate logical operator to produce the desired truth value.
      \begin{center}
        \begin{minipage}{3.5in}
          \begin{minted}[
            frame=lines,
            framesep=2mm,
            bgcolor=gray!15,
            baselinestretch=1.2
          ]{cpp}
  int numBooks = 40;
          \end{minted}
        \end{minipage}
      \end{center}
      \par\vskip 10pt
      \begin{enumerate}[(a)]
        \itemsep 15pt
        \item \mintinline{cpp}{(numBooks > 5)} \fillin[\mintinline{cpp}{&&} or \mintinline{cpp}{||}][1in] \mintinline{cpp}{(numBooks < 100)}
          -- this should be {\bf true}.
        \item \mintinline{cpp}{(numBooks < 5)} \fillin[\mintinline{cpp}{||}][1in] \mintinline{cpp}{(numBooks > 20)}
          -- this should be {\bf true}.
        \item \fillin[\mintinline{cpp}{!}][1in]\mintinline{cpp}{ (numBooks * 10 == 400)} 
          -- this should be {\bf false}.
      \end{enumerate}

    \item A {\it Boolean Expression} is an expression that uses relational operators and/or logical operators together with
      variables, literal values, and the {\it Boolean} values ``true'' and ``false''. Translate the following Boolean expressions 
      into an English statement.  The first one is done for you.
      \par\vskip 15pt
      \begin{enumerate}[(a)]
        \itemsep 15pt
        \item \mintinline{cpp}{(x == 2) && (y > 3)}
          \hfill\underline{\hspace{8pt}The variable x equals two and the variable y is bigger than three.\hspace{9pt}}
        \item \mintinline{cpp}{(x != 4) || (y <= 7)}
          \hfill\fillin[x is not equal to four or y is less than or equal to seven.][4.2in]
        \item \mintinline{cpp}{(x >= 2) && (x <= 10)}
          \hfill\fillin[x is between two and 10 inclusively][4.2in]
        \item \mintinline{cpp}{!((x == 2) && (y == 1))}
          \hfill\fillin[it is not the case that x is two and y is one.][4.2in]
      \end{enumerate}
      \par\vskip -40pt\ \
      
    \item Write a Boolean Expression for each English statement.\key\\[-2.5mm]
      \par\vskip 10pt

      \begin{enumerate}[(a)]
        \itemsep 15pt
        \item The string {\tt name} is not equal to ``Jane''.                     \hfill \fillin[\mintinline{cpp}{word != "Jane"}][2in]
        \item The value of $x$ is twice that of $y$ or $y$ is less than ten.      \hfill \fillin[\mintinline{cpp}{(x==2*y)||(y<=10)}][2in]
        \item The value of $z$ is between 0 and 5 excluding endpoints.            \hfill \fillin[\mintinline{cpp}{(z>0)&&(z<5)}][2in]
        \item It is not the case that $w$ is between 0 and 5 including endpoints. \hfill \fillin[\mintinline{cpp}{!((w>=0)&&(w<=5))}][2in]
      \end{enumerate}