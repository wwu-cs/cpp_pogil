\model{A Second Program with Pointers}
  \begin{center}
    \small
    \begin{minipage}{4.5in}
      \begin{cpplst}
#include <iostream>

using namespace std;

int main() {
  int numOne = 1892;
  int numTwo = 1973;
  int numThree = 2008;
  int *numArray[3] = {&numOne, &numTwo, &numThree};
  int *p;

  p = &numOne;
  cout << "Number: " << *p << endl;
  p = &numTwo;
  cout << "Number: " << *p << endl;
  p = &numThree;
  cout << "Number: " << *p << endl;
}
      \end{cpplst}
    \end{minipage}
  \end{center}
  
  {\it\large Refer to Model 2 above as your group develops consensus answers
    to the questions below.}

  \quest{20 min}
    
  \Q Without running any code, predict what the output of this
    program will be.
    \begin{answer}[0.75in]
      Number: 1892
      Number: 1973
      Number: 2008
    \end{answer}
    
  \Q The code for this model can be found in the file
    {\tt activity24b.cpp}.  Run this code and see if your
    predictions are correct.
    \begin{answer}[0.25in]
      Yes, the output matched the predictions.
    \end{answer}

  \newpage
    
  \Q When dealing with pointers in C++, the symbol {\tt *} is
    called the {\it dereference operator}.  When it is placed in
    front of a pointer variable, it retrieves the data to which
    the pointer variable points.      
    \begin{enumerate}
      \item On which lines in the model above is the dereference
        operator used?
        \begin{answer}[1in]
          Lines 13, 15, and 17
        \end{answer}
        \vskip -20pt\null

      \item How would the output of this program change if the
        dereference operator were \key\\[-2.5mm] removed from those lines?
        \begin{answer}[1in]
          The output would be memory addresses instead of the
          integer values.
        \end{answer}

      \item How does C++ tell the difference between a {\tt *}
        that is a dereference operator and a {\tt *} that indicates
        multiplication (i.e. \mintinline{cpp}|int x = 2 * y;|)?
        \begin{answer}[1in]
          By context.  If it is in front of a variable name, it is
          a dereference operator.  If it is between two values,
          it is a multiplication operator.
        \end{answer}
    \end{enumerate}

  \vskip -20pt

  \Q The symbol {\tt \&} is called a {\it reference operator}.    
    \begin{enumerate}        
      \item On which lines in the model above is the reference
        operator used?
        \begin{answer}[1in]
        Lines 12, 14, and 16.
        \end{answer}
       
      \item In your own words, describe what the reference
        operator does.
        \begin{answer}[1in]
          It returns the memory address (i.e. a reference) of the
          variable to which it is prepended.
        \end{answer}

      \item Is this usage of symbol {\tt \&} related to its usage
        to indicate ``pass-by-reference'' parameters in a function?
        Explain.
        \begin{answer}[1in]
          Yes.  In both cases, it is used to refer to the memory
          address of a variable rather than its data value.
        \end{answer}

      \item How would things changed if the reference operators 
        were removed from lines 12, 14, and 16 of the model?
        \begin{answer}[1in]
          The code would not compile because the pointer variable
          {\tt p} would be assigned integer values rather than
          memory addresses.
        \end{answer}
      \end{enumerate}

    \vskip -20pt
      
  \Q Consider the array {\tt numArray} defined in this model.
    \par\vskip 10pt
    \begin{enumerate}
      \item What is the type of this array?
        \begin{answer}[0.75in]
          It is an array of integer pointers (i.e. {\tt int*}).
        \end{answer}

      \item Describe the elements that are stored in this array.
        \begin{answer}[1in]
          Each element is a pointer to an integer variable
          ({\tt numOne}, {\tt numTwo}, and {\tt numThree}).
        \end{answer}

      \item Sketch a pointer diagram, similar to that seen in model 1,
        that depicts the values\key\\[-2.5mm] and pointer relationships of all
        five variables defined in model 2.
        \begin{answer}[1.6in]
          Answers will vary
        \end{answer}
    \end{enumerate}
    
  \Q Suppose a new variable was defined as follows:
    \cpp{int** p2 = &numArray[1]}.
    \begin{enumerate}
      \item Describe both the type and purpose of the variable {\tt p2}.
        \begin{answer}[1in]
          It is a pointer to a pointer to an integer.  It
          points to the second element of the {\tt numArray}
          array, which is itself a pointer to the integer variable
          {\tt numTwo}.
        \end{answer}

      \item What would be the output of each of the following commands?
        \begin{enumerate}
          \itemsep 10pt
          \item \cpp{cout << p2 << endl;}
            \hfill\ans[2.5in]{a memory address}

          \item \cpp{cout << *p2 << endl;}
            \hfill\ans[2.5in]{a memory address}

          \item \cpp{cout << **p2 << endl;}
            \hfill\ans[2.5in]{1973}
        \end{enumerate}
        
      \item Sketch a pointer diagram depicting the relationship 
        between the variable {\tt p2} and the rest of the
        variables in this model.
        \begin{answer}[1.25in]
          Answers will vary
        \end{answer}
    \end{enumerate}