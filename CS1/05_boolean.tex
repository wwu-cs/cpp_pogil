\documentclass{exam}
%\documentclass[answers]{exam}
\hbadness=99999
\setlength{\textheight}{9.5in}
\setlength{\textwidth}{6.5in}
\setlength{\topmargin}{-0.75in}
\setlength{\oddsidemargin}{0in}
\setlength{\evensidemargin}{0in}

\usepackage{amsmath}
%\usepackage{amsfonts}
\usepackage{amssymb}
\usepackage{enumerate}
\usepackage[table]{xcolor}
\usepackage{graphicx}
\usepackage{tikz}
%\usepackage{pgfplots}
\usepackage{multicol}

% for syntax highlighting
\usepackage{minted}
\usemintedstyle[cpp]{xcode}

% for overlay of output
\usepackage[overlay,showboxes]{textpos}

\pagestyle{plain}

\setlength\columnsep{50pt}
\newcommand{\key}{\hfill
      \raisebox{-.3\height}{\includegraphics[width=0.6in]{figures/key.png}}}

\begin{document}
  \thispagestyle{empty}
  \setlength{\parindent}{0pt}

  \begin{center}
    \Large Activity \#5: Boolean Expressions \\[5pt]
    \large Recorder's Report\\[20pt]
    \normalsize
    \begin{tabular}{lrp{0.1in}lr}
      Manager:  & \fillin[][2.0in] & & Presenter: & \fillin[][2.0in]\\[15pt]
      Recorder: & \fillin[][2.0in] & & Driver:    & \fillin[][2.0in]\\[15pt]
      Date:     & \fillin[][2.0in] & & Score:     & Satisfactory \hspace{10pt} /
      \hspace{10pt} Not Satisfactory
    \end{tabular}
  \end{center}
  \par\vskip 15pt
  
  Record your team's answers to the key questions (marked with
  \raisebox{-.3\height}{\includegraphics[width=0.5in]{figures/key.png}})
  below.
  \begin{enumerate}[(a)]
    \itemsep 1.75in
    \item Model 1, Question \#5
    \item Model 2, Question \#11
    \item Model 3, Question \#15
  \end{enumerate}

  \clearpage\pagenumbering{arabic} 
  
  \begin{center}
    \Large Activity \#5: Boolean Expressions \\[5pt]
    \large Activity Guide\\[20pt]
  \end{center}

  \begin{center}
    \fbox{
      \begin{minipage}{5.5in}
        {\bf Learning Objectives:} Students will be able to:
        \begin{itemize}
          \item Content:\\[-20pt]
            \begin{itemize}
              \itemsep 0pt
              \item Explain sequential, branching, and looping programming structures.
              \item Explain how relational and logical operators are used in programming.
              \item Use conditional operators with strings and numeric values.
            \end{itemize}
          \item Process\\[-20pt]
            \begin{itemize}
              \itemsep 0pt
              \item Write correct Boolean expressions and compound
                expressions.\\[-5pt]
            \end{itemize}
        \end{itemize}
      \end{minipage}
      }
  \end{center}
  \par\vskip 10pt
  
  
  {\bf\large Model 1: Programming Structures} \\[5pt]
  \ifprintanswers\vskip -30pt\null\fi
  
  \begin{center}
    \renewcommand{\arraystretch}{1.8}
    \begin{tabular}{|c|c|c|}
      \hline      
      \rowcolor{orange!20} \large Sequential Structure & \large Branching Structure & \large Looping Structure \\
      \hline
      \begin{minipage}{0.6in}
        \par\vskip 10pt
        \begin{tikzpicture}
          % three boxes
          \draw (0,0) -- (1.5,0) -- (1.5,1) -- (0,1) -- (0,0);
          \draw (0,1.5) -- (1.5,1.5) -- (1.5,2.5) -- (0,2.5) -- (0,1.5);
          \draw (0,3) -- (1.5,3) -- (1.5,4) -- (0,4) -- (0,3);
          % four arrows
          \draw[-stealth] (0.75,4.25) -- (0.75,4);
          \draw[-stealth] (0.75,3) -- (0.75,2.5);
          \draw[-stealth] (0.75,1.5) -- (0.75,1);
          \draw[-stealth] (0.75,0) -- (0.75,-0.25);
        \end{tikzpicture}
        \par\vskip 4pt\ \
      \end{minipage}
      &
      \begin{minipage}{1.8in}
        \par\vskip 10pt
        \begin{tikzpicture}
          % three boxes
          \draw (1.5,0) -- (3,0) -- (3,1) -- (1.5,1) -- (1.5,0);
          \draw (0,1.75) -- (1.5,1.75) -- (1.5,2.75) -- (0,2.75) -- (0,1.75);
          \draw (3,1.75) -- (4.5,1.75) -- (4.5,2.75) -- (3,2.75) -- (3,1.75);
          % a triangle
          \draw (2.25,3) -- (3.25,3.5) -- (2.25,4) -- (1.25,3.5) -- (2.25,3);
          % and six arrows
          \draw[-stealth] (2.25,4.25) -- (2.25,4);
          \draw[-stealth] (1.25,3.5) node[above left] {\scriptsize False} -- (0.75,3.5) -- (0.75,2.75);
          \draw[-stealth] (3.25,3.5) node[above right] {\scriptsize True} -- (3.75,3.5) -- (3.75,2.75);
          \draw[-stealth] (0.75,1.75) -- (0.75,1.35) -- (2.25,1.35) -- (2.25,1);
          \draw[-stealth] (3.75,1.75) -- (3.75,1.35) -- (2.25,1.35) -- (2.25,1);
          \draw[-stealth] (2.25,0) -- (2.25,-0.25);
        \end{tikzpicture}
        \par\vskip 4pt\ \
      \end{minipage}
      &
      \begin{minipage}{1.7in}
        \centering\par\vskip 10pt
        \begin{tikzpicture}
          % two rectangles
          \draw (0.25,1.5) -- (1.75,1.5) -- (1.75,2.5) -- (0.25,2.5) -- (0.25,1.5);
          \draw (0.25,0) -- (1.75,0) -- (1.75,1) -- (0.25,1) -- (0.25,0);
          % one triangle
          \draw (0,3.5) -- (1,3) -- (2,3.5) -- (1,4) -- (0,3.5);
          % five arrows
          \draw[-stealth] (1,4.25) -- (1,4);
          \draw[-stealth] (1,3) -- node[right] {\scriptsize True} (1,2.5);
          \draw[-stealth] (0.25,2) -- (-0.5,2) -- (-0.5,3.5) -- (0,3.5);
          \draw[-stealth] (2,3.5) -- (2.5,3.5) -- node[right] {\scriptsize False} (2.5,0.5) -- (1.75,0.5);
        \end{tikzpicture}
        \par\vskip 4pt\ \
      \end{minipage}
      \\
      \hline
    \end{tabular}
  \end{center}
  
  {\it\large Refer to Model 1 above as your team develops consensus answers
    to the questions below.}
    \par\vskip 10pt
    
  \begin{enumerate}
    \itemsep 20pt
    
    \item Based only on the pictures in the model above, what do the shapes represent?
      \begin{enumerate}[(a)]
        \item Rectangle
          \begin{solution}[0.4in]
            The rectangle represents a command or group of commands in a program.
          \end{solution}
        \item Diamond
          \begin{solution}[0.4in]
            The diamond represents a true/false decision that must be made.
          \end{solution}
      \end{enumerate}
      \ifprintanswers\vskip -30pt\null\fi
      
    \item Which structure best describes the types of programs you've written thus far?
      \begin{solution}[0.4in]
        The sequential structure -- our code just executes the statements in order  
      \end{solution}
      \ifprintanswers\vskip -30pt\null\fi
      
    \item Which structure(s) allows the programmer to create programs that decide what code to execute?
      \begin{solution}[0.4in]
        Either the branching or the looping structure.
      \end{solution}

\newpage

    \item A {\it relational operator} is used to test for a particular relationship between two
      values.  It returns either {\bf true}, if the relationship holds, or {\bf false}, if it 
      does not.  What relationship does each operator below test for?  If unsure, use the
      file {\tt activity05a.cpp} to try them out.
      
      \begin{enumerate}[(a)]
        \itemsep 10pt
        \begin{multicols}{2}
          \item $<$  \hspace{10pt} \fillin[less than][2in]
          \item $<=$ \hspace{3pt}  \fillin[less than or equal][2in]
          \item $!=$ \hspace{5pt}  \fillin[not equal][2in]
          \item $>$  \hspace{10pt} \fillin[greater than][2in]
          \item $>=$ \hspace{3pt}  \fillin[greater than or equal][2in]
          \item $==$ \hspace{3pt}  \fillin[equal to][2in]
        \end{multicols}
      \end{enumerate}
      \ifprintanswers\vskip -30pt\null\fi
      
    \item Use the variable values below to determine the value of the
      following expressions.\key\\[-5mm]
      
      \begin{center}
        \begin{minipage}{3.5in}
          \begin{minted}[
            frame=lines,
            framesep=2mm,
            bgcolor=gray!15,
            baselinestretch=1.2
          ]{cpp}
  int x = 4, y = 5, z = 4;
          \end{minted}
        \end{minipage}
      \end{center}
      \par\vskip 10pt
      
      \begin{enumerate}[(a)]
        \itemsep 15pt
        \begin{multicols}{2}
          \item \mintinline{cpp}|x > y|
            \hspace{14pt} \fillin[false][0.8in]
          \item \mintinline{cpp}|x < y|
            \hspace{14pt} \fillin[true][0.8in]
          \item \mintinline{cpp}|x == y|
            \hspace{9pt} \fillin[false][0.8in]
          \item \mintinline{cpp}|x != y|
            \hspace{10pt} \fillin[true][0.8in]
          \item \mintinline{cpp}|x >= z|
            \hspace{10pt} \fillin[true][0.8in]
          \item \mintinline{cpp}|x <= z|
            \hfill \fillin[true][0.8in]
          \item \mintinline{cpp}|x + y > 2 * x|
            \hfill \fillin[true][0.8in]
          \item \mintinline{cpp}|y * x - z != 4 % 4 + 15|
            \hfill \fillin[false][0.8in]
          \item \mintinline{cpp}|pow(x,2) == abs(-16)|
            \hfill \fillin[true][0.8in]
        \end{multicols}
      \end{enumerate}
 
    \item What are the two possible answers for each expression in that last question?
      \begin{solution}[0.5in]
        They can be true or false.
      \end{solution}
            
    \item Assume the following strings have been defined. Determine the results of the following
      expressions.
      
      \begin{center}
        \begin{minipage}{3.5in}
          \begin{minted}[
            frame=lines,
            framesep=2mm,
            bgcolor=gray!15,
            baselinestretch=1.2
          ]{cpp}
  string word1 = "hello";
  string word2 = "good-bye";
          \end{minted}
        \end{minipage}
      \end{center}
      \par\vskip 10pt
      
      \begin{enumerate}[(a)]
        \itemsep 15pt
        \begin{multicols}{2}
          \item \mintinline{cpp}|word1 == word2|
            \hspace{14pt} \fillin[false][1in]
          \item \mintinline{cpp}|word1 != word2|
            \hspace{14pt} \fillin[true][1in]
          \item \mintinline{cpp}|word1 < word2|
            \hspace{14pt} \fillin[false][1in]
          \item \mintinline{cpp}|word1 >= word2|
            \hspace{14pt} \fillin[true][1in]
        \end{multicols}
      \end{enumerate}


    \item How do relational operators work on strings?
      \begin{solution}[1in]
        \par
        They compare strings lexicographical.  Meaning, words that come first in the dictionary would be
        smaller than others.  To be equal, the strings must be identical.
      \end{solution}

\newpage
  
  {\bf\large Model 2: A Truth Table} \\
  
    \begin{center}
      \renewcommand{\arraystretch}{1.4}
      \begin{tabular}{|c|c|c|c|c|}
        \hline
        \rowcolor{orange!20} Condition \#1 & Condition \#2 & Negation (NOT) & Conjunction (AND) & Disjunction (OR) \\
        \rowcolor{orange!20} \mintinline{cpp}|p| & \mintinline{cpp}|q| & \mintinline{cpp}|! p| & \mintinline{cpp}|p && q| & \mintinline{cpp}{p || q} \\
        \hline
        True & True   & False & True  & True \\
        True & False  & False & False & True \\
        False & True  & True  & False & True \\
        False & False & True  & False & False \\
        \hline
      \end{tabular}
    \end{center}

  {\it\large Refer to Model 2 above as your team develops consensus answers
    to the questions below.}
    \par\vskip 10pt

    \item The symbols \mintinline{cpp}{&&}, \mintinline{cpp}{||}, and \mintinline{cpp}{!} are called
      {\it logical operators} because they combine conditions that are either true or false to create
      new compound conditions. Given the variable definitions below, fill in the
      appropriate logical operator to produce the desired truth value.
      \begin{center}
        \begin{minipage}{3.5in}
          \begin{minted}[
            frame=lines,
            framesep=2mm,
            bgcolor=gray!15,
            baselinestretch=1.2
          ]{cpp}
  int numBooks = 40;
          \end{minted}
        \end{minipage}
      \end{center}
      \par\vskip 10pt
      \begin{enumerate}[(a)]
        \itemsep 15pt
        \item \mintinline{cpp}{(numBooks > 5)} \fillin[\mintinline{cpp}{&&} or \mintinline{cpp}{||}][1in] \mintinline{cpp}{(numBooks < 100)}
          -- this should be {\bf true}.
        \item \mintinline{cpp}{(numBooks < 5)} \fillin[\mintinline{cpp}{||}][1in] \mintinline{cpp}{(numBooks > 20)}
          -- this should be {\bf true}.
        \item \fillin[\mintinline{cpp}{!}][1in]\mintinline{cpp}{ (numBooks * 10 == 400)} 
          -- this should be {\bf false}.
      \end{enumerate}

    \item A {\it Boolean Expression} is an expression that uses relational operators and/or logical operators together with
      variables, literal values, and the {\it Boolean} values ``true'' and ``false''. Translate the following Boolean expressions 
      into an English statement.  The first one is done for you.
      \par\vskip 15pt
      \begin{enumerate}[(a)]
        \itemsep 15pt
        \item \mintinline{cpp}{(x == 2) && (y > 3)}
          \hfill\underline{\hspace{8pt}The variable x equals two and the variable y is bigger than three.\hspace{9pt}}
        \item \mintinline{cpp}{(x != 4) || (y <= 7)}
          \hfill\fillin[x is not equal to four or y is less than or equal to seven.][4.2in]
        \item \mintinline{cpp}{(x >= 2) && (x <= 10)}
          \hfill\fillin[x is between two and 10 inclusively][4.2in]
        \item \mintinline{cpp}{!((x == 2) && (y == 1))}
          \hfill\fillin[it is not the case that x is two and y is one.][4.2in]
      \end{enumerate}
      \par\vskip -40pt\ \
      
    \item Write a Boolean Expression for each English statement.\key\\[-2.5mm]
      \par\vskip 10pt

      \begin{enumerate}[(a)]
        \itemsep 15pt
        \item The string {\tt name} is not equal to ``Jane''.                     \hfill \fillin[\mintinline{cpp}{word != "Jane"}][2in]
        \item The value of $x$ is twice that of $y$ or $y$ is less than ten.      \hfill \fillin[\mintinline{cpp}{(x==2*y)||(y<=10)}][2in]
        \item The value of $z$ is between 0 and 5 excluding endpoints.            \hfill \fillin[\mintinline{cpp}{(z>0)&&(z<5)}][2in]
        \item It is not the case that $w$ is between 0 and 5 including endpoints. \hfill \fillin[\mintinline{cpp}{!((w>=0)&&(w<=5))}][2in]
      \end{enumerate}
      
\newpage

  {\bf\large Model 3: A C++ Program} \\[-10pt]
  \begin{center}
    \begin{minipage}{5.5in}
      \begin{minted}[
        frame=lines,
        framesep=2mm,
        bgcolor=gray!15,
        baselinestretch=1.2,
        linenos
      ]{cpp}
#include <iostream>
using namespace std;

int main() {
  // ----------------------------------------- declare variables
  int a = 5, b = 2, c = 0;
  // ----------------------------------------- print output
  cout << "Line 8:  " << (a == b) << endl;
  cout << "Line 9:  " << (a != b) << endl;
  cout << "Line 10: " << (a = b) << endl;
  cout << "Line 11: " << ( (a > c) || ((b / c) == 1) ) << endl;
  cout << "Line 12: " << ( (a > c) && ((b / c) == 1) ) << endl;
}      
      \end{minted}
    \end{minipage}
  \end{center}
  \TPMargin{5pt}
  
  {\it\large Refer to Model 3 above as your group develops consensus answers
    to the questions below.}
    \par\ifprintanswers\vskip -20pt\null\else\vskip 10pt\fi

      \item This program can be found in {\tt activity05c.cpp}.  Run it and determine the output produced by each
        \mintinline{cpp}|cout| statement.
        \ifprintanswers\else\par\vskip 20pt\fi
        \begin{enumerate}[(a)]
          \ifprintanswers\itemsep 5pt\else\itemsep 15pt\fi
          \item The \mintinline{cpp}|cout| on line 8:  \hfill
            \fillin[Line 8:  0][4.5in]
          \item The \mintinline{cpp}|cout| on line 9:  \hfill
            \fillin[Line 9:  1][4.5in]
          \item The \mintinline{cpp}|cout| on line 10:  \hfill
            \fillin[Line 10: 2][4.5in]
          \item The \mintinline{cpp}|cout| on line 11:  \hfill 
            \fillin[Line 11: 1][4.5in]
          \item The \mintinline{cpp}|cout| on line 12:  \hfill 
            \fillin[Floating-point Error, Core Dump][4.5in]
        \end{enumerate}
                
      \item Based on the output observed in lines 8 and 9, how does C++ represent true and false?
        \begin{solution}[0.4in]
          True is represented as 1 and false as 0.
        \end{solution}
        \ifprintanswers\vskip -50pt\else\vskip -30pt\fi\null
        
      \item Based on the output from lines 8 and 10, how does C++ treat \mintinline{cpp}|=| and \mintinline{cpp}|==|
        differently?\key\\[-2.5mm]
        \begin{solution}[0.4in]
          \par
          A single = is an assignment statement (line 10 assigns the value of b to a) that
          returns the value being assigned.  The double == is a relational operator 
          and returns either true (1) or false (0).
        \end{solution}
        
      \item In the code above, line 11 prints a value, while line 12 produces an error.  Why do you think
        that is?
        \begin{solution}[0.4in]
          \par
          On line 11, only the first part, \mintinline{cpp}|a > c|, is evaluated because the second part of the OR
          statement does not need to be checked to tell that the whole thing will be true.  So we never divide by zero.
          But on line 12, the AND is only true if both sides are true, so we must check the second part, resulting in
          division by zero.
        \end{solution}                
        
        
  \end{enumerate}  
    
\end{document}
