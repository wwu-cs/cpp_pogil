\model{The Factorial Function}
  \begin{center}
    \renewcommand{\arraystretch}{1.2}
    \begin{tabular}{c|cccccc}
      $n$ & 0 & 1 & 2 & 3 & 4 & 5 \\
      \hline
      $n!$ & 1 & 1 & 2 & 6 & 24 & 120 \\
    \end{tabular}
  \end{center}
  
  {\it\large Refer to Model 1 above as your team develops consensus answers
    to the questions below.}

  \quest{15 min}

  \Q In mathematics, the {\it factorial} function for a natural
    number $n$ is denoted by $n!$.  It is the product of all positive
    integers less than or equal to $n$.  For example:
    \[
      5! = 5\times 4 \times 3 \times 2 \times 1 = 120
    \]
    Consider how to calculate $4!$.
    \begin{enumerate}
      \item Write out all of the numbers that need to be multiplied
        to get $4!$.
        \begin{answer}[0.5in]
          \[ 4! = 4\times 3\times 2\times 1 \]
        \end{answer}

      \item Rewrite the expression using $3!$ instead of $3\times
        2\times 1$.
        \begin{answer}[0.5in]
          \[ 4! = 4\times 3! \]            
        \end{answer}
    \end{enumerate}
    \par\vskip -40pt\null
    
  \Q Express the factorials as a product of a single
    natural number with a simpler factorial\key\\[-2mm].
    \begin{enumerate}
      \itemsep 10pt
      \begin{multicols}{2}
        \item $3! =$ \hspace{0.3in} \ans[2in]{$3 \times 2!$}
        \item $2! =$ \hspace{0.3in} \ans[2in]{$2 \times 1!$}
        \item $100! = $\hfill \ans[2in]{$100 \times 99!$}
        \item $n! = $\hfill \ans[2in]{$n\times (n-1)!$}
      \end{multicols}
    \end{enumerate}
    
  \item Now consider the very first natural number, 0.
    \begin{enumerate}
      \item Based on the model, what is the value of
        $0!$? \ans[1.5in]{1}\par\vskip 10pt

      \item Does it make sense to define $0!$ in terms of a simpler
        factorial?  Explain.        
        \begin{answer}[0.5in]
          No, we can't say that $0 \times -1! = 1$ because factorial
          is only defined for non-negative numbers.
        \end{answer}

      \newpage

      \item When we define the value of a function by
        referencing that same function for a simpler value, we will
        eventually reach a point where there are no simpler values
        and we have to just give a concrete value to the function.
        This is called a {\it base case}.  What is the base case for
        the factorial function?
        \begin{answer}[0.5in]
          The base case is $0! = 1$.
        \end{answer}
    \end{enumerate}

  \vskip -20pt
    
  \Q Suppose you already have a working implementation of the
    function declared below.
    \begin{center}
      \cpp{int factorial(int n);}
    \end{center}
    \begin{enumerate}
      \itemsep 10pt
      \item How could you compute $100!$ without calling
        \cpp{factorial(100)}?  Give a C++ command to do this.
        \begin{answer}[0.5in]
          \cpp{int result = 100 * factorial(99);}
        \end{answer}

      \item How could you compute $n!$ without calling
        \cpp{factorial(n)}?  Give a C++ command to do this.
        \begin{answer}[0.5in]
          \cpp{int result = n * factorial(n-1);}
        \end{answer}
    \end{enumerate}