% Source: https://drive.google.com/drive/folders/1JeCcwPkeQ1e5LSFeBPQm2Lo_3tSUscSw
% File: ".pdf"
% Access: 04-15-2022

% comment out for student version
%\ifdefined\Student\relax\else\def\Teacher{}\fi

\documentclass[12pt]{article}

\title{Activity \#15: Coding Style}
\author{Clif Kussmaul}
\newcommand{\activityeditor}{Preston Carman}
\newcommand{\activitysource}{\url{https://drive.google.com/drive/folders/1JeCcwPkeQ1e5LSFeBPQm2Lo_3tSUscSw}}
\date{Spring 2022}

\input{../../cspogil.sty}

\begin{document}

\begin{center}
  \Large Activity \#15: Coding Style \\[5pt]
  \large Recorder's Report\\[20pt]
  \normalsize
  \begin{tabular}{lrp{0.1in}lr}
      Manager:  & \ans{} &  & Reader: & \ans{}            \\[15pt]
      Recorder: & \ans{} &  & Driver: & \ans{}            \\[15pt]
      Date:     & \ans{} &  & Score:  & Satisfactory \hspace{10pt} /
      \hspace{10pt} Not Satisfactory
  \end{tabular}
\end{center}
\par\vskip 15pt

Record your team's answers to the key questions (marked with
\raisebox{-.3\height}{\includegraphics[width=0.5in]{../figures/key.png}})
below.
\begin{enumerate}[label=(\alph*)]
  \itemsep 1.25in
  \item Model 1, Question \#8
  \item Model 2, Question \#12
  \item Model 3, Question \#17
\end{enumerate}

\newpage
\maketitle

In this course, you will work in teams of 3--4 students to learn new concepts.
This activity will introduce you to the process of analyzing an algorithm complexity.

%\rolenames

\guide{
  \item Identify good C++ programming style for variables, expressions, and functions
  \item Articulate style guidelines for variables, expressions, and functions in C++
  \item Identify good style in coding practices

}{
  \item Rewrite code to improve its style \\[-5pt]
}{
No additional notes.
}


  \model{Variables and Expressions} \\
  \begin{center}
    \renewcommand{\arraystretch}{1.3}
    \begin{tabular}{|p{3in}|p{3in}|}
      \hline
      \rowcolor{orange!20} 
        \begin{minipage}{2.8in}\centering {\bf Code Block $X$}\end{minipage} & 
        \begin{minipage}{2.8in}\centering {\bf Code Block $Y$}\end{minipage} \\
      \hline
      \begin{minipage}{3in}
        \small
        \begin{cprlst}[
          frame=lines,
          framesep=2mm,
          bgcolor=gray!15,
          baselinestretch=1.2,
        ]{cpp}
int i1 = 10, i2 = 5;
double c1 = 9.99, c2 = 4.99; 
double d1 = 0.06, d2 = 0.99;
double s1 = i1*c1+i2*c2;
double t1 = s1+s1*d1+i1*d2;
        \end{cprlst}
        \vskip 70pt\null
      \end{minipage}
      &
      \begin{minipage}{3in}
        \small
        \begin{cprlst}[
          frame=lines,
          framesep=2mm,
          bgcolor=gray!15,
          baselinestretch=1.2,
        ]{cpp}
int numCD      = 10;
int numMP3     = 5;
double costCD  = 9.99;
double costMP3 = 4.99;
double shipCD  = 0.99;
double rateTax = 0.06;
double subCost = (numCD * costCD) + 
                 (numMP3 * costMP3);
double subShip = numCD * shipCD;
double subTax  = subCost * rateTax;
double total   = subCost + subTax + subShip;
        \end{cprlst} 
      \end{minipage}
      \\
      \hline
    \end{tabular} 
  \end{center}  
  \par\vskip 10pt
  
  {\it\large Refer to Model 1 above as your team develops consensus answers
    to the questions below.}
    \par\vskip 10pt
    
  \begin{enumerate}
    \itemsep 20pt

    \Q Determine which code block above ($X$ or $Y$) matches the
      description below.
      \par\vskip 15pt
      
      \begin{enumerate}[(a)]
        \itemsep 15pt
        \item Is shorter and would take less time to type: \hfill \ans[1in]{$X$} \hspace{1in}\null
        \item Uses more variables: \hfill \ans[1in]{$Y$} \hspace{1in}\null
        \item Would be easier to edit or debug: \hfill \ans[1in]{$Y$} \hspace{1in}\null
      \end{enumerate}
      % \ifprintanswers\vskip -20pt\null\fi
      
    \Q When we study foreign languages we learn about:
      \begin{itemize}
        \item {\it Syntax} -- The rules used to spell words and construct
          grammatically correct sentences
        \item {\it Semantics} -- The meanings of words and how to interpret sentences
        \item {\it Style} -- The distinctive voice of an author
          expressed within the rules of syntax and semantics          
      \end{itemize}
      All three of these are also present in a programming language.
      Are the differences between code block $X$ and $Y$ above
      differences in syntax, semantics, or style?  Explain.
      \begin{answer}[0.5in]
        \par
        They are differences in style since both are syntactically
        correct (both will compile) and semantically equivalent (the
        both compute the same value).
      \end{answer}
      
\newpage      
      
    \Q Each pair of code blocks below is syntactically and
      semantically equivalent.  Decide which one is better based only
      on the style.  Enter $X$, $Y$, or ? (if you think they are
      equally good).

      {\renewcommand{\arraystretch}{2.3}
      \begin{tabular}{|c|p{2.55in}|p{2.55in}|c|}
        \hline
        \rowcolor{orange!20} & Block $X$ & Block $Y$ & Choice \\
        \hline
        (a) & 
          \cpp{int i1 = 3, i2 = 7;} 
          & 
          \cpp{int numClosed = 3, numOpen = 7;} 
          & 
          \ans[0.4in]{$Y$} \\
        \hline
        (b) &
          \cpp{int numCat = 2, numDog = 5;} 
          & 
          \cpp{int numCat = 2, dogNum = 5;} 
          & 
          \ans[0.4in]{$X$} \\
        \hline
        (c) &
          \cpp{int numson = 3, isdone = true;}
          & 
          \cpp{int numSon = 3, isDone = true;} 
          & 
          \ans[0.4in]{$Y$} \\
        \hline
        (d) &
          \begin{minipage}{2.5in}
            \par\vskip 10pt
            \cpp{int costCoffee = 2, costTea = 1;}\\
            \cpp{int soldCoffee = 4, soldTea = 6;}\\
          \end{minipage}
          & 
          \begin{minipage}{2.5in}
            \par\vskip 10pt
            \cpp{int costCoffee = 2, costTea = 1;}\\
            \cpp{int soldTea = 6, soldCoffee = 4;}\\
          \end{minipage}
          & 
          \ans[0.4in]{$X$} \\
        \hline
      \end{tabular}}
      
    \Q Based on your answers above, describe what you think is a 
      good style for naming variables.
      \begin{answer}[1in]
        \par
        Variable names should be chosen to match the purpose of a
        variable.  If they are composed of multiple words, those words
        should be separated in some way (capitals, underscores, etc).
        Their names should be consistent (same order to compound
        words, etc).
      \end{answer}
      
    \Q Each pair of code blocks below is syntactically and
      semantically equivalent.  Decide which one is better based only on
      the style.  Enter $X$, $Y$, or ? (if you think they are equally good).
      
      {\renewcommand{\arraystretch}{2.3}
      \begin{tabular}{|c|p{2.5in}|p{2.5in}|c|}
        \hline
        \rowcolor{orange!20} & Block $X$ & Block $Y$ & Choice \\
        \hline
        (a) & 
          \cpp{s1=i1*c1+i2*c2;}
          & 
          \cpp{s1 = i1*c1 + i2*c2;}
          & 
          \ans[0.5in]{$Y$} \\
        \hline
        (b) &
          \cpp{s1=(i1*c1)+(i2*c2);}
          & 
          \cpp{s1=i1*c1+i2*c2;}
          & 
          \ans[0.5in]{$X$} \\
        \hline
        (c) &
          \cpp{s1 = c1*i1 + i2*c2;}
          & 
          \cpp{s1 = i1*c1 + i2*c2;}
          & 
          \ans[0.5in]{$Y$} \\
        \hline
        (d) &
          \begin{minipage}{2in}
            \begin{cprlst}[
              frame=none,
              bgcolor=white,
            ]{cpp}
float total = nCD*sCD +
  (nCD*cCD + nMP3*cMP3) *
  (1+rateTax);
            \end{cprlst} 
          \end{minipage}
          & 
          \begin{minipage}{2in}
            \begin{cprlst}[
              frame=none,
              bgcolor=white,
            ]{cpp}
float cost  = nCD*cCD + nMP3*cMP3;
float ship  = nCD*sCD;
float tax   = cost * rateTax;
float total = cost + tax + ship;
            \end{cprlst} 
          \end{minipage}
          & 
          \ans[0.5in]{$Y$} \\
        \hline
      \end{tabular}}
      
    \Q Based on your answers above, describe what you think is a
      good style for writing expressions.
      \begin{answer}[1in]
        \par
        Expressions should make use of white space to separate out
        different parts.  Using parentheses to group things, even if
        not required, is also a good idea.  It is better to split long
        expressions up into individual sub-computations that have
        meaning in the context of the problem being solved.
      \end{answer}

\newpage
      
    \Q Code that can be read and understood without separate
      documentation is called {\it self-documenting code}.  What are
      some way to make expressions self-documenting?
      \begin{answer}[1in]
        \begin{itemize}
          \item Use variable names that have meaning in the context of
            the problem being solved.
          \item Split long computations up into meaningful steps.
          \item Group pieces of the computation together so that the
            order of precedence is clearly seen.
        \end{itemize}
      \end{answer}
      % \par\ifprintanswers\vskip -40pt\else\vskip -20pt\fi\null

    \Q Rewrite the code below to be self-documenting.  Use
      your style recommendations above.\key\\[-2.5mm]
      \begin{center}
        \begin{minipage}{5.25in}
          \begin{cprlst}[
            frame=lines,
            framesep=2mm,
            bgcolor=gray!15,
            baselinestretch=1.2
          ]{cpp}
// calculate trip cost from 2-way flights, nights in hotels, and meals
double fc=500, ch=150, mc=30, nn=5;         // costs and number nights
double c = 2*fc+ch*nn+3*mc*nn;                // compute total cost
          \end{cprlst}
        \end{minipage}
      \end{center}
      \begin{answer}[2in]
        \scriptsize\vskip -35pt\null
        \begin{center}
          \begin{minipage}{4.25in}
            \begin{cprlst}[
              frame=lines,
              framesep=2mm,
              bgcolor=gray!15,
              baselinestretch=1.2,
              linenos
            ]{cpp}
// calculate trip cost from 2-way flights, nights in hotels, and meals
double flightPrice = 500;
double hotelPrice  = 150;
double mealPrice   = 30;
int    numNights   = 5;
double travelCost  = 2 * flightPrice;
double lodgingCost = numNights * hotelPrice;
double mealCost    = 3 * numNights * mealPrice;
double totalCost   = travelCost + lodgingcost + mealCost;
            \end{cprlst}
          \end{minipage}
          \vskip -10pt\null
        \end{center}
      \end{answer}
  \newpage
  \model{Function Calls} \\
  \begin{center}
    \renewcommand{\arraystretch}{1.3}
    \begin{tabular}{|p{3in}|p{3.5in}|}
      \hline
      \rowcolor{orange!20} 
        \begin{minipage}{2.8in}\centering {\bf Code Block $X$}\end{minipage} & 
        \begin{minipage}{2.8in}\centering {\bf Code Block $Y$}\end{minipage} \\
      \hline
      \begin{minipage}{3in}
        \scriptsize
        \begin{cpplst}
int i1 = f1("c"); int i2 = f2("m"); 
int c1 = f3("c"); int c2 = f4("m");
int s1 = g1(i1,c1)+g1(i2*c2); 
int t1 = s1 + h1(s1) + h2(i1,f5("c"));
        \end{cpplst}
        \vskip 30pt\null
      \end{minipage}
      &
      \begin{minipage}{3.5in}
        \scriptsize
        \begin{cpplst}
int numCD      = getNum ("CD");
int numMP3     = getNum ("MP3");
double costCD  = getCost("CD");
double costMP3 = getCost("MP3");
double subCost = getSubCost(numCD, costCD) +
                 getSubCost(numMP3, costMP3);
double subShip = getSubShip(numCD, getShip("CD"));
double total   = subCost + subShip + getTax(subCost);
        \end{cpplst} 
      \end{minipage}
      \\
      \hline
    \end{tabular} 
  \end{center}  
  
  {\it\large Refer to Model 2 above as your team develops consensus answers
    to the questions below.}

  \quest{15 min}

  \Q Determine which code block above ($X$ or $Y$) matches the
    description below.
    \begin{enumerate}
      \itemsep 10pt
      \item Is shorter and would take less time to type: \hfill \ans[1in]{$X$}
      \item Declares more variables: \hfill \ans[1in]{$Y$}
      \item Would be easier to update or debug if a problem is found: \hfill \ans[1in]{$Y$}
    \end{enumerate}

  \Q Each pair of code blocks below is syntactically and
    semantically equivalent.  Decide which one is better based only
    on the style.  Enter $X$, $Y$, or ? (if you think they are
    equally good).
    \begin{center}
      {\renewcommand{\arraystretch}{2.3}
      \begin{tabular}{|c|p{2.5in}|p{2.5in}|c|}
        \hline
        \rowcolor{orange!20} & Block $X$ & Block $Y$ & Choice \\
        \hline
        (a) & 
          \cpp{tree(pic,x,y);}
          & 
          \cpp{drawTree(pic,x,y);}
          & 
          \ans[0.5in]{$Y$} \\
        \hline
        (b) &
          \cpp{checkSize(a);}
          & 
          \cpp{hasValidSize(a);}
          & 
          \ans[0.5in]{$Y$} \\
        \hline
        (c) &
          \begin{minipage}{2in}
            \par\vskip 10pt
            \cpp{getCostToShip(item,num);}\\
            \cpp{getCostOfTax (item,num);}\\
          \end{minipage}
          & 
          \begin{minipage}{2in}
            \par\vskip 10pt
            \cpp{getShipCost (num,item);}\\
            \cpp{getCostOfTax(item,num);}\\
          \end{minipage}
          & 
          \ans[0.5in]{$X$} \\
        \hline
      \end{tabular}}
    \end{center}
    
  \Q Determine which type of C++ programming concept, a {\bf variable} or a {\bf function},
    best matches each description below.
    \begin{enumerate}
      \itemsep 10pt
      \item Is used to represent information \hfill \ans[2in]{variable}
      \item Is used to represent an action or a set of steps \hfill \ans[2in]{function}
      \item Should be named with a verb or a verb phrase \hfill \ans[2in]{function}
      \item Should be named with a noun or a noun phrase \hfill \ans[2in]{variable}
    \end{enumerate}

  \vskip -10pt
    
  \Q Based on the above, what do you think is good style for naming
    and calling\key\\[-2.5mm] functions?
    \begin{answer}[1in]
      \par
      Functions should be named using verbs.  Their parameter lists should be consistent (i.e. two
      functions with the same parameters should use the same order).  Their names should be
      descriptive of what they actually do.
    \end{answer}
  \newpage
  \model{Comments} \\
  \begin{center}
    \renewcommand{\arraystretch}{1.3}
    \begin{tabular}{|p{3in}|p{3in}|}
      \hline
      \rowcolor{orange!20} 
        \begin{minipage}{2.8in}\centering {\bf Code Block $X$}\end{minipage} & 
        \begin{minipage}{2.8in}\centering {\bf Code Block $Y$}\end{minipage} \\
      \hline
      \begin{minipage}{3in}
        \scriptsize
        \begin{cprlst}[
          frame=lines,
          framesep=2mm,
          bgcolor=gray!15,
          baselinestretch=1.2,
        ]{cpp}
// call getNum()        
int numCD      = getNum ("CD");
int numMP3     = getNum ("MP3");
// call getCost()
double costCD  = getCost("CD");
double costMP3 = getCost("MP3");
// call SubCost
double subCost = getSubCost(numCD, costCD) +
                 getSubCost(numMP3, costMP3);
// call getSubShip()
double subShip = getSubShip(numCD, getShip("CD"));
// call getTax()
double total   = subCost + subShip + getTax(subCost);        
        \end{cprlst}
      \end{minipage}
      &
      \begin{minipage}{3in}
        \scriptsize
        \begin{cprlst}[
          frame=lines,
          framesep=2mm,
          bgcolor=gray!15,
          baselinestretch=1.2,
        ]{cpp}
// get the number of each item type
int numCD      = getNum ("CD");
int numMP3     = getNum ("MP3");
// get the cost of each item type
double costCD  = getCost("CD");
double costMP3 = getCost("MP3");
// combine subtotals from each item
double subCost = getSubCost(numCD, costCD) +
                 getSubCost(numMP3, costMP3);
// add in shipping and tax to get grand total                 
double subShip = getSubShip(numCD, getShip("CD"));
double total   = subCost + subShip + getTax(subCost);        
        \end{cprlst} 
        \par\vskip 3pt\null
      \end{minipage}
      \\
      \hline
    \end{tabular} 
  \end{center}  
  \par\vskip 10pt
                                                
      
  {\it\large Refer to Model 3 above as your team develops consensus answers
    to the questions below.}

\newpage         

  \Q Determine which code block above ($X$ or $Y$) matches the
    description below.
    \par\vskip 15pt
    
    \begin{enumerate}[(a)]
      \itemsep 15pt
      \item Has the most lines with comments: \hfill \ans[1in]{$X$}
      \item Has comments that explain how and why the code works: \hfill \ans[1in]{$Y$}
      \item Has comments that could be generated automatically by a computer program: \hfill \ans[1in]{$X$}
    \end{enumerate}
    
    \Q Based on your observations above, describe what you think is good style for comments.
      \begin{answer}[1in]
        \par
        Comments should describe what the task being performed in the related lines of code
        actually is.  They should not use the same names and/or jargon that is used in the code.
      \end{answer}
      
    \Q Use your observations to suggest good comments for the two comment lines in the code
      below.
      \begin{center}
        \begin{tabular}{p{3in}p{3in}}
          \begin{minipage}{3in}
            \small
            \begin{cprlst}[
              frame=lines,
              framesep=2mm,
              bgcolor=gray!15,
              baselinestretch=1.2
            ]{cpp}
int getRange (vector<int> valList) {
  /* COMMENT A */
  int min = valList.at(0);
  int max = valList.at(0);
  /* COMMENT B */
  for(int i=0; i<valList.size(); i++) {
    if (valList.at(i) < min) 
      min = valList.at(i);
    if (valList.at(i) > max)
      max = valList.at(i);
  }
  return (min - max);
}
            \end{cprlst}
          \end{minipage}
          &
          \begin{minipage}{3in}
            \begin{enumerate}[(a)]
              \item \cpp{/* COMMENT A */}
                \begin{answer}[1in]
                  \par
                  Default minimum and maximum are first value in the vector
                \end{answer}
              \item \cpp{/* COMMENT B */}
                \begin{answer}[1in]
                  \par
                  Search through vector looking for smaller and/or larger values
                \end{answer}
            \end{enumerate}
          \end{minipage}
        \end{tabular}
      \end{center}

    \Q Good programming style is important not just for naming variables and functions or
      making comments.  It is also important in other coding practices.  For each
      pair of practices below, decide in which case it would be easier to find and fix any
      errors. Enter $X$, $Y$, or ? (if you think they are equally good).

      {\renewcommand{\arraystretch}{2}
      \begin{tabular}{|c|p{2.5in}|p{2.5in}|c|}
        \hline
        \rowcolor{orange!20} & Block $X$ & Block $Y$ & Choice \\
        \hline
        (a) & 10 existing lines of code (LoC)
            & 100 existing LoC
            & \ans[0.5in]{$X$} \\
        \hline
        (b) & 100 new LoC
            & 90 working LoC plus 10 new LoC
            & \ans[0.5in]{$Y$} \\
        \hline
        (c) & 1000 new LoC
            & 900 working LoC plus 100 new LoC
            & \ans[0.5in]{$Y$} \\
        \hline
        (d) & 1000 working LoC plus 10 new LoC
            & 500 working LoC plus 100 new LoC
            & \ans[0.5in]{$X$} \\
        \hline
        (e) & 1000 LoC all in one function
            & 1000 LoC split across 20 functions
            & \ans[0.5in]{$Y$} \\
        \hline
        (f) & 100 LoC with 10 \cpp{if-else} blocks
            & 100 LoC with 5 \cpp{if-else} blocks
            & \ans[0.5in]{?} \\
        \hline
      \end{tabular}}
      
      {\renewcommand{\arraystretch}{2.3}
      \begin{tabular}{|c|p{2.5in}|p{2.5in}|c|}
        \hline
        \rowcolor{orange!20} & Block $X$ & Block $Y$ & Choice \\
        \hline
        (g) & 100 LoC with four independent loops
            & 100 LoC two pairs of nested loops
            & \ans[0.5in]{$X$} \\
        \hline
        (h) &
          \begin{minipage}{2in}
            \small
            \begin{cprlst}[
              frame=none,
              bgcolor=white,
            ]{cpp}
// code with the structure
if (...) {
  if (...) {
    ...
  } else {
    ...
  }
} else {
  if (...) {
    ...
  } else {
    ...
  }
}
            \end{cprlst} 
          \end{minipage}
          & 
          \begin{minipage}{2in}
            \small
            \begin{cprlst}[
              frame=none,
              bgcolor=white,
            ]{cpp}
// code with the structure
if (...) {
  ...
} else if (...) {
  ...
} else if (...) {
  ...
} else if (...) {
  ...
}
            \end{cprlst}
            \par\vskip 35pt\null
          \end{minipage}
          &
          \ans[0.5in]{$X$} \\
        \hline
        (h) &
          \begin{minipage}{2in}
            \small
            \begin{cprlst}[
              frame=none,
              bgcolor=white,
            ]{cpp}
// code with the structure
for(int i=0; i<n; i++) {
  if (...) {
    ...
  } else {
    ...
  }
}
            \end{cprlst}
            \par\vskip 10pt\null
          \end{minipage}
          & 
          \begin{minipage}{2in}
            \small
            \begin{cprlst}[
              frame=none,
              bgcolor=white,
            ]{cpp}
// code with the structure
if (...) {
  for(int i; i<n; i++) {
    ...
  }
} else {  
  for(int i; i<n; i++) {
    ...
  }
}
            \end{cprlst}
          \end{minipage}
          &
          \ans[0.5in]{$X$} \\
        \hline
      \end{tabular}}
      \par\vskip -30pt\null

    \Q Based on your answers, summarize advice for good practices in writing and
      debugging code.\hspace{5pt}\key\\[-2.5mm]
      % \ifprintanswers\vskip -20pt\null\fi
      \begin{answer}[1in]
        \begin{itemize}
          \itemsep 0pt
          \item You should write code in small chunks, testing after each chunk
          \item You should break your code up into functions that accomplish a single task
          \item You should streamline \cpp{if-else} statements and loops to avoid
            nesting if possible
          \item If you must nest, you should try to avoid repeated blocks (i.e. the same
            \cpp{for} loop in two different parts of an \cpp{if} statement)
        \end{itemize}
      \end{answer}
      % \ifprintanswers\vskip -35pt\null\fi
      
    \Q Individually, think of a time that it was difficult for you to write or debug a
      program.  How could the advice above have helped you?
      \begin{answer}[0.5in]
        Answers will vary
        \par\vskip 0.75in
      \end{answer}
      % \ifprintanswers\vskip -35pt\null\fi
      
    \Q Share your answers to the previous question with your group and summarize any
      insights below.
      \begin{answer}[0.5in]
        Answers will vary
        \par\vskip 0.75in
      \end{answer}
  
  \vfill
  
  \begin{center}
    \it 
    Debugging is twice as hard as writing the code in the first place. Therefore, if you\\
    write the code as cleverly as possible, you are, by definition, not smart 
    enough to debug it.\\
    \rm
    - Brian Kernighan, author of The C Programming Language  
  \end{center}
    
\end{document}
