\model{Function Calls} \\
  \begin{center}
    \renewcommand{\arraystretch}{1.3}
    \begin{tabular}{|p{3in}|p{3in}|}
      \hline
      \rowcolor{orange!20} 
        \begin{minipage}{2.8in}\centering {\bf Code Block $X$}\end{minipage} & 
        \begin{minipage}{2.8in}\centering {\bf Code Block $Y$}\end{minipage} \\
      \hline
      \begin{minipage}{3in}
        \scriptsize
        \begin{cprlst}[
          frame=lines,
          framesep=2mm,
          bgcolor=gray!15,
          baselinestretch=1.2,
        ]{cpp}
int i1 = f1("c"); int i2 = f2("m"); 
int c1 = f3("c"); int c2 = f4("m");
int s1 = g1(i1,c1)+g1(i2*c2); 
int t1 = s1 + h1(s1) + h2(i1,f5("c"));
        \end{cprlst}
        \vskip 30pt\null
      \end{minipage}
      &
      \begin{minipage}{3in}
        \scriptsize
        \begin{cprlst}[
          frame=lines,
          framesep=2mm,
          bgcolor=gray!15,
          baselinestretch=1.2,
        ]{cpp}
int numCD      = getNum ("CD");
int numMP3     = getNum ("MP3");
double costCD  = getCost("CD");
double costMP3 = getCost("MP3");
double subCost = getSubCost(numCD, costCD) +
                 getSubCost(numMP3, costMP3);
double subShip = getSubShip(numCD, getShip("CD"));
double total   = subCost + subShip + getTax(subCost);
        \end{cprlst} 
      \end{minipage}
      \\
      \hline
    \end{tabular} 
  \end{center}  
  \par\vskip 10pt
  
  {\it\large Refer to Model 2 above as your team develops consensus answers
    to the questions below.}
    \par\vskip -20pt\null

  \Q Determine which code block above ($X$ or $Y$) matches the
    description below.
    \par\vskip 15pt
    
    \begin{enumerate}[(a)]
      \itemsep 15pt
      \item Is shorter and would take less time to type: \hfill \ans[1in]{$X$} \hspace{1in}\null
      \item Declares more variables: \hfill \ans[1in]{$Y$} \hspace{1in}\null
      \item Would be easier to update or debug if a problem is found: \hfill \ans[1in]{$Y$} \hspace{1in}\null
    \end{enumerate}

\newpage

    \Q Each pair of code blocks below is syntactically and
      semantically equivalent.  Decide which one is better based only
      on the style.  Enter $X$, $Y$, or ? (if you think they are
      equally good).

      {\renewcommand{\arraystretch}{2.3}
      \begin{tabular}{|c|p{2.5in}|p{2.5in}|c|}
        \hline
        \rowcolor{orange!20} & Block $X$ & Block $Y$ & Choice \\
        \hline
        (a) & 
          \cpp{tree(pic,x,y);}
          & 
          \cpp{drawTree(pic,x,y);}
          & 
          \ans[0.5in]{$Y$} \\
        \hline
        (b) &
          \cpp{checkSize(a);}
          & 
          \cpp{hasValidSize(a);}
          & 
          \ans[0.5in]{$Y$} \\
        \hline
        (c) &
          \begin{minipage}{2in}
            \par\vskip 10pt
            \cpp{getCostToShip(item,num);}\\
            \cpp{getCostOfTax (item,num);}\\
          \end{minipage}
          & 
          \begin{minipage}{2in}
            \par\vskip 10pt
            \cpp{getShipCost (num,item);}\\
            \cpp{getCostOfTax(item,num);}\\
          \end{minipage}
          & 
          \ans[0.5in]{$X$} \\
        \hline
      \end{tabular}}
      
  \Q Determine which type of C++ programming concept, a {\bf variable} or a {\bf function},
    best matches each description below.
    \par\vskip 15pt
    
    \begin{enumerate}[(a)]
      \itemsep 15pt
      \item Is used to represent information \hfill \ans[2in][variable]
      \item Is used to represent an action or a set of steps \hfill \ans[2in]{function}
      \item Should be named with a verb or a verb phrase \hfill \ans[2in]{function}
      \item Should be named with a noun or a noun phrase \hfill \ans[2in]{variable}
    \end{enumerate}
    \vskip -30pt\null
    
  \Q Based on the above, what do you think is good style for naming
    and calling functions?\key\\[-2.5mm]
    \begin{answer}[1in]
      \par
      Functions should be named using verbs.  Their parameter lists should be consistent (i.e. two
      functions with the same parameters should use the same order).  Their names should be
      descriptive of what they actually do.
    \end{answer}