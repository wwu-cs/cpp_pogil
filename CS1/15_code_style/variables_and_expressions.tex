{\bf\large Model 1: Variables and Expressions} \\[-5pt]
  \begin{center}
    \renewcommand{\arraystretch}{1.3}
    \begin{tabular}{|p{3in}|p{3in}|}
      \hline
      \rowcolor{orange!20} 
        \begin{minipage}{2.8in}\centering {\bf Code Block $X$}\end{minipage} & 
        \begin{minipage}{2.8in}\centering {\bf Code Block $Y$}\end{minipage} \\
      \hline
      \begin{minipage}{3in}
        \small
        \begin{minted}[
          frame=lines,
          framesep=2mm,
          bgcolor=gray!15,
          baselinestretch=1.2,
        ]{cpp}
int i1 = 10, i2 = 5;
double c1 = 9.99, c2 = 4.99; 
double d1 = 0.06, d2 = 0.99;
double s1 = i1*c1+i2*c2;
double t1 = s1+s1*d1+i1*d2;
        \end{minted}
        \vskip 70pt\null
      \end{minipage}
      &
      \begin{minipage}{3in}
        \small
        \begin{minted}[
          frame=lines,
          framesep=2mm,
          bgcolor=gray!15,
          baselinestretch=1.2,
        ]{cpp}
int numCD      = 10;
int numMP3     = 5;
double costCD  = 9.99;
double costMP3 = 4.99;
double shipCD  = 0.99;
double rateTax = 0.06;
double subCost = (numCD * costCD) + 
                 (numMP3 * costMP3);
double subShip = numCD * shipCD;
double subTax  = subCost * rateTax;
double total   = subCost + subTax + subShip;
        \end{minted} 
      \end{minipage}
      \\
      \hline
    \end{tabular} 
  \end{center}  
  \par\vskip 10pt
  
  {\it\large Refer to Model 1 above as your team develops consensus answers
    to the questions below.}
    \par\vskip 10pt
    
  \begin{enumerate}
    \itemsep 20pt

    \item Determine which code block above ($X$ or $Y$) matches the
      description below.
      \par\vskip 15pt
      
      \begin{enumerate}[(a)]
        \itemsep 15pt
        \item Is shorter and would take less time to type: \hfill \fillin[$X$][1in] \hspace{1in}\null
        \item Uses more variables: \hfill \fillin[$Y$][1in] \hspace{1in}\null
        \item Would be easier to edit or debug: \hfill \fillin[$Y$][1in] \hspace{1in}\null
      \end{enumerate}
      \ifprintanswers\vskip -20pt\null\fi
      
    \item When we study foreign languages we learn about:
      \begin{itemize}
        \item {\it Syntax} -- The rules used to spell words and construct
          grammatically correct sentences
        \item {\it Semantics} -- The meanings of words and how to interpret sentences
        \item {\it Style} -- The distinctive voice of an author
          expressed within the rules of syntax and semantics          
      \end{itemize}
      All three of these are also present in a programming language.
      Are the differences between code block $X$ and $Y$ above
      differences in syntax, semantics, or style?  Explain.
      \begin{solution}[0.5in]
        \par
        They are differences in style since both are syntactically
        correct (both will compile) and semantically equivalent (the
        both compute the same value).
      \end{solution}
      
\newpage      
      
    \item Each pair of code blocks below is syntactically and
      semantically equivalent.  Decide which one is better based only
      on the style.  Enter $X$, $Y$, or ? (if you think they are
      equally good).

      {\renewcommand{\arraystretch}{2.3}
      \begin{tabular}{|c|p{2.55in}|p{2.55in}|c|}
        \hline
        \rowcolor{orange!20} & Block $X$ & Block $Y$ & Choice \\
        \hline
        (a) & 
          \mintinline{cpp}|int i1 = 3, i2 = 7;| 
          & 
          \mintinline{cpp}|int numClosed = 3, numOpen = 7;| 
          & 
          \fillin[$Y$][0.4in] \\
        \hline
        (b) &
          \mintinline{cpp}|int numCat = 2, numDog = 5;| 
          & 
          \mintinline{cpp}|int numCat = 2, dogNum = 5;| 
          & 
          \fillin[$X$][0.4in] \\
        \hline
        (c) &
          \mintinline{cpp}|int numson = 3, isdone = true;|
          & 
          \mintinline{cpp}|int numSon = 3, isDone = true;| 
          & 
          \fillin[$Y$][0.4in] \\
        \hline
        (d) &
          \begin{minipage}{2.5in}
            \par\vskip 10pt
            \mintinline{cpp}|int costCoffee = 2, costTea = 1;|\\
            \mintinline{cpp}|int soldCoffee = 4, soldTea = 6;|\\
          \end{minipage}
          & 
          \begin{minipage}{2.5in}
            \par\vskip 10pt
            \mintinline{cpp}|int costCoffee = 2, costTea = 1;|\\
            \mintinline{cpp}|int soldTea = 6, soldCoffee = 4;|\\
          \end{minipage}
          & 
          \fillin[$X$][0.4in] \\
        \hline
      \end{tabular}}
      
    \item Based on your answers above, describe what you think is a 
      good style for naming variables.
      \begin{solution}[1in]
        \par
        Variable names should be chosen to match the purpose of a
        variable.  If they are composed of multiple words, those words
        should be separated in some way (capitals, underscores, etc).
        Their names should be consistent (same order to compound
        words, etc).
      \end{solution}
      
    \item Each pair of code blocks below is syntactically and
      semantically equivalent.  Decide which one is better based only on
      the style.  Enter $X$, $Y$, or ? (if you think they are equally good).
      
      {\renewcommand{\arraystretch}{2.3}
      \begin{tabular}{|c|p{2.5in}|p{2.5in}|c|}
        \hline
        \rowcolor{orange!20} & Block $X$ & Block $Y$ & Choice \\
        \hline
        (a) & 
          \mintinline{cpp}|s1=i1*c1+i2*c2;|
          & 
          \mintinline{cpp}|s1 = i1*c1 + i2*c2;|
          & 
          \fillin[$Y$][0.5in] \\
        \hline
        (b) &
          \mintinline{cpp}|s1=(i1*c1)+(i2*c2);|
          & 
          \mintinline{cpp}|s1=i1*c1+i2*c2;|
          & 
          \fillin[$X$][0.5in] \\
        \hline
        (c) &
          \mintinline{cpp}|s1 = c1*i1 + i2*c2;|
          & 
          \mintinline{cpp}|s1 = i1*c1 + i2*c2;|
          & 
          \fillin[$Y$][0.5in] \\
        \hline
        (d) &
          \begin{minipage}{2in}
            \begin{minted}[
              frame=none,
              bgcolor=white,
            ]{cpp}
float total = nCD*sCD +
  (nCD*cCD + nMP3*cMP3) *
  (1+rateTax);
            \end{minted} 
          \end{minipage}
          & 
          \begin{minipage}{2in}
            \begin{minted}[
              frame=none,
              bgcolor=white,
            ]{cpp}
float cost  = nCD*cCD + nMP3*cMP3;
float ship  = nCD*sCD;
float tax   = cost * rateTax;
float total = cost + tax + ship;
            \end{minted} 
          \end{minipage}
          & 
          \fillin[$Y$][0.5in] \\
        \hline
      \end{tabular}}
      
    \item Based on your answers above, describe what you think is a
      good style for writing expressions.
      \begin{solution}[1in]
        \par
        Expressions should make use of white space to separate out
        different parts.  Using parentheses to group things, even if
        not required, is also a good idea.  It is better to split long
        expressions up into individual sub-computations that have
        meaning in the context of the problem being solved.
      \end{solution}

\newpage
      
    \item Code that can be read and understood without separate
      documentation is called {\it self-documenting code}.  What are
      some way to make expressions self-documenting?
      \begin{solution}[1in]
        \begin{itemize}
          \item Use variable names that have meaning in the context of
            the problem being solved.
          \item Split long computations up into meaningful steps.
          \item Group pieces of the computation together so that the
            order of precedence is clearly seen.
        \end{itemize}
      \end{solution}
      \par\ifprintanswers\vskip -40pt\else\vskip -20pt\fi\null

    \item Rewrite the code below to be self-documenting.  Use
      your style recommendations above.\key\\[-2.5mm]
      \begin{center}
        \begin{minipage}{5.25in}
          \begin{minted}[
            frame=lines,
            framesep=2mm,
            bgcolor=gray!15,
            baselinestretch=1.2
          ]{cpp}
// calculate trip cost from 2-way flights, nights in hotels, and meals
double fc=500, ch=150, mc=30, nn=5;         // costs and number nights
double c = 2*fc+ch*nn+3*mc*nn;                // compute total cost
          \end{minted}
        \end{minipage}
      \end{center}
      \begin{solution}[2in]
        \scriptsize\vskip -35pt\null
        \begin{center}
          \begin{minipage}{4.25in}
            \begin{minted}[
              frame=lines,
              framesep=2mm,
              bgcolor=gray!15,
              baselinestretch=1.2,
              linenos
            ]{cpp}
// calculate trip cost from 2-way flights, nights in hotels, and meals
double flightPrice = 500;
double hotelPrice  = 150;
double mealPrice   = 30;
int    numNights   = 5;
double travelCost  = 2 * flightPrice;
double lodgingCost = numNights * hotelPrice;
double mealCost    = 3 * numNights * mealPrice;
double totalCost   = travelCost + lodgingcost + mealCost;
            \end{minted}
          \end{minipage}
          \vskip -10pt\null
        \end{center}
      \end{solution}