\model{Temperature Conversion}
  \begin{center}
    \small
    \begin{minipage}{5.5in}
      \fs
      \begin{cpplst}
class TempConvert {
  public:
    double convertTemp(double temp, string inScale, string outScale) {
      inScale = normalizeScale(inScale);
      outScale = normalizeScale(outScale);
      if (inScale == outScale) { return temp;
      } else if (inScale == "C") { return cToF(temp);
      } else { return fToC(temp);
      }
    }                                              
  private:
    double cToF(double c) { return 32 + c * 9 / 5; }
    double fToC(double f) { return (f - 32) * 5 / 9; }
    string normalizeScale(string s) {
      for_each(s.begin(),s.end(), [](char & c) { c = ::tolower(c); });
      if (s == "c" || s.substr(0, 4) == "cels") { return "C"; }
      if (s == "f" || s.substr(0, 4) == "fahr") { return "F"; }
      if (s == "k" || s.substr(0, 4) == "kelv") { return "K"; }
      return "?";
    }
};    
      \end{cpplst}      
    \end{minipage}
  \end{center}
  
  {\it\large Refer to Model 1 above as your group develops consensus answers
    to the questions below.}

  \quest{20 min}
    
  \Q The temperature conversion class above has one public method
    and three private methods.  Determine the {\bf return value} for each 
    method call given below.
    \begin{enumerate}
      \itemsep 10pt
      \item \cpp{normalizeScale("f")}\hfill\ans[2.5in]{The return value is "F".}
      \item \cpp{normalizeScale("Cel")}\hfill\ans[2.5in]{The return value is "C".}
      \item \cpp{convertTemp(-273.15, "C", "Fahrenheit")}\hfill\ans[2.5in]{The return value is -459.67.}
      \item \cpp{convertTemp(100, "F", "CeLsIuS")}\hfill\ans[2.5in]{The return value is 37.7778.}
    \end{enumerate}

  \newpage

  \Q The file {\tt activity26a.cpp} contains this class along with a program that
    utilizes it.  This program could run into trouble if the user supplies
    invalid input.  For example, a user could enter an invalid or
    unknown scale, or provide invalid temperatures (e.g. below
    the absolute zero of -273.15 Celsius).  Describe the problem
    that would be encountered with each set of user input below.
    Note that the semicolons are used to separate lines of input 
    and are not part of the input themselves.
    \begin{enumerate}
      \item {\tt 12} ; {\tt inches} ; {\tt fahr}
        \begin{answer}[0.5in]
          The problem is that "inches" is not a valid temperature scale,
          so the program will not be able to normalize it and will likely return "?" or cause an error.
        \end{answer}

      \item {\tt 16000000} ; {\tt C} ; {\tt K}
        \begin{answer}[0.5in]
          The problem is that 16000000 Celsius is an extremely high temperature,
          which may not be handled properly by the conversion methods, potentially leading to overflow or inaccuracies.
        \end{answer}

      \item {\tt -300} ; {\tt cels} ; {\tt fahr}
        \begin{answer}[0.5in]
          The problem is that -300 Celsius is below absolute zero,
          which is physically impossible, and the conversion method may not handle this invalid temperature correctly.
        \end{answer}
    \end{enumerate}

  \vskip -20pt
  
  \Q A program could respond to errors such as the ones
    above in several ways.  One way commonly used in simple programs
    is to {\it print an error message}.
    \begin{enumerate}
      \item In what method(s) should error messages about
        invalid or unknown temperature scales be printed?
        \begin{answer}[0.75in]
          In the normalizeScale method, since this is where the scale is being validated.
        \end{answer}

      \item In what method(s) should error messages
        about invalid temperatures be printed?
        \begin{answer}[0.75in]
          In the convertTemp method, since this is where the temperature
          value is being processed and converted.
        \end{answer}

      \item In what method(s) should error messages
        about unimplemented conversions (e.g. kelvin to
        fahrenheit) be printed?
        \begin{answer}[0.75in]
          In the convertTemp method, since this is where the conversion logic is handled
          and unimplemented conversions can be detected.
        \end{answer}

      \item Modify two of the methods identified
        above in the file {\tt activity26a.cpp} to print appropriate error
        messages for the user input from problem 2.
        \begin{answer}[1.5in]
          Answers will vary.
        \end{answer}
    \end{enumerate}
  
  \vskip -30pt
    
  \Q Many programs can not handle errors by simply printing out
    error messages.\key\\[-2.5mm] For example:
    \begin{itemize}
      \item Embedded systems, such as home appliances or automobiles, which do not
        have a traditional display and need to run unattended for days or even years.
      \item Computationally intense software for business or scientific applications
        that may run for hours or days on a cloud server without human monitoring.
    \end{itemize}
    Explain why printing messages may not be a good idea in software such as those
    mentioned above.
    \begin{answer}[0.75in]
      Answers will vary.
    \end{answer}