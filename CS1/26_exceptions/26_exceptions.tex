\documentclass{exam}
%\documentclass[answers]{exam}
\hbadness=99999
\usepackage[total={6.5in,9in}]{geometry}

\usepackage{enumerate}
\usepackage{amsmath}
\usepackage[table]{xcolor}
\usepackage{graphicx}
\usepackage{tikz}
%\usepackage{pgfplots}
\usepackage{multicol}

% for syntax highlighting
\usepackage{minted}
\usemintedstyle[cpp]{xcode}

% for overlay of output
\usepackage[overlay,showboxes]{textpos}

\pagestyle{plain}

\setlength\columnsep{50pt}
\newcommand{\key}{\hfill
      \raisebox{-.3\height}{\includegraphics[width=0.6in]{figures/key.png}}}

\begin{document}
  \thispagestyle{empty}
  \setlength{\parindent}{0pt}

  \begin{center}
    \Large Activity \#7: Exception Handling \\[5pt]
    \large Recorder's Report\\[20pt]
    \normalsize
    \begin{tabular}{lrp{0.1in}lr}
      Manager:  & \fillin[][2.0in] & & Presenter: & \fillin[][2.0in]\\[15pt]
      Recorder: & \fillin[][2.0in] & & Driver:    & \fillin[][2.0in]\\[15pt]
      Date:     & \fillin[][2.0in] & & Score:     & Satisfactory \hspace{10pt} /
      \hspace{10pt} Not Satisfactory
    \end{tabular}
  \end{center}
  \par\vskip 15pt
  
  Record your group's answers to the key questions (marked with
  \raisebox{-.3\height}{\includegraphics[width=0.5in]{figures/key.png}})
  below.
  \begin{enumerate}[(a)]
    \itemsep 1.75in
    \item Model 1, Question \#4
    \item Model 2, Question \#8
    \item Model 3, Question \#10
  \end{enumerate}

  \clearpage\pagenumbering{arabic} 
  
  \begin{center}
    \Large Activity \#7: Exception Handling \\[5pt]
    \large Activity Guide\\[20pt]
  \end{center}

  \begin{center}
    \fbox{
      \begin{minipage}{5.5in}
        {\bf Learning Objectives:} Students will be able to:
        \begin{itemize}
          \item Content:\\[-20pt]
            \begin{itemize}
              \itemsep 0pt
              \item Explain various methods for detecting and
                handling exceptions
              \item Explain the advantages of exception handling over
                printing error messages
            \end{itemize}
          \item Process \\[-20pt]
            \begin{itemize}
              \itemsep 0pt
              \item Write C++ code to throw and catch exceptions \\[-5pt]
            \end{itemize}
        \end{itemize}
      \end{minipage}
      }
  \end{center}
  \par\vskip 10pt
  
  
  {\bf\large Model 1: Temperature Conversion}\\[-10pt]
  \begin{center}
    \small
    \begin{minipage}{5.5in}
      \begin{minted}[
        frame=lines,
        framesep=2mm,
        bgcolor=gray!15,
        baselinestretch=1.2,
        linenos,
        firstnumber=6
      ]{cpp}
class TempConvert {
  public:
    double convertTemp(double temp, string inScale, string outScale) {
      inScale = normalizeScale(inScale);
      outScale = normalizeScale(outScale);
      if (inScale == outScale) { return temp;
      } else if (inScale == "C") { return cToF(temp);
      } else { return fToC(temp);
      }
    }                                              
  private:
    double cToF(double c) { return 32 + c * 9 / 5; }
    double fToC(double f) { return (f - 32) * 5 / 9; }
    string normalizeScale(string s) {
      for_each(s.begin(),s.end(), [](char & c) { c = ::tolower(c); });
      if (s == "c" || s.substr(0, 4) == "cels") { return "C"; }
      if (s == "f" || s.substr(0, 4) == "fahr") { return "F"; }
      if (s == "k" || s.substr(0, 4) == "kelv") { return "K"; }
      return "?";
    }
};    
      \end{minted}      
    \end{minipage}
  \end{center}
  \par\vskip 5pt
  
  {\it\large Refer to Model 1 above as your group develops consensus answers
    to the questions below.}
    \par\vskip 10pt
    
  \begin{enumerate}
    \itemsep 20pt
    
    \item The temperature conversion class above has one public method
      and three private methods.  Determine the {\bf return value} for each 
      method call given below.
      \par\vskip 15pt
      \begin{enumerate}
        \itemsep 15pt
        \item \mintinline{cpp}|normalizeScale("f")|\hfill\fillin[][2.5in]
        \item \mintinline{cpp}|normalizeScale("Cel")|\hfill\fillin[][2.5in]
        \item \mintinline{cpp}|convertTemp(-273.15,"C","Fahrenheit")|\hfill\fillin[][2.5in]
        \item \mintinline{cpp}|convertTemp(100,"F","CeLsIuS")|\hfill\fillin[][2.5in]
      \end{enumerate}

    \item The file {\tt activity07a.cpp} contains this class along with a program that
      utilizes it.  This program could run into trouble if the user supplies
      invalid input.  For example, a user could enter an invalid or
      unknown scale, or provide invalid temperatures (e.g. below
      the absolute zero of -273.15 Celsius).  Describe the problem
      that would be encountered with each set of user input below.
      Note that the semicolons are used to separate lines of input 
      and are not part of the input themselves.
      \par\vskip 10pt
      
      \begin{enumerate}
        \item {\tt 12} ; {\tt inches} ; {\tt fahr}
          \begin{solution}[0.5in]
          \end{solution}
        \item {\tt 16000000} ; {\tt C} ; {\tt K}
          \begin{solution}[0.5in]
          \end{solution}
        \item {\tt -300} ; {\tt cels} ; {\tt fahr}
          \begin{solution}[0.5in]
          \end{solution}
      \end{enumerate}
    
    \item A program could respond to errors such as the ones
      above in several ways.  One way commonly used in simple programs
      is to {\it print an error message}.
      \par\vskip 10pt
      \begin{enumerate}
        \item In what method(s) should error messages about
          invalid or unknown temperature scales be printed?
          \begin{solution}[0.75in]
          \end{solution}
        \item In what method(s) should error messages
          about invalid temperatures be printed?
          \begin{solution}[0.75in]
          \end{solution}
        \item In what method(s) should error messages
          about unimplemented conversions (e.g. kelvin to
          fahrenheit) be printed?
          \begin{solution}[0.75in]
          \end{solution}
        \item Modify two of the methods identified
          above in the file {\tt activity07a.cpp} to print appropriate error
          messages for the user input from problem 2.
          \begin{solution}[1.5in]
          \end{solution}
      \end{enumerate}
      
    \item Many programs can not handle errors by simply printing out
      error messages.  For example:\key\\[-5.5mm]
      \begin{itemize}
        \item Embedded systems, such as home appliances or automobiles, which do not
          have a traditional display and need to run unattended for days or even years.
        \item Computationally intense software for business or scientific applications
          that may run for hours or days on a cloud server without human monitoring.
      \end{itemize}
      Explain why printing messages may not be a good idea in software such as those
      mentioned above.
      \begin{solution}[0.75in]
      \end{solution}


  {\bf\large Model 2: Identifying Errors in Temperature Conversion}\\[-10pt]
  \begin{center}
    \small
    \begin{minipage}{5.5in}
      \begin{minted}[
        frame=lines,
        framesep=2mm,
        bgcolor=gray!15,
        baselinestretch=1.2,
        linenos,
        firstnumber=6
      ]{cpp}
class TempConvert {
public:
  double convertTemp(double temp, string inScale, string outScale) {
    inScale = normalizeScale(inScale);
    outScale = normalizeScale(outScale);
    if (inScale == outScale) { return temp;
    } else if (inScale == "C" && outScale == "F") { return cToF(temp);
    } else if (inScale == "F" && outScale == "C") { return fToC(temp);
    } else if (inScale == "K" && outScale == "C") { return kToC(temp);
    } else if (inScale == "K" && outScale == "F") { return cToF(kToC(temp));
    } else throw invalid_argument("Conversion Not Implemented");
  }
private:
  double cToF(double c) {
     if(c < 273.15) throw domain_error("Invalid Temp");
     return 32 + c * 9 / 5; 
  }
  double fToC(double f) {
     if(f < -459.67) throw domain_error("Invalid Temp");
     return (f - 32) * 5 / 9; 
  }   
  double kToC(double k) {
     if(k < 0) throw domain_error("Invalid Temp");
     return k + 273.15;
  }                             
  string normalizeScale(string s) {
    for_each(s.begin(),s.end(), [](char & c) { c = ::tolower(c); });
    if (s == "c" || s.substr(0, 4) == "cels") { return "C"; }
    if (s == "f" || s.substr(0, 4) == "fahr") { return "F"; }
    if (s == "k" || s.substr(0, 4) == "kelv") { return "K"; }          
    throw domain_error("Invalid Scale");
  }
};
      
      \end{minted}
    \end{minipage}
  \end{center}
  
  {\it\large Refer to Model 2 above as your group develops consensus answers
    to the questions below.}
    \par\vskip 10pt
    
    \item In this model, our class has been modified to
      support additional temperature conversions.  Describe what changes have been
      made to support conversions between Kelvin and the other
      temperature scales.
      \begin{solution}[1.25in]
      \end{solution}
      
    \item Robust error handling (also called {\it exception handling}) typically 
      separates the task of {\it identifying} exceptions from the task of 
      {\it handling} exceptions.  In the model above, we have added
      code to identify exceptions.
      \par\vskip 10pt
      
      \begin{enumerate}
        \item Which of the methods in this model include exception
          identification?
          \begin{solution}[0.75in]
          \end{solution}
        \item How many different types of exceptions are identified in
          the entire class?
          \begin{solution}[0.75in]
          \end{solution}
        \item In C++, what command is used to identify an exception?
          \begin{solution}[0.75in]
          \end{solution}
      \end{enumerate}
      
        
      \item While we have identified exceptions in this model, we have not 
        explicitly handled them, so C++ uses its default exception
        handling technique.  Run the code found in {\tt activity07b.cpp} and
        determine how the exceptions generated by each set of user
        input is handled.
      \par\vskip 10pt
      
      \begin{enumerate}
        \item {\tt 12} ; {\tt inches} ; {\tt fahr}
          \begin{solution}[0.5in]
          \end{solution}
        \item {\tt 16000000} ; {\tt C} ; {\tt K}
          \begin{solution}[0.5in]
          \end{solution}
        \item {\tt -300} ; {\tt cels} ; {\tt fahr}
          \begin{solution}[0.5in]
          \end{solution}
      \end{enumerate}
        
      \item Describe how you might wish to alter the default exception
        handling technique in the\key\\[-2.5mm] following contexts.  If you think
        the default is fine as it is, state that.
        \par\vskip 10pt        
        \begin{enumerate}
          \item If the class is being used to convert temperatures interactively, 
            as in this example.
            \begin{solution}[0.75in]
            \end{solution}
          \item If the class is being used to convert a large array of
            temperatures without human interaction.
            \begin{solution}[0.75in]
            \end{solution}
        \end{enumerate}

  {\bf\large Model 3: Handling Errors in Temperature Conversion}\\[-30pt]
  \begin{center}
    \small
    \begin{tabular}{p{2.95in}p{0.1in}p{2.95in}}
      \begin{minipage}{2.95in}
        \begin{minted}[
          frame=lines,
          framesep=2mm,
          bgcolor=gray!15,
          baselinestretch=1.2,
          linenos,
          firstnumber=44
        ]{cpp}
TempConvert tc;

double t[6] = {20,-30,170,25,-12,14 };
string u[6] = {"C","C","K","F","K","C"};
double newT[6];

for(int i=0; i<6; i++) {
  try {           
    newT[i] = tc.convertTemp(t[i],u[i],"F");
  } catch (logic_error &except) {
    newT[i] = NAN;
  }
}
        \end{minted}
        \centering Use Case \#1
      \end{minipage}
      & &
      \begin{minipage}{2.85in}
        \begin{minted}[
          frame=lines,
          framesep=2mm,
          bgcolor=gray!15,
          baselinestretch=1.2,
          linenos,
          firstnumber=22
        ]{cpp}
TempConvert tc;
double t,newT;
string u;
while(true) {
  try {
    cout << "Temp to Convert: ";
    cin >> t;
    cout << "Units: ";
    cin >> u;
    newT = tc.convertTemp(t,u,"F");
  }
  catch (logic_error &except) {
    cout << except.what() << endl;
    continue;
  }
  break;
}
        \end{minted}
      \end{minipage}
      \centering Use Case \#2
    \end{tabular}
  \end{center}
  \par\vskip -5pt
  
  {\it\large Refer to Model 3 above as your group develops consensus answers
    to the questions below.}
    \par\vskip -20pt\null
    
    \item The model above shows two different use cases for our
      temperature conversion class.  Answer the following general
      questions about these use cases.
      \par\vskip 10pt      
      \begin{enumerate}        
        \item What new keywords are used to handle the exceptions that
          were {\it thrown} in the class?
          \begin{solution}[0.5in]
          \end{solution}
        \item When is the code in in the \mintinline{cpp}|try| block run?
          \begin{solution}[0.75in]
          \end{solution}
        \item When is the code in the \mintinline{cpp}|catch| block run?
          \begin{solution}[0.75in]
          \end{solution}
      \end{enumerate}
        
    \item The code for use case \#1 can be found in the file {\tt
      activity07c.cpp}. Run it and answer\key\\[-2.5mm] the following questions.
      \par\vskip 10pt
      \begin{enumerate}
        \item What exception is thrown by the temperature conversion
          class and why is it thrown?
          \begin{solution}[0.75in]
          \end{solution}
        \item How is this exception handled by the program?
          \begin{solution}[0.75in]
          \end{solution}
        \item Why might this be a better way to handle the exception
          than the C++ default technique?
          \begin{solution}[0.75in]
          \end{solution}
      \end{enumerate}
        
    \item Modify the file {\tt activity7b.cpp} so that the main
      program looks like the one in use case \#2 of our model.  Then
      run it and answer the following questions.
      \par\vskip 10pt
      \begin{enumerate}
        \item Give appropriate user input to generate two different
          types of exceptions.
          \begin{solution}[0.5in]
          \end{solution}
        \item How are these exceptions handled in this use case?
          \begin{solution}[0.75in]
          \end{solution}
        \item Why might this be a better way to handle the exceptions
          than the C++ default technique?
          \begin{solution}[0.75in]
          \end{solution}
      \end{enumerate}        

  \end{enumerate}
       
\end{document}
