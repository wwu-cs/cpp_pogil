% Source: Unknown
% File: ".pdf"
% Access: Unknown

% comment out for student version
% \ifdefined\Student\relax\else\def\Teacher{}\fi

\documentclass[12pt]{article}

\title{Activity \#26: Exception Handling}
\author{Unknown}
\newcommand{\activityeditor}{Preston Carman}
\newcommand{\activitysource}{\urls{}}
\date{Spring 2022}

\input{../../cspogil.sty}

\begin{document}

  \begin{center}
    \maketitle
    \rolenames
  \end{center}
  
  \keyquestions{
    \item Model 1, Question \#4
    \item Model 2, Question \#8
    \item Model 3, Question \#10
  }

  \newpage
  \maketitle

  In this activity, you will work in teams of 3--4 students to learn new concepts.
  This activity will introduce you to exception handling in C++.

  \guide{
    \item Explain various methods for detecting and
      handling exceptions
    \item Explain the advantages of exception handling over
      printing error messages
  }{
    \item Write C++ code to throw and catch exceptions
  }{
    No additional notes.
  }

  \model{Temperature Conversion}
  \begin{center}
    \small
    \begin{minipage}{5.5in}
      \fs
      \begin{cpplst}
class TempConvert {
  public:
    double convertTemp(double temp, string inScale, string outScale) {
      inScale = normalizeScale(inScale);
      outScale = normalizeScale(outScale);
      if (inScale == outScale) { return temp;
      } else if (inScale == "C") { return cToF(temp);
      } else { return fToC(temp);
      }
    }                                              
  private:
    double cToF(double c) { return 32 + c * 9 / 5; }
    double fToC(double f) { return (f - 32) * 5 / 9; }
    string normalizeScale(string s) {
      for_each(s.begin(),s.end(), [](char & c) { c = ::tolower(c); });
      if (s == "c" || s.substr(0, 4) == "cels") { return "C"; }
      if (s == "f" || s.substr(0, 4) == "fahr") { return "F"; }
      if (s == "k" || s.substr(0, 4) == "kelv") { return "K"; }
      return "?";
    }
};    
      \end{cpplst}      
    \end{minipage}
  \end{center}
  
  {\it\large Refer to Model 1 above as your group develops consensus answers
    to the questions below.}

  \quest{20 min}
    
  \Q The temperature conversion class above has one public method
    and three private methods.  Determine the {\bf return value} for each 
    method call given below.
    \begin{enumerate}
      \itemsep 10pt
      \item \cpp{normalizeScale("f")}\hfill\ans[2.5in]{The return value is "F".}
      \item \cpp{normalizeScale("Cel")}\hfill\ans[2.5in]{The return value is "C".}
      \item \cpp{convertTemp(-273.15, "C", "Fahrenheit")}\hfill\ans[2.5in]{The return value is -459.67.}
      \item \cpp{convertTemp(100, "F", "CeLsIuS")}\hfill\ans[2.5in]{The return value is 37.7778.}
    \end{enumerate}

  \newpage

  \Q The file {\tt activity26a.cpp} contains this class along with a program that
    utilizes it.  This program could run into trouble if the user supplies
    invalid input.  For example, a user could enter an invalid or
    unknown scale, or provide invalid temperatures (e.g. below
    the absolute zero of -273.15 Celsius).  Describe the problem
    that would be encountered with each set of user input below.
    Note that the semicolons are used to separate lines of input 
    and are not part of the input themselves.
    \begin{enumerate}
      \item {\tt 12} ; {\tt inches} ; {\tt fahr}
        \begin{answer}[0.5in]
          The problem is that "inches" is not a valid temperature scale,
          so the program will not be able to normalize it and will likely return "?" or cause an error.
        \end{answer}

      \item {\tt 16000000} ; {\tt C} ; {\tt K}
        \begin{answer}[0.5in]
          The problem is that 16000000 Celsius is an extremely high temperature,
          which may not be handled properly by the conversion methods, potentially leading to overflow or inaccuracies.
        \end{answer}

      \item {\tt -300} ; {\tt cels} ; {\tt fahr}
        \begin{answer}[0.5in]
          The problem is that -300 Celsius is below absolute zero,
          which is physically impossible, and the conversion method may not handle this invalid temperature correctly.
        \end{answer}
    \end{enumerate}

  \vskip -20pt
  
  \Q A program could respond to errors such as the ones
    above in several ways.  One way commonly used in simple programs
    is to {\it print an error message}.
    \begin{enumerate}
      \item In what method(s) should error messages about
        invalid or unknown temperature scales be printed?
        \begin{answer}[0.75in]
          In the normalizeScale method, since this is where the scale is being validated.
        \end{answer}

      \item In what method(s) should error messages
        about invalid temperatures be printed?
        \begin{answer}[0.75in]
          In the convertTemp method, since this is where the temperature
          value is being processed and converted.
        \end{answer}

      \item In what method(s) should error messages
        about unimplemented conversions (e.g. kelvin to
        fahrenheit) be printed?
        \begin{answer}[0.75in]
          In the convertTemp method, since this is where the conversion logic is handled
          and unimplemented conversions can be detected.
        \end{answer}

      \item Modify two of the methods identified
        above in the file {\tt activity26a.cpp} to print appropriate error
        messages for the user input from problem 2.
        \begin{answer}[1.5in]
          Answers will vary.
        \end{answer}
    \end{enumerate}
  
  \vskip -30pt
    
  \Q Many programs can not handle errors by simply printing out
    error messages.\key\\[-2.5mm] For example:
    \begin{itemize}
      \item Embedded systems, such as home appliances or automobiles, which do not
        have a traditional display and need to run unattended for days or even years.
      \item Computationally intense software for business or scientific applications
        that may run for hours or days on a cloud server without human monitoring.
    \end{itemize}
    Explain why printing messages may not be a good idea in software such as those
    mentioned above.
    \begin{answer}[0.75in]
      Answers will vary.
    \end{answer}
  \newpage
  \model{Identifying Errors in Temperature Conversion}
  \begin{center}
    \small
    \begin{minipage}{5.5in}
      \fs
      \begin{cpplst}
class TempConvert {
public:
  double convertTemp(double temp, string inScale, string outScale) {
    inScale = normalizeScale(inScale);
    outScale = normalizeScale(outScale);
    if (inScale == outScale) { return temp;
    } else if (inScale == "C" && outScale == "F") { return cToF(temp);
    } else if (inScale == "F" && outScale == "C") { return fToC(temp);
    } else if (inScale == "K" && outScale == "C") { return kToC(temp);
    } else if (inScale == "K" && outScale == "F") { return cToF(kToC(temp));
    } else throw invalid_argument("Conversion Not Implemented");
  }
private:
  double cToF(double c) {
     if(c < 273.15) throw domain_error("Invalid Temp");
     return 32 + c * 9 / 5; 
  }
  double fToC(double f) {
     if(f < -459.67) throw domain_error("Invalid Temp");
     return (f - 32) * 5 / 9; 
  }   
  double kToC(double k) {
     if(k < 0) throw domain_error("Invalid Temp");
     return k + 273.15;
  }                             
  string normalizeScale(string s) {
    for_each(s.begin(),s.end(), [](char & c) { c = ::tolower(c); });
    if (s == "c" || s.substr(0, 4) == "cels") { return "C"; }
    if (s == "f" || s.substr(0, 4) == "fahr") { return "F"; }
    if (s == "k" || s.substr(0, 4) == "kelv") { return "K"; }          
    throw domain_error("Invalid Scale");
  }
};
      \end{cpplst}
    \end{minipage}
  \end{center}
  
  {\it\large Refer to Model 2 above as your group develops consensus answers
    to the questions below.}

  \quest{15 min}
    
  \Q In this model, our class has been modified to
    support additional temperature conversions.  Describe what changes have been
    made to support conversions between Kelvin and the other
    temperature scales.
    \begin{answer}[0.3in]
      \fs
      Two new private methods have been added: kToC to convert Kelvin to Celsius,
      and the convertTemp method has been modified to include cases for
      converting from Kelvin to Celsius and from Kelvin to Fahrenheit.
    \end{answer}

  \newpage
    
  \Q Robust error handling (also called {\it exception handling}) typically 
    separates the task of {\it identifying} exceptions from the task of 
    {\it handling} exceptions.  In the model above, we have added
    code to identify exceptions.
    \begin{enumerate}
      \item Which of the methods in this model include exception
        identification?
        \begin{answer}[0.75in]
          The methods convertTemp, cToF, fToC, kToC, and normalizeScale
          all include exception identification.
        \end{answer}

      \item How many different types of exceptions are identified in
        the entire class?
        \begin{answer}[0.75in]
          Two types of exceptions are identified: invalid temperature scales
          and invalid temperature values.
        \end{answer}

      \item In C++, what command is used to identify an exception?
        \begin{answer}[0.75in]
          The command used to identify an exception is {\tt throw}.
        \end{answer}
    \end{enumerate}

  \vskip -30pt
    
  \Q While we have identified exceptions in this model, we have not 
    explicitly handled them, so C++ uses its default exception
    handling technique.  Run the code found in {\tt activity26b.cpp} and
    determine how the exceptions generated by each set of user
    input is handled.
  \begin{enumerate}
    \item {\tt 12} ; {\tt inches} ; {\tt fahr}
      \begin{answer}[0.5in]
        The program terminates and prints an error message indicating
        that a domain error occurred due to an invalid scale.
      \end{answer}

    \item {\tt 16000000} ; {\tt C} ; {\tt K}
      \begin{answer}[0.5in]
        The program terminates and prints an error message indicating
        that an invalid argument error occurred because the conversion
        from Celsius to Kelvin is not implemented.
      \end{answer}

    \item {\tt -300} ; {\tt cels} ; {\tt fahr}
      \begin{answer}[0.5in]
        The program terminates and prints an error message indicating
        that a domain error occurred due to an invalid temperature
        (below absolute zero).
      \end{answer}
  \end{enumerate}

  \newpage
    
  \Q Describe how you might wish to alter the default exception
    handling technique\key\\[-2.5mm] in the following contexts.  If you think
    the default is fine as it is, state that.      
    \begin{enumerate}
      \item If the class is being used to convert temperatures interactively, 
        as in this example.
        \begin{answer}[0.75in]
          In this context, it might be better to catch the exceptions
          and provide user-friendly error messages, allowing the user
          to correct their input without terminating the program.
        \end{answer}

      \item If the class is being used to convert a large array of
        temperatures without human interaction.
        \begin{answer}[0.75in]
          In this context, it would be beneficial to log the errors
          for each invalid input and continue processing the rest of
          the array, rather than terminating the entire operation.
        \end{answer}
    \end{enumerate}
  \newpage
  \model{Handling Errors in Temperature Conversion}
  \begin{center}
    \scriptsize
    \begin{tabular}{p{2.95in}p{0.1in}p{2.95in}}
      \begin{minipage}{2.95in}
        \begin{cpplst}
TempConvert tc;

double t[6] = {20,-30,170,25,-12,14 };
string u[6] = {"C","C","K","F","K","C"};
double newT[6];

for(int i=0; i<6; i++) {
  try {           
    newT[i] = tc.convertTemp(t[i],u[i],"F");
  } catch (logic_error &except) {
    newT[i] = NAN;
  }
}
        \end{cpplst}
        \centering Use Case \#1
      \end{minipage}
      & &
      \begin{minipage}{2.85in}
        \begin{cpplst}
TempConvert tc;
double t, newT;
string u;
while(true) {
  try {
    cout << "Temp to Convert: ";
    cin >> t;
    cout << "Units: ";
    cin >> u;
    newT = tc.convertTemp(t,u,"F");
  }
  catch (logic_error &except) {
    cout << except.what() << endl;
    continue;
  }
  break;
}
        \end{cpplst}
      \end{minipage}
      \centering Use Case \#2
    \end{tabular}
  \end{center}
  
  {\it\large Refer to Model 3 above as your group develops consensus answers
    to the questions below.}

  \quest{15 min}
    
  \Q The model above shows two different use cases for our
    temperature conversion class.  Answer the following general
    questions about these use cases.
    \begin{enumerate}        
      \item What new keywords are used to handle the exceptions that
        were {\it thrown} in the class?
        \begin{answer}[0.5in]
          The new keywords used are {\tt try} and {\tt catch}.
        \end{answer}

      \item When is the code in the \cpp{try} block run?
        \begin{answer}[0.75in]
          When the program execution enters the {\tt try} block, the code within
          the block is executed normally until either it completes or an
          exception is thrown.
        \end{answer}

      \item When is the code in the \cpp{catch} block run?
        \begin{answer}[0.75in]
          The code in the {\tt catch} block is executed only if an exception
          is thrown within the corresponding {\tt try} block.
        \end{answer}
    \end{enumerate}

  \newpage
      
  \Q The code for use case \#1 can be found in the file {\tt
    activity26c.cpp}. Run it and\key\\[-2.5mm] answer the following questions.
    \begin{enumerate}
      \item What exception is thrown by the temperature conversion
        class and why is it thrown?
        \begin{answer}[0.75in]
          The exception thrown is a {\tt logic\_error} when an invalid temperature
          is provided for conversion, such as a negative value for Kelvin.
        \end{answer}

      \item How is this exception handled by the program?
        \begin{answer}[0.75in]
          The exception is caught in the {\tt catch} block, and the corresponding
          converted temperature is set to {\tt NAN} (not a number) to indicate
          that the conversion could not be performed.
        \end{answer}

      \item Why might this be a better way to handle the exception
        than the C++ default technique?
        \begin{answer}[0.75in]
          This method allows the program to continue running and process
          other temperature conversions even if one conversion fails,
          rather than terminating the program abruptly.
        \end{answer}
    \end{enumerate}

  \vskip -20pt
      
  \Q Modify the file {\tt activity26b.cpp} so that the main
    program looks like the one in use case \#2 of our model.  Then
    run it and answer the following questions.
    \begin{enumerate}
      \item Give appropriate user input to generate two different
        types of exceptions.
        \begin{answer}[0.5in]
          Answers will vary
        \end{answer}

      \item How are these exceptions handled in this use case?
        \begin{answer}[0.75in]
          Answers will vary
        \end{answer}

      \item Why might this be a better way to handle the exceptions
        than the C++ default technique?
        \begin{answer}[0.75in]
          Answers will vary
        \end{answer}
    \end{enumerate}
  
\end{document}
