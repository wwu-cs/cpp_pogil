\model{Identifying Errors in Temperature Conversion}
  \begin{center}
    \small
    \begin{minipage}{5.5in}
      \fs
      \begin{cpplst}
class TempConvert {
public:
  double convertTemp(double temp, string inScale, string outScale) {
    inScale = normalizeScale(inScale);
    outScale = normalizeScale(outScale);
    if (inScale == outScale) { return temp;
    } else if (inScale == "C" && outScale == "F") { return cToF(temp);
    } else if (inScale == "F" && outScale == "C") { return fToC(temp);
    } else if (inScale == "K" && outScale == "C") { return kToC(temp);
    } else if (inScale == "K" && outScale == "F") { return cToF(kToC(temp));
    } else throw invalid_argument("Conversion Not Implemented");
  }
private:
  double cToF(double c) {
     if(c < 273.15) throw domain_error("Invalid Temp");
     return 32 + c * 9 / 5; 
  }
  double fToC(double f) {
     if(f < -459.67) throw domain_error("Invalid Temp");
     return (f - 32) * 5 / 9; 
  }   
  double kToC(double k) {
     if(k < 0) throw domain_error("Invalid Temp");
     return k + 273.15;
  }                             
  string normalizeScale(string s) {
    for_each(s.begin(),s.end(), [](char & c) { c = ::tolower(c); });
    if (s == "c" || s.substr(0, 4) == "cels") { return "C"; }
    if (s == "f" || s.substr(0, 4) == "fahr") { return "F"; }
    if (s == "k" || s.substr(0, 4) == "kelv") { return "K"; }          
    throw domain_error("Invalid Scale");
  }
};
      \end{cpplst}
    \end{minipage}
  \end{center}
  
  {\it\large Refer to Model 2 above as your group develops consensus answers
    to the questions below.}

  \quest{15 min}
    
  \Q In this model, our class has been modified to
    support additional temperature conversions.  Describe what changes have been
    made to support conversions between Kelvin and the other
    temperature scales.
    \begin{answer}[0.3in]
      \fs
      Two new private methods have been added: kToC to convert Kelvin to Celsius,
      and the convertTemp method has been modified to include cases for
      converting from Kelvin to Celsius and from Kelvin to Fahrenheit.
    \end{answer}

  \newpage
    
  \Q Robust error handling (also called {\it exception handling}) typically 
    separates the task of {\it identifying} exceptions from the task of 
    {\it handling} exceptions.  In the model above, we have added
    code to identify exceptions.
    \begin{enumerate}
      \item Which of the methods in this model include exception
        identification?
        \begin{answer}[0.75in]
          The methods convertTemp, cToF, fToC, kToC, and normalizeScale
          all include exception identification.
        \end{answer}

      \item How many different types of exceptions are identified in
        the entire class?
        \begin{answer}[0.75in]
          Two types of exceptions are identified: invalid temperature scales
          and invalid temperature values.
        \end{answer}

      \item In C++, what command is used to identify an exception?
        \begin{answer}[0.75in]
          The command used to identify an exception is {\tt throw}.
        \end{answer}
    \end{enumerate}

  \vskip -30pt
    
  \Q While we have identified exceptions in this model, we have not 
    explicitly handled them, so C++ uses its default exception
    handling technique.  Run the code found in {\tt activity26b.cpp} and
    determine how the exceptions generated by each set of user
    input is handled.
  \begin{enumerate}
    \item {\tt 12} ; {\tt inches} ; {\tt fahr}
      \begin{answer}[0.5in]
        The program terminates and prints an error message indicating
        that a domain error occurred due to an invalid scale.
      \end{answer}

    \item {\tt 16000000} ; {\tt C} ; {\tt K}
      \begin{answer}[0.5in]
        The program terminates and prints an error message indicating
        that an invalid argument error occurred because the conversion
        from Celsius to Kelvin is not implemented.
      \end{answer}

    \item {\tt -300} ; {\tt cels} ; {\tt fahr}
      \begin{answer}[0.5in]
        The program terminates and prints an error message indicating
        that a domain error occurred due to an invalid temperature
        (below absolute zero).
      \end{answer}
  \end{enumerate}

  \newpage
    
  \Q Describe how you might wish to alter the default exception
    handling technique\key\\[-2.5mm] in the following contexts.  If you think
    the default is fine as it is, state that.      
    \begin{enumerate}
      \item If the class is being used to convert temperatures interactively, 
        as in this example.
        \begin{answer}[0.75in]
          In this context, it might be better to catch the exceptions
          and provide user-friendly error messages, allowing the user
          to correct their input without terminating the program.
        \end{answer}

      \item If the class is being used to convert a large array of
        temperatures without human interaction.
        \begin{answer}[0.75in]
          In this context, it would be beneficial to log the errors
          for each invalid input and continue processing the rest of
          the array, rather than terminating the entire operation.
        \end{answer}
    \end{enumerate}