% Source: https://drive.google.com/drive/folders/1JeCcwPkeQ1e5LSFeBPQm2Lo_3tSUscSw
% File: ".pdf"
% Access: 04-15-2022

% comment out for student version
\ifdefined\Student\relax\else\def\Teacher{}\fi

\documentclass[12pt]{article}

\title{Activity \#16: Arrays}
\author{unknown}
\newcommand{\activityeditor}{Preston Carman}
\newcommand{\activitysource}{\url{https://drive.google.com/drive/folders/1JeCcwPkeQ1e5LSFeBPQm2Lo_3tSUscSw}}
\date{Spring 2022}

\input{../../cspogil.sty}

\begin{document}

\begin{center}
  \maketitle \\[5pt]
  \rolenames
\end{center}

\keyquestions{
  \item Model 1, Question \#6
  \item Model 2, Question \#12.c
  \item Model 3, Question \#20
}

\newpage
\maketitle

In this course, you will work in teams of 3--4 students to learn new concepts.
This activity will introduce you to arrays in C++.

\guide{
  \item Explain the difference between vector and array syntax in C++
  \item Identify one- and two-dimensional arrays and understand their syntax in C++
  \item Understand how one- and two-dimensional arrays are passed to functions in C++

}{
  \item Translate code between vectors and one-dimensional arrays in C++
  \item Write code to manipulate two-dimensional arrays in C++ \\[-5pt]
}{
No additional notes.
}

\model{One Dimensional Arrays} \\[-15pt]
  \begin{center}
    \begin{minipage}{5.75in}
      \begin{cpplst}[
        frame=lines,
        framesep=2mm,
        bgcolor=gray!15,
        baselinestretch=1.2,
        linenos,
        firstnumber=6
      ]{cpp}
string homeworkNames[5] = {"Variables","If-Else","Loops","Vectors","Arrays"};
int pointsPossible[5] = {25, 35, 15, 20, 30};
int numStudents = 40;

for (int i=0; i<5; i++) {
  cout << homeworkNames[i] << ": " << pointsPossible[i] << endl;
}
cout << pointsPossible[5] << endl;

      \end{cpplst}
    \end{minipage}
  \end{center}  
  
  {\it\large Refer to Model 1 above as your team develops consensus answers
    to the questions below.}
    \par\vskip 10pt
    
  \begin{enumerate}
    \itemsep 20pt

    \Q Like vectors, arrays can be used to store a list of elements
      of the same type.  Answer the following questions about the arrays
      in the model above.
      \par\vskip 15pt
      
      \begin{enumerate}[(a)]
        \itemsep 12pt
        \item What are the names of the two arrays used in the model?
          \hfill \ans[2.25in]{\tt homeworkNames, pointsPossible}
        \item What are the types of the two arrays in the model?
          \hfill \ans[2.25in]{\cpp{string} and \cpp{int}}
        \item What are the sizes of the two arrays in the model?
          \hfill \ans[2.25in]{18 and 5}
        \item What word describes the relationship between the arrays?
          \hfill \ans[2.25in]{parallel}
        \item How would you access the last element of {\tt homeworkNames}?
          \hfill \ans[2.25in]{\cpp{homeworkNames[17]}}
      \end{enumerate}          
      
    \Q Based on the model, describe how the following symbols are
      used with arrays.
      
      \begin{enumerate}[(a)]
        \item The curly braces, \cpp{\{ \}.}
          \begin{answer}[0.5in]
            They are used to initialize the contents of the array when
            declaring it.
          \end{answer}
        \item The square brackets, \cpp{\[ \].}  Give two different ways they are used.
          \begin{answer}[0.75in]
            They are used to specify the size of the array when
            declaring it and to access an element in the array.
          \end{answer}
      \end{enumerate}
      
\newpage      
      
    \Q Without running it, predict what the code from this model
      will print out.
      % \ifprintanswers\vskip -20pt\null\fi
      \begin{answer}[0.75in]
        \small\tt\hspace{10pt}
        \begin{minipage}{4in}
          Variables: 25\\
          If-Else: 35\\
          Loops: 15\\
          Vectors: 20\\
          Arrays: 30\\
          <random integer>
        \end{minipage}
      \end{answer}
      % \ifprintanswers\vskip -30pt\null\fi
      
    \Q This code can be found in {\tt activity16a.cpp}.  Run it and
      explain what you see.
      \begin{answer}[0.5in]
        It prints out the first 5 elements of each array, but the
        value of \cpp{pointsPossible[5]} is undefined
        since the index 5 is past the array bounds of 0-4.
      \end{answer}
      % \ifprintanswers\vskip -30pt\null\fi
      
    \Q What happens if you change line 13 to \cpp{cout << pointsPossible}?
      \begin{answer}[0.5in]
        You get a memory address, such as {\tt 0x7ffd360a1840}.
      \end{answer}
      % \ifprintanswers\vskip -45pt\null\fi
            
    \Q Suppose that you were asked to rewrite this program to use
      vectors instead of arrays.\key\\[-15pt]
      \par\vskip 15pt
      
      \begin{enumerate}[(a)]
        \item How would you change lines 6 and 7 to
          declare vectors instead of arrays?
          \begin{answer}[0.75in]
            \scriptsize\vskip -25pt\null\hspace{60pt}
            \begin{minipage}{4.5in}
              \begin{cpplst}[
                frame=lines,
                framesep=2mm,
                bgcolor=gray!15,
                baselinestretch=1.2,
                linenos
              ]{cpp}
vector<string> homeworkNames(5) = {"Variables","If-Else","Loops","Vectors","Arrays"};
vector<int> pointsPossible(5) = {25, 35, 15, 20, 30};
              \end{cpplst}
            \end{minipage}
            \vskip -10pt\null          
          \end{answer}
          
        \item How would you change line 11 to print out the vector values?
          \begin{answer}[0.75in]
            \scriptsize\vskip -25pt\null\hspace{60pt}
            \begin{minipage}{4.5in}
              \begin{cpplst}[
                frame=lines,
                framesep=2mm,
                bgcolor=gray!15,
                baselinestretch=1.2,
                linenos,
                firstnumber=6
              ]{cpp}
cout << homeworkNames.at(i) << ": " << pointsPossible.at(i) << endl;
              \end{cpplst}
            \end{minipage}
            \vskip -10pt\null          
          \end{answer}
          
        \item Suppose line 13 was changed to \cpp{cout << pointsPossible.at(5) << endl;}.  What would happen when the
          program runs that command?
          \begin{answer}[0.5in]
            The program would crash with an out-of-range error.
          \end{answer}
      \end{enumerate}
      % \ifprintanswers\vskip -30pt\null\fi
      
    \Q The code below is designed to collect 100 names and then
      print out a random one.  It is built using vectors.  Rewrite it
      using arrays.
      
      \begin{tabular}{p{2.9in}p{2.9in}}
        \begin{minipage}{2.9in}
          \small
          \begin{cpplst}[
            frame=lines,
            framesep=2mm,
            bgcolor=gray!15,
            baselinestretch=1.2
          ]{cpp}
vector<string> names;
string tmpName;
for (int i=0; i<100; i++) {
  cout << "Name " << (i+1) << ": ";
  cin >> tmpName;
  names.push_back(tmpNames);
}
cout << names.at(rand() % 100) << endl;
          \end{cpplst}
        \end{minipage}
        &
        \begin{minipage}{2.9in}
          \begin{answer}
          \small
          \begin{cpplst}[
            frame=lines,
            framesep=2mm,
            bgcolor=gray!15,
            baselinestretch=1.2,
            linenos
          ]{cpp}
string names[100];
for (int i=0; i<100; i++) {
  cout << "Name " << (i+1) << ": ";
  cin >> names[i];
}
cout << names[rand() % 101] << endl;
          \end{cpplst}
          
          \end{answer}
        \end{minipage}
      \end{tabular}                                            
\newpage
{\bf\large Model 2: Two Dimensional Arrays} \\[-20pt]
  \begin{center}
    \begin{minipage}{4in}
      \begin{minted}[
        frame=lines,
        framesep=2mm,
        bgcolor=gray!15,
        baselinestretch=1.2
      ]{cpp}
int table[3][2] = { {1,2}, {3,4}, {5,6} };

for (int i=0; i<3; i++) {
  for (int j=0; j<2; j++) {
    cout << table[i][j] << " ";
  }
  cout << endl;
}
      \end{minted}
    \end{minipage}
  \end{center}  
  \TPMargin{5pt}
  \begin{textblock*}{1.15in}[0,0](4.5in,-1.5in)
    \textblockcolor{white}
    \begin{minipage}{1in}
      {\bf Output:} 
      \hrule\vskip 5pt
      \tt
      1 2 \\
      3 4 \\
      5 6
    \end{minipage}
  \end{textblock*}  
  \par\vskip 10pt
  
  {\it\large Refer to Model 2 above as your team develops consensus answers
    to the questions below.}
    \par\vskip -20pt\null

  \item A {\it two dimensional array} (2D array) uses two indices to identify entries.
    What is the difference in C++ syntax between declaring a 2D array and a one-dimensional
    (1D) array?
    \begin{solution}[0.5in]
      A 2D array has two pairs of square brackets, the first with the number of rows and the second with
      the number of columns in it.
    \end{solution}
    
  \item By convention, the first index is used for the row number
    and the second index is used for the column number.  Find the value of each entry
    in the 2D array given in the model.
    \par\vskip 15pt
    
    \begin{enumerate}[(a)]
      \itemsep 15pt
      \item What is the value of \mintinline{cpp}|table[2][1]| in the code above? \hfill\fillin[6][1.5in]
      \item What is the value of \mintinline{cpp}|table[1][0]| in the code above? \hfill\fillin[3][1.5in]
      \item What is the value of \mintinline{cpp}|table[0][1]| in the code above? \hfill\fillin[2][1.5in]
    \end{enumerate}
    
  \item Suppose a new 2D array was declared as \mintinline{cpp}|char tableTwo[3][3]|.  Use square
    brackets to fill in the indices for each of the elements.  The first one is done for you.
    
    \begin{center}
      \renewcommand{\arraystretch}{2.5}
      \begin{tabular}{|p{1.75in}|p{1.75in}|p{1.75in}|}
        \hline
          \tt \mintinline{cpp}|tableTwo[0][0]| = 'A' &
          \tt \fillin[\mintinline{cpp}{tableTwo[0][1]}][1.25in] = 'B' &
          \tt \fillin[\mintinline{cpp}{tableTwo[0][2]}][1.25in] = 'C' \\
        \hline
          \tt \fillin[\mintinline{cpp}{tableTwo[1][0]}][1.25in] = 'D' &
          \tt \fillin[\mintinline{cpp}{tableTwo[1][1]}][1.25in] = 'E' &
          \tt \fillin[\mintinline{cpp}{tableTwo[1][2]}][1.25in] = 'F' \\
        \hline
          \tt \fillin[\mintinline{cpp}{tableTwo[2][0]}][1.25in] = 'G' &
          \tt \fillin[\mintinline{cpp}{tableTwo[2][1]}][1.25in] = 'H' &
          \tt \fillin[\mintinline{cpp}{tableTwo[2][2]}][1.25in] = 'I' \\
        \hline
      \end{tabular}
    \end{center}
    
  \item Consider how you would write code to print the first row of {\tt tableTwo} above.  
    That is, to print {\tt A B C}.
    
    \begin{enumerate}[(a)]
      \item First give two lines of similar code that will print out the first two elements, {\tt A B}.
        \begin{solution}[1in]
          \scriptsize\vskip -25pt\null\hspace{1in}
          \begin{minipage}{2.5in}
            \begin{minted}[
              frame=lines,
              framesep=2mm,
              bgcolor=gray!15,
              baselinestretch=1.2,
              linenos
            ]{cpp}
cout << tableTwo[0][0] << " ";
cout << tableTwo[0][1] << " ";
            \end{minted}
          \end{minipage}
          \vskip -10pt\null          
        \end{solution}
      
\newpage

      \item Suppose the first row had 200 elements.  Give a few lines of code to print the entire first row.
        \begin{solution}[1in]
          \scriptsize\vskip -25pt\null\hspace{1in}
          \begin{minipage}{2.5in}
            \begin{minted}[
              frame=lines,
              framesep=2mm,
              bgcolor=gray!15,
              baselinestretch=1.2,
              linenos
            ]{cpp}
for (int i = 0; i < 200; i++) {
  cout << tableTwo[0][i] << " ";
}  
            \end{minted}
          \end{minipage}
          \vskip -10pt\null
        \end{solution}
          
      \item Suppose there were \mintinline{cpp}|int col| elements per row. Give code to print out the
        \mintinline{cpp}|int k|th row.
        \begin{solution}[1in]
          \scriptsize\vskip -25pt\null\hspace{1in}
          \begin{minipage}{2.5in}
            \begin{minted}[
              frame=lines,
              framesep=2mm,
              bgcolor=gray!15,
              baselinestretch=1.2,
              linenos
            ]{cpp}
for (int i = 0; i < col; i++) {
  cout << tableTwo[k][i] << " ";
}  
            \end{minted}
          \end{minipage}
          \vskip -10pt\null
        \end{solution}
    \end{enumerate}
    
  \item Now consider how you would adjust your code above to print out columns instead of rows.
    
    \begin{enumerate}[(a)]
      \item First give two lines of code that will print out the first two elements of the first column, {\tt A D}.
        \begin{solution}[1in]
          \scriptsize\vskip -25pt\null\hspace{1in}
          \begin{minipage}{2.5in}
            \begin{minted}[
              frame=lines,
              framesep=2mm,
              bgcolor=gray!15,
              baselinestretch=1.2,
              linenos
            ]{cpp}
cout << tableTwo[0][0] << " ";
cout << tableTwo[0][1] << " ";
            \end{minted}
          \end{minipage}
          \vskip -10pt\null          
        \end{solution}
        
      \item Suppose the first column had 200 elements.  Give code to print the entire first column.
        \begin{solution}[1in]
          \scriptsize\vskip -25pt\null\hspace{1in}
          \begin{minipage}{2.5in}
            \begin{minted}[
              frame=lines,
              framesep=2mm,
              bgcolor=gray!15,
              baselinestretch=1.2,
              linenos
            ]{cpp}
for (int i = 0; i < 200; i++) {
  cout << tableTwo[i][0] << " ";
}  
            \end{minted}
          \end{minipage}
          \vskip -10pt\null
        \end{solution}
        \par\vskip -25pt\null
          
      \item If there were \mintinline{cpp}|int row| elements per column, write code to print out the
        \mintinline{cpp}|int k|th column.\hspace{5pt}\key\\[-2.5mm]
        \begin{solution}[1in]
          \scriptsize\vskip -25pt\null\hspace{1in}
          \begin{minipage}{2.5in}
            \begin{minted}[
              frame=lines,
              framesep=2mm,
              bgcolor=gray!15,
              baselinestretch=1.2,
              linenos
            ]{cpp}
for (int i = 0; i < row; i++) {
  cout << tableTwo[i][k] << " ";
}  
            \end{minted}
          \end{minipage}
          \vskip -10pt\null
        \end{solution}
    \end{enumerate}

  \item The file {\tt activity16b.cpp} contains code that initializes a 2D array with 10 rows and 10
    columns, whose entries are integers from 1 to 100.  Use a nested for loop to print out
    the following entries in the array.  You may wish to use \mintinline{cpp}|setw(4)| from the {\tt
    iomanip} library.
    
    \begin{center}
      \begin{tabular}{p{2.9in}p{2.9in}}
        \begin{minipage}{2.9in}
          \begin{minted}[]{text}
  1
 11  12
 21  22  23
 31  32  33  34
 41  42  43  44  45
 51  52  53  54  55  56
 61  62  63  64  65  66  67
 71  72  73  74  75  76  77  78
 81  82  83  84  85  86  87  88  89
 91  92  93  94  95  96  97  98  99 100
          \end{minted}
        \end{minipage}
        &
        \begin{minipage}{2.9in}
          \begin{solution}
            \scriptsize\vskip -25pt\null
            \begin{minipage}{2.75in}
              \begin{minted}[
                frame=lines,
                framesep=2mm,
                bgcolor=gray!15,
                baselinestretch=1.2,
              ]{cpp}
for (int row = 0; row < 10; row++) {
  for (int col = 0; col <= row; col++) {
    cout << setw(4) << myTable[row][col];
  }
  cout << endl;
}  
            \end{minted}
          \end{minipage}
          \vskip -10pt\null
          \end{solution}
        \end{minipage}
      \end{tabular}
    \end{center}

  
\newpage
{\bf\large Model 3: Arrays as Function Parameters} \\[-20pt]
  \begin{center}
    \begin{tabular}{p{2.8in}p{0.1in}p{2.7in}}
      \begin{minipage}{2.8in}
        \scriptsize
        \begin{minted}[
          frame=lines,
          framesep=2mm,
          bgcolor=gray!15,
          baselinestretch=1.2,
          linenos,
          firstnumber=5
        ]{cpp}        
const int NUM_ROWS = 3;
const int NUM_COLS = 3;

// initialize the game board to spaces
void initBoard(char board[][NUM_COLS], int rows);

// print the game board
void printBoard(char board[][NUM_COLS], int rows);

// prompt player `turn' for row/col of play
void getMove(char turn, int &row, int &col);

// check if single array isall the same char
bool isWinningRow(char row[], int size, char &winner);

// check if a board has a winning row/col/diagonal
bool isWinningBoard(char board[][NUM_COLS], 
                    int rows, char &winner);
                    
int main() {
  char board[NUM_ROWS][NUM_COLS];
  char winner,playAgain;
  int row, col;
        \end{minted}
      \end{minipage}
      & &
      \begin{minipage}{2.7in}
        \scriptsize
        \begin{minted}[
          frame=lines,
          framesep=2mm,
          bgcolor=gray!15,
          baselinestretch=1.2,
          linenos,
          firstnumber=28
        ]{cpp}
  do { 
    char turn = 'O';
    initBoard(board,NUM_ROWS);
    do {
      if(turn == 'X') turn = 'O'; else turn = 'X';
      printBoard(board,NUM_ROWS);
      do {
        getMove(turn,row,col);
      } while( board[row][col] != ' ');
      board[row][col] = turn;
    } while(!isWinningBoard(board,NUM_ROWS,winner));
    printBoard(board,NUM_ROWS);
    if (winner != 'D') {
      cout << "Congrats player " << winner << endl;
    } else {
      cout << "It's a draw!" << endl;
    }
    cout << "Play again (y/n)? ";
    cin >> playAgain;
  } while (playAgain == 'y');
  return 0;
}   
        \end{minted}
        \vskip 2pt\null
      \end{minipage}
    \end{tabular} 
  \end{center}  
  \par\vskip 10pt

  {\it\large Refer to Model 3 above as your team develops consensus answers
    to the questions below.}
    \ifprintanswers\vskip -20pt\null\fi          

  \item Determine if the functions prototyped in the model above have 1D, 2D, or no array parameters.
    \par\vskip 15pt
    
    \begin{enumerate}[(a)]
      \itemsep 15pt
      \begin{multicols}{2}
        \item \mintinline{cpp}|void initBoard(...)|\hfill\fillin[2D][0.75in]
        \item \mintinline{cpp}|void printBoard(...)|\hfill\fillin[2D][0.75in]
        \item \mintinline{cpp}|void getMove(...)|\hfill\fillin[none][0.75in]
        \item \mintinline{cpp}|bool isWinningRow(...)|\hfill\fillin[1D][0.75in]
        \item \mintinline{cpp}|bool isWinningBoard(...)|\hfill\fillin[2D][0.75in]
      \end{multicols}
    \end{enumerate}
    
    \item Based on your observations above, answer the following questions about passing arrays to functions.
      
      \begin{enumerate}[(a)]
        \item How do you specify that a given parameter is a 1D array?
          \begin{solution}[0.5in]
            You give the type and name followed by square brackets.  E.g. \mintinline{cpp}|char theRow[]|
          \end{solution}
        \item How do you specify that a given parameter is a 2D array?
          \begin{solution}[0.5in]
            Type and name followed by two pairs of \mintinline{cpp}|[]| with size in 2nd. E.g. \mintinline{cpp}|char board[][3]|
          \end{solution}
        \item How do you pass an array as an argument in a function call?
          \begin{solution}[0.5in]
            You use the name of the array only (no \mintinline{cpp}|[]|). E.g. \mintinline{cpp}|printBoard(board,NUM_ROWS)|
          \end{solution}
        \item What other parameter must you always pass with an array parameter?
          \begin{solution}[0.5in]
            You must pass the number of rows in the array, whether it is 1D or 2D.
          \end{solution}
      \end{enumerate}

\newpage

    \item This program, together with the function definitions, is in {\tt activity16c.cpp}.
      What does it do?
      \begin{solution}[0.5in]
        It implements a simple tic-tac-toe game.
      \end{solution}
      
    \item How is the 2D array defined on line 25 used in this program?  In particular, suppose the
      following output was produced by \mintinline{cpp}|printBoard()|.  What would the values of each
      array entry be?
      
      \begin{center}
        \begin{tabular}{p{1.4in}p{4.4in}}
          \begin{minipage}{1.4in}
            \begin{minted}[]{text}
 O | O | X
---+---+---
   | X |  
---+---+---
   |   |              
            \end{minted}
          \end{minipage}
          &
          \begin{minipage}{4.4in}
            \begin{solution}[1.25in]
              The array contains the X's and O's that have been played in the slots. For this game
              board, we would have:
              \scriptsize
              \begin{itemize}
                \begin{multicols}{3}
                  \item \mintinline{cpp}|board[0][0] = 'O'|
                  \item \mintinline{cpp}|board[1][0] = ' '|
                  \item \mintinline{cpp}|board[2][0] = ' '|
                  \item \mintinline{cpp}|board[0][1] = 'O'| 
                  \item \mintinline{cpp}|board[1][1] = 'X'|
                  \item \mintinline{cpp}|board[2][1] = ' '|
                  \item \mintinline{cpp}|board[0][2] = 'X'|
                  \item \mintinline{cpp}|board[1][2] = ' '|
                  \item \mintinline{cpp}|board[2][2] = ' '|
                \end{multicols}
              \end{itemize}
            \end{solution}
          \end{minipage}
        \end{tabular}
      \end{center}
      
    \item What happens if you change the values of the constants on lines 5 and 6 from \mintinline{cpp}|3| to
      \mintinline{cpp}|4|?
      \begin{solution}[0.5in]
        We are now playing tic-tac-toe on a 4 by 4 board instead of 3 by 3.
      \end{solution}
      
    \item Why do you think we always have a size parameter together with any array parameter of a function?
      \begin{solution}[0.75in]
        To indicate how many rows the array has (or at least how many should be used).
      \end{solution}
      \par\vskip -35pt\null
      
    \item Take another look at the function \mintinline{cpp}|void initBoard| declared in this
      model.\key\\[-20pt]
      \par\vskip 15pt
      \begin{enumerate}[(a)]
        \itemsep 15pt
        \item What does this function do?
          \begin{solution}[0.5in]
            The function sets all entries in the array to be spaces.
          \end{solution}
        \item On what line is the \mintinline{cpp}|char board[][]| variable passed to this function
          declared?
          \hfill\fillin[line 25][1in]
        \item Is the \mintinline{cpp}|char board[][]| variable modified in this function local or global?
          \hfill\fillin[local][1in]
        \item Based on the effect of the function, was the array passed by value or by reference?
          \hfill\fillin[by reference][1in]
        \item Was an {\tt \&} used in the function header to make the array pass-by-reference?
          \hfill\fillin[no][1in]
        \item What can you conclude about how array parameters are passed?
          \begin{solution}[0.5in]
            Array parameters are always passed by reference, even without an {\tt \&}.
          \end{solution}
      \end{enumerate}
    
\end{document}
