\model{Arrays as Function Parameters}
  \begin{center}
    \begin{tabular}{p{3in}p{0.1in}p{2.7in}}
      \begin{minipage}{3in}
        \scriptsize
        \begin{cpplst}       
const int NUM_ROWS = 3;
const int NUM_COLS = 3;

// initialize the game board to spaces
void initBoard(char board[][NUM_COLS], int rows);

// print the game board
void printBoard(char board[][NUM_COLS], int rows);

// prompt player `turn' for row/col of play
void getMove(char turn, int &row, int &col);

// check if single array isall the same char
bool isWinningRow(char row[], int size, char &winner);

// check if a board has a winning row/col/diagonal
bool isWinningBoard(char board[][NUM_COLS], 
                    int rows, char &winner);
                    
        \end{cpplst}
      \end{minipage}
      & &
      \begin{minipage}{2.7in}
        \scriptsize
        \begin{cpplst}
int main() {
  char board[NUM_ROWS][NUM_COLS];
  char winner,playAgain;
  int row, col;
  do {
    char turn = 'O';
    initBoard(board,NUM_ROWS);
    do {
      if(turn == 'X') turn = 'O'; else turn = 'X';
      printBoard(board,NUM_ROWS);
      do {
        getMove(turn,row,col);
      } while( board[row][col] != ' ');
      board[row][col] = turn;
    } while(!isWinningBoard(board,NUM_ROWS,winner));
    printBoard(board,NUM_ROWS);
    if (winner != 'D') {
      cout << "Congrats player " << winner << endl;
    } else {
      cout << "It's a draw!" << endl;
    }
    cout << "Play again (y/n)? ";
    cin >> playAgain;
  } while (playAgain == 'y');
  return 0;
  }   
        \end{cpplst}
      \end{minipage}
    \end{tabular} 
  \end{center}  

  {\it\large Refer to Model 3 above as your team develops consensus answers
    to the questions below.}        

  \Q Determine if the functions prototyped in the model above have 1D, 2D, or no array parameters.
    \begin{enumerate}
      \itemsep 10pt
      \begin{multicols}{2}
        \item \cpp{void initBoard(...)}\hfill\ans[0.75in]{2D}
        \item \cpp{void printBoard(...)}\hfill\ans[0.75in]{2D}
        \item \cpp{void getMove(...)}\hfill\ans[0.75in]{none}
        \item \cpp{bool isWinningRow(...)}\hfill\ans[0.75in]{1D}
        \item \cpp{bool isWinningBoard(...)}\hfill\ans[0.75in]{2D}
      \end{multicols}
    \end{enumerate}
    
  \Q Based on your observations above, answer the following questions about passing arrays to functions.
    \begin{enumerate}
      \item How do you specify that a given parameter is a 1D array?
        \begin{answer}[0.5in]
          You give the type and name followed by square brackets.  E.g. \cpp{char theRow[]}
        \end{answer}

      \item How do you specify that a given parameter is a 2D array?
        \begin{answer}[0.5in]
          Type and name followed by two pairs of \cpp{[]} with size in 2nd. E.g. \cpp{char board[][3]}
        \end{answer}

      \item How do you pass an array as an argument in a function call?
        \begin{answer}[0.5in]
          You use the name of the array only (no \cpp{[]}). E.g. \cpp{printBoard(board,NUM_ROWS)}
        \end{answer}

      \item What other parameter must you always pass with an array parameter?
        \begin{answer}[0.5in]
          You must pass the number of rows in the array, whether it is 1D or 2D.
        \end{answer}
    \end{enumerate}

  \Q This program, together with the function definitions, is in {\tt activity16c.cpp}.
    What does it do?
    \begin{answer}[0.5in]
      It implements a simple tic-tac-toe game.
    \end{answer}
    
  \Q How is the 2D array defined on line 2R used in this program?  In particular, suppose the
    following output was produced by \cpp{printBoard()}.  What would the values of each
    array entry be?
    \begin{center}
      \begin{tabular}{p{1.4in}p{4.4in}}
        \begin{minipage}{1.4in}
          \begin{cpplst}[]{text}
 O | O | X
---+---+---
   | X |  
---+---+---
   |   |              
          \end{cpplst}
        \end{minipage}
        &
        \begin{minipage}{4.4in}
          \begin{answer}[1.25in]
            The array contains the X's and O's that have been played in the slots. For this game
            board, we would have:
            \scriptsize
            \begin{itemize}
              \begin{multicols}{3}
                \item \cpp{board[0][0] = 'O'}
                \item \cpp{board[1][0] = ' '}
                \item \cpp{board[2][0] = ' '}
                \item \cpp{board[0][1] = 'O'} 
                \item \cpp{board[1][1] = 'X'}
                \item \cpp{board[2][1] = ' '}
                \item \cpp{board[0][2] = 'X'}
                \item \cpp{board[1][2] = ' '}
                \item \cpp{board[2][2] = ' '}
              \end{multicols}
            \end{itemize}
          \end{answer}
        \end{minipage}
      \end{tabular}
    \end{center}
      
  \Q What happens if you change the values of the constants on lines 1 and 2 from \cpp{3} to
    \cpp{4}?
    \begin{answer}[0.5in]
      We are now playing tic-tac-toe on a 4 by 4 board instead of 3 by 3.
    \end{answer}
    
  \Q Why do you think we always have a size parameter together with any array parameter of a function?
    \begin{answer}[0.75in]
      To indicate how many rows the array has (or at least how many should be used).
    \end{answer}
    \par\vskip -35pt\null
    
  \Q Take another look at the function \cpp{void initBoard} declared in this
    model.\key\\[-20pt]
    \begin{enumerate}
      \itemsep 10pt
      \item What does this function do?
        \begin{answer}[0.5in]
          The function sets all entries in the array to be spaces.
        \end{answer}

      \item On what line is the \cpp{char board[][]} variable passed to this function
        declared?
        \begin{answer}[0.2in]
          line 7R
        \end{answer}

      \item Is the \cpp{char board[][]} variable modified in this function local or global?
        \begin{answer}[0.2in]
          local
        \end{answer}

      \item Based on the effect of the function, was the array passed by value or by reference?
        \begin{answer}[0.2in]
          by reference
        \end{answer}

      \item Was an {\tt \&} used in the function header to make the array pass-by-reference?
        \begin{answer}[0.2in]
          no
        \end{answer}

      \item What can you conclude about how array parameters are passed?
        \begin{answer}[0.5in]
          Array parameters are always passed by reference, even without an {\tt \&}.
        \end{answer}
    \end{enumerate}