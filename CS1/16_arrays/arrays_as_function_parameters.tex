{\bf\large Model 3: Arrays as Function Parameters} \\[-20pt]
  \begin{center}
    \begin{tabular}{p{2.8in}p{0.1in}p{2.7in}}
      \begin{minipage}{2.8in}
        \scriptsize
        \begin{minted}[
          frame=lines,
          framesep=2mm,
          bgcolor=gray!15,
          baselinestretch=1.2,
          linenos,
          firstnumber=5
        ]{cpp}        
const int NUM_ROWS = 3;
const int NUM_COLS = 3;

// initialize the game board to spaces
void initBoard(char board[][NUM_COLS], int rows);

// print the game board
void printBoard(char board[][NUM_COLS], int rows);

// prompt player `turn' for row/col of play
void getMove(char turn, int &row, int &col);

// check if single array isall the same char
bool isWinningRow(char row[], int size, char &winner);

// check if a board has a winning row/col/diagonal
bool isWinningBoard(char board[][NUM_COLS], 
                    int rows, char &winner);
                    
int main() {
  char board[NUM_ROWS][NUM_COLS];
  char winner,playAgain;
  int row, col;
        \end{minted}
      \end{minipage}
      & &
      \begin{minipage}{2.7in}
        \scriptsize
        \begin{minted}[
          frame=lines,
          framesep=2mm,
          bgcolor=gray!15,
          baselinestretch=1.2,
          linenos,
          firstnumber=28
        ]{cpp}
  do { 
    char turn = 'O';
    initBoard(board,NUM_ROWS);
    do {
      if(turn == 'X') turn = 'O'; else turn = 'X';
      printBoard(board,NUM_ROWS);
      do {
        getMove(turn,row,col);
      } while( board[row][col] != ' ');
      board[row][col] = turn;
    } while(!isWinningBoard(board,NUM_ROWS,winner));
    printBoard(board,NUM_ROWS);
    if (winner != 'D') {
      cout << "Congrats player " << winner << endl;
    } else {
      cout << "It's a draw!" << endl;
    }
    cout << "Play again (y/n)? ";
    cin >> playAgain;
  } while (playAgain == 'y');
  return 0;
}   
        \end{minted}
        \vskip 2pt\null
      \end{minipage}
    \end{tabular} 
  \end{center}  
  \par\vskip 10pt

  {\it\large Refer to Model 3 above as your team develops consensus answers
    to the questions below.}
    \ifprintanswers\vskip -20pt\null\fi          

  \item Determine if the functions prototyped in the model above have 1D, 2D, or no array parameters.
    \par\vskip 15pt
    
    \begin{enumerate}[(a)]
      \itemsep 15pt
      \begin{multicols}{2}
        \item \mintinline{cpp}|void initBoard(...)|\hfill\fillin[2D][0.75in]
        \item \mintinline{cpp}|void printBoard(...)|\hfill\fillin[2D][0.75in]
        \item \mintinline{cpp}|void getMove(...)|\hfill\fillin[none][0.75in]
        \item \mintinline{cpp}|bool isWinningRow(...)|\hfill\fillin[1D][0.75in]
        \item \mintinline{cpp}|bool isWinningBoard(...)|\hfill\fillin[2D][0.75in]
      \end{multicols}
    \end{enumerate}
    
    \item Based on your observations above, answer the following questions about passing arrays to functions.
      
      \begin{enumerate}[(a)]
        \item How do you specify that a given parameter is a 1D array?
          \begin{solution}[0.5in]
            You give the type and name followed by square brackets.  E.g. \mintinline{cpp}|char theRow[]|
          \end{solution}
        \item How do you specify that a given parameter is a 2D array?
          \begin{solution}[0.5in]
            Type and name followed by two pairs of \mintinline{cpp}|[]| with size in 2nd. E.g. \mintinline{cpp}|char board[][3]|
          \end{solution}
        \item How do you pass an array as an argument in a function call?
          \begin{solution}[0.5in]
            You use the name of the array only (no \mintinline{cpp}|[]|). E.g. \mintinline{cpp}|printBoard(board,NUM_ROWS)|
          \end{solution}
        \item What other parameter must you always pass with an array parameter?
          \begin{solution}[0.5in]
            You must pass the number of rows in the array, whether it is 1D or 2D.
          \end{solution}
      \end{enumerate}

\newpage

    \item This program, together with the function definitions, is in {\tt activity16c.cpp}.
      What does it do?
      \begin{solution}[0.5in]
        It implements a simple tic-tac-toe game.
      \end{solution}
      
    \item How is the 2D array defined on line 25 used in this program?  In particular, suppose the
      following output was produced by \mintinline{cpp}|printBoard()|.  What would the values of each
      array entry be?
      
      \begin{center}
        \begin{tabular}{p{1.4in}p{4.4in}}
          \begin{minipage}{1.4in}
            \begin{minted}[]{text}
 O | O | X
---+---+---
   | X |  
---+---+---
   |   |              
            \end{minted}
          \end{minipage}
          &
          \begin{minipage}{4.4in}
            \begin{solution}[1.25in]
              The array contains the X's and O's that have been played in the slots. For this game
              board, we would have:
              \scriptsize
              \begin{itemize}
                \begin{multicols}{3}
                  \item \mintinline{cpp}|board[0][0] = 'O'|
                  \item \mintinline{cpp}|board[1][0] = ' '|
                  \item \mintinline{cpp}|board[2][0] = ' '|
                  \item \mintinline{cpp}|board[0][1] = 'O'| 
                  \item \mintinline{cpp}|board[1][1] = 'X'|
                  \item \mintinline{cpp}|board[2][1] = ' '|
                  \item \mintinline{cpp}|board[0][2] = 'X'|
                  \item \mintinline{cpp}|board[1][2] = ' '|
                  \item \mintinline{cpp}|board[2][2] = ' '|
                \end{multicols}
              \end{itemize}
            \end{solution}
          \end{minipage}
        \end{tabular}
      \end{center}
      
    \item What happens if you change the values of the constants on lines 5 and 6 from \mintinline{cpp}|3| to
      \mintinline{cpp}|4|?
      \begin{solution}[0.5in]
        We are now playing tic-tac-toe on a 4 by 4 board instead of 3 by 3.
      \end{solution}
      
    \item Why do you think we always have a size parameter together with any array parameter of a function?
      \begin{solution}[0.75in]
        To indicate how many rows the array has (or at least how many should be used).
      \end{solution}
      \par\vskip -35pt\null
      
    \item Take another look at the function \mintinline{cpp}|void initBoard| declared in this
      model.\key\\[-20pt]
      \par\vskip 15pt
      \begin{enumerate}[(a)]
        \itemsep 15pt
        \item What does this function do?
          \begin{solution}[0.5in]
            The function sets all entries in the array to be spaces.
          \end{solution}
        \item On what line is the \mintinline{cpp}|char board[][]| variable passed to this function
          declared?
          \hfill\fillin[line 25][1in]
        \item Is the \mintinline{cpp}|char board[][]| variable modified in this function local or global?
          \hfill\fillin[local][1in]
        \item Based on the effect of the function, was the array passed by value or by reference?
          \hfill\fillin[by reference][1in]
        \item Was an {\tt \&} used in the function header to make the array pass-by-reference?
          \hfill\fillin[no][1in]
        \item What can you conclude about how array parameters are passed?
          \begin{solution}[0.5in]
            Array parameters are always passed by reference, even without an {\tt \&}.
          \end{solution}
      \end{enumerate}