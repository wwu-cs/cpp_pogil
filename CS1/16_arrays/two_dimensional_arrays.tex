{\bf\large Model 2: Two Dimensional Arrays} \\[-20pt]
  \begin{center}
    \begin{minipage}{4in}
      \begin{minted}[
        frame=lines,
        framesep=2mm,
        bgcolor=gray!15,
        baselinestretch=1.2
      ]{cpp}
int table[3][2] = { {1,2}, {3,4}, {5,6} };

for (int i=0; i<3; i++) {
  for (int j=0; j<2; j++) {
    cout << table[i][j] << " ";
  }
  cout << endl;
}
      \end{minted}
    \end{minipage}
  \end{center}  
  \TPMargin{5pt}
  \begin{textblock*}{1.15in}[0,0](4.5in,-1.5in)
    \textblockcolor{white}
    \begin{minipage}{1in}
      {\bf Output:} 
      \hrule\vskip 5pt
      \tt
      1 2 \\
      3 4 \\
      5 6
    \end{minipage}
  \end{textblock*}  
  \par\vskip 10pt
  
  {\it\large Refer to Model 2 above as your team develops consensus answers
    to the questions below.}
    \par\vskip -20pt\null

  \item A {\it two dimensional array} (2D array) uses two indices to identify entries.
    What is the difference in C++ syntax between declaring a 2D array and a one-dimensional
    (1D) array?
    \begin{solution}[0.5in]
      A 2D array has two pairs of square brackets, the first with the number of rows and the second with
      the number of columns in it.
    \end{solution}
    
  \item By convention, the first index is used for the row number
    and the second index is used for the column number.  Find the value of each entry
    in the 2D array given in the model.
    \par\vskip 15pt
    
    \begin{enumerate}[(a)]
      \itemsep 15pt
      \item What is the value of \mintinline{cpp}|table[2][1]| in the code above? \hfill\fillin[6][1.5in]
      \item What is the value of \mintinline{cpp}|table[1][0]| in the code above? \hfill\fillin[3][1.5in]
      \item What is the value of \mintinline{cpp}|table[0][1]| in the code above? \hfill\fillin[2][1.5in]
    \end{enumerate}
    
  \item Suppose a new 2D array was declared as \mintinline{cpp}|char tableTwo[3][3]|.  Use square
    brackets to fill in the indices for each of the elements.  The first one is done for you.
    
    \begin{center}
      \renewcommand{\arraystretch}{2.5}
      \begin{tabular}{|p{1.75in}|p{1.75in}|p{1.75in}|}
        \hline
          \tt \mintinline{cpp}|tableTwo[0][0]| = 'A' &
          \tt \fillin[\mintinline{cpp}{tableTwo[0][1]}][1.25in] = 'B' &
          \tt \fillin[\mintinline{cpp}{tableTwo[0][2]}][1.25in] = 'C' \\
        \hline
          \tt \fillin[\mintinline{cpp}{tableTwo[1][0]}][1.25in] = 'D' &
          \tt \fillin[\mintinline{cpp}{tableTwo[1][1]}][1.25in] = 'E' &
          \tt \fillin[\mintinline{cpp}{tableTwo[1][2]}][1.25in] = 'F' \\
        \hline
          \tt \fillin[\mintinline{cpp}{tableTwo[2][0]}][1.25in] = 'G' &
          \tt \fillin[\mintinline{cpp}{tableTwo[2][1]}][1.25in] = 'H' &
          \tt \fillin[\mintinline{cpp}{tableTwo[2][2]}][1.25in] = 'I' \\
        \hline
      \end{tabular}
    \end{center}
    
  \item Consider how you would write code to print the first row of {\tt tableTwo} above.  
    That is, to print {\tt A B C}.
    
    \begin{enumerate}[(a)]
      \item First give two lines of similar code that will print out the first two elements, {\tt A B}.
        \begin{solution}[1in]
          \scriptsize\vskip -25pt\null\hspace{1in}
          \begin{minipage}{2.5in}
            \begin{minted}[
              frame=lines,
              framesep=2mm,
              bgcolor=gray!15,
              baselinestretch=1.2,
              linenos
            ]{cpp}
cout << tableTwo[0][0] << " ";
cout << tableTwo[0][1] << " ";
            \end{minted}
          \end{minipage}
          \vskip -10pt\null          
        \end{solution}
      
\newpage

      \item Suppose the first row had 200 elements.  Give a few lines of code to print the entire first row.
        \begin{solution}[1in]
          \scriptsize\vskip -25pt\null\hspace{1in}
          \begin{minipage}{2.5in}
            \begin{minted}[
              frame=lines,
              framesep=2mm,
              bgcolor=gray!15,
              baselinestretch=1.2,
              linenos
            ]{cpp}
for (int i = 0; i < 200; i++) {
  cout << tableTwo[0][i] << " ";
}  
            \end{minted}
          \end{minipage}
          \vskip -10pt\null
        \end{solution}
          
      \item Suppose there were \mintinline{cpp}|int col| elements per row. Give code to print out the
        \mintinline{cpp}|int k|th row.
        \begin{solution}[1in]
          \scriptsize\vskip -25pt\null\hspace{1in}
          \begin{minipage}{2.5in}
            \begin{minted}[
              frame=lines,
              framesep=2mm,
              bgcolor=gray!15,
              baselinestretch=1.2,
              linenos
            ]{cpp}
for (int i = 0; i < col; i++) {
  cout << tableTwo[k][i] << " ";
}  
            \end{minted}
          \end{minipage}
          \vskip -10pt\null
        \end{solution}
    \end{enumerate}
    
  \item Now consider how you would adjust your code above to print out columns instead of rows.
    
    \begin{enumerate}[(a)]
      \item First give two lines of code that will print out the first two elements of the first column, {\tt A D}.
        \begin{solution}[1in]
          \scriptsize\vskip -25pt\null\hspace{1in}
          \begin{minipage}{2.5in}
            \begin{minted}[
              frame=lines,
              framesep=2mm,
              bgcolor=gray!15,
              baselinestretch=1.2,
              linenos
            ]{cpp}
cout << tableTwo[0][0] << " ";
cout << tableTwo[0][1] << " ";
            \end{minted}
          \end{minipage}
          \vskip -10pt\null          
        \end{solution}
        
      \item Suppose the first column had 200 elements.  Give code to print the entire first column.
        \begin{solution}[1in]
          \scriptsize\vskip -25pt\null\hspace{1in}
          \begin{minipage}{2.5in}
            \begin{minted}[
              frame=lines,
              framesep=2mm,
              bgcolor=gray!15,
              baselinestretch=1.2,
              linenos
            ]{cpp}
for (int i = 0; i < 200; i++) {
  cout << tableTwo[i][0] << " ";
}  
            \end{minted}
          \end{minipage}
          \vskip -10pt\null
        \end{solution}
        \par\vskip -25pt\null
          
      \item If there were \mintinline{cpp}|int row| elements per column, write code to print out the
        \mintinline{cpp}|int k|th column.\hspace{5pt}\key\\[-2.5mm]
        \begin{solution}[1in]
          \scriptsize\vskip -25pt\null\hspace{1in}
          \begin{minipage}{2.5in}
            \begin{minted}[
              frame=lines,
              framesep=2mm,
              bgcolor=gray!15,
              baselinestretch=1.2,
              linenos
            ]{cpp}
for (int i = 0; i < row; i++) {
  cout << tableTwo[i][k] << " ";
}  
            \end{minted}
          \end{minipage}
          \vskip -10pt\null
        \end{solution}
    \end{enumerate}

  \item The file {\tt activity16b.cpp} contains code that initializes a 2D array with 10 rows and 10
    columns, whose entries are integers from 1 to 100.  Use a nested for loop to print out
    the following entries in the array.  You may wish to use \mintinline{cpp}|setw(4)| from the {\tt
    iomanip} library.
    
    \begin{center}
      \begin{tabular}{p{2.9in}p{2.9in}}
        \begin{minipage}{2.9in}
          \begin{minted}[]{text}
  1
 11  12
 21  22  23
 31  32  33  34
 41  42  43  44  45
 51  52  53  54  55  56
 61  62  63  64  65  66  67
 71  72  73  74  75  76  77  78
 81  82  83  84  85  86  87  88  89
 91  92  93  94  95  96  97  98  99 100
          \end{minted}
        \end{minipage}
        &
        \begin{minipage}{2.9in}
          \begin{solution}
            \scriptsize\vskip -25pt\null
            \begin{minipage}{2.75in}
              \begin{minted}[
                frame=lines,
                framesep=2mm,
                bgcolor=gray!15,
                baselinestretch=1.2,
              ]{cpp}
for (int row = 0; row < 10; row++) {
  for (int col = 0; col <= row; col++) {
    cout << setw(4) << myTable[row][col];
  }
  cout << endl;
}  
            \end{minted}
          \end{minipage}
          \vskip -10pt\null
          \end{solution}
        \end{minipage}
      \end{tabular}
    \end{center}

  