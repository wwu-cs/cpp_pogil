{\bf\large Model 1: One Dimensional Arrays} \\[-15pt]
  \begin{center}
    \begin{minipage}{5.75in}
      \begin{minted}[
        frame=lines,
        framesep=2mm,
        bgcolor=gray!15,
        baselinestretch=1.2,
        linenos,
        firstnumber=6
      ]{cpp}
string homeworkNames[5] = {"Variables","If-Else","Loops","Vectors","Arrays"};
int pointsPossible[5] = {25, 35, 15, 20, 30};
int numStudents = 40;

for (int i=0; i<5; i++) {
  cout << homeworkNames[i] << ": " << pointsPossible[i] << endl;
}
cout << pointsPossible[5] << endl;

      \end{minted}
    \end{minipage}
  \end{center}  
  
  {\it\large Refer to Model 1 above as your team develops consensus answers
    to the questions below.}
    \par\vskip 10pt
    
  \begin{enumerate}
    \itemsep 20pt

    \item Like vectors, arrays can be used to store a list of elements
      of the same type.  Answer the following questions about the arrays
      in the model above.
      \par\vskip 15pt
      
      \begin{enumerate}[(a)]
        \itemsep 12pt
        \item What are the names of the two arrays used in the model?
          \hfill \fillin[\tt homeworkNames, pointsPossible][2.25in]
        \item What are the types of the two arrays in the model?
          \hfill \fillin[\mintinline{cpp}|string| and \mintinline{cpp}|int|][2.25in]
        \item What are the sizes of the two arrays in the model?
          \hfill \fillin[18 and 5][2.25in]
        \item What word describes the relationship between the arrays?
          \hfill \fillin[parallel][2.25in]
        \item How would you access the last element of {\tt homeworkNames}?
          \hfill\fillin[\mintinline{cpp}{homeworkNames[17]}][2.25in]
      \end{enumerate}          
      
    \item Based on the model, describe how the following symbols are
      used with arrays.
      
      \begin{enumerate}[(a)]
        \item The curly braces, \mintinline{cpp}|{ }|.
          \begin{solution}[0.5in]
            They are used to initialize the contents of the array when
            declaring it.
          \end{solution}
        \item The square brackets, \mintinline{cpp}|[ ]|.  Give two different ways they are used.
          \begin{solution}[0.75in]
            They are used to specify the size of the array when
            declaring it and to access an element in the array.
          \end{solution}
      \end{enumerate}
      
\newpage      
      
    \item Without running it, predict what the code from this model
      will print out.
      \ifprintanswers\vskip -20pt\null\fi
      \begin{solution}[0.75in]
        \small\tt\hspace{10pt}
        \begin{minipage}{4in}
          Variables: 25\\
          If-Else: 35\\
          Loops: 15\\
          Vectors: 20\\
          Arrays: 30\\
          <random integer>
        \end{minipage}
      \end{solution}
      \ifprintanswers\vskip -30pt\null\fi
      
    \item This code can be found in {\tt activity16a.cpp}.  Run it and
      explain what you see.
      \begin{solution}[0.5in]
        It prints out the first 5 elements of each array, but the
        value of \mintinline{cpp}|pointsPossible[5]| is undefined
        since the index 5 is past the array bounds of 0-4.
      \end{solution}
      \ifprintanswers\vskip -30pt\null\fi
      
    \item What happens if you change line 13 to \mintinline{cpp}|cout << pointsPossible|?
      \begin{solution}[0.5in]
        You get a memory address, such as {\tt 0x7ffd360a1840}.
      \end{solution}
      \ifprintanswers\vskip -45pt\null\fi
            
    \item Suppose that you were asked to rewrite this program to use
      vectors instead of arrays.\key\\[-15pt]
      \par\vskip 15pt
      
      \begin{enumerate}[(a)]
        \item How would you change lines 6 and 7 to
          declare vectors instead of arrays?
          \begin{solution}[0.75in]
            \scriptsize\vskip -25pt\null\hspace{60pt}
            \begin{minipage}{4.5in}
              \begin{minted}[
                frame=lines,
                framesep=2mm,
                bgcolor=gray!15,
                baselinestretch=1.2,
                linenos
              ]{cpp}
vector<string> homeworkNames(5) = {"Variables","If-Else","Loops","Vectors","Arrays"};
vector<int> pointsPossible(5) = {25, 35, 15, 20, 30};
              \end{minted}
            \end{minipage}
            \vskip -10pt\null          
          \end{solution}
          
        \item How would you change line 11 to print out the vector values?
          \begin{solution}[0.75in]
            \scriptsize\vskip -25pt\null\hspace{60pt}
            \begin{minipage}{4.5in}
              \begin{minted}[
                frame=lines,
                framesep=2mm,
                bgcolor=gray!15,
                baselinestretch=1.2,
                linenos,
                firstnumber=6
              ]{cpp}
cout << homeworkNames.at(i) << ": " << pointsPossible.at(i) << endl;
              \end{minted}
            \end{minipage}
            \vskip -10pt\null          
          \end{solution}
          
        \item Suppose line 13 was changed to \mintinline{cpp}|cout << pointsPossible.at(5) << endl;|.  What would happen when the
          program runs that command?
          \begin{solution}[0.5in]
            The program would crash with an out-of-range error.
          \end{solution}
      \end{enumerate}
      \ifprintanswers\vskip -30pt\null\fi
      
    \item The code below is designed to collect 100 names and then
      print out a random one.  It is built using vectors.  Rewrite it
      using arrays.
      
      \begin{tabular}{p{2.9in}p{2.9in}}
        \begin{minipage}{2.9in}
          \small
          \begin{minted}[
            frame=lines,
            framesep=2mm,
            bgcolor=gray!15,
            baselinestretch=1.2
          ]{cpp}
vector<string> names;
string tmpName;
for (int i=0; i<100; i++) {
  cout << "Name " << (i+1) << ": ";
  cin >> tmpName;
  names.push_back(tmpNames);
}
cout << names.at(rand() % 100) << endl;
          \end{minted}
        \end{minipage}
        &
        \begin{minipage}{2.9in}
          \begin{solution}
          \small
          \begin{minted}[
            frame=lines,
            framesep=2mm,
            bgcolor=gray!15,
            baselinestretch=1.2,
            linenos
          ]{cpp}
string names[100];
for (int i=0; i<100; i++) {
  cout << "Name " << (i+1) << ": ";
  cin >> names[i];
}
cout << names[rand() % 101] << endl;
          \end{minted}
          
          \end{solution}
        \end{minipage}
      \end{tabular}