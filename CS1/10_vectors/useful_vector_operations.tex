\model{Useful Vector Operations}
  \begin{center}
    \small
    \renewcommand{\arraystretch}{1.4}    
    \begin{tabular}{|c|c|c|}
      \hline
      \rowcolor{orange!20} Original {\tt myVec} & Operation & Resulting {\tt myVec}  \\
      \hline
        \begin{minipage}{1.5in}
          \centering\vskip 10pt
          \begin{tabular}{|c|c|c|c|}
            \hline
            {\tt myVec:} & 8 & 2 & 6 \\
            \hline
          \end{tabular}
          \vskip 5pt\null
        \end{minipage}
        &
        \cpp{myVec.resize(4)}
        &
        \begin{minipage}{1.8in}
          \centering\vskip 10pt
          \begin{tabular}{|r|c|c|c|c|c|}
            \hline
            {\tt myVec:} & 8 & 2 & 6 & \\
            \hline
          \end{tabular}
          \vskip 5pt\null
        \end{minipage}
      \\ \hline
        \begin{minipage}{1.5in}
          \centering\vskip 10pt
          \begin{tabular}{|r|c|c|c|}
            \hline
            {\tt myVec:} & 8 & 2 & 6 \\
            \hline
          \end{tabular}
          \vskip 5pt\null
        \end{minipage}
        &
        \cpp{myVec.push_back(4)}
        &
        \begin{minipage}{1.5in}
          \centering\vskip 10pt
          \begin{tabular}{|r|c|c|c|c|}
            \hline
            {\tt myVec:} & 8 & 2 & 6 & 4 \\
            \hline
          \end{tabular}
          \vskip 5pt\null
        \end{minipage}
      \\ \hline
        \begin{minipage}{1.5in}
          \centering\vskip 10pt
          \begin{tabular}{|r|c|c|c|}
            \hline
            {\tt myVec:} & 8 & 2 & 6 \\
            \hline
          \end{tabular}
          \vskip 5pt\null
        \end{minipage}
        &
        \cpp{myVec.pop_back()}
        &
        \begin{minipage}{1.5in}
          \centering\vskip 10pt
          \begin{tabular}{|r|c|c|}
            \hline
            {\tt myVec:} & 8 & 2 \\
            \hline
          \end{tabular}
          \vskip 5pt\null
        \end{minipage}
      \\ \hline
        \begin{minipage}{1.5in}
          \centering\vskip 10pt
          \begin{tabular}{|r|c|c|c|}
            \hline
            {\tt myVec:} & 8 & 2 & 6 \\
            \hline
          \end{tabular}
          \vskip 5pt\null
        \end{minipage}
        &
        \cpp{myVec.insert(myVec.begin()+1,9)}
        &
        \begin{minipage}{1.5in}
          \centering\vskip 10pt
          \begin{tabular}{|r|c|c|c|c|}
            \hline
            {\tt myVec:} & 8 & 9 & 2 & 6\\
            \hline
          \end{tabular}
          \vskip 5pt\null
        \end{minipage}
      \\ \hline
        \begin{minipage}{1.5in}
          \centering\vskip 10pt
          \begin{tabular}{|r|c|c|c|}
            \hline
            {\tt myVec:} & 8 & 2 & 6 \\
            \hline
          \end{tabular}
          \vskip 5pt\null
        \end{minipage}
        &
        \cpp{myVec.erase(myVec.end()-2)}
        &
        \begin{minipage}{1.5in}
          \centering\vskip 10pt
          \begin{tabular}{|r|c|c|}
            \hline
            {\tt myVec:} & 8 & 6 \\
            \hline
          \end{tabular}
          \vskip 5pt\null
        \end{minipage}
      \\ \hline
        \begin{minipage}{1.5in}
          \centering\vskip 10pt
          \begin{tabular}{|r|c|c|c|}
            \hline
            {\tt myVec:} & 8 & 2 & 6\\
            \hline
          \end{tabular}
          \vskip 5pt\null
        \end{minipage}
        &
        \cpp{myVec.clear()}
        &
        \begin{minipage}{1.5in}
          \centering\vskip 10pt
          \begin{tabular}{|r|c|}
            \hline
            {\tt myVec:} \\
            \hline
          \end{tabular}
          \vskip 5pt\null
        \end{minipage}
      \\ \hline      
    \end{tabular}
  \end{center}
      
  {\it\large Refer to Model 3 above as your team develops consensus answers
    to the questions below.}

  \quest{20 min}

  \Q In your own words, describe what each of the following
    vector operations does.
    \par\vskip 10pt
    \begin{enumerate}
      \item \cpp{myVec.resize(n)}
        \begin{answer}[0.5in]
          Changes the size of the vector to {\tt n}.
        \end{answer}

      \item \cpp{myVec.push_back(value)}
        \begin{answer}[0.5in]
          Adds a value to the end of the vector, resizing it if
          necessary.
        \end{answer}

      \item \cpp{myVec.pop_back()}
        \begin{answer}[0.5in]
          Removes a value from the end of the vector (and returns
          the value).
        \end{answer}

      \item \cpp{myVec.begin()}
        \begin{answer}[0.5in]
          Points to the beginning of the vector.
        \end{answer}

      \item \cpp{myVec.insert(location,value)}
        \begin{answer}[0.5in]
          Inserts a value at the given location, moving everything
          over to make room.
        \end{answer}

      \item \cpp{myVec.end()}
        \begin{answer}[0.5in]
          Identifies the end of the vector (one slot past the end
          actually).
        \end{answer}

      \item \cpp{myVec.erase(location)}
        \begin{answer}[0.5in]
          Removes a value from the given location, shifting later
          values left.
        \end{answer}

      \item \cpp{myVec.clear()}
        \begin{answer}[0.5in]
          Removes all values from the vector and sets its size to zero.
        \end{answer}
    \end{enumerate}
    
  \vskip -20pt
    
  \Q Suppose the vector {\tt myVec} is defined as shown below.
    Sketch a diagram of the contents of the vector after each set of
    vector operations.  Start over with the original {\tt myVec} for
    each part of this problem.  This code can be found in {\tt activity10c.cpp}.
    \begin{center}
      \begin{tabular}{|r|c|c|c|c|c|}
        \hline
        {\tt myVec:} & 25 & 12 & 73 & 19 & 42 \\
        \hline
      \end{tabular}
    \end{center}
    
    \begin{enumerate}
      \item First set of commands\par
        \begin{center}
          \begin{tabular}{p{3in}p{3in}}
            \begin{minipage}{3in}
              \small
              \begin{cpplst}
  myVec.pop_back();
  myVec.insert(myVec.begin()+2,55);
  myVec.erase(myVec.begin()+1);
                \end{cpplst}
              \end{minipage}
              &
              \begin{minipage}{3in}
                \begin{answer}
                  \begin{tabular}{|r|c|c|c|c|c|}
                    % \tiny
                    \hline
                    {\tt myVec:} & 25 & 55 & 73 & 19  \\
                    \hline
                  \end{tabular}
                \end{answer}
              \end{minipage}
            \end{tabular}
          \end{center}
        
        \item Second set of commands        
          \begin{center}
            \begin{tabular}{p{3in}p{3.0in}}
              \begin{minipage}{3in}
                \small
                \begin{cpplst}
  myVec.resize(7);
  myVec.at(5) = 31;
  myVec.insert(myVec.end()-3,90);
  myVec.push_back(27);
                \end{cpplst}
              \end{minipage}
              &
              \begin{minipage}{3.0in}
                \begin{answer}
                  \scriptsize
                  {\tt myVec:}
                  \begin{tabular}{|c|c|c|c|c|c|c|c|c|}
                    \hline
                     25 & 12 & 73 & 19 & 90 & 42 & 31 & 0 & 27  \\
                    \hline
                  \end{tabular}
                \end{answer}
              \end{minipage}
            \end{tabular}
          \end{center}
          
      \end{enumerate}
    
    \vskip 10pt
      
    \Q Suppose the vector {\tt myVec} is defined as shown below.
      Give a sequence of command\key\\[-2.5mm] that 
      would produce the given vectors below.
      \begin{center}
        \begin{tabular}{|r|c|c|c|c|c|}
          \hline
          {\tt myVec:} & x & A & c & Y & w \\
          \hline
        \end{tabular}
      \end{center}
      \par\vskip 20pt

      \begin{enumerate}
        \item \begin{tabular}{|r|c|c|c|c|c|c|}
            \hline
            {\tt myVec:} & A & h & c & n & w & V\\
            \hline
          \end{tabular}
          \begin{answer}[1.75in]
            Answers may vary, but the following would work.
            \begin{center}
              \begin{minipage}{3.5in}
                \small
                \begin{cpplst}
  myVec.erase(myVec.begin());
  myVec.erase(myVec.begin()+2);
  myVec.insert(myVec.begin()+1,'h');
  myVec.insert(myVec.begin()+3,'n');
  myVec.push_back('V');
                \end{cpplst}
              \end{minipage}
            \end{center}
          \end{answer}

        \item \begin{tabular}{|r|c|c|c|c|c|}
            \hline
            {\tt myVec:} & x & A & w & Z\\
            \hline
          \end{tabular}
          \begin{answer}[1.75in]
            Answers may vary, but the following would work.
            \begin{center}
              \begin{minipage}{3.5in}
                \small
                \begin{cpplst}
  myVec.erase(myVec.begin()+2);                
  myVec.erase(myVec.begin()+2);
  myVec.push_back('Z');
                \end{cpplst}
              \end{minipage}
            \end{center}
          \end{answer}

        \item \begin{tabular}{|r|c|c|c|c|c|c|c|c|}
            \hline
            {\tt myVec:} & F & x & c & Y & & w & & \\
            \hline
          \end{tabular}
          \begin{answer}[1.75in]
            Answers may vary, but the following would work.
            \begin{center}
              \begin{minipage}{3.5in}
                \small
                \begin{cpplst}
  myVec.insert(myVec.begin(),'F');
  myVec.erase(myVec.begin()+2);
  myVec.insert(myVec.end()-1,' ');
  myVec.push_back(' ');
  myVec.push_back(' ');
                \end{cpplst}
              \end{minipage}
            \end{center}
          \end{answer}
      \end{enumerate}