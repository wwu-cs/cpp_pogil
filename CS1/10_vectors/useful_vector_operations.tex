\model{Useful Vector Operations}
  
  \begin{center}
    \small
    \renewcommand{\arraystretch}{1.4}    
    \begin{tabular}{|c|c|c|}
      \hline
      \rowcolor{orange!20} Original {\tt myVec} & Operation & Resulting {\tt myVec}  \\
      \hline
        \begin{minipage}{1.5in}
          \centering\vskip 10pt
          \begin{tabular}{r|c|c|c|}
            \hhline{~|-|-|-|}
            {\tt myVec:} & 8 & 2 & 6 \\
            \hhline{~|-|-|-|}
          \end{tabular}
          \vskip 5pt\null
        \end{minipage}
        &
        \mintinline{cpp}|myVec.resize(4)|
        &
        \begin{minipage}{1.5in}
          \centering\vskip 10pt
          \begin{tabular}{r|c|c|c|c|c|}
            \hhline{~|-|-|-|-|-|}
            {\tt myVec:} & 8 & 2 & 6 & \\
            \hhline{~|-|-|-|-|-|}
          \end{tabular}
          \vskip 5pt\null
        \end{minipage}
      \\ \hline
        \begin{minipage}{1.5in}
          \centering\vskip 10pt
          \begin{tabular}{r|c|c|c|}
            \hhline{~|-|-|-|}
            {\tt myVec:} & 8 & 2 & 6 \\
            \hhline{~|-|-|-|}
          \end{tabular}
          \vskip 5pt\null
        \end{minipage}
        &
        \mintinline{cpp}{myVec.push_back(4)}
        &
        \begin{minipage}{1.5in}
          \centering\vskip 10pt
          \begin{tabular}{r|c|c|c|c|}
            \hhline{~|-|-|-|-|}
            {\tt myVec:} & 8 & 2 & 6 & 4 \\
            \hhline{~|-|-|-|-|}
          \end{tabular}
          \vskip 5pt\null
        \end{minipage}
      \\ \hline
        \begin{minipage}{1.5in}
          \centering\vskip 10pt
          \begin{tabular}{r|c|c|c|}
            \hhline{~|-|-|-|}
            {\tt myVec:} & 8 & 2 & 6 \\
            \hhline{~|-|-|-|}
          \end{tabular}
          \vskip 5pt\null
        \end{minipage}
        &
        \mintinline{cpp}{myVec.pop_back()}
        &
        \begin{minipage}{1.5in}
          \centering\vskip 10pt
          \begin{tabular}{r|c|c|}
            \hhline{~|-|-|}
            {\tt myVec:} & 8 & 2 \\
            \hhline{~|-|-|}
          \end{tabular}
          \vskip 5pt\null
        \end{minipage}
      \\ \hline
        \begin{minipage}{1.5in}
          \centering\vskip 10pt
          \begin{tabular}{r|c|c|c|}
            \hhline{~|-|-|-|}
            {\tt myVec:} & 8 & 2 & 6 \\
            \hhline{~|-|-|-|}
          \end{tabular}
          \vskip 5pt\null
        \end{minipage}
        &
        \mintinline{cpp}{myVec.insert(myVec.begin()+1,9)}
        &
        \begin{minipage}{1.5in}
          \centering\vskip 10pt
          \begin{tabular}{r|c|c|c|c|}
            \hhline{~|-|-|-|-|}
            {\tt myVec:} & 8 & 9 & 2 & 6\\
            \hhline{~|-|-|-|-|}
          \end{tabular}
          \vskip 5pt\null
        \end{minipage}
      \\ \hline
        \begin{minipage}{1.5in}
          \centering\vskip 10pt
          \begin{tabular}{r|c|c|c|}
            \hhline{~|-|-|-|}
            {\tt myVec:} & 8 & 2 & 6 \\
            \hhline{~|-|-|-|}
          \end{tabular}
          \vskip 5pt\null
        \end{minipage}
        &
        \mintinline{cpp}{myVec.erase(myVec.end()-2)}
        &
        \begin{minipage}{1.5in}
          \centering\vskip 10pt
          \begin{tabular}{r|c|c|}
            \hhline{~|-|-|}
            {\tt myVec:} & 8 & 6 \\
            \hhline{~|-|-|}
          \end{tabular}
          \vskip 5pt\null
        \end{minipage}
      \\ \hline
        \begin{minipage}{1.5in}
          \centering\vskip 10pt
          \begin{tabular}{r|c|c|c|}
            \hhline{~|-|-|-|}
            {\tt myVec:} & 8 & 2 & 6\\
            \hhline{~|-|-|-|}
          \end{tabular}
          \vskip 5pt\null
        \end{minipage}
        &
        \mintinline{cpp}{myVec.clear()}
        &
        \begin{minipage}{1.5in}
          \centering\vskip 10pt
          \begin{tabular}{r|}
            \hhline{~|}
            {\tt myVec:} \\
            \hhline{~|}
          \end{tabular}
          \vskip 5pt\null
        \end{minipage}
      \\ \hline      
    \end{tabular}
  \end{center}
      
  {\it\large Refer to Model 3 above as your team develops consensus answers
    to the questions below.}

\newpage

    \item In your own words, describe what each of the following
      vector operations does.
      \par\vskip 10pt
      
      \begin{enumerate}[(a)]
        \item \mintinline{cpp}|myVec.resize(n)|\ifprintanswers\vskip -8pt\fi
          \begin{solution}[0.5in]
            Changes the size of the vector to {\tt n}.
          \end{solution}
        \item \mintinline{cpp}|myVec.push_back(value)|\ifprintanswers\vskip -8pt\fi
          \begin{solution}[0.5in]
            Adds a value to the end of the vector, resizing it if
            necessary.
          \end{solution}
        \item \mintinline{cpp}|myVec.pop_back()|\ifprintanswers\vskip -8pt\fi
          \begin{solution}[0.5in]
            Removes a value from the end of the vector (and returns
            the value).
          \end{solution}
        \item \mintinline{cpp}|myVec.begin()|\ifprintanswers\vskip -8pt\fi
          \begin{solution}[0.5in]
            Points to the beginning of the vector.
          \end{solution}
        \item \mintinline{cpp}|myVec.insert(location,value)|\ifprintanswers\vskip -8pt\fi
          \begin{solution}[0.5in]
            Inserts a value at the given location, moving everything
            over to make room.
          \end{solution}
        \item \mintinline{cpp}|myVec.end()|\ifprintanswers\vskip -8pt\fi
          \begin{solution}[0.5in]
            Identifies the end of the vector (one slot past the end
            actually).
          \end{solution}
        \item \mintinline{cpp}|myVec.erase(location)|\ifprintanswers\vskip -8pt\fi
          \begin{solution}[0.5in]
            Removes a value from the given location, shifting later
            values left.
          \end{solution}
        \item \mintinline{cpp}|myVec.clear()|\ifprintanswers\vskip -8pt\fi
          \begin{solution}[0.5in]
            Removes all values from the vector and sets its size to zero.
          \end{solution}
      \end{enumerate}
      
      
    \item Suppose the vector {\tt myVec} is defined as shown below.
      Sketch a diagram of the contents of the vector after each set of
      vector operations.  Start over with the original {\tt myVec} for
      each part of this problem.  This code can be found in {\tt activity10c.cpp}.
      \begin{center}
        \begin{tabular}{r|c|c|c|c|c|}
          \hhline{~|-|-|-|-|-|}
          {\tt myVec:} & 25 & 12 & 73 & 19 & 42 \\
          \hhline{~|-|-|-|-|-|}
        \end{tabular}
      \end{center}
      
      \begin{enumerate}[(a)]
        \item First set of commands\par
          \begin{center}
            \begin{tabular}{p{2.5in}p{3.0in}}
              \begin{minipage}{2.5in}
                \small
                \begin{minted}[
                  frame=lines,
                  framesep=2mm,
                  bgcolor=gray!15,
                  baselinestretch=1.2,
                  linenos,
                  firstnumber=10
                ]{cpp}
  myVec.pop_back();
  myVec.insert(myVec.begin()+2,55);
  myVec.erase(myVec.begin()+1);
                \end{minted}
              \end{minipage}
              &
              \begin{minipage}{2.5in}
                \begin{solution}
                  \begin{tabular}{r|c|c|c|c|c|}
                    \hhline{~|-|-|-|-|-|}
                    {\tt myVec:} & 25 & 55 & 73 & 19  \\
                    \hhline{~|-|-|-|-|-|}
                  \end{tabular}
                \end{solution}
              \end{minipage}
            \end{tabular}
          \end{center}
        
        \item Second set of commands        
          \begin{center}
            \begin{tabular}{p{2.5in}p{3.0in}}
              \begin{minipage}{2.5in}
                \small
                \begin{minted}[
                  frame=lines,
                  framesep=2mm,
                  bgcolor=gray!15,
                  baselinestretch=1.2,
                  linenos,
                  firstnumber=15
                ]{cpp}
  myVec.resize(7);
  myVec.at(5) = 31;
  myVec.insert(myVec.end()-3,90);
  myVec.push_back(27);
                \end{minted}
              \end{minipage}
              &
              \begin{minipage}{3.0in}
                \begin{solution}
                  {\tt myVec:}
                  \begin{tabular}{|c|c|c|c|c|c|c|c|c|}
                    \hhline{|-|-|-|-|-|-|-|-|-|}
                     25 & 12 & 73 & 19 & 90 & 42 & 31 & 0 & 27  \\
                    \hhline{|-|-|-|-|-|-|-|-|-|}
                  \end{tabular}
                \end{solution}
              \end{minipage}
            \end{tabular}
          \end{center}
          
      \end{enumerate}
      
\newpage      
      
    \item Suppose the vector {\tt myVec} is defined as shown below.
      Give a sequence of command that \key\\[-2.5mm] 
      would produce the given vectors below.
      \begin{center}
        \begin{tabular}{r|c|c|c|c|c|}
          \hhline{~|-|-|-|-|-|}
          {\tt myVec:} & x & A & c & Y & w \\
          \hhline{~|-|-|-|-|-|}
        \end{tabular}
      \end{center}
      \par\vskip 20pt

      \begin{enumerate}[(a)]
        \item \begin{tabular}{r|c|c|c|c|c|c|}
            \hhline{~|-|-|-|-|-|-|}
            {\tt myVec:} & A & h & c & n & w & V\\
            \hhline{~|-|-|-|-|-|-|}
          \end{tabular}
          \begin{solution}[1.75in]
            Answers may vary, but the following would work.
            \begin{center}
              \begin{minipage}{3.5in}
                \small
                \begin{minted}[
                  frame=lines,
                  framesep=2mm,
                  bgcolor=gray!15,
                  baselinestretch=1.2,
                ]{cpp}
  myVec.erase(myVec.begin());
  myVec.erase(myVec.begin()+2);
  myVec.insert(myVec.begin()+1,'h');
  myVec.insert(myVec.begin()+3,'n');
  myVec.push_back('V');
                \end{minted}
              \end{minipage}
            \end{center}
          \end{solution}

        \item \begin{tabular}{r|c|c|c|c|c|}
            \hhline{~|-|-|-|-|-|}
            {\tt myVec:} & x & A & w & Z\\
            \hhline{~|-|-|-|-|-|}
          \end{tabular}
          \begin{solution}[1.75in]
            Answers may vary, but the following would work.
            \begin{center}
              \begin{minipage}{3.5in}
                \small
                \begin{minted}[
                  frame=lines,
                  framesep=2mm,
                  bgcolor=gray!15,
                  baselinestretch=1.2,
                ]{cpp}
  myVec.erase(myVec.begin()+2);                
  myVec.erase(myVec.begin()+2);
  myVec.push_back('Z');
                \end{minted}
              \end{minipage}
            \end{center}
          \end{solution}

        \item \begin{tabular}{r|c|c|c|c|c|c|c|c|}
            \hhline{~|-|-|-|-|-|-|-|-|}
            {\tt myVec:} & F & x & c & Y & & w & & \\
            \hhline{~|-|-|-|-|-|-|-|-|}
          \end{tabular}
          \begin{solution}[1.75in]
            Answers may vary, but the following would work.
            \begin{center}
              \begin{minipage}{3.5in}
                \small
                \begin{minted}[
                  frame=lines,
                  framesep=2mm,
                  bgcolor=gray!15,
                  baselinestretch=1.2,
                ]{cpp}
  myVec.insert(myVec.begin(),'F');
  myVec.erase(myVec.begin()+2);
  myVec.insert(myVec.end()-1,' ');
  myVec.push_back(' ');
  myVec.push_back(' ');
                \end{minted}
              \end{minipage}
            \end{center}
          \end{solution}

      \end{enumerate}