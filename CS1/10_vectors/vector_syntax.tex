\model{ Vector Syntax}\\
  \begin{center}
    \begin{minipage}{2.5in}
      \scriptsize
      \begin{cpplst}
vector<int> quizScores = {8,6};
vector<string> profNames(4);

profNames.at(0) = "Carman";
profNames.at(1) = "Foster";
profNames.at(2) = "Duncan";

cout << quizScores.size() << endl;
cout << profNames.at(1) << endl;
      \end{cpplst}
    \end{minipage}
    &
    \begin{minipage}{2.5in}
      \begin{tabular}{|c|c|c|}
        \hline
        {\tt quizScores:}  & 8 & 6 \\
        \hline
        index & 0 & 1 \\
        \hline
      \end{tabular}
      \begin{tabular}{|c|c|c|c|c|}
        \hline
        {\tt profNames:} & Carman & Foster & Duncan & \\
        \hline
        index & 0 & 1 & 2 & 3 \\
        \hline
      \end{tabular}
      \vskip 10pt
      {\bf Output:} 
      \hrule\vskip 5pt
      2\\
      Foster
    \end{minipage}
  \end{center}
  
  {\it\large Refer to Model 1 above as your team develops consensus answers
    to the questions below.}

  \quest{15 min}
  
  \Q A {\it vector} is an ordered list of related variables, all of the
    same type.  Each value in a vector is known as an {\it element}.  The code
    above can be found in {\tt activity10a.cpp}.  Answer the following questions 
    about the vectors declared on line 1.
    \begin{enumerate}
      \itemsep 10pt
      \item What is the name of this vector? \hfill
        \ans[2.5in]{\tt quizScores}

      \item What type of variable does this vector hold? \hfill
        \ans[2.5in]{It is a vector of  \cpp{int}s}
    \end{enumerate}
    
  \Q A second vector is declared on line 2 in the model.  Answer
    the following questions about it.
    \begin{enumerate}
      \itemsep 10pt
      \item What is the name of this vector? \hfill
        \ans[2.5in]{}\tt profNames}

      \item What type of variable does this vector hold? \hfill
        \ans[2.5in]{It is a vector of  \cpp{string}s}
    \end{enumerate}
  
  \Q The {\it size} of a vector is the number of variables that
    it holds.  What is the size of each vectors above? 
    \begin{answer}[0.5in]
      The vector declared on line 1 has size 2 and the vector
      declared on line 2 has size 4.
    \end{answer}
    
  \Q Based on the code above, what are two different ways to
    initialize the variables stored in a vector?
    \begin{answer}[0.5in]
      % \par
      You can set the values when the vector is declared
      using the vector<type> name = value_list
      syntax, or you can set each value using the
      vectorName.at(index) = value command.
    \end{answer}

    \newpage

  \Q The {\it index} of an element in a vector is its position in
    the vector.  Indexes start at 0 in C++.  Answer the following
    questions related to indexes.
    \begin{enumerate}
      \itemsep 10pt
      \item What is stored at index 0 in the vector {\tt quizScores}?\hfill
        \ans[1in]{8}

      \item At what index is the name ``Duncan'' stored in the vector {\tt profNames}?\hfill
        \ans[1in]{Index 2}

      \item What is the maximum allowed index for the vector {\tt profNames}?\hfill
        \ans[1in]{3}

      \item What is stored at index 2 in the vector {\tt quizScores}?\hfill
        \ans[1in]{\small Out of Bounds}
    \end{enumerate}
              
  \Q Write a single line of C++ code to declare a new vector of
    doubles named {\tt homeworkAvg} that contains the values 
    82.4 at index 2, 91.6 and index 0, and 73.9 at index 1.
    \begin{answer}[0.5in]
      \cpp{vector<double> homeworkAgv = {91.6, 73.9, 82.4};}
    \end{answer}
    
  \Q The expression \cpp{myVector.at(i)} returns the value of the
    variable at index {\tt i} in the vector {\tt myVector}.  It can
    also be used to set the value at that index, as seen in lines 9-11
    of the model.
    
    \begin{enumerate}
      \item Write a single C++ expression for the average of the two values in {\tt quizScores}.
        \begin{answer}[0.75in]
          \cpp{( quizScores.at(0) + quizScores.at(1) ) / 2.0}
        \end{answer}

      \item Write a sequence of C++ commands that changes the contents
        of the vector {\tt profNames} to those shown below without
        using the string literals \cpp{"Foster"} or
        \cpp{"Carman"}.
        \begin{center}
          \begin{tabular}{|c|c|c|c|c|}
            \hline
            {\tt profNames:} & Aamodt & Carman & Duncan & Foster \\
            \hline
            Index & 0 & 1 & 2 & 3 \\
            \hline
          \end{tabular}
        \end{center}
        \begin{answer}[1in]
          \scriptsize
          \begin{center}
            \begin{minipage}{2.25in}
              \begin{cpplst}
profNames.at(3) = profNames.at(1);
profNames.at(1) = profNames.at(0);
profNames.at(0) = "Aamodt";
                \end{cpplst}
              \end{minipage}
            \end{center}
          \end{answer}          
      \end{enumerate}

  \newpage
      
  \Q The expression {\tt myVector.size()} returns the size of the
    vector {\tt myVector}.  Use this function to write a 
    {\tt cout} command for each line of output below.
    
    \begin{enumerate}
      \item {\tt There are 4 CS professors at WWU.}
        \begin{answer}[0.5in]
          \par
          \cpp{cout << "There are " << profNames.size() << " CS professors at WWU." << endl;}            
        \end{answer}
      \item {\tt Total number of quiz points: 20} \hspace{10pt} (assume 10 points per quiz)
        \begin{answer}[0.5in]
          \par
          \cpp{cout << "Total number of quiz points: " << 10*quizScores.size() << endl;}
        \end{answer}
    \end{enumerate}

  \vskip -30pt
    
  \Q True/False:  {\tt myVector.at(myVector.size())}
  returns the last value in the vector.
  \vskip 5pt
  \begin{center}
    \ans[0.3in]{} True
    \ans[0.3in]{\checkmark} False
  \end{center}
