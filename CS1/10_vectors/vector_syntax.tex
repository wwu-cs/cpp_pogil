\model{ Vector Syntax}\\

  \begin{center}
    \begin{tabular}{p{2.5in}p{3.1in}}
      \begin{minipage}{2.5in}
        \scriptsize
        \begin{cprlst}[
          frame=lines,
          framesep=2mm,
          bgcolor=gray!15,
          baselinestretch=1.2,
          linenos,
          firstnumber=8
        ]{cpp}
  vector<int> quizScores = {8,6};
  vector<string> profNames(4);
  
  profNames.at(0) = "Carman";
  profNames.at(1) = "Foster";
  profNames.at(2) = "Duncan";
  
  cout << quizScores.size() << endl;
  cout << profNames.at(1) << endl;
        \end{cprlst}
      \end{minipage}
      &
      \begin{minipage}{3.1in}
        \null\par\vskip 50pt
        \begin{tabular}{p{0.8in}|c|c|c|}
          \hhline{~|-|-|-|}
          {\tt quizScores:} & 8 & 6 \\
          \hhline{~|-|-|-|}
          \multicolumn{1}{c}{} & \multicolumn{1}{c}{0} & \multicolumn{1}{c}{1} \\
        \end{tabular}
        \par\vskip 5pt
        \begin{tabular}{p{0.8in}|c|c|c|c|}
          \hhline{~|-|-|-|-|}
          {\tt profNames:} & Carman & Foster & Duncan & \\
          \hhline{~|-|-|-|-|}
          \multicolumn{1}{c}{} & \multicolumn{1}{c}{0} & \multicolumn{1}{c}{1} & \multicolumn{1}{c}{2} & \multicolumn{1}{c}{3} \\
        \end{tabular}
      \end{minipage}
    \end{tabular}
  \end{center}
  \TPMargin{5pt}
  \begin{textblock*}{1.5in}[0,0](2.6in,-1.5in)
    \textblockcolor{white}
    \begin{minipage}{1.35in}
      \scriptsize
      {\bf Output:} 
      \hrule\vskip 5pt
      2\\
      Foster
    \end{minipage}
  \end{textblock*}
  
  
  {\it\large Refer to Model 1 above as your team develops consensus answers
    to the questions below.}
    \par\vskip 10pt
    

    \Q A {\it vector} is an ordered list of related variables, all of the
      same type.  Each value in a vector is known as an {\it element}.  The code
      above can be found in {\tt activity10a.cpp}.  Answer the following questions 
      about the vectors declared on line 8.
      \par\vskip 20pt
      
      \begin{enumerate}[(a)]
        \itemsep 15pt
        \item What is the name of this vector? \hfill
          \ans[2.5in]{\tt quizScores}
        \item What type of variable does this vector hold? \hfill
          \ans[2.5in]{It is a vector of  \cpp{int}s}
      \end{enumerate}
      
    \Q A second vector is declared on line 9 in the model.  Answer
      the following questions about it.
      \par\vskip 20pt
      
      \begin{enumerate}[(a)]
        \itemsep 15pt
        \item What is the name of this vector? \hfill
          \ans[2.5in][\tt profNames]
        \item What type of variable does this vector hold? \hfill
          \ans[2.5in]{It is a vector of  \cpp{string}s}
      \end{enumerate}
    
    \Q The {\it size} of a vector is the number of variables that
      it holds.  What is the size of each vectors above? 
      \begin{answer}[0.5in]
        The vector declared on line 1 has size 2 and the vector
        declared on line 2 has size 4.
      \end{answer}
      
    \Q Based on the code above, what are two different ways to
      initialize the variables stored in a vector?
      \begin{answer}[1in]
        \par
        You can set the values when the vector is declared
        using the \cpp{vector<type> name = { value_list }}
        syntax, or you can set each value using the
        \cpp{vectorName.at(index) = value} command.
      \end{answer}
      
      
\newpage

    \Q The {\it index} of an element in a vector is its position in
      the vector.  Indexes start at 0 in C++.  Answer the following
      questions related to indexes.
      \par\vskip 20pt
      
      \begin{enumerate}[(a)]
        \itemsep 10pt
        \item What is stored at index 0 in the vector {\tt quizScores}?\hfill
          \ans[1in]{8}
        \item At what index is the name ``Duncan'' stored in the vector {\tt profNames}?\hfill
          \ans[1in]{Index 2}
        \item What is the maximum allowed index for the vector {\tt profNames}?\hfill
          \ans[1in]{3}
        \item What is stored at index 2 in the vector {\tt quizScores}?\hfill
          \ans[1in]{\small Out of Bounds}
      \end{enumerate}
                
    \Q Write a single line of C++ code to declare a new vector of
      doubles named {\tt homeworkAvg} that contains the values 
      82.4 at index 2, 91.6 and index 0, and 73.9 at index 1.
      \begin{answer}[0.5in]
        \cpp{vector<double> homeworkAgv = {91.6, 73.9, 82.4};}
      \end{answer}
      
    \Q The expression \cpp{myVector.at(i)} returns the value of the
      variable at index {\tt i} in the vector {\tt myVector}.  It can
      also be used to set the value at that index, as seen in lines 4-6
      of the model.
      
      \begin{enumerate}[(a)]
        \item Write a single C++ expression for the average of the two values in {\tt quizScores}.
          \begin{answer}[0.75in]
            \cpp{( quizScores.at(0) + quizScores.at(1) ) / 2.0}
          \end{answer}
        \item Write a sequence of C++ commands that changes the contents
          of the vector {\tt profNames} to those shown below without
          using the string literals \cpp{"Foster"} or
          \cpp{"Carman"}.
          \begin{center}
            \begin{tabular}{p{0.8in}|c|c|c|c|}
              \hhline{~|-|-|-|-|}
              {\tt profNames:} & Aamodt & Carman & Duncan & Foster \\
              \hhline{~|-|-|-|-|}
              \multicolumn{1}{c}{} & \multicolumn{1}{c}{0} & \multicolumn{1}{c}{1} & \multicolumn{1}{c}{2} & \multicolumn{1}{c}{3} \\
            \end{tabular}
          \end{center}
          \begin{answer}[1in]
            \scriptsize\vskip -35pt\null
            \begin{center}
              \begin{minipage}{2.25in}
                \begin{cprlst}[
                  frame=lines,
                  framesep=2mm,
                  bgcolor=gray!15,
                  baselinestretch=1.2,
                ]{cpp}
profNames.at(3) = profNames.at(1);
profNames.at(1) = profNames.at(0);
profNames.at(0) = "Aamodt";
                \end{cprlst}
              \end{minipage}
            \end{center}\vskip -20pt\null
          \end{answer}          
      \end{enumerate}
      
    \Q The expression {\tt myVector.size()} returns the size of the
      vector {\tt myVector}.  Use this function to write a 
      {\tt cout} command for each line of output below.
      
      \begin{enumerate}[(a)]
        \item {\tt There are 4 CS professors at WWU.}
          \begin{answer}[0.5in]
            \par
            \cpp{cout << "There are " << profNames.size() << " CS professors at WWU." << endl;}            
          \end{answer}
        \item {\tt Total number of quiz points: 20} \hspace{10pt} (assume 10 points per quiz)
          \begin{answer}[0.5in]
            \par
            \cpp{cout << "Total number of quiz points: " << 10*quizScores.size() << endl;}
          \end{answer}
      \end{enumerate}
      \par\vskip -40pt\null
      
    \Q True/False:  {\tt myVector.at(myVector.size())}
    returns the last value in the vector.%\ifprintanswers\hspace{5pt}(false)\fi\key
