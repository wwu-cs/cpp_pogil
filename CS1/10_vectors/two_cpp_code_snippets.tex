\model{Two C++ Code Snippets}
  \begin{center}
    \begin{minipage}{3.2in}
      \small
      \begin{cpplst}
// put first 100 powers of two in x
vector<double> x(100);
for (int i = 0; i < x.size(); i++) {
  x.at(i) = pow(2,i);
}
      \end{cpplst}
    \end{minipage}
    & &
    \begin{minipage}{3.5in}
      \small
      \begin{cpplst}
// sum the elements of y
int sum = 0;
for (int i = 0; i < y.size(); i++) {
  sum += y.at(i);
}
      \end{cpplst}
    \end{minipage}      
  \end{center}

  {\it\large Refer to Model 2 above as your team develops consensus answers
    to the questions below.}

  \quest{15 min}

  \Q One of the biggest advantages of a vector is the ability to process them 
    using loops.  That is, to perform the same task for multiple elements.  The file
    {\tt activity10b.cpp} contains the loops from the model above.
    \begin{enumerate}
      \item Find the value of each expression involving {\tt x} declared and initialized
        in the first code snippet.
        \begin{enumerate}
          \itemsep 10pt
          \begin{multicols}{2}
            \item {\tt x.at(1) + x.at(2)} \hspace{0.15in}
              \ans[1in]{6}

            \item {\tt x.at(2) * x.at(3)} \hspace{0.15in}
              \ans[1in]{32}

            \item {\tt x.at(4) / x.at(1)} \hfill
              \ans[1in]{8}

            \item {\tt x.at(x.at(2))} \hfill
              \ans[1in]{16}
          \end{multicols}
        \end{enumerate}

      \item How many values are saved in the vector {\tt y} in the second code snippet?  Does it matter?
        \begin{answer}[0.75in]
          We don't know, but it doesn't matter.  We can still sum them all.
        \end{answer}
    \end{enumerate}

  \newpage
    
  \Q A {\it code trace} is a method for hand simulating the
    execution of your code in order to\key\\[-2.5mm] manually verify that it
    works before you compile it.  Fill in the table to trace the code below and determine the value of the {\tt data} vector
    and {\tt accumulator} variable after the code has finished.
    \begin{center}
      \begin{tabular}{p{3.7in}p{3in}}
        \begin{minipage}{3.7in}
          \begin{cpplst}
vecor<int> data = {5,26,13,12,37,15,16,4,1,3};
int accumulator = 0;
for (int i = 0; i < data.size(); i++) {
  if (data.at(i) % 2 == 1 && 
      i + 1 < data.size()) {
    data.at(i) *= -1;
    accumulator += data.at(i+1);
  }
}            
          \end{cpplst}            
          {\tt data} =
          \ans[3in]{\tt \{-5,26,-13,12,-37,-15,16,4,-1,3\}}\vskip 5pt\\
          {\tt accumulator} = \ans[0.5in]{72}
        \end{minipage}
        &
        \begin{minipage}{3in}
          \begin{center}
            \renewcommand{\arraystretch}{1.6}\small
            \begin{tabular}{|c|c|c|}
              \hline
              \bf i & \bf\tt data.at(i) & \bf\tt accumulator \\
              \hline
              0 & \ans[0.5in]{-5} & \ans[0.5in]{26} \\
              \hline
              1 & \ans[0.5in]{26} & \ans[0.5in]{26} \\
              \hline
              2 & \ans[0.5in]{-13} & \ans[0.5in]{38} \\
              \hline
              3 & \ans[0.5in]{12} & \ans[0.5in]{38} \\
              \hline
              4 & \ans[0.5in]{-37} & \ans[0.5in]{53} \\
              \hline
              5 & \ans[0.5in]{-15} & \ans[0.5in]{69} \\
              \hline
              6 & \ans[0.5in]{16} & \ans[0.5in]{69} \\
              \hline
              7 & \ans[0.5in]{4} & \ans[0.5in]{69} \\
              \hline
              8 & \ans[0.5in]{-1} & \ans[0.5in]{72} \\
              \hline
              9 & \ans[0.5in]{3} & \ans[0.5in]{72} \\
              \hline
            \end{tabular}
          \end{center}
        \end{minipage}
      \end{tabular}
    \end{center}

  \Q Suppose the vector \cpp{vector<double> a} and
    \cpp{vector<double> b} have been declared and filled
    with elements.  Write code to find the pairwise maximum value in
    these vectors and place it in a vector named {\tt myMax} which
    you declare.  So, for example, \cpp{myMax.at(0)} should
    contain the larger of \cpp{a.at(0)} and
    \cpp{b.at(0)} and so on.
    \begin{answer}[1in]
      \tiny
      \begin{center}
        \begin{minipage}{2.5in}
          \begin{cpplst}
vector<double> myMax(x.size());
for (int i=0; i<a.size(); i++) {
  if (a.at(i) > b.at(i)) {
    myMax.at(i) = a.at(i);
  } else {
    myMax.at(i) = b.at(i);
  }
}
          \end{cpplst}
        \end{minipage}
      \end{center}
    \end{answer}

  \Q In a certain class 40\% of your final grade comes from your
    homework average and 60\% comes from your exam average.  Suppose 
    that vectors \cpp{vector<int> homeworkScores}
    and \\ \cpp{vector<int> examScores} have been defined
    and contain your individual homework and exam scores.  Write 
    C++ code to compute your class grade and store it in
    \cpp{double finalGrade}.  Don't forget to typecast
    if needed!
    \begin{answer}[1.2in]
      \tiny
      \begin{center}
        \begin{minipage}{4.5in}
          \begin{cpplst}
int homeworkSum = 0;
for (int i=0; i<homeworkScores.size(); i++) {
  homeworkSum += homeworkScores.at(i);
}
int examSum = 0;
for (int i=0; i<examScores.size(); i++) {
  examSum += examScores.at(i);
}
double homeworkGrade = static_cast<double>(homeworkSum) / homeworkScores.size();
double examGrade = static_cast<double>(examSum) / examScores.size();
double finalGrade = 0.40 * homeworkGrade + 0.60 * examGrade;
          \end{cpplst}
        \end{minipage}
      \end{center}
    \end{answer}