  {\bf\large Model 2: Two C++ Code Snippets} \\[-20pt]
  \begin{center}
    \begin{tabular}{p{2.7in}p{0.25in}p{2.7in}}
      \begin{minipage}{2.7in}
        \small
        \begin{minted}[
          frame=lines,
          framesep=2mm,
          bgcolor=gray!15,
          baselinestretch=1.2,
          linenos,
          firstnumber=8
        ]{cpp}
  // put first 100 powers of two in x
  vector<double> x(100);
  for (int i = 0; i < x.size(); i++) {
    x.at(i) = pow(2,i);
  }
        \end{minted}
      \end{minipage}
      & &
      \begin{minipage}{2.7in}
        \small
        \begin{minted}[
          frame=lines,
          framesep=2mm,
          bgcolor=gray!15,
          baselinestretch=1.2,
          linenos,firstnumber=17
        ]{cpp}
  // sum the elements of y
  int sum = 0;
  for (int i = 0; i < y.size(); i++) {
    sum += y.at(i);
  }
        \end{minted}
      \end{minipage}      
    \end{tabular}
  \end{center}
  \TPMargin{5pt}


  {\it\large Refer to Model 2 above as your team develops consensus answers
    to the questions below.}

    \item One of the biggest advantages of a vector is the ability to process them 
      using loops.  That is, to perform the same task for multiple elements.  The file
      {\tt activity10b.cpp} contains the loops from the model above.
      \par\vskip 15pt
      
      \begin{enumerate}[(a)]
        \item Find the value of each expression involving {\tt x} declared and initialized
          in the first code snippet.
          \par\vskip 20pt
          \begin{enumerate}[i.]
            \itemsep 15pt
            \begin{multicols}{2}
              \item {\tt x.at(1) + x.at(2)} \hspace{0.15in}
                \fillin[6][1in]
              \item {\tt x.at(2) * x.at(3)} \hspace{0.15in}
                \fillin[32][1in]
              \item {\tt x.at(4) / x.at(1)} \hfill
                \fillin[8][1in]
              \item {\tt x.at(x.at(2))} \hfill
                \fillin[16][1in]
            \end{multicols}
          \end{enumerate}
          \par\vskip 15pt
        \item How many values are saved in the vector {\tt y} in the second code snippet?  Does it matter?
          \begin{solution}[0.75in]
            We don't know, but it doesn't matter.  We can still sum them all.
          \end{solution}
      \end{enumerate}
      \par\vskip -30pt\null
      
    \item A {\it code trace} is a method for hand simulating the
      execution of your code in order to\key\\[-2.5mm] manually verify that it
      works before you compile it.  Fill in the table to trace the code below and determine the value of the {\tt data} vector
      and {\tt accumulator} variable after the code has finished.
      \par\vskip 20pt
      
      \begin{center}
        \begin{tabular}{p{2.5in}p{3in}}
          \begin{minipage}{2.5in}
            \scriptsize
            \begin{minted}[
              frame=lines,
              framesep=2mm,
              bgcolor=gray!15,
              baselinestretch=1.2,
              linenos
            ]{cpp}
vector<int> data = {5,26,13,12,37,15,16,4,1,3};
int accumulator = 0;
for (int i = 0; i < data.size(); i++) {
  if (data.at(i) % 2 == 1 && 
      i + 1 < data.size()) {
    data.at(i) *= -1;
    accumulator += data.at(i+1);
  }
}            
            \end{minted}            
            \par\vskip 20pt
            \begin{center}
              {\tt data} =
              \fillin[\tt \{-5,26,-13,12,-37,-15,16,4,-1,3\}][1.75in]\par\vskip 20pt
              {\tt accumulator} = \fillin[72][0.5in]
            \end{center}
          \end{minipage}
          &
          \begin{minipage}{3in}
            \begin{center}
              \renewcommand{\arraystretch}{1.6}\small
              \begin{tabular}{|c|c|c|}
                \hline
                \bf i & \bf\tt data.at(i) & \bf\tt accumulator \\
                \hline
                0 & \ifprintanswers -5\fi & \ifprintanswers 26\fi \\
                \hline
                1 & \ifprintanswers 26\fi & \ifprintanswers 26\fi \\
                \hline
                2 & \ifprintanswers -13\fi & \ifprintanswers 38\fi \\
                \hline
                3 & \ifprintanswers 12\fi & \ifprintanswers 38\fi \\
                \hline
                4 & \ifprintanswers -37\fi & \ifprintanswers 53\fi \\
                \hline
                5 & \ifprintanswers -15\fi & \ifprintanswers 69\fi \\
                \hline
                6 & \ifprintanswers 16\fi & \ifprintanswers 69\fi \\
                \hline
                7 & \ifprintanswers 4\fi & \ifprintanswers 69\fi \\
                \hline
                8 & \ifprintanswers -1\fi & \ifprintanswers 72\fi \\
                \hline
                9 & \ifprintanswers 3\fi & \ifprintanswers 72\fi \\
                \hline
              \end{tabular}
            \end{center}
          \end{minipage}
        \end{tabular}
      \end{center}

\newpage

    \item Suppose the vector \mintinline{cpp}|vector<double> a| and
      \mintinline{cpp}|vector<double> b| have been declared and filled
      with elements.  Write code to find the pairwise maximum value in
      these vectors and place it in a vector named {\tt myMax} which
      you declare.  So, for example, \mintinline{cpp}|myMax.at(0)| should
      contain the larger of \mintinline{cpp}|a.at(0)| and
      \mintinline{cpp}|b.at(0)| and so on.
      
      \begin{solution}[1.5in]
        \scriptsize\vskip -35pt\null
        \begin{center}
          \begin{minipage}{2.5in}
            \begin{minted}[
              frame=lines,
              framesep=2mm,
              bgcolor=gray!15,
              baselinestretch=1.2
            ]{cpp}
  vector<double> myMax(x.size());
  for (int i=0; i<a.size(); i++) {
    if (a.at(i) > b.at(i)) {
      myMax.at(i) = a.at(i);
    } else {
      myMax.at(i) = b.at(i);
    }
  }
            \end{minted}
          \end{minipage}
        \end{center}\vskip -20pt\null
      \end{solution}

    \item In a certain class 40\% of your final grade comes from your
      homework average and 60\% comes from your exam average.  Suppose 
      that vectors \mintinline{cpp}|vector<int> homeworkScores|
      and \mintinline{cpp}|vector<int> examScores| have been defined
      and contain your individual homework and exam scores.  Write 
      C++ code to compute your class grade and store it in
      \mintinline{cpp}|double finalGrade|.  Don't forget to typecast
      if needed!

      \begin{solution}[1.5in]
        \scriptsize\vskip -35pt\null
        \begin{center}
          \begin{minipage}{4.5in}
            \begin{minted}[
              frame=lines,
              framesep=2mm,
              bgcolor=gray!15,
              baselinestretch=1.2,
            ]{cpp}
  int homeworkSum = 0;
  for (int i=0; i<homeworkScores.size(); i++) {
    homeworkSum += homeworkScores.at(i);
  }
  int examSum = 0;
  for (int i=0; i<examScores.size(); i++) {
    examSum += examScores.at(i);
  }
  double homeworkGrade = static_cast<double>(homeworkSum) / homeworkScores.size();
  double examGrade = static_cast<double>(examSum) / examScores.size();
  double finalGrade = 0.40 * homeworkGrade + 0.60 * examGrade;
            \end{minted}
          \end{minipage}
        \end{center}\vskip -20pt\null
      \end{solution}
