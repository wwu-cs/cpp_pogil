\documentclass{exam}
%\documentclass[answers]{exam}
\hbadness=99999
\setlength{\textheight}{9.5in}
\setlength{\textwidth}{6.5in}
\setlength{\topmargin}{-0.75in}
\setlength{\oddsidemargin}{0in}
\setlength{\evensidemargin}{0in}

\usepackage{amsmath}
\usepackage{amssymb}
\usepackage{enumerate}
\usepackage[table]{xcolor}
\usepackage{hhline}
\usepackage{graphicx}
\usepackage{tikz}
%\usepackage{pgfplots}
\usepackage{multicol}
\usepackage{fancyvrb}

% for syntax highlighting
\usepackage{minted}
\usemintedstyle[cpp]{xcode}

% for overlay of output
\usepackage[overlay,showboxes]{textpos}

\pagestyle{plain}

\setlength\columnsep{50pt}
\newcommand{\key}{\hfill
      \raisebox{-.3\height}{\includegraphics[width=0.6in]{figures/key.png}}}

\begin{document}
  \thispagestyle{empty}
  \setlength{\parindent}{0pt}

  \begin{center}
    \Large Activity \#12: Value Returning Functions \\[5pt]
    \large Recorder's Report\\[20pt]
    \normalsize
    \begin{tabular}{lrp{0.1in}lr}
      Manager:  & \fillin[][2.0in] & & Presenter: & \fillin[][2.0in]\\[15pt]
      Recorder: & \fillin[][2.0in] & & Driver:    & \fillin[][2.0in]\\[15pt]
      Date:     & \fillin[][2.0in] & & Score:     & Satisfactory \hspace{10pt} /
      \hspace{10pt} Not Satisfactory
    \end{tabular}
  \end{center}
  \par\vskip 15pt
  
  Record your team's answers to the key questions (marked with
  \raisebox{-.3\height}{\includegraphics[width=0.5in]{figures/key.png}})
  below.
  \begin{enumerate}[(a)]
    \itemsep 1.75in
    \item Model 1, Question \#4
    \item Model 2, Question \#9
    \item Model 3, Question \#11
  \end{enumerate}

  \clearpage\pagenumbering{arabic} 
  
  \begin{center}
    \Large Activity \#12: Value-Returning Functions \\[5pt]
    \large Activity Guide\\[20pt]
  \end{center}

  \begin{center}
    \fbox{
      \begin{minipage}{5.5in}
        {\bf Learning Objectives:} Students will be able to:
        \begin{itemize}
          \item Content:\\[-20pt]
            \begin{itemize}
              \itemsep 0pt
              \item Explain the concept and purpose of a value-returning function
              \item Combine function calls with looping and branching statements
              \item Explain programs that use the same function multiple times
              \item Use tests for programs which include functions
            \end{itemize}
          \item Process\\[-20pt]
            \begin{itemize}
              \itemsep 0pt
              \item Write code that includes function definitions and function calls
              \item Write programs using functions together with looping and branching statements\\[-5pt]
            \end{itemize}
        \end{itemize}
      \end{minipage}
      }
  \end{center}
  \par\vskip 10pt
  

  {\bf\large Model 1: A C++ Program} \\[-15pt]
  \begin{center}
    \begin{minipage}{5.8in}
      \begin{minted}[
        frame=lines,
        framesep=2mm,
        bgcolor=gray!15,
        baselinestretch=1.2,
        linenos,
        firstnumber=4
      ]{cpp}
int getSmaller(int num1, int num2) {
  int smaller;
  if (num1 < num2) {
    smaller = num1;
  } else {
    smaller = num2;
  }
  return smaller;
}

int main() {
  int userNum1, userNum2;
  cout << "Enter two numbers separated by a space: ";
  cin >> userNum1 >> userNum2;
  int smallerNum = getSmaller(userNum1, userNum2);
  cout << "The smaller number is " << smallerNum << endl;
}
      \end{minted}
    \end{minipage}
  \end{center}
  \TPMargin{5pt}
  \begin{textblock*}{3.1in}[0,0](3.65in,-3.25in)
    \textblockcolor{white}
    \begin{minipage}{2.95in}
      {\bf Output:} 
      \hrule\vskip 5pt
      Enter two numbers separated by a space: 52 36\\
      The smaller number is 36
    \end{minipage}
  \end{textblock*}
  
  
  {\it\large Refer to Model 1 above as your team develops consensus answers
    to the questions below.}
    \par\vskip 10pt
    
  \begin{enumerate}
    \itemsep 20pt

    \item What does the function defined on lines 4-12 do?
      \ifprintanswers\vskip -10pt\fi
      \begin{solution}[0.25in]
        It returns the smaller of the two integer arguments passed to it.
      \end{solution}
      \ifprintanswers\vskip -35pt\null\fi
      
    \item So far you have created `void` functions which do not send any
      information back to the calling code. A {\it value-returning} function
      does send information of a particular type back to the calling code.
      \par\ifprintanswers\vskip 10pt\else\vskip 20pt\fi
      
      \begin{enumerate}[(a)]
        \itemsep 15pt
        \item What type of information does the function {\tt getSmaller} return? \hfill
          \fillin[An integer][1.8in]
        \item On what line of code is the function call to {\tt getSmaller}? \hfill
          \fillin[Line 15][1.8in]
        \item On what line does the function {\tt getSmaller} send back a value? \hfill
          \fillin[Line 8][1.8in]
      \end{enumerate}

\newpage      

    \item In a {\it void function} the {\it function call} is on a line by itself.
      Why is the function call in this model on the right-hand-side of an
      assignment statement?
      \begin{solution}[0.5in]
        Because it returns a value that is being assigned to the variable {\tt
        smallerNum}.
      \end{solution}      
      
    \item If you wanted to alter this function to select and return the
      smaller of two {\tt  double} values,\key\\[-2.5mm] what changes would you
      need to make?
      \begin{solution}[1in]
        You need to change the \mintinline{cpp}|int| type to a
        \mintinline{cpp}|double| in the following places.
        \begin{itemize}
          \begin{multicols}{2}
            \item In the three instances on line 1
            \item In the one instance on line 2
          \end{multicols}
        \end{itemize}
      \end{solution}
      
  
  {\bf\large Model 2: A C++ Function} \\[-20pt]
  \begin{center}
    \begin{minipage}{4.5in}
      \begin{minted}[
        frame=lines,
        framesep=2mm,
        bgcolor=gray!15,
        baselinestretch=1.2,
        linenos
      ]{cpp}
void getHypotenuse(double a, double b) {
  double square = pow(a,2) + pow(b,2);
  double squareRoot = sqrt(square);
  cout << "The hypotenuse length is " << squareRoot << endl;
}
      \end{minted}
    \end{minipage}
  \end{center}


  {\it\large Refer to Model 2 above as your team develops consensus answers
    to the questions below.}

    \item What does this function do?
      \ifprintanswers\vskip -20pt\null\fi
      \begin{solution}[0.35in]
        It solves for the length of the hypotenuse of a right triangle given
        the length of the two legs and prints it out.
      \end{solution}
      \ifprintanswers\vskip -35pt\null\fi
      
    \item Is this a {\it void function} or a {\it value-returning} function?
      \hfill\fillin[void][2in]

    \item Suppose you wanted to turn this into a value-returning function
      that returns the hypotenuse length.
      \par\vskip 15pt
    
      \begin{enumerate}[(a)]
        \itemsep 15pt
        \item How would you change line 1 in the model? \hfill
          \fillin[\mintinline{cpp}|double getHypotenuse(double a, double b)|][3in]
        \item What command would you add to the end? \hfill
          \fillin[\mintinline{cpp}|return squareRoot;|][3in]
        \item What line would you remove and why? \hfill
          \fillin[Line 4 -- function doesn't produce output][3in]
      \end{enumerate}
      
    \item What is the difference between using {\tt cout} in a function and using
      {\tt return} in a function?
      \begin{solution}[0.5in]
        Using {\tt cout} prints values out to the screen while using {\tt return}
        sends the value back to the caller, where it can decide what to do with it.
      \end{solution}
    
\newpage
    \null\vskip -60pt\null
    \item Below is the framework for a program that drills students on addition
      problems.\key\\[-2.5mm]  In particular, the program should do the following:
      \begin{itemize}
        \item Display five addition problems, one at a time, and let the student answer each
        \item Print the correct answer if the user enters an incorrect answer
        \item Print a congratulatory message if the student's answer is correct
        \item Keep track of the number of problems the student answers correctly
        \item Print a special message if the user gets all five problems right
      \end{itemize}
      Fill in the missing code to make this program function
      as described.  The code is in {\tt activity12b.cpp}.
      \vskip -30pt\null
    
      \begin{center}
        \begin{tabular}{p{2.75in}p{0.1in}p{2.8in}}
          \begin{minipage}{2.75in}
            \begin{minted}[
              frame=lines,
              framesep=2mm,
              bgcolor=gray!15,
              baselinestretch=1.2,
              linenos,
              firstnumber=5
            ]{cpp}
// Print celebratory rocket

/* ANSWER A */ printRocket() {
  cout << "Blast-off!" << endl;
  cout << "  ^"    << endl;
  cout << " /*\\"  << endl;
  cout << "/***\\" << endl;
  cout << "|***|"  << endl;
  cout << "|***|"  << endl;
  cout << "|#|#|"  << endl;
  /* ANSWER B */
}

// give problem and check answer
bool giveProblem(/* ANSWER C */) {
  int studentAns;
  int correctAns = num1+num2;
  cout << num1 << "+" << num2 << "=";
  cin >> studentAns;
  /* ANSWER D */
}
            \end{minted}
          \end{minipage}
          & &
          \begin{minipage}{2.8in}
            \begin{minted}[
              frame=lines,
              framesep=2mm,
              bgcolor=gray!15,
              baselinestretch=1.2,
              linenos,
              firstnumber=27
            ]{cpp}          
int main() {
  srand(time(0));  // seed random num
  int numCorrect = 0;
  for(/* ANSWER E */) {
    int num1 = rand() % 10 + 1;
    int num2 = rand() % 10 + 1;
    if (/* ANSWER F */) {
      cout << "Correct!" << endl;
      numCorrect++;
    } else {
      cout << "Incorrect! It is "
           << (num1 + num2) << endl;
    }
  }
  if( /* ANSWER G */ ) {
    /* ANSWER H */
  } else {
    cout << "You got " << numCorrect
         << " correct." << endl;
  }
}
            \end{minted}
          \end{minipage}
        \end{tabular}
      \end{center}
      
      \begin{enumerate}[A.]
        \begin{multicols}{2}
          \item Line 7: The type of {\tt printRocket}:\par
            \begin{minipage}{2.75in}
              \begin{solution}[0.55in]
                {\tt void}
              \end{solution}
            \end{minipage}
          \item Line 15: A {\tt return} statement:\par
            \begin{minipage}{2.75in}
              \begin{solution}[0.55in]
                {\tt return} or empty
              \end{solution}
            \end{minipage}
          \item Line 19: Parameters for {\tt giveProblem}:\par
            \begin{minipage}{2.75in}
              \begin{solution}[0.55in]
                \mintinline{cpp}|int num1, int num2|
              \end{solution}
            \end{minipage}
          \item Line 24: A {\tt return} statement:\par
            \begin{minipage}{2.75in}
              \begin{solution}[0.55in]
                \small
                \mintinline{cpp}|return studentAns==correctAns|
              \end{solution}
            \end{minipage}
          \item Line 30: Setup for {\tt for} loop:\par
            \begin{minipage}{2.75in}
              \begin{solution}[0.55in]
                \mintinline{cpp}|int i=0; i<5; i++|
              \end{solution}
            \end{minipage}
          \item Line 33: Function call in {\tt if} condition:\par
            \begin{minipage}{2.75in}
              \begin{solution}[0.55in]
                \mintinline{cpp}|giveProblem(num1,num2)|
              \end{solution}
            \end{minipage}
          \item Line 41: Condition for {\tt if} statement:\par
            \begin{minipage}{2.75in}
              \begin{solution}[0.55in]
                \mintinline{cpp}|numCorrect==5|
              \end{solution}
            \end{minipage}
          \item Line 42: Command if all 5 correct:\par
            \begin{minipage}{2.75in}
              \begin{solution}[0.55in]
                \mintinline{cpp}|printRocket();|
              \end{solution}
            \end{minipage}                      
        \end{multicols}
      \end{enumerate}

    
\newpage


  {\bf\large Model 3: Program Output and the {\tt main} Function} \\[-10pt]

  \begin{center}
    \begin{tabular}{p{3in}p{2.7in}}
      \begin{minipage}{3in}
        \small
        \begin{minted}[
          frame=lines,
          framesep=2mm,
          bgcolor=gray!15,
          baselinestretch=1.2,
          linenos,
          firstnumber=7
        ]{cpp}
int main() {
  int userNum,compNum;
  char tryAgain;
  srand(time(0));
  do {
    do {
      cout << "Enter a number from 1 to 5: ";
      cin >> userNum;
      if (userNum < 1 || userNum > 5) {
        cout << "Invalid Number!" << endl;
      }
    } while (userNum < 1 || userNum > 5);
    compNum = rand() % 5 + 1;
    cout << "Computer number: " 
         << compNum << endl;
    cout << "Your number: " 
         << userNum << endl;
    cout << getMessage(userNum,compNum);
    cout << "Try again (y/n)? ";
    cin >> tryAgain;    
  } while (tryAgain=='y');
}
        \end{minted}
      \end{minipage}
      &
      \fbox{\begin{minipage}{2.4in}
        \small
        {\bf Output:} 
        \hrule\vskip 5pt\tt
        Enter a number from 1 to 5: 7\\
        Invalid Number!\\
        Enter a number from 1 to 5: 3\\
        Computer number: 3\\
        Your number: 3\\
        Congratulations! You guessed it.\\
        Try again (y/n)? y\\
        \\
        Enter a number from 1 to 5: 3\\
        Computer number: 4\\
        Your number: 3\\
        I'm sorry, your number is smaller.\\
        Try again (y/n)? y\\
        \\    
        Enter a number from 1 to 5: 3\\
        Computer number: 1\\
        Your number: 3\\
        I'm sorry, your number is larger.\\
        Try again (y/n)? n\\
      \end{minipage}}
    \end{tabular}
  \end{center}
      
  {\it\large Refer to Model 3 above as your team develops consensus answers
    to the questions below.}

    \item There is one user-defined function that is missing from this program.
      \par\vskip 15pt
      
      \begin{enumerate}[(a)]
        \itemsep 15pt
        \item What is the name of the missing function? \hfill
          \fillin[\tt getMessage][3in]
        \item Is the function {\tt void} or value-returning? \hfill
          \fillin[value-returning][3in]
        \item What are the type(s) of its parameters?\hfill
          \fillin[It has two integer parameters][3in]
        \item Write the header for the function. \hfill
          \fillin[\mintinline{cpp}|string getMessage(int uNum, int cNum)|][3in]
      \end{enumerate}
      \par\vskip -40pt\null
      
    \item Does the missing function contain any {\tt cout} statements?
      Explain your reasoning.\key
      \ifprintanswers\vskip -20pt\null\fi
      \begin{solution}[0.5in]
        No.  It returns the value to print out, but the {\tt cout}
        happens in the {\tt main} function, not in the missing function.
      \end{solution}
      
    \item Suppose line 24 in the model was changed to just
      \mintinline{cpp}|getMessage(userNum,compNum);|.  How would this
      change your answers to questions 10 and 11 above?
      \ifprintanswers\vskip -20pt\null\fi
      \begin{solution}[0.5in]
        The function now produces the output and would have a {\tt
        void} return type.
      \end{solution}
           
\newpage

    \item Now write the function definition assuming that the function header
      is as shown below.
      \begin{center}
        \mintinline{cpp}|string getMessage(int userNum, int compNum)|
      \end{center}
      \begin{solution}[2.25in]
        \scriptsize\vskip -35pt\null
        \begin{center}
          \begin{minipage}{3.5in}
            \begin{minted}[
              frame=lines,
              framesep=2mm,
              bgcolor=gray!15,
              baselinestretch=1.2,
              linenos
            ]{cpp}
string getMessage(int userNum,int compNum) {
  string message;
  if (userNum < compNum) {
    message = "I'm sorry, your number is smaller.\n";
  } else if (userNum > compNum) {
    message = "I'm sorry, your number is larger.\n";
  } else {
    message = "Congratulations!  You picked the number.\n";
  }
  return message;
}
            \end{minted}
          \end{minipage}
        \end{center}\vskip -20pt\null
      \end{solution}          
  
    \item Write a different version of this same function assuming that
      line 24 in the model was changed to just \mintinline{cpp}|getMessage(userNum,compNum);| 
      and using the following function header.
      \begin{center}
        \mintinline{cpp}|void getMessage(int userNum, int compNum)|
      \end{center}
      \begin{solution}[2.25in]
        \scriptsize\vskip -35pt\null
        \begin{center}
          \begin{minipage}{3.5in}
            \begin{minted}[
              frame=lines,
              framesep=2mm,
              bgcolor=gray!15,
              baselinestretch=1.2,
              linenos
            ]{cpp}
void getMessage(int userNum,int compNum) {
  if (userNum < compNum) {
    cout << "I'm sorry, your number is smaller." << endl;
  } else if (userNum > compNum) {
    cout << "I'm sorry, your number is larger." << endl;
  } else {
    cout << "Congratulations!  You picked the number." << endl;
  }
}
            \end{minted}
          \end{minipage}
        \end{center}\vskip -20pt\null
      \end{solution}          

  \end{enumerate}  
    
\end{document}
