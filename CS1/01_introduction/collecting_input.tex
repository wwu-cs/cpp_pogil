  \model{Collecting Input}
  \begin{center}
    \begin{minipage}{3.5in}
      \begin{cprlst}[
        frame=lines,
        framesep=2mm,
        bgcolor=gray!15,
        baselinestretch=1.2,
        linenos
      ]{cpp}
#include <iostream>
#include <string>
using namespace std;

int main() {
  string name;
  cout << "What is your name? ";
  cin >> name;
  cout << "Your name is " << name;
}      
      \end{cprlst}
    \end{minipage}
  \end{center}
  \par\vskip 10pt
  
  {\it\large Refer to Model 3 above as your group develops consensus answers
    to the questions below.}
    \par\vskip 10pt

  \begin{enumerate}
    \itemsep 10pt
    \setcounter{enumi}{9}
    \item You will find this code in the file {\tt activity01c.cpp}.  Execute it and
      determine what the program does.
      \begin{answer}[1in]
        It prompts the user for their name, for example ``Jon'', and then prints out {\it
        Your name is Jon}.
      \end{answer}
    
    \Q The word {\it name} in this code identifies a {\it variable} (a name given to a
      memory location used to store data).  What happens to the data that the user of
      this program enters?
      \begin{answer}[0.5in]
      \end{answer}
      
    \Q In C++, each variable has a {\it type} which specifies what sort of data it can
      store.  What is the type of the variable in the program above?  What sort of data
      can it store?
      \begin{answer}[0.5in]
        It is a string.  That means it can store a string of arbitrary characters.
      \end{answer}

    \Q Suppose you wished to also store the users age.  Explain the errors that occur
      when you attempt to create the following {\it integer} (int) variables to this
      program.\par\vskip 15pt
      \begin{enumerate}[(a)]
        \itemsep 15pt
        \item \cpp{int age?;} \hfill 
          \ans[4.5in]{not a valid variable name (no questionmarks allowed)}
        \item \cpp{int your age;} \hfill 
          \ans[4.5in]{not a valid variable name (no spaces allowed)}
        \item \cpp{int 2age;} \hfill 
          \ans[4.5in]{not a valid variable name (can't start with digit)}
        \item \cpp{int int;} \hfill 
          \ans[4.5in]{not a valid variable name (`int' is a keyword)}
        \item \cpp{int the.age;} \hfill
          \ans[4.5in]{not a valid variable name (no periods allowed)}
      \end{enumerate}
      
\newpage      
      
    \Q The following are valid variable names for age.  Based on this list, and
      the errors you \key\\[-2.5mm] found above, write two rules that valid variable names must
      follow in C++.
      \begin{center}
        \tt
        age \hspace{10pt}
        age2 \hspace{10pt}
        myAge \hspace{10pt}
        the\_age 
      \end{center}
      \begin{answer}[1in]
        \begin{itemize}
          \item May contain only characters a-z, a-Z, 0-9, and \_
          \item Must start with a letter or \_
        \end{itemize}
      \end{answer}

    \Q Suppose you need a variable to store the cost of an item.  The following names
      are suggested.  Are they valid?  Are they good choices?
      \par\vskip 15pt
      \begin{enumerate}[(a)]
        \itemsep 15pt
        \item {\tt price} \hfill \ans[4.5in]{yes to both}
        \item {\tt priceoftheitem} \hfill \ans[4.5in]{valid, probably to long}
        \item {\tt x} \hfill \ans[4.5in]{valid but not meaningful}
        \item {\tt itemPrice} \hfill \ans[4.5in]{yes to both}
      \end{enumerate}
      \par\vskip 15pt
      
    \Q Modify the C++ program found in {\tt activity01c.cpp} so that it prompts
      the user for two integers and then prints out the sum of those two integers
      as shown in the example output below.  Use meaningful variable names and 
      comments in your code.
      
      \begin{textblock*}{3.5in}[-0.5,0](0in,0.25in)
        \textblockcolor{white}
        \begin{minipage}{3.3in}
          {\bf Output:} 
          \hrule\vskip 5pt
          Enter first number: 7\\
          Enter second number: 12\\
          7 + 12 = 19
        \end{minipage}
      \end{textblock*}
      
      \par\vskip 1in
      \begin{answer}
        Answers will vary.
      \end{answer}

  \end{enumerate}
  