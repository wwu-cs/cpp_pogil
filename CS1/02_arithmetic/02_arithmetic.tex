\documentclass{exam}
%\documentclass[answers]{exam}
\hbadness=99999
\setlength{\textheight}{9.5in}
\setlength{\textwidth}{6.5in}
\setlength{\topmargin}{-0.75in}
\setlength{\oddsidemargin}{0in}
\setlength{\evensidemargin}{0in}

\usepackage{amsmath}
%\usepackage{amsfonts}
\usepackage{amssymb}
\usepackage{enumerate}
\usepackage[table]{xcolor}
\usepackage{graphicx}
\usepackage{tikz}
%\usepackage{pgfplots}
\usepackage{multicol}

% for syntax highlighting
\usepackage{minted}
\usemintedstyle[cpp]{xcode}

% for overlay of output
\usepackage[overlay,showboxes]{textpos}

\pagestyle{plain}

\setlength\columnsep{50pt}
\newcommand{\key}{\hfill
      \raisebox{-.3\height}{\includegraphics[width=0.6in]{figures/key.png}}}

\begin{document}
  \thispagestyle{empty}
  \setlength{\parindent}{0pt}

  \begin{center}
    \Large Activity \#2: Arithmetic Operations and Assignment \\[5pt]
    \large Recorder's Report\\[20pt]
    \normalsize
    \begin{tabular}{lrp{0.1in}lr}
      Manager:  & \fillin[][2.0in] & & Presenter: & \fillin[][2.0in]\\[15pt]
      Recorder: & \fillin[][2.0in] & & Driver:    & \fillin[][2.0in]\\[15pt]
      Date:     & \fillin[][2.0in] & & Score:     & Satisfactory \hspace{10pt} /
      \hspace{10pt} Not Satisfactory
    \end{tabular}
  \end{center}
  \par\vskip 15pt
  
  Record your team's answers to the key questions (marked with
  \raisebox{-.3\height}{\includegraphics[width=0.5in]{figures/key.png}})
  below.
  \begin{enumerate}[(a)]
    \itemsep 1.75in
    \item Model 1, Question \#3
    \item Model 2, Question \#9
    \item Model 3, Question \#12
  \end{enumerate}

  \clearpage\pagenumbering{arabic} 
  
  \begin{center}
    \Large Activity \#2: Arithmetic Operations and Assignment \\[5pt]
    \large Activity Guide\\[20pt]
  \end{center}

  \begin{center}
    \fbox{
      \begin{minipage}{5.5in}
        {\bf Learning Objectives:} Students will be able to:
        \begin{itemize}
          \item Content:\\[-20pt]
            \begin{itemize}
              \itemsep 0pt
              \item Explain each how to print content to the screen using C++
              \item Explain the meaning and use of an assignment statement
              \item Explain the use of ``+'' with string variables
              \item Explain the difference between integer and floating point division
            \end{itemize}
          \item Process\\[-20pt]
            \begin{itemize}
              \itemsep 0pt
              \item Create input and output statements in C++
              \item Create C++ code that performs mathematical operations
              \item Create C++ code that uses assignment statements
              \item Create C++ code that formats numeric output\\[-5pt]
            \end{itemize}
        \end{itemize}
      \end{minipage}
      }
  \end{center}
  \par\vskip 10pt
  
  
  {\bf\large Model 1: A C++ Program} \\[-10pt]
  \begin{center}
    \begin{minipage}{3.5in}
      \begin{minted}[
        frame=lines,
        framesep=2mm,
        bgcolor=gray!15,
        baselinestretch=1.2,
        linenos
      ]{cpp}
#include <iostream>
using namespace std;

int main() {
  cout << 16 + 3 << endl;
  cout << 16 - 3 << endl;
  cout << 16 * 3 << endl;
  cout << 16 / 3 << endl;
  cout << 16 % 3 << endl;
}      
      \end{minted}
    \end{minipage}
  \end{center}
  
  {\it\large Refer to Model 1 above as your team develops consensus answers
    to the questions below.}
    \par\vskip 10pt
    
  \begin{enumerate}
    \itemsep 20pt
    
    \item Starting with the file {\tt activity02a.cpp}, enter and run the C++ program
      above.  What is the output of each of the following lines?\par\vskip 15pt
      \begin{enumerate}[(a)]
        \itemsep 10pt
        \item[5:] {\tt cout << 16 + 3 << endl;} \hfill
          \fillin[It prints {\it 19}][3.5in]
        \item[6:] {\tt cout << 16 - 3 << endl;} \hfill
          \fillin[It prints {\it 13}][3.5in]
        \item[7:] {\tt cout << 16 * 3 << endl;} \hfill
          \fillin[It prints {\it 48}][3.5in]
        \item[8:] {\tt cout << 16 / 3 << endl;} \hfill
          \fillin[It prints {\it 5}][3.5in]
        \item[9:] {\tt cout << 16 \% 3 << endl;} \hfill
          \fillin[It prints {\it 1}][3.5in]
      \end{enumerate}
      
    \item Were any of these a surprise to your team?
      \begin{solution}[0.25in]
      \end{solution}
      
\newpage      
      
    \item Name the arithmetic operation represented by each symbol in C++.\key\\[-2.5mm]
      \par\vskip 20pt
      \begin{enumerate}[(a)]
        \begin{multicols}{2}
          \itemsep 15pt
          \item {\tt +} \hfill \fillin[Addition][2.25in]
          \item {\tt -} \hfill \fillin[Subtraction][2.25in]
          \item {\tt *} \hfill \fillin[Multiplication][2.25in]
          \item {\tt /} \hfill \fillin[Integer Division][2.25in]
          \item {\tt \%} \hfill \fillin[Modulus (remainder)][2.25in]
        \end{multicols}
      \end{enumerate}

    \item An {\it assignment statement} uses the ``='' sign to store the result of an
      operation performed on the right-hand side into the memory location named by 
      the variable on the left-hand side.\par\vskip 5pt
      
      Enter and execute the following two lies of C++ code (modify the {\tt activity02a.cpp}
      file to do this).
      
      \begin{center}
        \begin{minipage}{3.5in}
          \begin{minted}[
            frame=lines,
            framesep=2mm,
            bgcolor=gray!15,
            baselinestretch=1.2,
            linenos,
            firstnumber=5
          ]{cpp}
  int age = 15;
  cout << "Your age is: " << age;
      \end{minted}
    \end{minipage}
  \end{center}

      \begin{enumerate}[(a)]
        \itemsep 10pt
        \item What does the {\it assignment statement} on line 1 of this code chunk do?
          \begin{solution}[0.5in]
            It declares an integer variable named ``age'' and assigns it the value of 15.
          \end{solution}
        \item Split up the {\it variable declaration} and {\it assignment statement} into 
          two separate lines of code.
          \begin{solution}[0.5in]
            The lines would be
            \begin{enumerate}[1:]
              \itemsep -2pt
              \item {\tt int age;}
              \item {\tt age = 15;}
            \end{enumerate}            
          \end{solution}
        \item What happens if you replace the assignment statement only (without altering the
          declaration) with \mintinline{cpp}|age = "fifteen";|?
          \begin{solution}[0.5in]
            You get a run-time error because you can't save a string into an integer variable.
          \end{solution}
      \end{enumerate}
      \ifprintanswers\par\vskip -24pt\null\fi
      
    \item Enter and execute the following three lines of C++ code (again, use the {\tt
      activity02a.cpp} file), but {\bf remember to add  \mintinline{cpp}|#include <string>| in
      the header}.
      \ifprintanswers\par\vskip -24pt\null\fi
      \begin{center}
        \begin{minipage}{3.5in}
          \begin{minted}[
            frame=lines,
            framesep=2mm,
            bgcolor=gray!15,
            baselinestretch=1.2,
            linenos,
            firstnumber=6
          ]{cpp}
  string schoolName = "Walla Walla";
  string schoolType = "University";
  string fullName = schoolName + schoolType;
  cout << fullName << endl;
      \end{minted}
    \end{minipage}
  \end{center}
  
  \begin{enumerate}[(a)]
  
    \item What value is stored in the {\tt fullName} variable at line 3?
      \begin{solution}[0.5in]
        ``Walla WallaUniversity''
      \end{solution}
      
\newpage

    \item How does the ``+'' sign behave differently when used with strings instead of numbers?
      \begin{solution}[0.5in]
        It concatenates the strings instead of adding values.
      \end{solution}           
      
    \item How could you fix the output so that the words are all separated?
      \begin{solution}[1in]
        For example, we could change line 3 to: \\
        {\tt string fullName = schoolName + " " + schoolType;}
      \end{solution}
      
    \item When two strings are ``glued'' together, they are said to be {\it concatenated}.
      What would happen if you tried to concatenate a string with an integer?
      \begin{solution}[1in]
        A syntax error.
      \end{solution}
      
  \end{enumerate}
  
  
  {\bf\large Model 2: Some Arithmetic Expressions in C++ Syntax} \\[-10pt]
    \begin{center}
      \renewcommand{\arraystretch}{1.5}
      \begin{tabular}{|c|c|c|}
        \hline
        \rowcolor{orange!20} Mathematical Expression & C++ Expression & Value \\
        \hline
        $3+2 \times 5$               & {\tt 3+2*5}     & 13 \\
        \hline
        $\frac{3+5}{2}$              & {\tt (3+5)/2}   & 4 \\
        \hline
        $6^2 + 3 \times \frac{4}{2}$ & {\tt 6*6+3*4/2} & 42 \\
        \hline
      \end{tabular}
    \end{center}

  {\it\large Refer to Model 2 above as your team develops consensus answers
    to the questions below.}
    \par\vskip 10pt
    
    \item If the parentheses were removed from the C++ Expression in the second row
      so that it was {\tt 3+5/2}, what would its value be?
      \begin{solution}[0.5in]
        The new value would be $3+\frac{5}{2} = 3 + 2 = 5$
      \end{solution}

    \item What is stored in memory after each assignment statement below is executed?  Assume
      all variables have been declared as integers.
    
      \begin{center}
        \renewcommand{\arraystretch}{1.5}
        \begin{tabular}{|c||l|c|}
          \hline
          \rowcolor{gray!10} Assignment Statement & \multicolumn{2}{c|}{Computer Memory} \\
          \hline
          \mintinline{cpp}|answer = 6*6+3*4/2;|   & {\tt answer} & \ifprintanswers 42\fi \\
          \hline
          \mintinline{cpp}|answer = answer + 1;|  & {\tt answer} & \ifprintanswers 43\fi \\
          \hline
          \mintinline{cpp}|final = answer % 4|    & {\tt final} & \ifprintanswers 3\fi \\
          \hline          
        \end{tabular}
      \end{center}
      
\newpage

    \item Convert each expression to a C++ arithmetic expression.  Do not compute
      its value.
      \par\vskip 15pt      
      \begin{enumerate}[(a)]
        \itemsep 20pt
        \item $\displaystyle 3(6-1)$             \hfill \fillin[{\tt 3*(6-1)}][4in]
        \item $\displaystyle 12-\frac{3+2}{6-1}$ \hfill \fillin[{\tt 12-(3+2)/(6-1)}][4in]\\[-10pt]
        \item $\displaystyle \frac{3+2^2}{5}$    \hfill \fillin[{\tt (3+2*2)/5}][4in]
        \item Eight to the fourth power          \hfill \fillin[{\tt 8*8*8*8}][4in]
      \end{enumerate}\vskip -35pt\ \

    \item What is the value of each C++ arithmetic expression below?\key\\[-2.5mm]
      \par\vskip 15pt      
      \begin{enumerate}[(a)]
        \itemsep 20pt
        \item {\tt 4 + 3 \% 2}          \hfill \fillin[{\tt 4 + 3 \% 2 = 4 + 1 = 5}][4in]
        \item {\tt 3 + 9 \% 2 * (-1+3)} \hfill \fillin[{\tt 3 + 1 * 2 = 3 + 2 = 5}][4in]\\[-10pt]
      \end{enumerate}
      \par\vskip 20pt
    

  {\bf\large Model 3: A C++ Division Program} \\[-10pt]
  \begin{center}
    \begin{minipage}{5.5in}
      \begin{minted}[
        frame=lines,
        framesep=2mm,
        bgcolor=gray!15,
        baselinestretch=1.2,
        linenos
      ]{cpp}
#include <iostream>
using namespace std;

int main() {
  int a;
  double b;
  cout << "First number: ";
  cin >> a;
  cout << "Second number: ";
  cin >> b;
  cout << "Quotient: " << a << "/" << b << " = " << (a/b) << endl;
}      
      \end{minted}
    \end{minipage}
  \end{center}
  \TPMargin{5pt}
  \begin{textblock*}{2.1in}[0,0](4.25in,-2.5in)
    \textblockcolor{white}
    \begin{minipage}{1.8in}
    {\bf Example Output:} 
    \hrule\vskip 5pt
      First number: 10 \\
      Second number: 2.5 \\
      Quotient: 10/2.5 = 4
    \end{minipage}
  \end{textblock*}
  \par\vskip 10pt
  
  {\it\large Refer to Model 3 above as your group develops consensus answers
    to the questions below.}
    \par\vskip 10pt
    
      \item This program can be found in {\tt activity02c.cpp}.  Run it several times with the
        following inputs and write down the resulting output.\par\vskip 10pt
        \begin{enumerate}[(a)]
          \itemsep 10pt
          \item First number: 10, Second number: 4\hfill
            \fillin[Quotient: 10/4 = 2.5][3.5in]
          \item First number: 10, Second number: 3 \hfill
            \fillin[Quotient: 10/3 = 3.33333][3.5in]
          \item First number: 9.5, No second number \hfill 
            \fillin[Quotient: 9/0.5 = 18][3.5in]
          \item First number: 10, Second number: x \hfill
            \fillin[Quotient: 10/0 = inf][3.5in]
        \end{enumerate}
        
      \item A {\it floating point} number is a number that contains a floating decimal point
        (as opposed to integers, which have no decimals). For example, 5.5 or 0.001.  In C++
        we often use the {\tt double} variable type to hold floating point values.  
        
        \begin{enumerate}
          \item How did the division operation change (as compared to earlier models) because one of 
            the variables was of type \mintinline{cpp}|double|?
            \begin{solution}[1in]
              It now does decimal division, returning the decimal number of times the divisor
              goes into the dividend instead of just the integer number of times.
            \end{solution}
            
          \item How would your results in problem 10 differ if both variables were of type \mintinline{cpp}|int|?
            \begin{solution}[1in]
              If both variables were of type \mintinline{cpp}|int| then we would get:
              \begin{enumerate}[(a)]
                \begin{multicols}{4}
                  \item 2
                  \item 3
                  \item error
                  \item error
                \end{multicols}
              \end{enumerate}             
            \end{solution}
            
          \item How would your results in problem 10 differ if both variables were of type  \mintinline{cpp}|double|?
            \begin{solution}[1in]
              \begin{enumerate}[(a)]
                \begin{multicols}{4}
                  \item same
                  \item same
                  \item need 2nd \#
                  \item same
                \end{multicols}
              \end{enumerate}
            \end{solution}
            
        \end{enumerate}
        
      \item Write a conjecture as to when the division operation in C++ will return decimals and\key\\[-2.5mm]
        when it will only return an integer.  Use the phrase {\it floating point} in your answer.
        \begin{solution}[0.5in]
          It will return decimals if one or both of the values involved is a floating point
          number.  Otherwise, it returns an integer.
        \end{solution}
        
      \item Create a program that asks asks for a number of cookies and a number of children
        and then prints out how many cookies each child will get (assuming they are split
        evenly) and how many are left over.  Be creative and professional in prompting the
        user and displaying the results.
        \begin{solution}
          Answers will vary.
        \end{solution}        
        
  \end{enumerate}  
    
\end{document}
