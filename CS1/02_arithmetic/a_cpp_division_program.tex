\model{C++ Division Program}
  \begin{minipage}[t]{0.5\textwidth}
    \begin{cpplst}
#include <iostream>
using namespace std;

int main() {
  int a;
  double b;
  cout << "First number: ";
  cin >> a;
  cout << "Second number: ";
  cin >> b;
  cout << "Quotient: " << a << "/" << b << " = " << (a/b) << endl;
}      
      \end{cpplst}
    \end{minipage}
    \hfill
    \begin{minipage}[t]{0.4\textwidth}
      {\bf Example Output:} 
      \hrule\vskip 5pt
        First number: 10 \\
        Second number: 2.5 \\
        Quotient: 10/2.5 = 4
    \end{minipage}
  \par\vskip -5pt
  
  {\it\large Refer to Model 3 above as your group develops consensus answers
    to the questions below.}
    \par\vskip 10pt

    \quest{20 min}
    
      \Q This program can be found in {\tt activity02c.cpp}.  Run it several times with the
        following inputs and write down the resulting output.\par\vskip 10pt
        \begin{enumerate}[(a)]
          \itemsep 10pt
          \item First number: 10, Second number: 4\hfill
            \ans[3.5in]{Quotient: 10/4 = 2.5}
          \item First number: 10, Second number: 3 \hfill
            \ans[3.5in]{Quotient: 10/3 = 3.33333}
          \item First number: 9.5, No second number \hfill 
            \ans[3.5in]{Quotient: 9/0.5 = 18}
          \item First number: 10, Second number: x \hfill
            \ans[3.5in]{Quotient: 10/0 = inf}
        \end{enumerate}
        
      \Q A {\it floating point} number is a number that contains a floating decimal point
        (as opposed to integers, which have no decimals). For example, 5.5 or 0.001.  In C++
        we often use the {\tt double} variable type to hold floating point values.  
        
        \begin{enumerate}
          \item How did the division operation change (as compared to earlier models) because one of 
            the variables was of type \cpp{double}?
            \begin{answer}[1in]
              It now does decimal division, returning the decimal number of times the divisor
              goes into the dividend instead of just the integer number of times.
            \end{answer}
            
          \item How would your results in problem 10 differ if both variables were of type \cpp{int}?
            \begin{answer}[1in]
              If both variables were of type \cpp{int} then we would get:
              \begin{enumerate}[(a)]
                \begin{multicols}{4}
                  \item 2
                  \item 3
                  \item error
                  \item error
                \end{multicols}
              \end{enumerate}             
            \end{answer}
            
          \item How would your results in problem 10 differ if both variables were of type  \cpp{double}?
            \begin{answer}[1in]
              \begin{enumerate}[(a)]
                \begin{multicols}{4}
                  \item same
                  \item same
                  \item need 2nd \#
                  \item same
                \end{multicols}
              \end{enumerate}
            \end{answer}
            
        \end{enumerate}
        
      \Q Write a conjecture as to when the division operation in C++ will return decimals\key\\[-2.5mm] and
        when it will only return an integer.  Use the phrase {\it floating point} in your answer.
        \begin{answer}[0.5in]
          It will return decimals if one or both of the values involved is a floating point
          number.  Otherwise, it returns an integer.
        \end{answer}
        
      \Q Create a program that asks asks for a number of cookies and a number of children
        and then prints out how many cookies each child will get (assuming they are split
        evenly) and how many are left over.  Be creative and professional in prompting the
        user and displaying the results.
        \begin{answer}
          Answers will vary.
        \end{answer}        