\model{A C++ Program}
  \begin{center}
    \begin{minipage}{3.5in}
      \begin{cpplst}
#include <iostream>

using namespace std;

int main() {
  cout << 16 + 3 << endl;
  cout << 16 - 3 << endl;
  cout << 16 * 3 << endl;
  cout << 16 / 3 << endl;
  cout << 16 % 3 << endl;
}      
      \end{cpplst}
    \end{minipage}
  \end{center}
  
  {\it\large Refer to Model 1 above as your team develops consensus answers
    to the questions below.}
    \par\vskip 10pt

  \quest{20 min}
    
  \Q Starting with the file {\tt activity02a.cpp}, enter and run the C++ program
    above.  What is the output of each of the following lines?
    \begin{enumerate}
      \itemsep 10pt
      \item {\tt cout << 16 + 3 << endl;} \hfill
        \ans[3.5in]{It prints {\it 19}}

      \item {\tt cout << 16 - 3 << endl;} \hfill
        \ans[3.5in]{It prints {\it 13}}

      \item {\tt cout << 16 * 3 << endl;} \hfill
        \ans[3.5in]{It prints {\it 48}}

      \item {\tt cout << 16 / 3 << endl;} \hfill
        \ans[3.5in]{It prints {\it 5}}

      \item {\tt cout << 16 \% 3 << endl;} \hfill
        \ans[3.5in]{It prints {\it 1}}
    \end{enumerate}
      
  \Q Were any of these a surprise to your team?
    \begin{answer}[0.25in]
      Answers will vary
    \end{answer}
      
  \Q Name the arithmetic operation represented by each symbol in C++.\key\\[-2.5mm]
    \par\vskip 10pt
    \begin{enumerate}
      \begin{multicols}{2}
        \itemsep 10pt
        \item {\tt +}
          \ans[1.5in]{Addition}

        \item {\tt -}
          \ans[1.5in]{Subtraction}

        \item {\tt *}
          \ans[1.5in]{Multiplication}

        \item {\tt /}
          \ans[1.5in]{Integer Division}

        \item {\tt \%}
          \ans[1.8in]{Modulus (remainder)}
      \end{multicols}
    \end{enumerate}

  \Q An {\it assignment statement} uses the ``='' sign to store the result of an
    operation performed on the right-hand side into the memory location named by 
    the variable on the left-hand side.
    Enter and execute the following two lies of C++ code (modify the {\tt activity02a.cpp}
    file to do this).
    \begin{center}
      \begin{minipage}{3.5in}
        \begin{cpplst}
  int age = 15;
  cout << "Your age is: " << age;
        \end{cpplst}
      \end{minipage}
    \end{center}

    \begin{enumerate}
      \itemsep 10pt
      \item What does the {\it assignment statement} on line 1 of this code chunk do?
        \begin{answer}[0.5in]
          It declares an integer variable named ``age'' and assigns it the value of 15.
        \end{answer}

      \item Split up the {\it variable declaration} and {\it assignment statement} into 
        two separate lines of code.
        \begin{answer}[0.5in]
          \fs
          The lines would be
          \begin{enumerate}
            \itemsep -2pt
            \item {\tt int age;}
            \item {\tt age = 15;}
          \end{enumerate}            
        \end{answer}

      \item What happens if you replace the assignment statement only (without altering the
        declaration) with \cpp{age = "fifteen";}?
        \begin{answer}[0.5in]
          You get a run-time error because you can't save a string into an integer variable.
        \end{answer}
    \end{enumerate}
    \vskip-20pt
    
  \Q Enter and execute the following three lines of C++ code (again, use the {\tt
      activity02a.cpp} file), but {\bf remember to add  \cpp{#include <string>} in
      the header}.
      \par\vskip 10pt
      \begin{center}
        \begin{minipage}{3.5in}
          \begin{cpplst}
  string schoolName = "Walla Walla";
  string schoolType = "University";
  string fullName = schoolName + schoolType;
  cout << fullName << endl;
          \end{cpplst}
        \end{minipage}
      \end{center}
      \begin{enumerate}
        \item What value is stored in the {\tt fullName} variable at line 3?
          \begin{answer}[0.5in]
            ``Walla WallaUniversity''
          \end{answer}

        \item How does the ``+'' sign behave differently when used with strings instead of numbers?
          \begin{answer}[0.5in]
            It concatenates the strings instead of adding values.
          \end{answer}           
        
        \item How could you fix the output so that the words are all separated?
          \begin{answer}[1in]
            For example, we could change line 3 to: \\
            {\tt string fullName = schoolName + " " + schoolType;}
          \end{answer}
        
        \item When two strings are ``glued'' together, they are said to be {\it concatenated}.
          What would happen if you tried to concatenate a string with an integer?
          \begin{answer}[1in]
            A syntax error.
          \end{answer}
      \end{enumerate}