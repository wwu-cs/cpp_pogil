\model{A C++ Program} \\
  \begin{center}
    \begin{minipage}{3.5in}
      \begin{cprlst}[
        frame=lines,
        framesep=2mm,
        bgcolor=gray!15,
        baselinestretch=1.2,
        linenos
      ]{cpp}
#include <iostream>
using namespace std;

int main() {
  cout << 16 + 3 << endl;
  cout << 16 - 3 << endl;
  cout << 16 * 3 << endl;
  cout << 16 / 3 << endl;
  cout << 16 % 3 << endl;
}      
      \end{cprlst}
    \end{minipage}
  \end{center}
  
  {\it\large Refer to Model 1 above as your team develops consensus answers
    to the questions below.}
    \par\vskip 10pt
    
  \begin{enumerate}
    \itemsep 20pt
    
    \Q Starting with the file {\tt activity02a.cpp}, enter and run the C++ program
      above.  What is the output of each of the following lines?\par\vskip 15pt
      \begin{enumerate}[(a)]
        \itemsep 10pt
        \item[5:] {\tt cout << 16 + 3 << endl;} \hfill
          \ans[3.5in]{It prints {\it 19}}
        \item[6:] {\tt cout << 16 - 3 << endl;} \hfill
          \ans[3.5in]{It prints {\it 13}}
        \item[7:] {\tt cout << 16 * 3 << endl;} \hfill
          \ans[3.5in]{It prints {\it 48}}
        \item[8:] {\tt cout << 16 / 3 << endl;} \hfill
          \ans[3.5in]{It prints {\it 5}}
        \item[9:] {\tt cout << 16 \% 3 << endl;} \hfill
          \ans[3.5in]{It prints {\it 1}}
      \end{enumerate}
      
    \Q Were any of these a surprise to your team?
      \begin{answer}[0.25in]
      \end{answer}
      
\newpage      
      
    \Q Name the arithmetic operation represented by each symbol in C++.\key\\[-2.5mm]
      \par\vskip 20pt
      \begin{enumerate}[(a)]
        \begin{multicols}{2}
          \itemsep 15pt
          \item {\tt +} \hfill \ans[2.25in]{Addition}
          \item {\tt -} \hfill \ans[2.25in]{Subtraction}
          \item {\tt *} \hfill \ans[2.25in]{Multiplication}
          \item {\tt /} \hfill \ans[2.25in]{Integer Division}
          \item {\tt \%} \hfill \ans[2.25in]{Modulus (remainder)}
        \end{multicols}
      \end{enumerate}

    \Q An {\it assignment statement} uses the ``='' sign to store the result of an
      operation performed on the right-hand side into the memory location named by 
      the variable on the left-hand side.\par\vskip 5pt
      
      Enter and execute the following two lies of C++ code (modify the {\tt activity02a.cpp}
      file to do this).
      
      \begin{center}
        \begin{minipage}{3.5in}
          \begin{cprlst}[
            frame=lines,
            framesep=2mm,
            bgcolor=gray!15,
            baselinestretch=1.2,
            linenos,
            firstnumber=5
          ]{cpp}
  int age = 15;
  cout << "Your age is: " << age;
      \end{cprlst}
    \end{minipage}
  \end{center}

      \begin{enumerate}[(a)]
        \itemsep 10pt
        \item What does the {\it assignment statement} on line 1 of this code chunk do?
          \begin{answer}[0.5in]
            It declares an integer variable named ``age'' and assigns it the value of 15.
          \end{answer}
        \item Split up the {\it variable declaration} and {\it assignment statement} into 
          two separate lines of code.
          \begin{answer}[0.5in]
            The lines would be
            \begin{enumerate}[1:]
              \itemsep -2pt
              \item {\tt int age;}
              \item {\tt age = 15;}
            \end{enumerate}            
          \end{answer}
        \item What happens if you replace the assignment statement only (without altering the
          declaration) with \cpp{age = "fifteen";}?
          \begin{answer}[0.5in]
            You get a run-time error because you can't save a string into an integer variable.
          \end{answer}
      \end{enumerate}
      \ifprintanswers\par\vskip -24pt\null\fi
      
    \Q Enter and execute the following three lines of C++ code (again, use the {\tt
      activity02a.cpp} file), but {\bf remember to add  \cpp{#include <string>} in
      the header}.
      \ifprintanswers\par\vskip -24pt\null\fi
      \begin{center}
        \begin{minipage}{3.5in}
          \begin{cprlst}[
            frame=lines,
            framesep=2mm,
            bgcolor=gray!15,
            baselinestretch=1.2,
            linenos,
            firstnumber=6
          ]{cpp}
  string schoolName = "Walla Walla";
  string schoolType = "University";
  string fullName = schoolName + schoolType;
  cout << fullName << endl;
      \end{cprlst}
    \end{minipage}
  \end{center}
  
  \begin{enumerate}[(a)]
  
    \Q What value is stored in the {\tt fullName} variable at line 3?
      \begin{answer}[0.5in]
        ``Walla WallaUniversity''
      \end{answer}
      
\newpage

    \Q How does the ``+'' sign behave differently when used with strings instead of numbers?
      \begin{answer}[0.5in]
        It concatenates the strings instead of adding values.
      \end{answer}           
      
    \Q How could you fix the output so that the words are all separated?
      \begin{answer}[1in]
        For example, we could change line 3 to: \\
        {\tt string fullName = schoolName + " " + schoolType;}
      \end{answer}
      
    \Q When two strings are ``glued'' together, they are said to be {\it concatenated}.
      What would happen if you tried to concatenate a string with an integer?
      \begin{answer}[1in]
        A syntax error.
      \end{answer}