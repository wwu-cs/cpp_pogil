\documentclass{exam}
%\documentclass[answers]{exam}
\hbadness=99999
\setlength{\textheight}{9.5in}
\setlength{\textwidth}{6.5in}
\setlength{\topmargin}{-0.75in}
\setlength{\oddsidemargin}{0in}
\setlength{\evensidemargin}{0in}

\usepackage{amsmath}
\usepackage{amssymb}
\usepackage{enumerate}
\usepackage[table]{xcolor}
\usepackage{hhline}
\usepackage{graphicx}
\usepackage{tikz}
%\usepackage{pgfplots}
\usepackage{multicol}
\usepackage{fancyvrb}

% for syntax highlighting
\usepackage{minted}
\usemintedstyle[cpp]{xcode}

% for overlay of output
\usepackage[overlay,showboxes]{textpos}

\pagestyle{plain}

\setlength\columnsep{50pt}
\newcommand{\key}{\hfill
      \raisebox{-.3\height}{\includegraphics[width=0.6in]{figures/key.png}}}

\begin{document}
  \thispagestyle{empty}
  \setlength{\parindent}{0pt}

  \begin{center}
    \Large Activity \#13: Parameter Passing \\[5pt]
    \large Recorder's Report\\[20pt]
    \normalsize
    \begin{tabular}{lrp{0.1in}lr}
      Manager:  & \fillin[][2.0in] & & Presenter: & \fillin[][2.0in]\\[15pt]
      Recorder: & \fillin[][2.0in] & & Driver:    & \fillin[][2.0in]\\[15pt]
      Date:     & \fillin[][2.0in] & & Score:     & Satisfactory \hspace{10pt} /
      \hspace{10pt} Not Satisfactory
    \end{tabular}
  \end{center}
  \par\vskip 15pt
  
  Record your team's answers to the key questions (marked with
  \raisebox{-.3\height}{\includegraphics[width=0.5in]{figures/key.png}})
  below.
  \begin{enumerate}[(a)]
    \itemsep 1.75in
    \item Model 1, Question \#6
    \item Model 2, Question \#12
    \item Model 3, Question \#20
  \end{enumerate}

  \clearpage\pagenumbering{arabic} 
  
  \begin{center}
    \Large Activity \#13: Parameter Passing \\[5pt]
    \large Activity Guide\\[20pt]
  \end{center}

  \begin{center}
    \fbox{
      \begin{minipage}{5.5in}
        {\bf Learning Objectives:} Students will be able to:
        \begin{itemize}
          \item Content:\\[-20pt]
            \begin{itemize}
              \itemsep 0pt
              \item Explain the use of default arguments in C++ functions
              \item Explain the difference between pass-by-reference and pass-by-value parameters
              \item Identify a function's signature
            \end{itemize}
          \item Process\\[-20pt]
            \begin{itemize}
              \itemsep 0pt
              \item Write functions that use default arguments values
              \item Write functions including pass-by-value and pass-by-reference parameters
              \item Write functions with the same name but different signatures\\[-5pt]
            \end{itemize}
        \end{itemize}
      \end{minipage}
      }
  \end{center}
  \par\vskip 10pt
  

  {\bf\large Model 1: A C++ Function Prototype} \\[-15pt]
  \begin{center}
    \begin{minipage}{4.5in}
      \begin{minted}[
        frame=lines,
        framesep=2mm,
        bgcolor=gray!15,
        baselinestretch=1.2,
        linenos,
        firstnumber=4
      ]{cpp}
int sum(int a, int b=0, int c=0, int d=0);

int main() {
  cout << "Test One: " << sum(1,2,3,4) << endl;
  cout << "Test Two: " << sum(1,2,3) << endl;
  cout << "Test Three: " << sum(1,2) << endl; 
}
      \end{minted}
    \end{minipage}
  \end{center}
  \TPMargin{5pt}
  \begin{textblock*}{1.8in}[0,0](4.75in,-1.45in)
    \textblockcolor{white}
    \begin{minipage}{1.65in}
      {\bf Output:} 
      \hrule\vskip 5pt\tt
\mintinline{html}|Test One: 10|\\
\mintinline{html}|Test Two: 6|\\
\mintinline{html}|Test Three: 3|
    \end{minipage}
  \end{textblock*}
  
  
  {\it\large Refer to Model 1 above as your team develops consensus answers
    to the questions below.}
    \par\vskip 10pt
    
  \begin{enumerate}
    \itemsep 20pt

    \item A prototype for the function {\tt sum} is defined on line 4
      of the model above.  Assume that this function is defined
      elsewhere so as to produce the output shown given the {\tt main}
      program.
      \par\ifprintanswers\vskip 10pt\else\vskip 20pt\fi
      
      \begin{enumerate}[(a)]
        \itemsep 15pt
        \item How many parameters does the function {\tt sum} have? \hfill
          \fillin[four][1.5in]
        \item How many arguments does the call to this function on line 7 have? \hfill
          \fillin[four][1.5in]
        \item How many arguments does the call to this function on line 8 have? \hfill
          \fillin[three][1.5in]
        \item How many arguments does the call to this function on line 9 have? \hfill
          \fillin[two][1.5in]
      \end{enumerate}
      
    \item What are the values of the parameters {\tt a}, {\tt b}, {\tt c}, and {\tt d}
      in the body of the function {\tt sum} for each call?
      \par\vskip 15pt
      
      \begin{enumerate}[(a)]
        \itemsep 15pt
        \item Line 7: \mintinline{cpp}|sum(1,2,3,4)| \hfill 
            {\tt a =} \fillin[1][0.5in], \hspace{5pt} 
            {\tt b =} \fillin[2][0.5in], \hspace{5pt}
            {\tt c =} \fillin[3][0.5in], \hspace{5pt} and
            {\tt d =} \fillin[4][0.5in]
        \item Line 8: \mintinline{cpp}|sum(1,2,3)| \hfill
            {\tt a =} \fillin[1][0.5in], \hspace{5pt} 
            {\tt b =} \fillin[2][0.5in], \hspace{5pt}
            {\tt c =} \fillin[3][0.5in], \hspace{5pt} and
            {\tt d =} \fillin[0][0.5in]
        \item Line 9: \mintinline{cpp}|sum(1,2)| \hfill 
            {\tt a =} \fillin[1][0.5in], \hspace{5pt} 
            {\tt b =} \fillin[2][0.5in], \hspace{5pt}
            {\tt c =} \fillin[0][0.5in], \hspace{5pt} and
            {\tt d =} \fillin[0][0.5in]
      \end{enumerate}
   

\newpage      

    \item A {\it default argument} is a value provided in a function
      declaration that is automatically assigned by the compiler when
      function calls don't provide a value for the argument.  Which
      parameters in the {\tt sum} function have default arguments?
      How can you tell?
      \begin{solution}[0.5in]
        Parameters {\tt b}, {\tt c}, and {\tt d} have default
        arguments.  You can tell because the all say 
        ``\mintinline{cpp}|=0|''.
      \end{solution}
      \ifprintanswers\vskip -35pt\null\fi
      
    \item Rewrite the prototype of the {\tt sum} function from line
      one so that the function calls below return the indicated value.
      We will use the notation \mintinline{cpp}|sum(1,2,3,4)|
      $\rightarrow$ 10 to indicate the function call returns 10.
      \par\vskip 15pt
      
      \begin{enumerate}[(a)]
        \itemsep 15pt
        \item \mintinline{cpp}|sum(0,0,0)| $\rightarrow$ 2 \hfill
          \fillin[\mintinline{cpp}|int sum(int a,int b, int c, int d=2)|][4.5in]
        \item \mintinline{cpp}|sum(3,2)| $\rightarrow$ 8 \hfill
          \fillin[\mintinline{cpp}|int sum(int a,int b, int c=6, int d=2)|][4.5in]
        \item \mintinline{cpp}|sum(1)| $\rightarrow$ 10 \hfill
          \fillin[\mintinline{cpp}|int sum(int a,int b=1, int c=6, int d=2)|][4.5in]
      \end{enumerate}
      
    \item The file {\tt activity13a.cpp} contains the code from the
      model as well as the function definition.  Replace the function
      prototype on line 4 of that file with each of the following. Place a check 
      next to the prototypes for which the program compiles without errors.
      \par\vskip 5pt
      
      \begin{enumerate}[(a)]
        \small
        \itemsep 5pt
          \item \fillin[][0.25in]             \mintinline{cpp}|int sum(int a, int b=0, int c, int d=0)|
          \item \fillin[$\checkmark$][0.25in] \mintinline{cpp}|int sum(int a, int b, int c=0, int d=0)|
          \item \fillin[][0.25in]             \mintinline{cpp}|int sum(int a=0, int b=0, int c, int d)|
          \item \fillin[$\checkmark$][0.25in] \mintinline{cpp}|int sum(int a=0, int b=0, int c=0, int d=0)|
      \end{enumerate}
      \vskip -50pt\null
      
    \item Based on these results, what rule does C++
      enforces in regard to default arguments?\key
      \begin{solution}[0.5in]
        Once a parameter has a default argument, every parameter after that must also.
      \end{solution}
      \ifprintanswers\vskip -35pt\null\fi
      \vskip -20pt\null
      
    \item Follow the steps below to write a function that returns the product of
      between two and four arguments, as shown in the examples below.
      Use the file {\tt activity13a.cpp} to test your code.
      \begin{itemize}
        \small
        \begin{multicols}{3}
          \item\mintinline{cpp}|product(2,2,2,3)| $\rightarrow$ 24
          \item\mintinline{cpp}|product(3,3,5)| $\rightarrow$ 45
          \item\mintinline{cpp}|product(5,5)| $\rightarrow$ 25
        \end{multicols}
      \end{itemize}
      
      \begin{enumerate}[(a)]
        \item First, write the function definition based on the
          function header shown below.  Enter your definition below the
          {\tt main} program in the {\tt activity13a.cpp} file.
          \begin{center}
            \mintinline{cpp}|int product(int a, int b, int c, int d)|
          \end{center}
          \begin{solution}[1.5in]
            \scriptsize\vskip -35pt\null
            \begin{center}
              \begin{minipage}{3.5in}
                \begin{minted}[
                  frame=lines,
                  framesep=2mm,
                  bgcolor=gray!15,
                  baselinestretch=1.2,
                  linenos
                ]{cpp}
int product(int a, int b, int c, int d) {
  return a*b*c*d;
}
                \end{minted}
              \end{minipage}
            \end{center}\vskip -20pt\null
          \end{solution}          
        \item Next, add a function prototype above the {\tt main}
          program.  Include appropriate default arguments.
          \begin{solution}[0.5in]
            \scriptsize\vskip -35pt\null
            \begin{center}
              \begin{minipage}{3.5in}
                \begin{minted}[
                  frame=lines,
                  framesep=2mm,
                  bgcolor=gray!15,
                  baselinestretch=1.2,
                ]{cpp}
int product(int a, int b, int c=1, int d=1);
                \end{minted}
              \end{minipage}
            \end{center}\vskip -20pt\null
          \end{solution}
        \item Finally, add the three function calls above to the {\tt main} program.
          Do they work as expected?
      \end{enumerate}

\newpage
  
  {\bf\large Model 2: Two C++ Functions and a Main Program} \\[-20pt]
  \begin{center}
    \begin{tabular}{p{2in}p{0.25in}p{3.4in}}
      \begin{minipage}{2in}
        \begin{minted}[
          frame=lines,
          framesep=2mm,
          bgcolor=gray!15,
          baselinestretch=1.2,
          linenos,
          firstnumber=4
        ]{cpp}
int addOneA(int x) {
  x = x + 1;
  return x;
}

int addOneB(int &x) {
  x = x + 1;
  return x;
}

        \end{minted}
      \end{minipage}
      & &
      \begin{minipage}{3.4in}
        \begin{minted}[
          frame=lines,
          framesep=2mm,
          bgcolor=gray!15,
          baselinestretch=1.2,
          linenos,
          firstnumber=14
        ]{cpp}
int main() {
  int a = 1;
  cout << "Function: " << addOneA(a) << ", ";
  cout << "Argument: " << a << endl;
  
  int b = 1;
  cout << "Function: " << addOneB(b) << ", ";
  cout << "Argument: " << b << endl;
}
        \end{minted}
      \end{minipage}
    \end{tabular}
  \end{center}

  {\it\large Refer to Model 2 above as your team develops consensus answers
    to the questions below.}

    \item Without running them, determine what the functions {\tt addOneA} and 
      {\tt addOneB} do.
      \ifprintanswers\vskip -20pt\null\fi
      \begin{solution}[0.35in]
        They both add one to the parameter and then return that value.
      \end{solution}
      \ifprintanswers\vskip -35pt\null\fi
      
    \item How do the function headers differ between these two functions?
      \ifprintanswers\vskip -20pt\null\fi
      \begin{solution}[0.35in]
        The second function header has an {\tt \& }before the parameter name.
      \end{solution}
      \ifprintanswers\vskip -35pt\null\fi

    \item The code for this model can be found in {\tt activity13b.cpp}.  Run it and
      record the output.
      \ifprintanswers\vskip -20pt\null\fi
      \begin{solution}[0.75in]
        The output is:\par
        Function: 2, Argument 1\\
        Function: 2, Argument 2
      \end{solution}
      \ifprintanswers\vskip -35pt\null\fi
      
    \item In this class we will see two ways to pass arguments to a parameter in C++.
      \begin{itemize}
        \itemsep 5pt
        \item When an argument is {\it passed by value}, a copy of the argument
          value is made in the function parameter variable and any changes
          made to that parameter variable inside the function are thrown away 
          when the function is done.
        \item When an argument is {\it passed by reference}, the parameter
          variable is a reference to the actual argument variable outside
          the function.  Any changes made to the parameter variable inside 
          the function are actually made to the argument variable and persist
          when the function is done.
      \end{itemize}
      Based on these two definitions, which function in this model uses pass by
      reference?
      \begin{solution}[0.5in]
        The function {\tt addOneB} since the value of {\tt b} in the {\tt main} program
        is changed but the value {\tt a} in the {\tt main} program is not.
      \end{solution}
      \ifprintanswers\vskip -55pt\else\vskip -30pt\fi\null
      
    \item By looking at the function header, how can you determine if an argument will
      be passed \key\\[-2.5mm] by value or passed by reference?
      \begin{solution}[0.5in]
        If the parameter name has a {\tt \&} in front of it, the argument will be
        passed by reference.  Otherwise it will be passed by value.
      \end{solution}
    
\newpage

    \item Consider the following C++ function.
      \begin{center}
        \begin{minipage}{5in}
          \begin{minted}[
            frame=lines,
            framesep=2mm,
            bgcolor=gray!15,
            baselinestretch=1.2
          ]{cpp}
// Normally sine and cosine expect angles given in radians  
// This function returns the sine and cosine of an angle in degrees
void getSinCos(double degrees, double &sinOut, double &cosOut) {
  const double PI = 3.14159265358979323846;
  double radians = degrees * PI / 180.0;
  sinOut = sin(radians);
  cosOut = cos(radians);
}
          \end{minted}
        \end{minipage}        
      \end{center}
      
      \begin{enumerate}[(a)]
        \itemsep 15pt
        \item Which parameter(s) take pass by value arguments? \hfill
          \fillin[{\tt degrees}][2in]
        \item Which parameter(s) take pass by reference arguments? \hfill
          \fillin[{\tt sinOut} and {\tt cosOut}][2in]
        \item What is the return type of this function? \hfill
          \fillin[\mintinline{cpp}|void|][2in]
        \item What does the comment mean when it says ``This function returns the sine
          and cosine{\ldots}?''          
          \begin{solution}[0.5in]
            It means that the function changes the values of the argument variables
            referred to by {\tt sinOut} and {\tt cosOut}.
          \end{solution}
          \ifprintanswers\vskip -25pt\null\fi          
        \item Why couldn't we use a \mintinline{cpp}|return| statement at the end of
          the function to return the sine and cosine?
          \ifprintanswers\vskip -10pt\null\fi
          \begin{solution}[0.5in]
            Because a function can only return one value.
          \end{solution}
          \ifprintanswers\vskip -25pt\null\fi
      \end{enumerate}
      
      
    \item The function below is meant to swap the values of the integer argument variables
      passed in to it.
      \ifprintanswers\vskip -25pt\null\fi
      \begin{center}
        \begin{tabular}{p{3in}p{2.75in}}
          \begin{minipage}{3in}
            \small
            \begin{minted}[
              frame=lines,
              framesep=2mm,
              bgcolor=gray!15,
              baselinestretch=1.2,
              linenos,
            ]{cpp}
/* function header */ {
  int temp = x; // save x in temporary var
  x = y;        // replace x with y
  y = temp;     // put original x value in y
}

int main() {
  int a = 2;
  int b = 3;
  swap(a,b);
  cout << a << ", " << b << endl;
}
            \end{minted}
          \end{minipage}        
          &
          \begin{minipage}{2.75in}
            \begin{enumerate}[(a)]
              \item Write a function header for line 1 that makes the program output {\tt 2,3}.
                \begin{solution}[0.65in]
                  \mintinline{cpp}|void swap(int x, int y)|
                \end{solution}
              \item Write a function header for line 1 that makes the program output {\tt 3,2}.
                \begin{solution}[0.65in]
                  \mintinline{cpp}|void swap(int &x, int &y)|
                \end{solution}
            \end{enumerate}
          \end{minipage}
        \end{tabular}
      \end{center}
      
      \begin{enumerate}[(a)]
        \setcounter{enumii}{2}
        \item Using the function header from (b), which function calls below 
          are valid?  Check all that apply.
          \begin{enumerate}[i.]
            \begin{multicols}{3}
              \item \fillin[$\checkmark$][0.25in] \mintinline{cpp}|swap(b,a)|
              \item \fillin[][0.25in] \mintinline{cpp}|swap(a,2)|
              \item \fillin[][0.25in] \mintinline{cpp}|swap(3,2)|
            \end{multicols}
          \end{enumerate}
      \end{enumerate}
          
\newpage

  {\bf\large Model 3: Similar C++ Function Prototypes} \\[-10pt]
  \ifprintanswers\vskip -20pt\null\fi

  \begin{center}
    \begin{minipage}{3in}
      \begin{minted}[
        frame=lines,
        framesep=2mm,
        bgcolor=gray!15,
        baselinestretch=1.2,
        linenos,
        firstnumber=5
      ]{cpp}
int Maximum(int, int);
int Maximum(int, int, int);
double Maximum(double, double);
double Maximum(double, double, double);
      \end{minted}        
    \end{minipage}
  \end{center}
      
  {\it\large Refer to Model 3 above as your team develops consensus answers
    to the questions below.}
    \ifprintanswers\vskip -20pt\null\fi
    
    \item What is the name of the function whose prototype is given on each of the
      following lines?
      \par\vskip 15pt
      \begin{enumerate}[(a)]
        \itemsep 15pt
        \begin{multicols}{2}
          \item Line 1 \hfill \fillin[\tt Maximum][2in]
          \item Line 2 \hfill \fillin[\tt Maximum][2in]
          \item Line 3 \hfill \fillin[\tt Maximum][2in]
          \item Line 4 \hfill \fillin[\tt Maximum][2in]
        \end{multicols}
      \end{enumerate}

    \item On which line number is the function prototype to which each
      function call below matches?  Try running the code in {\tt
      activity13c.cpp} and reading any compile error messages to help you decide.
      \par\vskip 15pt      
      \begin{enumerate}[(a)]
        \itemsep 15pt
        \item \mintinline{cpp}|cout << Maximum(5,1,3);| \hfill
          \fillin[Line 6][3in]
        \item \mintinline{cpp}|cout << Maximum(5,1.5,3);| \hfill
          \fillin[Ambiguous (either line 6 or line 8)][3in]
        \item \mintinline{cpp}|cout << Maximum(5,3);| \hfill
          \fillin[Line 5][3in]
        \item \mintinline{cpp}|cout << Maximum(2.3,4.7);| \hfill
          \fillin[Line 8][3in]
      \end{enumerate}
      
    \item How could you tell which function call go with which prototype?
      \ifprintanswers\vskip -20pt\null\fi
      \begin{solution}[0.5in]
        Match the number, type, and order of the arguments with
        the parameters.  If there is no exact match, typecasting may
        be used but must be unambiguous (i.e. only one possible match).
      \end{solution}
      \ifprintanswers\vskip -35pt\null\fi
      
    \item A function's {\it signature} is its name and the number, type, and
      order of its parameters.  Give prototypes for at least two additional functions
      with the name {\tt Maximum} that have different function signatures.
      \ifprintanswers\vskip -20pt\null\fi
      \begin{solution}[0.35in]
        \mintinline{cpp}|string Maximum(string,string)| 
        and 
        \mintinline{cpp}|int Maximum(int,int,int,int)|\hfill
      \end{solution}
      \ifprintanswers\vskip -35pt\null\fi
      
    \item Give at least one new (different) prototype for a function named 
      {\tt Maximum} that has the same signature as one of those in the model.
      \ifprintanswers\vskip -20pt\null\fi
      \begin{solution}[0.35in]
        \mintinline{cpp}|void Maximum(int,int)|
      \end{solution}
      \ifprintanswers\vskip -55pt\null\else\vskip -30pt\null\fi
      
    \item Add your prototypes from the last two problems to the file {\tt
      activity13c.cpp} to see which\key\\[-2.5mm] the compiler accepts and which produce errors.
      Based on that, make a conjecture about when you can have functions with the same
      name in C++.
      \ifprintanswers\vskip -20pt\null\fi
      \begin{solution}[0.5in]
        Functions can have the same name if their signatures are different.
      \end{solution}
      
           

  \end{enumerate}  
    
\end{document}
