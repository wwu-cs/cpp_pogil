\model{A C++ Function Prototype} \\[-15pt]
  \begin{center}
    \begin{minipage}{4.5in}
      \begin{cpplst}
int sum(int a, int b=0, int c=0, int d=0);

int main() {
  cout << "Test One: " << sum(1,2,3,4) << endl;
  cout << "Test Two: " << sum(1,2,3) << endl;
  cout << "Test Three: " << sum(1,2) << endl; 
}
      \end{cpplst}
    \end{minipage}
  \end{center}
  \TPMargin{5pt}
  \begin{textblock*}{1.8in}[0,0](4.75in,-1.45in)
    \textblockcolor{white}
    \begin{minipage}{1.65in}
      {\bf Output:} 
      \hrule\vskip 5pt\tt
\cpp{html}{Test One: 10}\\
\cpp{html}{Test Two: 6}\\
\cpp{html}{Test Three: 3}
    \end{minipage}
  \end{textblock*}
  
  
  {\it\large Refer to Model 1 above as your team develops consensus answers
    to the questions below.}
    \par\vskip 10pt
    

    \Q A prototype for the function {\tt sum} is defined on line 4
      of the model above.  Assume that this function is defined
      elsewhere so as to produce the output shown given the {\tt main}
      program.
      % \par\ifprintanswers\vskip 10pt\else\vskip 20pt\fi
      
      \begin{enumerate}[(a)]
        \itemsep 15pt
        \item How many parameters does the function {\tt sum} have? \hfill
          \ans[1.5in]{four}
        \item How many arguments does the call to this function on line 7 have? \hfill
          \ans[1.5in]{four}
        \item How many arguments does the call to this function on line 8 have? \hfill
          \ans[1.5in]{three}
        \item How many arguments does the call to this function on line 9 have? \hfill
          \ans[1.5in]{two}
      \end{enumerate}
      
    \Q What are the values of the parameters {\tt a}, {\tt b}, {\tt c}, and {\tt d}
      in the body of the function {\tt sum} for each call?
      \par\vskip 15pt
      
      \begin{enumerate}[(a)]
        \itemsep 15pt
        \item Line 7: \cpp{sum(1,2,3,4)} \hfill 
            {\tt a =} \ans[0.5in]{1}, \hspace{5pt} 
            {\tt b =} \ans[0.5in]{2}, \hspace{5pt}
            {\tt c =} \ans[0.5in]{3}, \hspace{5pt} and
            {\tt d =} \ans[0.5in]{4}
        \item Line 8: \cpp{sum(1,2,3)} \hfill
            {\tt a =} \ans[0.5in]{1}, \hspace{5pt} 
            {\tt b =} \ans[0.5in]{2}, \hspace{5pt}
            {\tt c =} \ans[0.5in]{3}, \hspace{5pt} and
            {\tt d =} \ans[0.5in]{0}
        \item Line 9: \cpp{sum(1,2)} \hfill 
            {\tt a =} \ans[0.5in]{1}, \hspace{5pt} 
            {\tt b =} \ans[0.5in]{2}, \hspace{5pt}
            {\tt c =} \ans[0.5in]{0}, \hspace{5pt} and
            {\tt d =} \ans[0.5in]{0}
      \end{enumerate}
   

\newpage      

    \Q A {\it default argument} is a value provided in a function
      declaration that is automatically assigned by the compiler when
      function calls don't provide a value for the argument.  Which
      parameters in the {\tt sum} function have default arguments?
      How can you tell?
      \begin{answer}[0.5in]
        Parameters {\tt b}, {\tt c}, and {\tt d} have default
        arguments.  You can tell because the all say 
        ``\cpp{=0}''.
      \end{answer}
      % \ifprintanswers\vskip -35pt\null\fi
      
    \Q Rewrite the prototype of the {\tt sum} function from line
      one so that the function calls below return the indicated value.
      We will use the notation \cpp{sum(1,2,3,4)}
      $\rightarrow$ 10 to indicate the function call returns 10.
      \par\vskip 15pt
      
      \begin{enumerate}[(a)]
        \itemsep 15pt
        \item \cpp{sum(0,0,0)} $\rightarrow$ 2 \hfill
          \ans[4.5in]{\cpp{int sum(int a,int b, int c, int d=2)}
        \item \cpp{sum(3,2)} $\rightarrow$ 8 \hfill
          \ans[4.5in]{\cpp{int sum(int a,int b, int c=6, int d=2)}
        \item \cpp{sum(1)} $\rightarrow$ 10 \hfill
          \ans[4.5in]{\cpp{int sum(int a,int b=1, int c=6, int d=2)}
      \end{enumerate}
      
    \Q The file {\tt activity13a.cpp} contains the code from the
      model as well as the function definition.  Replace the function
      prototype on line 4 of that file with each of the following. Place a check 
      next to the prototypes for which the program compiles without errors.
      \par\vskip 5pt
      
      \begin{enumerate}[(a)]
        \small
        \itemsep 5pt
          \item \ans[0.25in]{\cpp{int sum(int a, int b=0, int c, int d=0)}}
          \item \ans[$\checkmark$][0.25in]{\cpp{int sum(int a, int b, int c=0, int d=0)}}
          \item \ans[0.25in]{\cpp{int sum(int a=0, int b=0, int c, int d)}}
          \item \ans[$\checkmark$][0.25in]{\cpp{int sum(int a=0, int b=0, int c=0, int d=0)}}
      \end{enumerate}
      \vskip -50pt\null
      
    \Q Based on these results, what rule does C++
      enforces in regard to default arguments?\key
      \begin{answer}[0.5in]
        Once a parameter has a default argument, every parameter after that must also.
      \end{answer}
      % \ifprintanswers\vskip -35pt\null\fi
      \vskip -20pt\null
      
    \Q Follow the steps below to write a function that returns the product of
      between two and four arguments, as shown in the examples below.
      Use the file {\tt activity13a.cpp} to test your code.
      \begin{itemize}
        \small
        \begin{multicols}{3}
          \item\cpp{product(2,2,2,3)} $\rightarrow$ 24
          \item\cpp{product(3,3,5)} $\rightarrow$ 45
          \item\cpp{product(5,5)} $\rightarrow$ 25
        \end{multicols}
      \end{itemize}
      
      \begin{enumerate}[(a)]
        \item First, write the function definition based on the
          function header shown below.  Enter your definition below the
          {\tt main} program in the {\tt activity13a.cpp} file.
          \begin{center}
            \cpp{int product(int a, int b, int c, int d)}
          \end{center}
          \begin{answer}[1.5in]
            \scriptsize\vskip -35pt\null
            \begin{center}
              \begin{minipage}{3.5in}
                \begin{cpplst}
int product(int a, int b, int c, int d) {
  return a*b*c*d;
}
                \end{cpplst}
              \end{minipage}
            \end{center}\vskip -20pt\null
          \end{answer}          
        \item Next, add a function prototype above the {\tt main}
          program.  Include appropriate default arguments.
          \begin{answer}[0.5in]
            \scriptsize\vskip -35pt\null
            \begin{center}
              \begin{minipage}{3.5in}
                \begin{cpplst}
int product(int a, int b, int c=1, int d=1);
                \end{cpplst}
              \end{minipage}
            \end{center}\vskip -20pt\null
          \end{answer}
        \item Finally, add the three function calls above to the {\tt main} program.
          Do they work as expected?
      \end{enumerate}
