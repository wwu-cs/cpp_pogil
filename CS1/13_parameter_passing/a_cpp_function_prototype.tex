{\bf\large Model 1: A C++ Function Prototype} \\[-15pt]
  \begin{center}
    \begin{minipage}{4.5in}
      \begin{minted}[
        frame=lines,
        framesep=2mm,
        bgcolor=gray!15,
        baselinestretch=1.2,
        linenos,
        firstnumber=4
      ]{cpp}
int sum(int a, int b=0, int c=0, int d=0);

int main() {
  cout << "Test One: " << sum(1,2,3,4) << endl;
  cout << "Test Two: " << sum(1,2,3) << endl;
  cout << "Test Three: " << sum(1,2) << endl; 
}
      \end{minted}
    \end{minipage}
  \end{center}
  \TPMargin{5pt}
  \begin{textblock*}{1.8in}[0,0](4.75in,-1.45in)
    \textblockcolor{white}
    \begin{minipage}{1.65in}
      {\bf Output:} 
      \hrule\vskip 5pt\tt
\mintinline{html}|Test One: 10|\\
\mintinline{html}|Test Two: 6|\\
\mintinline{html}|Test Three: 3|
    \end{minipage}
  \end{textblock*}
  
  
  {\it\large Refer to Model 1 above as your team develops consensus answers
    to the questions below.}
    \par\vskip 10pt
    
  \begin{enumerate}
    \itemsep 20pt

    \item A prototype for the function {\tt sum} is defined on line 4
      of the model above.  Assume that this function is defined
      elsewhere so as to produce the output shown given the {\tt main}
      program.
      \par\ifprintanswers\vskip 10pt\else\vskip 20pt\fi
      
      \begin{enumerate}[(a)]
        \itemsep 15pt
        \item How many parameters does the function {\tt sum} have? \hfill
          \fillin[four][1.5in]
        \item How many arguments does the call to this function on line 7 have? \hfill
          \fillin[four][1.5in]
        \item How many arguments does the call to this function on line 8 have? \hfill
          \fillin[three][1.5in]
        \item How many arguments does the call to this function on line 9 have? \hfill
          \fillin[two][1.5in]
      \end{enumerate}
      
    \item What are the values of the parameters {\tt a}, {\tt b}, {\tt c}, and {\tt d}
      in the body of the function {\tt sum} for each call?
      \par\vskip 15pt
      
      \begin{enumerate}[(a)]
        \itemsep 15pt
        \item Line 7: \mintinline{cpp}|sum(1,2,3,4)| \hfill 
            {\tt a =} \fillin[1][0.5in], \hspace{5pt} 
            {\tt b =} \fillin[2][0.5in], \hspace{5pt}
            {\tt c =} \fillin[3][0.5in], \hspace{5pt} and
            {\tt d =} \fillin[4][0.5in]
        \item Line 8: \mintinline{cpp}|sum(1,2,3)| \hfill
            {\tt a =} \fillin[1][0.5in], \hspace{5pt} 
            {\tt b =} \fillin[2][0.5in], \hspace{5pt}
            {\tt c =} \fillin[3][0.5in], \hspace{5pt} and
            {\tt d =} \fillin[0][0.5in]
        \item Line 9: \mintinline{cpp}|sum(1,2)| \hfill 
            {\tt a =} \fillin[1][0.5in], \hspace{5pt} 
            {\tt b =} \fillin[2][0.5in], \hspace{5pt}
            {\tt c =} \fillin[0][0.5in], \hspace{5pt} and
            {\tt d =} \fillin[0][0.5in]
      \end{enumerate}
   

\newpage      

    \item A {\it default argument} is a value provided in a function
      declaration that is automatically assigned by the compiler when
      function calls don't provide a value for the argument.  Which
      parameters in the {\tt sum} function have default arguments?
      How can you tell?
      \begin{solution}[0.5in]
        Parameters {\tt b}, {\tt c}, and {\tt d} have default
        arguments.  You can tell because the all say 
        ``\mintinline{cpp}|=0|''.
      \end{solution}
      \ifprintanswers\vskip -35pt\null\fi
      
    \item Rewrite the prototype of the {\tt sum} function from line
      one so that the function calls below return the indicated value.
      We will use the notation \mintinline{cpp}|sum(1,2,3,4)|
      $\rightarrow$ 10 to indicate the function call returns 10.
      \par\vskip 15pt
      
      \begin{enumerate}[(a)]
        \itemsep 15pt
        \item \mintinline{cpp}|sum(0,0,0)| $\rightarrow$ 2 \hfill
          \fillin[\mintinline{cpp}|int sum(int a,int b, int c, int d=2)|][4.5in]
        \item \mintinline{cpp}|sum(3,2)| $\rightarrow$ 8 \hfill
          \fillin[\mintinline{cpp}|int sum(int a,int b, int c=6, int d=2)|][4.5in]
        \item \mintinline{cpp}|sum(1)| $\rightarrow$ 10 \hfill
          \fillin[\mintinline{cpp}|int sum(int a,int b=1, int c=6, int d=2)|][4.5in]
      \end{enumerate}
      
    \item The file {\tt activity13a.cpp} contains the code from the
      model as well as the function definition.  Replace the function
      prototype on line 4 of that file with each of the following. Place a check 
      next to the prototypes for which the program compiles without errors.
      \par\vskip 5pt
      
      \begin{enumerate}[(a)]
        \small
        \itemsep 5pt
          \item \fillin[][0.25in]             \mintinline{cpp}|int sum(int a, int b=0, int c, int d=0)|
          \item \fillin[$\checkmark$][0.25in] \mintinline{cpp}|int sum(int a, int b, int c=0, int d=0)|
          \item \fillin[][0.25in]             \mintinline{cpp}|int sum(int a=0, int b=0, int c, int d)|
          \item \fillin[$\checkmark$][0.25in] \mintinline{cpp}|int sum(int a=0, int b=0, int c=0, int d=0)|
      \end{enumerate}
      \vskip -50pt\null
      
    \item Based on these results, what rule does C++
      enforces in regard to default arguments?\key
      \begin{solution}[0.5in]
        Once a parameter has a default argument, every parameter after that must also.
      \end{solution}
      \ifprintanswers\vskip -35pt\null\fi
      \vskip -20pt\null
      
    \item Follow the steps below to write a function that returns the product of
      between two and four arguments, as shown in the examples below.
      Use the file {\tt activity13a.cpp} to test your code.
      \begin{itemize}
        \small
        \begin{multicols}{3}
          \item\mintinline{cpp}|product(2,2,2,3)| $\rightarrow$ 24
          \item\mintinline{cpp}|product(3,3,5)| $\rightarrow$ 45
          \item\mintinline{cpp}|product(5,5)| $\rightarrow$ 25
        \end{multicols}
      \end{itemize}
      
      \begin{enumerate}[(a)]
        \item First, write the function definition based on the
          function header shown below.  Enter your definition below the
          {\tt main} program in the {\tt activity13a.cpp} file.
          \begin{center}
            \mintinline{cpp}|int product(int a, int b, int c, int d)|
          \end{center}
          \begin{solution}[1.5in]
            \scriptsize\vskip -35pt\null
            \begin{center}
              \begin{minipage}{3.5in}
                \begin{minted}[
                  frame=lines,
                  framesep=2mm,
                  bgcolor=gray!15,
                  baselinestretch=1.2,
                  linenos
                ]{cpp}
int product(int a, int b, int c, int d) {
  return a*b*c*d;
}
                \end{minted}
              \end{minipage}
            \end{center}\vskip -20pt\null
          \end{solution}          
        \item Next, add a function prototype above the {\tt main}
          program.  Include appropriate default arguments.
          \begin{solution}[0.5in]
            \scriptsize\vskip -35pt\null
            \begin{center}
              \begin{minipage}{3.5in}
                \begin{minted}[
                  frame=lines,
                  framesep=2mm,
                  bgcolor=gray!15,
                  baselinestretch=1.2,
                ]{cpp}
int product(int a, int b, int c=1, int d=1);
                \end{minted}
              \end{minipage}
            \end{center}\vskip -20pt\null
          \end{solution}
        \item Finally, add the three function calls above to the {\tt main} program.
          Do they work as expected?
      \end{enumerate}
