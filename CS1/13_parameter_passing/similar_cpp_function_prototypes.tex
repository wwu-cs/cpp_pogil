{\bf\large Model 3: Similar C++ Function Prototypes} \\[-10pt]
  \ifprintanswers\vskip -20pt\null\fi

  \begin{center}
    \begin{minipage}{3in}
      \begin{minted}[
        frame=lines,
        framesep=2mm,
        bgcolor=gray!15,
        baselinestretch=1.2,
        linenos,
        firstnumber=5
      ]{cpp}
int Maximum(int, int);
int Maximum(int, int, int);
double Maximum(double, double);
double Maximum(double, double, double);
      \end{minted}        
    \end{minipage}
  \end{center}
      
  {\it\large Refer to Model 3 above as your team develops consensus answers
    to the questions below.}
    \ifprintanswers\vskip -20pt\null\fi
    
    \item What is the name of the function whose prototype is given on each of the
      following lines?
      \par\vskip 15pt
      \begin{enumerate}[(a)]
        \itemsep 15pt
        \begin{multicols}{2}
          \item Line 1 \hfill \fillin[\tt Maximum][2in]
          \item Line 2 \hfill \fillin[\tt Maximum][2in]
          \item Line 3 \hfill \fillin[\tt Maximum][2in]
          \item Line 4 \hfill \fillin[\tt Maximum][2in]
        \end{multicols}
      \end{enumerate}

    \item On which line number is the function prototype to which each
      function call below matches?  Try running the code in {\tt
      activity13c.cpp} and reading any compile error messages to help you decide.
      \par\vskip 15pt      
      \begin{enumerate}[(a)]
        \itemsep 15pt
        \item \mintinline{cpp}|cout << Maximum(5,1,3);| \hfill
          \fillin[Line 6][3in]
        \item \mintinline{cpp}|cout << Maximum(5,1.5,3);| \hfill
          \fillin[Ambiguous (either line 6 or line 8)][3in]
        \item \mintinline{cpp}|cout << Maximum(5,3);| \hfill
          \fillin[Line 5][3in]
        \item \mintinline{cpp}|cout << Maximum(2.3,4.7);| \hfill
          \fillin[Line 8][3in]
      \end{enumerate}
      
    \item How could you tell which function call go with which prototype?
      \ifprintanswers\vskip -20pt\null\fi
      \begin{solution}[0.5in]
        Match the number, type, and order of the arguments with
        the parameters.  If there is no exact match, typecasting may
        be used but must be unambiguous (i.e. only one possible match).
      \end{solution}
      \ifprintanswers\vskip -35pt\null\fi
      
    \item A function's {\it signature} is its name and the number, type, and
      order of its parameters.  Give prototypes for at least two additional functions
      with the name {\tt Maximum} that have different function signatures.
      \ifprintanswers\vskip -20pt\null\fi
      \begin{solution}[0.35in]
        \mintinline{cpp}|string Maximum(string,string)| 
        and 
        \mintinline{cpp}|int Maximum(int,int,int,int)|\hfill
      \end{solution}
      \ifprintanswers\vskip -35pt\null\fi
      
    \item Give at least one new (different) prototype for a function named 
      {\tt Maximum} that has the same signature as one of those in the model.
      \ifprintanswers\vskip -20pt\null\fi
      \begin{solution}[0.35in]
        \mintinline{cpp}|void Maximum(int,int)|
      \end{solution}
      \ifprintanswers\vskip -55pt\null\else\vskip -30pt\null\fi
      
    \item Add your prototypes from the last two problems to the file {\tt
      activity13c.cpp} to see which\key\\[-2.5mm] the compiler accepts and which produce errors.
      Based on that, make a conjecture about when you can have functions with the same
      name in C++.
      \ifprintanswers\vskip -20pt\null\fi
      \begin{solution}[0.5in]
        Functions can have the same name if their signatures are different.
      \end{solution}