\model{Two C++ Functions and a Main Program}
  \begin{center}
    \begin{tabular}{p{2.5in}p{0.0025in}p{4in}}
      \begin{minipage}{2.5in}
        \begin{cpplst}
int addOneA(int x) {
  x = x + 1;
  return x;
}

int addOneB(int &x) {
  x = x + 1;
  return x;
}

        \end{cpplst}
      \end{minipage}
      & &
      \begin{minipage}{4in}
        \begin{cpplst}
int main() {
  int a = 1;
  cout << "Function: " << addOneA(a) << ", ";
  cout << "Argument: " << a << endl;
  
  int b = 1;
  cout << "Function: " << addOneB(b) << ", ";
  cout << "Argument: " << b << endl;
}
        \end{cpplst}
      \end{minipage}
    \end{tabular}
  \end{center}

  {\it\large Refer to Model 2 above as your team develops consensus answers
    to the questions below.}

  \quest{20 min}

  \Q Without running them, determine what the functions {\tt addOneA} and 
    {\tt addOneB} do.
    \begin{answer}[0.35in]
      They both add one to the parameter and then return that value.
    \end{answer}
    
  \Q How do the function headers differ between these two functions?
    \begin{answer}[0.35in]
      The second function header has an {\tt \& }before the parameter name.
    \end{answer}
    
  \Q The code for this model can be found in {\tt activity13a.cpp}.  Run it and
    record the output.
    \begin{answer}[0.75in]
      The output is:\par
      Function: 2, Argument 1\\
      Function: 2, Argument 2
    \end{answer}
    
  \Q In this class we will see two ways to pass arguments to a parameter in C++.
    \begin{itemize}
      \itemsep 10pt
      \item When an argument is {\it passed by value}, a copy of the argument
        value is made in the function parameter variable and any changes
        made to that parameter variable inside the function are thrown away 
        when the function is done.
      \item When an argument is {\it passed by reference}, the parameter
        variable is a reference to the actual argument variable outside
        the function.  Any changes made to the parameter variable inside 
        the function are actually made to the argument variable and persist
        when the function is done.
    \end{itemize}

  \Q Based on these two definitions, which function in this model uses pass by
  reference?
  \begin{answer}[0.5in]
    The function {\tt addOneB} since the value of {\tt b} in the {\tt main} program
    is changed but the value {\tt a} in the {\tt main} program is not.
  \end{answer}
    
  \Q By looking at the function header, how can you determine if an argument will
    be \key\\[-2.5mm] passed by value or passed by reference?
    \begin{answer}[0.5in]
      If the parameter name has a {\tt \&} in front of it, the argument will be
      passed by reference. Otherwise it will be passed by value.
    \end{answer}
  
  \Q Consider the following C++ function.
    \begin{center}
      \begin{minipage}{6in}
        \begin{cpplst}
// Normally sine and cosine expect angles given in radians  
// This function returns the sine and cosine of an angle in degrees
void getSinCos(double degrees, double &sinOut, double &cosOut) {
  const double PI = 3.14159265358979323846;
  double radians = degrees * PI / 180.0;
  sinOut = sin(radians);
  cosOut = cos(radians);
}
        \end{cpplst}
      \end{minipage}        
    \end{center}
      
    \begin{enumerate}
      \itemsep 10pt
      \item Which parameter(s) take pass by value arguments? \hfill
        \ans[2in]{degrees}

      \item Which parameter(s) take pass by reference arguments? \hfill
         \ans[2in]{sinOut and cosOut}

      \item What is the return type of this function? \hfill
         \ans[2in]{void}

      \item What does the comment mean when it says ``This function returns the sine
        and cosine{\ldots}?''          
        \begin{answer}[0.5in]
          It means that the function changes the values of the argument variables
          referred to by {\tt sinOut} and {\tt cosOut}.
        \end{answer}
        
      \item Why couldn't we use a \cpp{return} statement at the end of
        the function to return the sine and cosine?
        \begin{answer}[0.5in]
          Because a function can only return one value.
        \end{answer}
    \end{enumerate}

  \newpage
    
  \Q The function below is meant to swap the values of the integer argument variables
    passed in to it.
    \begin{center}
      \begin{tabular}{p{4in}p{2.75in}}
        \begin{minipage}{4in}
          \small
          \begin{cpplst}
/* function header */ {
  int temp = x; // save x in temporary var
  x = y;        // replace x with y
  y = temp;     // put original x value in y
}

int main() {
  int a = 2;
  int b = 3;
  swap(a,b);
  cout << a << ", " << b << endl;
}
        \end{cpplst}
      \end{minipage}        
      &
      \begin{minipage}{2.75in}
        \begin{enumerate}
          \item Write a function header for line 1 that makes the program output {\tt 2,3}.
            \begin{answer}[0.65in]
              \cpp{void swap(int x, int y)}
            \end{answer}
          \item Write a function header for line 1 that makes the program output {\tt 3,2}.
            \begin{answer}[0.65in]
              \cpp{void swap(int &x, int &y)}
            \end{answer}
          \end{enumerate}
        \end{minipage}
      \end{tabular}
    \end{center}
      
    \Q Using the function header from (b), which function calls below 
      are valid?  Check all that apply.
      \begin{enumerate}[i.]
        \begin{multicols}{3}
          \item \ans[0.25in]{\checkmark} \cpp{swap(b,a)}
          \item \ans[0.25in]{} \cpp{swap(a,2)}
          \item \ans[0.25in]{} \cpp{swap(3,2)}
        \end{multicols}
      \end{enumerate}