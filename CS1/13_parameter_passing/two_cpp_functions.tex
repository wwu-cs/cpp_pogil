{\bf\large Model 2: Two C++ Functions and a Main Program} \\[-20pt]
  \begin{center}
    \begin{tabular}{p{2in}p{0.25in}p{3.4in}}
      \begin{minipage}{2in}
        \begin{minted}[
          frame=lines,
          framesep=2mm,
          bgcolor=gray!15,
          baselinestretch=1.2,
          linenos,
          firstnumber=4
        ]{cpp}
int addOneA(int x) {
  x = x + 1;
  return x;
}

int addOneB(int &x) {
  x = x + 1;
  return x;
}

        \end{minted}
      \end{minipage}
      & &
      \begin{minipage}{3.4in}
        \begin{minted}[
          frame=lines,
          framesep=2mm,
          bgcolor=gray!15,
          baselinestretch=1.2,
          linenos,
          firstnumber=14
        ]{cpp}
int main() {
  int a = 1;
  cout << "Function: " << addOneA(a) << ", ";
  cout << "Argument: " << a << endl;
  
  int b = 1;
  cout << "Function: " << addOneB(b) << ", ";
  cout << "Argument: " << b << endl;
}
        \end{minted}
      \end{minipage}
    \end{tabular}
  \end{center}

  {\it\large Refer to Model 2 above as your team develops consensus answers
    to the questions below.}

    \item Without running them, determine what the functions {\tt addOneA} and 
      {\tt addOneB} do.
      \ifprintanswers\vskip -20pt\null\fi
      \begin{solution}[0.35in]
        They both add one to the parameter and then return that value.
      \end{solution}
      \ifprintanswers\vskip -35pt\null\fi
      
    \item How do the function headers differ between these two functions?
      \ifprintanswers\vskip -20pt\null\fi
      \begin{solution}[0.35in]
        The second function header has an {\tt \& }before the parameter name.
      \end{solution}
      \ifprintanswers\vskip -35pt\null\fi

    \item The code for this model can be found in {\tt activity13b.cpp}.  Run it and
      record the output.
      \ifprintanswers\vskip -20pt\null\fi
      \begin{solution}[0.75in]
        The output is:\par
        Function: 2, Argument 1\\
        Function: 2, Argument 2
      \end{solution}
      \ifprintanswers\vskip -35pt\null\fi
      
    \item In this class we will see two ways to pass arguments to a parameter in C++.
      \begin{itemize}
        \itemsep 5pt
        \item When an argument is {\it passed by value}, a copy of the argument
          value is made in the function parameter variable and any changes
          made to that parameter variable inside the function are thrown away 
          when the function is done.
        \item When an argument is {\it passed by reference}, the parameter
          variable is a reference to the actual argument variable outside
          the function.  Any changes made to the parameter variable inside 
          the function are actually made to the argument variable and persist
          when the function is done.
      \end{itemize}
      Based on these two definitions, which function in this model uses pass by
      reference?
      \begin{solution}[0.5in]
        The function {\tt addOneB} since the value of {\tt b} in the {\tt main} program
        is changed but the value {\tt a} in the {\tt main} program is not.
      \end{solution}
      \ifprintanswers\vskip -55pt\else\vskip -30pt\fi\null
      
    \item By looking at the function header, how can you determine if an argument will
      be passed \key\\[-2.5mm] by value or passed by reference?
      \begin{solution}[0.5in]
        If the parameter name has a {\tt \&} in front of it, the argument will be
        passed by reference.  Otherwise it will be passed by value.
      \end{solution}
    
\newpage

    \item Consider the following C++ function.
      \begin{center}
        \begin{minipage}{5in}
          \begin{minted}[
            frame=lines,
            framesep=2mm,
            bgcolor=gray!15,
            baselinestretch=1.2
          ]{cpp}
// Normally sine and cosine expect angles given in radians  
// This function returns the sine and cosine of an angle in degrees
void getSinCos(double degrees, double &sinOut, double &cosOut) {
  const double PI = 3.14159265358979323846;
  double radians = degrees * PI / 180.0;
  sinOut = sin(radians);
  cosOut = cos(radians);
}
          \end{minted}
        \end{minipage}        
      \end{center}
      
      \begin{enumerate}[(a)]
        \itemsep 15pt
        \item Which parameter(s) take pass by value arguments? \hfill
          \fillin[{\tt degrees}][2in]
        \item Which parameter(s) take pass by reference arguments? \hfill
          \fillin[{\tt sinOut} and {\tt cosOut}][2in]
        \item What is the return type of this function? \hfill
          \fillin[\mintinline{cpp}|void|][2in]
        \item What does the comment mean when it says ``This function returns the sine
          and cosine{\ldots}?''          
          \begin{solution}[0.5in]
            It means that the function changes the values of the argument variables
            referred to by {\tt sinOut} and {\tt cosOut}.
          \end{solution}
          \ifprintanswers\vskip -25pt\null\fi          
        \item Why couldn't we use a \mintinline{cpp}|return| statement at the end of
          the function to return the sine and cosine?
          \ifprintanswers\vskip -10pt\null\fi
          \begin{solution}[0.5in]
            Because a function can only return one value.
          \end{solution}
          \ifprintanswers\vskip -25pt\null\fi
      \end{enumerate}
      
      
    \item The function below is meant to swap the values of the integer argument variables
      passed in to it.
      \ifprintanswers\vskip -25pt\null\fi
      \begin{center}
        \begin{tabular}{p{3in}p{2.75in}}
          \begin{minipage}{3in}
            \small
            \begin{minted}[
              frame=lines,
              framesep=2mm,
              bgcolor=gray!15,
              baselinestretch=1.2,
              linenos,
            ]{cpp}
/* function header */ {
  int temp = x; // save x in temporary var
  x = y;        // replace x with y
  y = temp;     // put original x value in y
}

int main() {
  int a = 2;
  int b = 3;
  swap(a,b);
  cout << a << ", " << b << endl;
}
            \end{minted}
          \end{minipage}        
          &
          \begin{minipage}{2.75in}
            \begin{enumerate}[(a)]
              \item Write a function header for line 1 that makes the program output {\tt 2,3}.
                \begin{solution}[0.65in]
                  \mintinline{cpp}|void swap(int x, int y)|
                \end{solution}
              \item Write a function header for line 1 that makes the program output {\tt 3,2}.
                \begin{solution}[0.65in]
                  \mintinline{cpp}|void swap(int &x, int &y)|
                \end{solution}
            \end{enumerate}
          \end{minipage}
        \end{tabular}
      \end{center}
      
      \begin{enumerate}[(a)]
        \setcounter{enumii}{2}
        \item Using the function header from (b), which function calls below 
          are valid?  Check all that apply.
          \begin{enumerate}[i.]
            \begin{multicols}{3}
              \item \fillin[$\checkmark$][0.25in] \mintinline{cpp}|swap(b,a)|
              \item \fillin[][0.25in] \mintinline{cpp}|swap(a,2)|
              \item \fillin[][0.25in] \mintinline{cpp}|swap(3,2)|
            \end{multicols}
          \end{enumerate}
      \end{enumerate}