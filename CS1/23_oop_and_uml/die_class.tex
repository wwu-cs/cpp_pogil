\model{The Die Class}
  \begin{center}
    \footnotesize
    \renewcommand{\arraystretch}{1.2}
    \begin{tabular}{p{1.5in}p{2.5in}p{0.8in}}
      \begin{minipage}{1.5in}
        \begin{tabular}{|l|}
          \hline
          \rowcolor{orange!20}\multicolumn{1}{|c|}{\bf Die}\\
          \hline
          \rowcolor{white} \tt -face: int\\
          \hline
          \tt +Die()\\
          \tt +getFace(): int\\
          \tt +roll(): int\\
          \hline
        \end{tabular}
      \end{minipage}
      & 
      \begin{minipage}{2.5in}
        \begin{cpplst}
class Die {
  public:
    Die() { this->face = 1; }
    int getFace() const {
      return this->face;
    }
    int roll() {
      this->face = rand() % 6 + 1;
      return this->face;
    }
  private:
    int face;
};
        \end{cpplst}   
      \end{minipage}
      &
      \begin{minipage}{0.8in}
        \centering
        \includegraphics[width=1in]{../../figures/dice.png}
      \end{minipage}
    \end{tabular}
  \end{center}

  {\it\large Refer to Model 1 above as your team develops consensus answers
    to the questions below.}

  \quest{20 min}

  \Q In the {\it Unified Modeling Language} (UML), a class diagram
    (on the left in our model) provides a way of graphically 
    illustrating a class's design, independent of programming language.
    \begin{enumerate}
      \item What are the attributes of {\tt Die} and what are its
        methods?
        \begin{answer}[1in]
          The only attribute is {\tt face} and the methods are {\tt
          Die}, {\tt roll}, and {\tt getFace}.
        \end{answer}

      \item How are attributes and methods distinguished in the
        class diagram?
        \begin{answer}[1in]
          The attributes are in the first block of the diagram below
          the title.  The methods are in the second block after the
          attributes.
        \end{answer}
    \end{enumerate}

  \vskip -20pt

  \Q Notice that several special symbols are used in the class
    diagram.  Compare the diagram with the given C++ implementation
    of the class to help you answer the following questions.
    \begin{enumerate}
      \item In the class diagram, what do the ``{\tt -}'' and
        ``{\tt +}'' symbols represent?
        \begin{answer}[0.75in]
          The plus means that the attribute or method is public.
          The minus means it is private. 
        \end{answer}

      \item What does the ``{\tt :}'' represent?
        \begin{answer}[0.75in]
          The colon refers to the data type.
        \end{answer}
    \end{enumerate}

  \vskip -20pt

  \Q The file {\tt activity23a.cpp} contains the class from this
    model along with an example {\tt main} program that utilizes it.    
    \begin{enumerate}
      \item What would you change to make this a 5-sided die?        
        \begin{answer}[0.75in]
          You would change the code on line 12 to read
          \begin{center}
            \cpp{this->face = rand() \% 5 + 1;}
          \end{center}
        \end{answer}

      \item How would you change the class diagram to reflect this
        change in the code?
        \begin{answer}[0.75in]
          You would not change it at all.
        \end{answer}
    \end{enumerate}
    \vskip -40pt\null

  \Q Suppose you wanted to generalize this class so that it could
    represent a die with any\key\\[-2.5mm] number of sides, set by the constructor.  
    \begin{enumerate}
      \item What new attribute(s) and/or method(s) would you need to add?
        \begin{answer}[0.75in]
          We would need to add an attribute for {\tt numFaces}.
        \end{answer}

      \item What methods would you need to alter?
        \begin{answer}[0.75in]
          We would need to alter the constructor to set the number
          of faces, and we would need to change the {\tt roll} method
          to use the number of faces in computing a random value.
        \end{answer}

      \item What would the class diagram look like after these changes?
        \begin{answer}[0.75in]
          \begin{center}
            \begin{tabular}{|l|}
              \hline
              \rowcolor{orange!20}\multicolumn{1}{|c|}{\bf Die}\\
              \hline
              \rowcolor{white} \tt -face: int\\
              \tt -numFaces: int\\
              \hline
              \tt +Die(numFaces: int)\\
              \tt +getFace(): int\\
              \tt +roll(): int\\
              \hline
            \end{tabular}
          \end{center}
        \end{answer}
    \end{enumerate}