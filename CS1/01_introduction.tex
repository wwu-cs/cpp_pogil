\documentclass{exam}
%\documentclass[answers]{exam}
\hbadness=99999
\setlength{\textheight}{9.5in}
\setlength{\textwidth}{6.5in}
\setlength{\topmargin}{-0.75in}
\setlength{\oddsidemargin}{0in}
\setlength{\evensidemargin}{0in}

\usepackage{amsmath}
%\usepackage{amsfonts}
\usepackage{amssymb}
\usepackage{enumerate}
\usepackage[table]{xcolor}
\usepackage{graphicx}
\usepackage{tikz}
%\usepackage{pgfplots}
\usepackage{multicol}

% for syntax highlighting
\usepackage{minted}
\usemintedstyle[cpp]{xcode}

% for overlay of output
\usepackage[overlay,showboxes]{textpos}

\pagestyle{plain}

\setlength\columnsep{50pt}
\newcommand{\key}{\hfill
      \raisebox{-.3\height}{\includegraphics[width=0.6in]{figures/key.png}}}

\begin{document}
  \thispagestyle{empty}
  \setlength{\parindent}{0pt}

  \begin{center}
    \Large Activity \#1: Introduction to C++ \\[5pt]
    \large Recorder's Report\\[20pt]
    \normalsize
    \begin{tabular}{lrp{0.1in}lr}
      Manager:  & \fillin[][2.0in] & & Presenter: & \fillin[][2.0in]\\[15pt]
      Recorder: & \fillin[][2.0in] & & Driver:    & \fillin[][2.0in]\\[15pt]
      Date:     & \fillin[][2.0in] & & Score:     & Satisfactory \hspace{10pt} /
      \hspace{10pt} Not Satisfactory
    \end{tabular}
  \end{center}
  \par\vskip 15pt
  
  Record your group's answers to the key questions (marked with
  \raisebox{-.3\height}{\includegraphics[width=0.5in]{figures/key.png}})
  below.
  \begin{enumerate}[(a)]
    \itemsep 1.75in
    \item Model 1, Question \#5
    \item Model 2, Question \#9
    \item Model 3, Question \#14
  \end{enumerate}

  \clearpage\pagenumbering{arabic} 
  
  \begin{center}
    \Large Activity \#1: Introduction to C++ \\[5pt]
    \large Activity Guide\\[20pt]
  \end{center}

  \begin{center}
    \fbox{
      \begin{minipage}{5.5in}
        {\bf Learning Objectives:} Students will be able to:
        \begin{itemize}
          \item Content:\\[-20pt]
            \begin{itemize}
              \itemsep 0pt
              \item Explain how to print content to the screen using C++
              \item Explain how to create a comment in C++ code
              \item Determine the difference between a {\it string literal} and a {\it integer}
              \item Explain how to input data into a variable in C++
              \item Explain the meaning and purpose of a variable
            \end{itemize}
          \item Process\\[-20pt]
            \begin{itemize}
              \itemsep 0pt
              \item Create input and output statements in C++
              \item Create C++ code that displays the results to calculated addition facts
              \item Create C++ code that prompts the user for data and stores it in a variable
              \item Select valid and meaningful variable names
              \item Discuss problems and programs with all group members\\[-5pt]
            \end{itemize}
        \end{itemize}
      \end{minipage}
      }
  \end{center}
  \par\vskip 10pt
  
  
  {\bf\large Model 1: A C++ Program} \\
  \begin{center}
    \begin{minipage}{3.5in}
      \begin{minted}[
        frame=lines,
        framesep=2mm,
        bgcolor=gray!15,
        baselinestretch=1.2,
        linenos
      ]{cpp}
#include <iostream>
using namespace std;

int main() {
  cout << "Go!$\backslash$n";
}      
      \end{minted}
    \end{minipage}
  \end{center}
  \par\vskip 10pt
  
  {\it\large Refer to Model 1 above as your group develops consensus answers
    to the questions below.}
    \par\vskip 10pt
    
  \begin{enumerate}
    \itemsep 20pt
    
    \item The file {\tt activity01a.cpp} contains this C++ program. Run it and determine what it does.
      \begin{solution}[0.5in]
        It prints out the text {\it Go!}
      \end{solution}
      
    \item Replace line 5 with each of the following.  What is produced?  Indicate if
      there is a problem.\par\vskip 20pt      
      \begin{enumerate}[(a)]
        \itemsep 10pt
        \item {\tt cout << "Hello, my name is Jon";} \hfill
          \fillin[It prints {\it Hello, my name is Jon}][3in]
        \item {\tt cout << Hello, my name is Jon;} \hfill
          \fillin[It produces an error][3in]
        \item {\tt cout << "Hello,$\backslash$nmy name is Jon";} \hfill
          \fillin[Same as (a), but with a line break][3in]
      \end{enumerate}
      
\newpage      
      
    \item A {\it string literal} is a sequence of characters surrounded by double
      quotation marks (" ").  Use this term to describe the difference between part (a)
      and part (b) above.
      \begin{solution}[1in]
        Part (a) prints out the string literal "Hello, my name is Jon" while in part (b)
        the lack of quotation marks confuses the compiler.
      \end{solution}
      
    \item What caused the different output format for statements (a) and (c) in question 2?
      \begin{solution}[1in]
        The $\backslash$n in part (c) caused a line break instead of the space we
        got in part (a)
      \end{solution}

    \item What do you think the following C++ statements output?\key\\[-2.5mm]
      \par\vskip 20pt
      \begin{enumerate}[(a)]
        \itemsep 10pt
        \item {\tt cout << 2+5;} \hfill
          \fillin[It prints {\it 7}][2.5in]
        \item {\tt cout << 2*5;} \hfill
          \fillin[It prints {\it 10}][2.5in]
        \item {\tt cout << "2+5";} \hfill
          \fillin[It prints {\it 2+5}][2.5in]
        \item {\tt cout << "Age:" << 20;} \hfill
          \fillin[It prints {\it Age: 20}][2.5in]
      \end{enumerate}
              
    \item With reference to the output for each statement in question 5,
      \begin{enumerate}[(a)]
        \itemsep 10pt
        \item What is the difference in the output for statements (a) and (c)?
          \begin{solution}[1in]
            Statement (a) prints computes the value of 2+5 and prints it out, while
            statement (c) prints out the literal characters {\it 2+5}.
          \end{solution}          
        \item What caused the difference?
          \begin{solution}[0.5in]
            The quotation marks around the {\it 2+5} in statement (c) made it into a
            string literal.
          \end{solution}
        \item Which statement(s) contained a {\it string literal}? Explain. \hfill
          \fillin[Statement (c), it has quotes around it][2.5in]
        \item What does the extra {\tt <<} do in part (d) of question 4?  How does it
          affect the spacing of the output?
          \begin{solution}[1in]
            \par
            It outputs the second item right after the first, resulting in {\it Age:} and
            {\it 20} being put together.\par
            It does not add any extra space between the two items.
          \end{solution}
      \end{enumerate}
      
  \end{enumerate}

  \newpage

  {\bf\large Model 2: Another C++ Program} \\
  \begin{center}
    \begin{minipage}{4.5in}
      \begin{minted}[
        frame=lines,
        framesep=2mm,
        bgcolor=gray!15,
        baselinestretch=1.2,
        linenos
      ]{cpp}
// Programmer: Jane Doe
// Date: 9/19/2019
// Description: This program prints a welcome statement
#include <iostream>
using namespace std;

int main() {
  cout << "Hello, Jane\n";
  cout << "Enjoy programming in C++\n";
}      
      \end{minted}
    \end{minipage}
  \end{center}
  \TPMargin{5pt}
  \begin{textblock*}{2.75in}[0,0](3.5in,-1.5in)
    \textblockcolor{white}
    \begin{minipage}{2.5in}
    {\bf Output:} 
    \hrule\vskip 5pt
    Hello, Jane\\
    Enjoy programming in C++
    \end{minipage}
  \end{textblock*}
  \par\vskip 10pt
  
  {\it\large Refer to Model 2 above as your group develops consensus answers
    to the questions below.}
    \par\vskip 10pt
    
    \begin{enumerate}
      \itemsep 10pt
      \setcounter{enumi}{6}
      \item What do the first three lines of this program do?
        \begin{solution}[0.5in]
          The do not do anything.
        \end{solution}
        
      \item What would happen if placed {\tt //} in front of the code: \hspace{5pt}
        \mintinline{cpp}|cout << "Hello, Jane\n"|\hspace{5pt} in this program?
        \begin{solution}[0.75in]
          The output would now only say {\it Enjoy programming in C++}
        \end{solution}
        
      \item A {\it comment} is text in a program that explains
        or annotates the source code, but does  \key\\[-2.5mm] not affect how the program runs.
        Complete the statements on lines 6-7 so that the C++ program below
        produces the desired output.  Add a comment on line 4.  This
        code is in the file {\tt activity01b.cpp}
        
        
  \begin{center}
    \begin{minipage}{4.5in}
      \begin{minted}[
        frame=lines,
        framesep=2mm,
        bgcolor=gray!15,
        baselinestretch=1.2,
        linenos
      ]{cpp}
#include <iostream>
using namespace std;

int main() {

  cout <<                               
  cout <<                               
  cout <<                               
}      
      \end{minted}
    \end{minipage}
  \end{center}
  \TPMargin{5pt}
  \begin{textblock*}{2.2in}[0,0](4.5in,-2in)
    \textblockcolor{white}
    \begin{minipage}{2in}
    {\bf Output:} 
    \hrule\vskip 5pt
    Congratulations!\\
    You just created\\
    your first C++ program
    \end{minipage}
  \end{textblock*}

  \end{enumerate}

\newpage
          
  {\bf\large Model 3: Collecting Input} \\
  \begin{center}
    \begin{minipage}{3.5in}
      \begin{minted}[
        frame=lines,
        framesep=2mm,
        bgcolor=gray!15,
        baselinestretch=1.2,
        linenos
      ]{cpp}
#include <iostream>
#include <string>
using namespace std;

int main() {
  string name;
  cout << "What is your name? ";
  cin >> name;
  cout << "Your name is " << name;
}      
      \end{minted}
    \end{minipage}
  \end{center}
  \par\vskip 10pt
  
  {\it\large Refer to Model 3 above as your group develops consensus answers
    to the questions below.}
    \par\vskip 10pt

  \begin{enumerate}
    \itemsep 10pt
    \setcounter{enumi}{9}
    \item You will find this code in the file {\tt activity01c.cpp}.  Execute it and
      determine what the program does.
      \begin{solution}[1in]
        It prompts the user for their name, for example ``Jon'', and then prints out {\it
        Your name is Jon}.
      \end{solution}
    
    \item The word {\it name} in this code identifies a {\it variable} (a name given to a
      memory location used to store data).  What happens to the data that the user of
      this program enters?
      \begin{solution}[0.5in]
      \end{solution}
      
    \item In C++, each variable has a {\it type} which specifies what sort of data it can
      store.  What is the type of the variable in the program above?  What sort of data
      can it store?
      \begin{solution}[0.5in]
        It is a string.  That means it can store a string of arbitrary characters.
      \end{solution}

    \item Suppose you wished to also store the users age.  Explain the errors that occur
      when you attempt to create the following {\it integer} (int) variables to this
      program.\par\vskip 15pt
      \begin{enumerate}[(a)]
        \itemsep 15pt
        \item \mintinline{cpp}|int age?;| \hfill 
          \fillin[not a valid variable name (no questionmarks allowed)][4.5in]
        \item \mintinline{cpp}|int your age;| \hfill 
          \fillin[not a valid variable name (no spaces allowed)][4.5in]
        \item \mintinline{cpp}|int 2age;| \hfill 
          \fillin[not a valid variable name (can't start with digit)][4.5in]
        \item \mintinline{cpp}|int int;| \hfill 
          \fillin[not a valid variable name (`int' is a keyword)][4.5in]
        \item \mintinline{cpp}|int the.age;| \hfill
          \fillin[not a valid variable name (no periods allowed)][4.5in]
      \end{enumerate}
      
\newpage      
      
    \item The following are valid variable names for age.  Based on this list, and
      the errors you \key\\[-2.5mm] found above, write two rules that valid variable names must
      follow in C++.
      \begin{center}
        \tt
        age \hspace{10pt}
        age2 \hspace{10pt}
        myAge \hspace{10pt}
        the\_age 
      \end{center}
      \begin{solution}[1in]
        \begin{itemize}
          \item May contain only characters a-z, a-Z, 0-9, and \_
          \item Must start with a letter or \_
        \end{itemize}
      \end{solution}

    \item Suppose you need a variable to store the cost of an item.  The following names
      are suggested.  Are they valid?  Are they good choices?
      \par\vskip 15pt
      \begin{enumerate}[(a)]
        \itemsep 15pt
        \item {\tt price} \hfill \fillin[yes to both][4.5in]
        \item {\tt priceoftheitem} \hfill \fillin[valid, probably to long][4.5in]
        \item {\tt x} \hfill \fillin[valid but not meaningful][4.5in]
        \item {\tt itemPrice} \hfill \fillin[yes to both][4.5in]
      \end{enumerate}
      \par\vskip 15pt
      
    \item Modify the C++ program found in {\tt activity01c.cpp} so that it prompts
      the user for two integers and then prints out the sum of those two integers
      as shown in the example output below.  Use meaningful variable names and 
      comments in your code.
      
      \begin{textblock*}{3.5in}[-0.5,0](0in,0.25in)
        \textblockcolor{white}
        \begin{minipage}{3.3in}
          {\bf Output:} 
          \hrule\vskip 5pt
          Enter first number: 7\\
          Enter second number: 12\\
          7 + 12 = 19
        \end{minipage}
      \end{textblock*}
      
      \par\vskip 1in
      \begin{solution}
        Answers will vary.
      \end{solution}

  \end{enumerate}
  
  
    
\end{document}
