\documentclass{exam}
%\documentclass[answers]{exam}
\hbadness=99999
\setlength{\textheight}{9.5in}
\setlength{\textwidth}{6.5in}
\setlength{\topmargin}{-0.75in}
\setlength{\oddsidemargin}{0in}
\setlength{\evensidemargin}{0in}

\usepackage{amsmath}
\usepackage{amssymb}
\usepackage{enumerate}
\usepackage[table]{xcolor}
\usepackage{hhline}
\usepackage{graphicx}
\usepackage{tikz}
%\usepackage{pgfplots}
\usepackage{multicol}
\usepackage{fancyvrb}

% for syntax highlighting
\usepackage{minted}
\usemintedstyle[cpp]{xcode}

% for overlay of output
\usepackage[overlay,showboxes]{textpos}

\pagestyle{plain}

\setlength\columnsep{50pt}
\newcommand{\key}{\hfill
      \raisebox{-.3\height}{\includegraphics[width=0.6in]{figures/key.png}}}

\begin{document}
  \thispagestyle{empty}
  \setlength{\parindent}{0pt}

  \begin{center}
    \Large Activity \#10: Vectors \\[5pt]
    \large Recorder's Report\\[20pt]
    \normalsize
    \begin{tabular}{lrp{0.1in}lr}
      Manager:  & \fillin[][2.0in] & & Presenter: & \fillin[][2.0in]\\[15pt]
      Recorder: & \fillin[][2.0in] & & Driver:    & \fillin[][2.0in]\\[15pt]
      Date:     & \fillin[][2.0in] & & Score:     & Satisfactory \hspace{10pt} /
      \hspace{10pt} Not Satisfactory
    \end{tabular}
  \end{center}
  \par\vskip 15pt
  
  Record your team's answers to the key questions (marked with
  \raisebox{-.3\height}{\includegraphics[width=0.5in]{figures/key.png}})
  below.
  \begin{enumerate}[(a)]
    \itemsep 1.75in
    \item Model 1, Question \#9
    \item Model 2, Question \#11 (final vector and accumulator only)
    \item Model 3, Question \#16.a
  \end{enumerate}

  \clearpage\pagenumbering{arabic} 
  
  \begin{center}
    \Large Activity \#10: Vectors \\[5pt]
    \large Activity Guide\\[20pt]
  \end{center}

  \begin{center}
    \fbox{
      \begin{minipage}{5.5in}
        {\bf Learning Objectives:} Students will be able to:
        \begin{itemize}
          \item Content:\\[-20pt]
            \begin{itemize}
              \itemsep 0pt
              \item Read and write nested {\tt for} loops.
              \item Identify inner and outer loops.
            \end{itemize}
          \item Process\\[-20pt]
            \begin{itemize}
              \itemsep 0pt
              \item Write code that uses nested {\tt for} loops. \\[-5pt]
            \end{itemize}
        \end{itemize}
      \end{minipage}
      }
  \end{center}
  \par\vskip 10pt
  

  {\bf\large Model 1: Vector Syntax} \\[-15pt]
  \begin{center}
    \begin{tabular}{p{2.5in}p{3.1in}}
      \begin{minipage}{2.5in}
        \scriptsize
        \begin{minted}[
          frame=lines,
          framesep=2mm,
          bgcolor=gray!15,
          baselinestretch=1.2,
          linenos,
          firstnumber=8
        ]{cpp}
  vector<int> quizScores = {8,6};
  vector<string> profNames(4);
  
  profNames.at(0) = "Carman";
  profNames.at(1) = "Foster";
  profNames.at(2) = "Duncan";
  
  cout << quizScores.size() << endl;
  cout << profNames.at(1) << endl;
        \end{minted}
      \end{minipage}
      &
      \begin{minipage}{3.1in}
        \null\par\vskip 50pt
        \begin{tabular}{p{0.8in}|c|c|c|}
          \hhline{~|-|-|-|}
          {\tt quizScores:} & 8 & 6 \\
          \hhline{~|-|-|-|}
          \multicolumn{1}{c}{} & \multicolumn{1}{c}{0} & \multicolumn{1}{c}{1} \\
        \end{tabular}
        \par\vskip 5pt
        \begin{tabular}{p{0.8in}|c|c|c|c|}
          \hhline{~|-|-|-|-|}
          {\tt profNames:} & Carman & Foster & Duncan & \\
          \hhline{~|-|-|-|-|}
          \multicolumn{1}{c}{} & \multicolumn{1}{c}{0} & \multicolumn{1}{c}{1} & \multicolumn{1}{c}{2} & \multicolumn{1}{c}{3} \\
        \end{tabular}
      \end{minipage}
    \end{tabular}
  \end{center}
  \TPMargin{5pt}
  \begin{textblock*}{1.5in}[0,0](2.6in,-1.5in)
    \textblockcolor{white}
    \begin{minipage}{1.35in}
      \scriptsize
      {\bf Output:} 
      \hrule\vskip 5pt
      2\\
      Foster
    \end{minipage}
  \end{textblock*}
  
  
  {\it\large Refer to Model 1 above as your team develops consensus answers
    to the questions below.}
    \par\vskip 10pt
    
  \begin{enumerate}
    \itemsep 20pt
    
    \item A {\it vector} is an ordered list of related variables, all of the
      same type.  Each value in a vector is known as an {\it element}.  The code
      above can be found in {\tt activity10a.cpp}.  Answer the following questions 
      about the vectors declared on line 8.
      \par\vskip 20pt
      
      \begin{enumerate}[(a)]
        \itemsep 15pt
        \item What is the name of this vector? \hfill
          \fillin[\tt quizScores][2.5in]
        \item What type of variable does this vector hold? \hfill
          \fillin[It is a vector of  \mintinline{cpp}|int|s][2.5in]
      \end{enumerate}
      
    \item A second vector is declared on line 9 in the model.  Answer
      the following questions about it.
      \par\vskip 20pt
      
      \begin{enumerate}[(a)]
        \itemsep 15pt
        \item What is the name of this vector? \hfill
          \fillin[\tt profNames][2.5in]
        \item What type of variable does this vector hold? \hfill
          \fillin[It is a vector of  \mintinline{cpp}|string|s][2.5in]
      \end{enumerate}
    
    \item The {\it size} of a vector is the number of variables that
      it holds.  What is the size of each vectors above? 
      \begin{solution}[0.5in]
        The vector declared on line 1 has size 2 and the vector
        declared on line 2 has size 4.
      \end{solution}
      
    \item Based on the code above, what are two different ways to
      initialize the variables stored in a vector?
      \begin{solution}[1in]
        \par
        You can set the values when the vector is declared
        using the \mintinline{cpp}|vector<type> name = { value_list }|
        syntax, or you can set each value using the
        \mintinline{cpp}|vectorName.at(index) = value| command.
      \end{solution}
      
      
\newpage

    \item The {\it index} of an element in a vector is its position in
      the vector.  Indexes start at 0 in C++.  Answer the following
      questions related to indexes.
      \par\vskip 20pt
      
      \begin{enumerate}[(a)]
        \itemsep 10pt
        \item What is stored at index 0 in the vector {\tt quizScores}?\hfill
          \fillin[8][1in]
        \item At what index is the name ``Duncan'' stored in the vector {\tt profNames}?\hfill
          \fillin[Index 2][1in]
        \item What is the maximum allowed index for the vector {\tt profNames}?\hfill
          \fillin[3][1in]
        \item What is stored at index 2 in the vector {\tt quizScores}?\hfill
          \fillin[\small Out of Bounds][1in]
      \end{enumerate}
                
    \item Write a single line of C++ code to declare a new vector of
      doubles named {\tt homeworkAvg} that contains the values 
      82.4 at index 2, 91.6 and index 0, and 73.9 at index 1.
      \begin{solution}[0.5in]
        \mintinline{cpp}|vector<double> homeworkAgv = {91.6, 73.9, 82.4};|
      \end{solution}
      
    \item The expression \mintinline{cpp}|myVector.at(i)| returns the value of the
      variable at index {\tt i} in the vector {\tt myVector}.  It can
      also be used to set the value at that index, as seen in lines 4-6
      of the model.
      
      \begin{enumerate}[(a)]
        \item Write a single C++ expression for the average of the two values in {\tt quizScores}.
          \begin{solution}[0.75in]
            \mintinline{cpp}|( quizScores.at(0) + quizScores.at(1) ) / 2.0|
          \end{solution}
        \item Write a sequence of C++ commands that changes the contents
          of the vector {\tt profNames} to those shown below without
          using the string literals \mintinline{cpp}|"Foster"| or
          \mintinline{cpp}|"Carman"|.
          \begin{center}
            \begin{tabular}{p{0.8in}|c|c|c|c|}
              \hhline{~|-|-|-|-|}
              {\tt profNames:} & Aamodt & Carman & Duncan & Foster \\
              \hhline{~|-|-|-|-|}
              \multicolumn{1}{c}{} & \multicolumn{1}{c}{0} & \multicolumn{1}{c}{1} & \multicolumn{1}{c}{2} & \multicolumn{1}{c}{3} \\
            \end{tabular}
          \end{center}
          \begin{solution}[1in]
            \scriptsize\vskip -35pt\null
            \begin{center}
              \begin{minipage}{2.25in}
                \begin{minted}[
                  frame=lines,
                  framesep=2mm,
                  bgcolor=gray!15,
                  baselinestretch=1.2,
                ]{cpp}
profNames.at(3) = profNames.at(1);
profNames.at(1) = profNames.at(0);
profNames.at(0) = "Aamodt";
                \end{minted}
              \end{minipage}
            \end{center}\vskip -20pt\null
          \end{solution}          
      \end{enumerate}
      
    \item The expression {\tt myVector.size()} returns the size of the
      vector {\tt myVector}.  Use this function to write a 
      {\tt cout} command for each line of output below.
      
      \begin{enumerate}[(a)]
        \item {\tt There are 4 CS professors at WWU.}
          \begin{solution}[0.5in]
            \par
            \mintinline{cpp}|cout << "There are " << profNames.size() << " CS professors at WWU." << endl;|            
          \end{solution}
        \item {\tt Total number of quiz points: 20} \hspace{10pt} (assume 10 points per quiz)
          \begin{solution}[0.5in]
            \par
            \mintinline{cpp}|cout << "Total number of quiz points: " << 10*quizScores.size() << endl;|
          \end{solution}
      \end{enumerate}
      \par\vskip -40pt\null
      
    \item True/False:  {\tt myVector.at(myVector.size())}
    returns the last value in the vector.\ifprintanswers\hspace{5pt}(false)\fi\key

\newpage
  
  {\bf\large Model 2: Two C++ Code Snippets} \\[-20pt]
  \begin{center}
    \begin{tabular}{p{2.7in}p{0.25in}p{2.7in}}
      \begin{minipage}{2.7in}
        \small
        \begin{minted}[
          frame=lines,
          framesep=2mm,
          bgcolor=gray!15,
          baselinestretch=1.2,
          linenos,
          firstnumber=8
        ]{cpp}
  // put first 100 powers of two in x
  vector<double> x(100);
  for (int i = 0; i < x.size(); i++) {
    x.at(i) = pow(2,i);
  }
        \end{minted}
      \end{minipage}
      & &
      \begin{minipage}{2.7in}
        \small
        \begin{minted}[
          frame=lines,
          framesep=2mm,
          bgcolor=gray!15,
          baselinestretch=1.2,
          linenos,firstnumber=17
        ]{cpp}
  // sum the elements of y
  int sum = 0;
  for (int i = 0; i < y.size(); i++) {
    sum += y.at(i);
  }
        \end{minted}
      \end{minipage}      
    \end{tabular}
  \end{center}
  \TPMargin{5pt}


  {\it\large Refer to Model 2 above as your team develops consensus answers
    to the questions below.}

    \item One of the biggest advantages of a vector is the ability to process them 
      using loops.  That is, to perform the same task for multiple elements.  The file
      {\tt activity10b.cpp} contains the loops from the model above.
      \par\vskip 15pt
      
      \begin{enumerate}[(a)]
        \item Find the value of each expression involving {\tt x} declared and initialized
          in the first code snippet.
          \par\vskip 20pt
          \begin{enumerate}[i.]
            \itemsep 15pt
            \begin{multicols}{2}
              \item {\tt x.at(1) + x.at(2)} \hspace{0.15in}
                \fillin[6][1in]
              \item {\tt x.at(2) * x.at(3)} \hspace{0.15in}
                \fillin[32][1in]
              \item {\tt x.at(4) / x.at(1)} \hfill
                \fillin[8][1in]
              \item {\tt x.at(x.at(2))} \hfill
                \fillin[16][1in]
            \end{multicols}
          \end{enumerate}
          \par\vskip 15pt
        \item How many values are saved in the vector {\tt y} in the second code snippet?  Does it matter?
          \begin{solution}[0.75in]
            We don't know, but it doesn't matter.  We can still sum them all.
          \end{solution}
      \end{enumerate}
      \par\vskip -30pt\null
      
    \item A {\it code trace} is a method for hand simulating the
      execution of your code in order to\key\\[-2.5mm] manually verify that it
      works before you compile it.  Fill in the table to trace the code below and determine the value of the {\tt data} vector
      and {\tt accumulator} variable after the code has finished.
      \par\vskip 20pt
      
      \begin{center}
        \begin{tabular}{p{2.5in}p{3in}}
          \begin{minipage}{2.5in}
            \scriptsize
            \begin{minted}[
              frame=lines,
              framesep=2mm,
              bgcolor=gray!15,
              baselinestretch=1.2,
              linenos
            ]{cpp}
vector<int> data = {5,26,13,12,37,15,16,4,1,3};
int accumulator = 0;
for (int i = 0; i < data.size(); i++) {
  if (data.at(i) % 2 == 1 && 
      i + 1 < data.size()) {
    data.at(i) *= -1;
    accumulator += data.at(i+1);
  }
}            
            \end{minted}            
            \par\vskip 20pt
            \begin{center}
              {\tt data} =
              \fillin[\tt \{-5,26,-13,12,-37,-15,16,4,-1,3\}][1.75in]\par\vskip 20pt
              {\tt accumulator} = \fillin[72][0.5in]
            \end{center}
          \end{minipage}
          &
          \begin{minipage}{3in}
            \begin{center}
              \renewcommand{\arraystretch}{1.6}\small
              \begin{tabular}{|c|c|c|}
                \hline
                \bf i & \bf\tt data.at(i) & \bf\tt accumulator \\
                \hline
                0 & \ifprintanswers -5\fi & \ifprintanswers 26\fi \\
                \hline
                1 & \ifprintanswers 26\fi & \ifprintanswers 26\fi \\
                \hline
                2 & \ifprintanswers -13\fi & \ifprintanswers 38\fi \\
                \hline
                3 & \ifprintanswers 12\fi & \ifprintanswers 38\fi \\
                \hline
                4 & \ifprintanswers -37\fi & \ifprintanswers 53\fi \\
                \hline
                5 & \ifprintanswers -15\fi & \ifprintanswers 69\fi \\
                \hline
                6 & \ifprintanswers 16\fi & \ifprintanswers 69\fi \\
                \hline
                7 & \ifprintanswers 4\fi & \ifprintanswers 69\fi \\
                \hline
                8 & \ifprintanswers -1\fi & \ifprintanswers 72\fi \\
                \hline
                9 & \ifprintanswers 3\fi & \ifprintanswers 72\fi \\
                \hline
              \end{tabular}
            \end{center}
          \end{minipage}
        \end{tabular}
      \end{center}

\newpage

    \item Suppose the vector \mintinline{cpp}|vector<double> a| and
      \mintinline{cpp}|vector<double> b| have been declared and filled
      with elements.  Write code to find the pairwise maximum value in
      these vectors and place it in a vector named {\tt myMax} which
      you declare.  So, for example, \mintinline{cpp}|myMax.at(0)| should
      contain the larger of \mintinline{cpp}|a.at(0)| and
      \mintinline{cpp}|b.at(0)| and so on.
      
      \begin{solution}[1.5in]
        \scriptsize\vskip -35pt\null
        \begin{center}
          \begin{minipage}{2.5in}
            \begin{minted}[
              frame=lines,
              framesep=2mm,
              bgcolor=gray!15,
              baselinestretch=1.2
            ]{cpp}
  vector<double> myMax(x.size());
  for (int i=0; i<a.size(); i++) {
    if (a.at(i) > b.at(i)) {
      myMax.at(i) = a.at(i);
    } else {
      myMax.at(i) = b.at(i);
    }
  }
            \end{minted}
          \end{minipage}
        \end{center}\vskip -20pt\null
      \end{solution}

    \item In a certain class 40\% of your final grade comes from your
      homework average and 60\% comes from your exam average.  Suppose 
      that vectors \mintinline{cpp}|vector<int> homeworkScores|
      and \mintinline{cpp}|vector<int> examScores| have been defined
      and contain your individual homework and exam scores.  Write 
      C++ code to compute your class grade and store it in
      \mintinline{cpp}|double finalGrade|.  Don't forget to typecast
      if needed!

      \begin{solution}[1.5in]
        \scriptsize\vskip -35pt\null
        \begin{center}
          \begin{minipage}{4.5in}
            \begin{minted}[
              frame=lines,
              framesep=2mm,
              bgcolor=gray!15,
              baselinestretch=1.2,
            ]{cpp}
  int homeworkSum = 0;
  for (int i=0; i<homeworkScores.size(); i++) {
    homeworkSum += homeworkScores.at(i);
  }
  int examSum = 0;
  for (int i=0; i<examScores.size(); i++) {
    examSum += examScores.at(i);
  }
  double homeworkGrade = static_cast<double>(homeworkSum) / homeworkScores.size();
  double examGrade = static_cast<double>(examSum) / examScores.size();
  double finalGrade = 0.40 * homeworkGrade + 0.60 * examGrade;
            \end{minted}
          \end{minipage}
        \end{center}\vskip -20pt\null
      \end{solution}


  {\bf\large Model 3: Useful Vector Operations} \\[-5pt]
  
  \begin{center}
    \small
    \renewcommand{\arraystretch}{1.4}    
    \begin{tabular}{|c|c|c|}
      \hline
      \rowcolor{orange!20} Original {\tt myVec} & Operation & Resulting {\tt myVec}  \\
      \hline
        \begin{minipage}{1.5in}
          \centering\vskip 10pt
          \begin{tabular}{r|c|c|c|}
            \hhline{~|-|-|-|}
            {\tt myVec:} & 8 & 2 & 6 \\
            \hhline{~|-|-|-|}
          \end{tabular}
          \vskip 5pt\null
        \end{minipage}
        &
        \mintinline{cpp}|myVec.resize(4)|
        &
        \begin{minipage}{1.5in}
          \centering\vskip 10pt
          \begin{tabular}{r|c|c|c|c|c|}
            \hhline{~|-|-|-|-|-|}
            {\tt myVec:} & 8 & 2 & 6 & \\
            \hhline{~|-|-|-|-|-|}
          \end{tabular}
          \vskip 5pt\null
        \end{minipage}
      \\ \hline
        \begin{minipage}{1.5in}
          \centering\vskip 10pt
          \begin{tabular}{r|c|c|c|}
            \hhline{~|-|-|-|}
            {\tt myVec:} & 8 & 2 & 6 \\
            \hhline{~|-|-|-|}
          \end{tabular}
          \vskip 5pt\null
        \end{minipage}
        &
        \mintinline{cpp}{myVec.push_back(4)}
        &
        \begin{minipage}{1.5in}
          \centering\vskip 10pt
          \begin{tabular}{r|c|c|c|c|}
            \hhline{~|-|-|-|-|}
            {\tt myVec:} & 8 & 2 & 6 & 4 \\
            \hhline{~|-|-|-|-|}
          \end{tabular}
          \vskip 5pt\null
        \end{minipage}
      \\ \hline
        \begin{minipage}{1.5in}
          \centering\vskip 10pt
          \begin{tabular}{r|c|c|c|}
            \hhline{~|-|-|-|}
            {\tt myVec:} & 8 & 2 & 6 \\
            \hhline{~|-|-|-|}
          \end{tabular}
          \vskip 5pt\null
        \end{minipage}
        &
        \mintinline{cpp}{myVec.pop_back()}
        &
        \begin{minipage}{1.5in}
          \centering\vskip 10pt
          \begin{tabular}{r|c|c|}
            \hhline{~|-|-|}
            {\tt myVec:} & 8 & 2 \\
            \hhline{~|-|-|}
          \end{tabular}
          \vskip 5pt\null
        \end{minipage}
      \\ \hline
        \begin{minipage}{1.5in}
          \centering\vskip 10pt
          \begin{tabular}{r|c|c|c|}
            \hhline{~|-|-|-|}
            {\tt myVec:} & 8 & 2 & 6 \\
            \hhline{~|-|-|-|}
          \end{tabular}
          \vskip 5pt\null
        \end{minipage}
        &
        \mintinline{cpp}{myVec.insert(myVec.begin()+1,9)}
        &
        \begin{minipage}{1.5in}
          \centering\vskip 10pt
          \begin{tabular}{r|c|c|c|c|}
            \hhline{~|-|-|-|-|}
            {\tt myVec:} & 8 & 9 & 2 & 6\\
            \hhline{~|-|-|-|-|}
          \end{tabular}
          \vskip 5pt\null
        \end{minipage}
      \\ \hline
        \begin{minipage}{1.5in}
          \centering\vskip 10pt
          \begin{tabular}{r|c|c|c|}
            \hhline{~|-|-|-|}
            {\tt myVec:} & 8 & 2 & 6 \\
            \hhline{~|-|-|-|}
          \end{tabular}
          \vskip 5pt\null
        \end{minipage}
        &
        \mintinline{cpp}{myVec.erase(myVec.end()-2)}
        &
        \begin{minipage}{1.5in}
          \centering\vskip 10pt
          \begin{tabular}{r|c|c|}
            \hhline{~|-|-|}
            {\tt myVec:} & 8 & 6 \\
            \hhline{~|-|-|}
          \end{tabular}
          \vskip 5pt\null
        \end{minipage}
      \\ \hline
        \begin{minipage}{1.5in}
          \centering\vskip 10pt
          \begin{tabular}{r|c|c|c|}
            \hhline{~|-|-|-|}
            {\tt myVec:} & 8 & 2 & 6\\
            \hhline{~|-|-|-|}
          \end{tabular}
          \vskip 5pt\null
        \end{minipage}
        &
        \mintinline{cpp}{myVec.clear()}
        &
        \begin{minipage}{1.5in}
          \centering\vskip 10pt
          \begin{tabular}{r|}
            \hhline{~|}
            {\tt myVec:} \\
            \hhline{~|}
          \end{tabular}
          \vskip 5pt\null
        \end{minipage}
      \\ \hline      
    \end{tabular}
  \end{center}
      
  {\it\large Refer to Model 3 above as your team develops consensus answers
    to the questions below.}

\newpage

    \item In your own words, describe what each of the following
      vector operations does.
      \par\vskip 10pt
      
      \begin{enumerate}[(a)]
        \item \mintinline{cpp}|myVec.resize(n)|\ifprintanswers\vskip -8pt\fi
          \begin{solution}[0.5in]
            Changes the size of the vector to {\tt n}.
          \end{solution}
        \item \mintinline{cpp}|myVec.push_back(value)|\ifprintanswers\vskip -8pt\fi
          \begin{solution}[0.5in]
            Adds a value to the end of the vector, resizing it if
            necessary.
          \end{solution}
        \item \mintinline{cpp}|myVec.pop_back()|\ifprintanswers\vskip -8pt\fi
          \begin{solution}[0.5in]
            Removes a value from the end of the vector (and returns
            the value).
          \end{solution}
        \item \mintinline{cpp}|myVec.begin()|\ifprintanswers\vskip -8pt\fi
          \begin{solution}[0.5in]
            Points to the beginning of the vector.
          \end{solution}
        \item \mintinline{cpp}|myVec.insert(location,value)|\ifprintanswers\vskip -8pt\fi
          \begin{solution}[0.5in]
            Inserts a value at the given location, moving everything
            over to make room.
          \end{solution}
        \item \mintinline{cpp}|myVec.end()|\ifprintanswers\vskip -8pt\fi
          \begin{solution}[0.5in]
            Identifies the end of the vector (one slot past the end
            actually).
          \end{solution}
        \item \mintinline{cpp}|myVec.erase(location)|\ifprintanswers\vskip -8pt\fi
          \begin{solution}[0.5in]
            Removes a value from the given location, shifting later
            values left.
          \end{solution}
        \item \mintinline{cpp}|myVec.clear()|\ifprintanswers\vskip -8pt\fi
          \begin{solution}[0.5in]
            Removes all values from the vector and sets its size to zero.
          \end{solution}
      \end{enumerate}
      
      
    \item Suppose the vector {\tt myVec} is defined as shown below.
      Sketch a diagram of the contents of the vector after each set of
      vector operations.  Start over with the original {\tt myVec} for
      each part of this problem.  This code can be found in {\tt activity10c.cpp}.
      \begin{center}
        \begin{tabular}{r|c|c|c|c|c|}
          \hhline{~|-|-|-|-|-|}
          {\tt myVec:} & 25 & 12 & 73 & 19 & 42 \\
          \hhline{~|-|-|-|-|-|}
        \end{tabular}
      \end{center}
      
      \begin{enumerate}[(a)]
        \item First set of commands\par
          \begin{center}
            \begin{tabular}{p{2.5in}p{3.0in}}
              \begin{minipage}{2.5in}
                \small
                \begin{minted}[
                  frame=lines,
                  framesep=2mm,
                  bgcolor=gray!15,
                  baselinestretch=1.2,
                  linenos,
                  firstnumber=10
                ]{cpp}
  myVec.pop_back();
  myVec.insert(myVec.begin()+2,55);
  myVec.erase(myVec.begin()+1);
                \end{minted}
              \end{minipage}
              &
              \begin{minipage}{2.5in}
                \begin{solution}
                  \begin{tabular}{r|c|c|c|c|c|}
                    \hhline{~|-|-|-|-|-|}
                    {\tt myVec:} & 25 & 55 & 73 & 19  \\
                    \hhline{~|-|-|-|-|-|}
                  \end{tabular}
                \end{solution}
              \end{minipage}
            \end{tabular}
          \end{center}
        
        \item Second set of commands        
          \begin{center}
            \begin{tabular}{p{2.5in}p{3.0in}}
              \begin{minipage}{2.5in}
                \small
                \begin{minted}[
                  frame=lines,
                  framesep=2mm,
                  bgcolor=gray!15,
                  baselinestretch=1.2,
                  linenos,
                  firstnumber=15
                ]{cpp}
  myVec.resize(7);
  myVec.at(5) = 31;
  myVec.insert(myVec.end()-3,90);
  myVec.push_back(27);
                \end{minted}
              \end{minipage}
              &
              \begin{minipage}{3.0in}
                \begin{solution}
                  {\tt myVec:}
                  \begin{tabular}{|c|c|c|c|c|c|c|c|c|}
                    \hhline{|-|-|-|-|-|-|-|-|-|}
                     25 & 12 & 73 & 19 & 90 & 42 & 31 & 0 & 27  \\
                    \hhline{|-|-|-|-|-|-|-|-|-|}
                  \end{tabular}
                \end{solution}
              \end{minipage}
            \end{tabular}
          \end{center}
          
      \end{enumerate}
      
\newpage      
      
    \item Suppose the vector {\tt myVec} is defined as shown below.
      Give a sequence of command that \key\\[-2.5mm] 
      would produce the given vectors below.
      \begin{center}
        \begin{tabular}{r|c|c|c|c|c|}
          \hhline{~|-|-|-|-|-|}
          {\tt myVec:} & x & A & c & Y & w \\
          \hhline{~|-|-|-|-|-|}
        \end{tabular}
      \end{center}
      \par\vskip 20pt

      \begin{enumerate}[(a)]
        \item \begin{tabular}{r|c|c|c|c|c|c|}
            \hhline{~|-|-|-|-|-|-|}
            {\tt myVec:} & A & h & c & n & w & V\\
            \hhline{~|-|-|-|-|-|-|}
          \end{tabular}
          \begin{solution}[1.75in]
            Answers may vary, but the following would work.
            \begin{center}
              \begin{minipage}{3.5in}
                \small
                \begin{minted}[
                  frame=lines,
                  framesep=2mm,
                  bgcolor=gray!15,
                  baselinestretch=1.2,
                ]{cpp}
  myVec.erase(myVec.begin());
  myVec.erase(myVec.begin()+2);
  myVec.insert(myVec.begin()+1,'h');
  myVec.insert(myVec.begin()+3,'n');
  myVec.push_back('V');
                \end{minted}
              \end{minipage}
            \end{center}
          \end{solution}

        \item \begin{tabular}{r|c|c|c|c|c|}
            \hhline{~|-|-|-|-|-|}
            {\tt myVec:} & x & A & w & Z\\
            \hhline{~|-|-|-|-|-|}
          \end{tabular}
          \begin{solution}[1.75in]
            Answers may vary, but the following would work.
            \begin{center}
              \begin{minipage}{3.5in}
                \small
                \begin{minted}[
                  frame=lines,
                  framesep=2mm,
                  bgcolor=gray!15,
                  baselinestretch=1.2,
                ]{cpp}
  myVec.erase(myVec.begin()+2);                
  myVec.erase(myVec.begin()+2);
  myVec.push_back('Z');
                \end{minted}
              \end{minipage}
            \end{center}
          \end{solution}

        \item \begin{tabular}{r|c|c|c|c|c|c|c|c|}
            \hhline{~|-|-|-|-|-|-|-|-|}
            {\tt myVec:} & F & x & c & Y & & w & & \\
            \hhline{~|-|-|-|-|-|-|-|-|}
          \end{tabular}
          \begin{solution}[1.75in]
            Answers may vary, but the following would work.
            \begin{center}
              \begin{minipage}{3.5in}
                \small
                \begin{minted}[
                  frame=lines,
                  framesep=2mm,
                  bgcolor=gray!15,
                  baselinestretch=1.2,
                ]{cpp}
  myVec.insert(myVec.begin(),'F');
  myVec.erase(myVec.begin()+2);
  myVec.insert(myVec.end()-1,' ');
  myVec.push_back(' ');
  myVec.push_back(' ');
                \end{minted}
              \end{minipage}
            \end{center}
          \end{solution}

      \end{enumerate}
      

  \end{enumerate}  
    
\end{document}
