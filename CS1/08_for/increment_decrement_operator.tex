 {\bf\large Model 3: Increment / Decrement Operators} \\[-5pt]
  \begin{center}
    \renewcommand{\arraystretch}{1.4}
    \begin{tabular}{|c|c|c|c|}
      \hline
      \rowcolor{orange!20} Initial {\tt x} Value & Statement & Final {\tt y} value & Final {\tt x} Value \\
      \hline
      2 & \mintinline{cpp}|y = x++;| & 2 & 3 \\
      \hline
      2 & \mintinline{cpp}|y = ++x;| & 3 & 3 \\
      \hline
      2 & \mintinline{cpp}|y = x--;| & 2 & 1 \\
      \hline
      2 & \mintinline{cpp}|y = --x;| & 1 & 1 \\
      \hline
    \end{tabular}
  \end{center}
  \par\vskip 10pt
  
  {\it\large Refer to Model 3 above as your team develops consensus answers
    to the questions below.}

    \item Based on the model above, describe what each operator does.
      \par\vskip 20pt
      
      \begin{enumerate}[(a)]
        \itemsep 15pt
        \item \mintinline{cpp}|++| \hspace{20pt} 
          \fillin[Adds one to the variable involved.][5.25in]
        \item \mintinline{cpp}|--| \hspace{20pt}
          \fillin[Subtracts one from the variable involved.][5.25in]
      \end{enumerate}

\newpage

    \item Rewrite the following {\tt for} loop to use an {\it increment operator}.
    
      \begin{center}
        \begin{tabular}{p{2.75in}p{2.75in}}
          \begin{minipage}{2.75in}
            \begin{minted}[
              frame=lines,
              framesep=2mm,
              bgcolor=gray!15,
              baselinestretch=1.2,
            ]{cpp}
  for (int i=5; i<=10; i+=1) {
    cout << i << " ";
  }
            \end{minted}
          \end{minipage}
          &
          \begin{minipage}{2.75in}
            \begin{solution}
              \scriptsize
              \begin{minted}[
                frame=lines,
                framesep=2mm,
                bgcolor=gray!15,
                baselinestretch=1.2,
              ]{cpp}
  for (int i=5; i<=10; i++) {
    cout << i << " ";
  }
              \end{minted}
            \end{solution}
          \end{minipage}                  
        \end{tabular}
      \end{center}


    \item Rewrite the following {\tt while} loop to use an {\it decrement operator}.
    
      \begin{center}
        \begin{tabular}{p{2.75in}p{2.75in}}
          \begin{minipage}{2.75in}
            \begin{minted}[
              frame=lines,
              framesep=2mm,
              bgcolor=gray!15,
              baselinestretch=1.2,
            ]{cpp}
  int counter = 10;            
  while (counter > 0) {
    cout << "text" << endl;
    counter = counter - 1;
  }
            \end{minted}
          \end{minipage}
          &
          \begin{minipage}{2.75in}
            \begin{solution}
              \scriptsize
              \begin{minted}[
                frame=lines,
                framesep=2mm,
                bgcolor=gray!15,
                baselinestretch=1.2,
              ]{cpp}
  int counter = 10;            
  while (counter > 0) {
    cout << "text" << endl;
    counter--;
  }
              \end{minted}
            \end{solution}
          \end{minipage}                  
        \end{tabular}
      \end{center}

    \item Why are increment and decrement operators especially useful for counters?
      \begin{solution}[0.5in]
        Because they are a compact way to add or subtract one to a counter.
      \end{solution}
    
    \item Note that there are two versions of the {\it increment operator}: a {\it pre-increment} version \key\\[-2.5mm]
      (i.e. {\tt ++x}) and a {\it post-increment} version (i.e. {\tt x++}).  The same is true of the decrement operator.
      Use the file {\tt activity08c.cpp} to assist as you complete the missing entries in the table below.
      
      \begin{center}
        \renewcommand{\arraystretch}{1.8}
        \begin{tabular}{|c|c|c|c|}
          \hline
          \rowcolor{orange!20} Initial {\tt x} Value & Expression & Expression Value & Final {\tt x} Value \\
          \hline
          5 & {\tt 3 + 2*(x++)} & \ifprintanswers 13\fi & 6 \\
          \hline
          3 & {\tt 5 - (++x) / 2} & 3 & \ifprintanswers 4\fi  \\
          \hline
          \ifprintanswers 1\fi & {\tt 6 * (--x)} & 0 & 0\\
          \hline
          -3 & \ifprintanswers {\tt 2+(x--)}\fi & -1 & -4\\
          \hline
        \end{tabular}
      \end{center}
      
    \item Based on your work above, what is the difference between the pre- and post- versions of these operators?
      \begin{solution}[0.5in]
        \par
        The pre- version changes the value of the variable before it is used in the expression.  The post- version changes the
        value after it has been used in the expression.
      \end{solution}

  \end{enumerate}  