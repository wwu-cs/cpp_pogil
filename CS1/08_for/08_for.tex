\documentclass{exam}
%\documentclass[answers]{exam}
\hbadness=99999
\setlength{\textheight}{9.5in}
\setlength{\textwidth}{6.5in}
\setlength{\topmargin}{-0.75in}
\setlength{\oddsidemargin}{0in}
\setlength{\evensidemargin}{0in}

\usepackage{amsmath}
%\usepackage{amsfonts}
\usepackage{amssymb}
\usepackage{enumerate}
\usepackage[table]{xcolor}
\usepackage{graphicx}
\usepackage{tikz}
%\usepackage{pgfplots}
\usepackage{multicol}

% for syntax highlighting
\usepackage{minted}
\usemintedstyle[cpp]{xcode}

% for overlay of output
\usepackage[overlay,showboxes]{textpos}

\pagestyle{plain}

\setlength\columnsep{50pt}
\newcommand{\key}{\hfill
      \raisebox{-.3\height}{\includegraphics[width=0.6in]{figures/key.png}}}

\begin{document}
  \thispagestyle{empty}
  \setlength{\parindent}{0pt}

  \begin{center}
    \Large Activity \#8: For Loops \\[5pt]
    \large Recorder's Report\\[20pt]
    \normalsize
    \begin{tabular}{lrp{0.1in}lr}
      Manager:  & \fillin[][2.0in] & & Presenter: & \fillin[][2.0in]\\[15pt]
      Recorder: & \fillin[][2.0in] & & Driver:    & \fillin[][2.0in]\\[15pt]
      Date:     & \fillin[][2.0in] & & Score:     & Satisfactory \hspace{10pt} /
      \hspace{10pt} Not Satisfactory
    \end{tabular}
  \end{center}
  \par\vskip 15pt
  
  Record your team's answers to the key questions (marked with
  \raisebox{-.3\height}{\includegraphics[width=0.5in]{figures/key.png}})
  below.
  \begin{enumerate}[(a)]
    \itemsep 1.75in
    \item Model 1, Question \#5
    \item Model 2, Question \#13 (write the whole code snippet)
    \item Model 3, Question \#18
  \end{enumerate}

  \clearpage\pagenumbering{arabic} 
  
  \begin{center}
    \Large Activity \#8: For Loops \\[5pt]
    \large Activity Guide\\[20pt]
  \end{center}

  \begin{center}
    \fbox{
      \begin{minipage}{5.5in}
        {\bf Learning Objectives:} Students will be able to:
        \begin{itemize}
          \item Content:\\[-20pt]
            \begin{itemize}
              \itemsep 0pt
              \item Explain the difference between a {\tt while} loop and a {\tt for} loop.
              \item Explain the syntax of a {\tt for} loop in C++.
              \item Explain how an {\bf accumulator} is used in a {\tt for} loop.
              \item Explain how the {\bf increment operator} and {\bf decrement operator} work.
            \end{itemize}
          \item Process\\[-20pt]
            \begin{itemize}
              \itemsep 0pt
              \item Write code that includes {\tt for} loops. \\[-5pt]
            \end{itemize}
        \end{itemize}
      \end{minipage}
      }
  \end{center}
  \par\vskip 10pt
  

  {\bf\large Model 1: Two C++ Programs} \\[-15pt]
  \begin{center}
    \small
    \begin{tabular}{p{2.5in}p{0.5in}p{2.5in}}
      \begin{minipage}{2.5in}
        \begin{minted}[
          frame=lines,
          framesep=2mm,
          bgcolor=gray!15,
          baselinestretch=1.2,
          linenos,
          firstnumber=7
        ]{cpp}
  string name;
  cout << "Enter your name: ";
  cin >> name;
  int x = 0;
  while (x < 10) {
    cout << name << endl;
    x = x + 1;
  }
  cout << "Nice to meet you!" << endl;
        \end{minted}
      \end{minipage}
      & &
      \begin{minipage}{2.5in}
        \begin{minted}[
          frame=lines,
          framesep=2mm,
          bgcolor=gray!15,
          baselinestretch=1.2,
          linenos,
          firstnumber=7
        ]{cpp}
  string name;
  cout << "Enter your name: ";
  cin >> name;
  for (int x = 0; x < 10; x += 1) {
    cout << name << endl;
  }
  cout << "Nice to meet you!" << endl;
        \end{minted}
      \end{minipage}
    \end{tabular}
  \end{center}
  \TPMargin{5pt}
  
  
  {\it\large Refer to Model 1 above as your team develops consensus answers
    to the questions below.}
    \par\vskip 10pt
    
  \begin{enumerate}
    \itemsep 20pt
    
    \item What is the output of each program?  Note that you saw the
      first program in the previous activity and {\tt activity08a.cpp}
      contains the second program.
      \begin{solution}[0.5in]
        The output of the two programs is identical -- the entered
        name is printed ten times.
      \end{solution}
      
    \item The loop shown on the right is called a {\tt for} loop.
      Identify the code that makes up each part of the loop and the line
      on which it appears.
      \par\vskip 15pt
      
      \begin{enumerate}[(a)]
        \itemsep 15pt
        \item The Initialization Statement: \hfill
          \fillin[\mintinline{cpp}|int x = 0| on line 10][4in]
        \item The Test Condition: \hfill
          \fillin[\mintinline{cpp}|x < 10| on line 10][4in]
        \item The Update Statement: \hfill
          \fillin[\mintinline{cpp}|x += 1| on line 10][4in]
      \end{enumerate}
    

    \item You should have noted above that both programs produce the
      same output.  Which is more concise?
      \begin{solution}[0.5in]
        The {\tt for} loop is more concise.
      \end{solution}
      
\newpage

    \item What output will each of the following code snippets
      produce?  You can use {\tt activity08a.cpp} to check your
      answers.
      \par\vskip 10pt
      \begin{enumerate}[(a)]
        \item \begin{tabular}{p{3in}p{2.5in}}
          \begin{minipage}{3in}
            \begin{minted}[
              frame=lines,
              framesep=2mm,
              bgcolor=gray!15,
              baselinestretch=1.2,
            ]{cpp}
  for (int i = 0; i < 5; i += 1) {
    cout << i << " ";
  }
            \end{minted}
          \end{minipage}
          &
          \fillin[\tt 1 2 3 4 5][2.5in]
        \end{tabular}
        
        \item \begin{tabular}{p{3in}p{2.5in}}
          \begin{minipage}{3in}
            \begin{minted}[
              frame=lines,
              framesep=2mm,
              bgcolor=gray!15,
              baselinestretch=1.2,
            ]{cpp}
  for (int i = 1; i < 5; i += 1) {
    cout << i << " ";
  }
            \end{minted}
          \end{minipage}
          &
          \fillin[\tt 1 2 3 4][2.5in]
        \end{tabular}

        \item \begin{tabular}{p{3in}p{2.5in}}
          \begin{minipage}{3in}
            \begin{minted}[
              frame=lines,
              framesep=2mm,
              bgcolor=gray!15,
              baselinestretch=1.2,
            ]{cpp}
  for (int i = 2; i <= 6; i += 1) {
    cout << i << " ";
  }
            \end{minted}
          \end{minipage}
          &
          \fillin[\tt 2 3 4 5 6][2.5in]
        \end{tabular}

        \item \begin{tabular}{p{3in}p{2.5in}}
          \begin{minipage}{3in}
            \begin{minted}[
              frame=lines,
              framesep=2mm,
              bgcolor=gray!15,
              baselinestretch=1.2,
            ]{cpp}
  for (int i = 2; i <= 6; i += 2) {
    cout << i << " ";
  }
            \end{minted}
          \end{minipage}
          &
          \fillin[\tt 2 4 6][2.5in]
        \end{tabular}
      \end{enumerate}
      \par\vskip -40pt\null
      
    \item Complete the missing code in the {\tt for} loops below so
      that they print the indicated output.\key\\[-2.5mm]

      \begin{enumerate}[(a)]
        \begin{multicols}{2}
          \item Even numbers from 100 to  200, inclusive\par
            \begin{minipage}{2.6in}
              \ifprintanswers
              \begin{minted}[
                frame=lines,
                framesep=2mm,
                bgcolor=gray!15,
                baselinestretch=1.5,
              ]{cpp}
  for (int i=100; i<=200; i+=1) {
    cout << i << " ";
  }
              \end{minted}
              \else\large
              \begin{minted}[
                frame=lines,
                framesep=2mm,
                bgcolor=gray!15,
                baselinestretch=1.5,
              ]{cpp}
  for (                     ) {
    cout << i << " ";
  }
              \end{minted}
              \fi
            \end{minipage}
          \item {\tt 5 4 3 2 1 0}\par
            \begin{minipage}{2.6in}
              \ifprintanswers
              \begin{minted}[
                frame=lines,
                framesep=2mm,
                bgcolor=gray!15,
                baselinestretch=1.5,
              ]{cpp}
  for (int i=5; i>=0; i-=1) {
    cout << i << " ";
  }
              \end{minted}
              \else\large
              \begin{minted}[
                frame=lines,
                framesep=2mm,
                bgcolor=gray!15,
                baselinestretch=1.5,
              ]{cpp}
  for (                     ) {
    cout << i << " ";
  }
              \end{minted}
              \fi
            \end{minipage}          
        \end{multicols}
      \end{enumerate}
      
  \item Based on your solutions above, answer the following questions about the {\tt for} loop.
    \begin{enumerate}[(a)]
      \item Why do we start a for loop with something like
        \mintinline{cpp}|int i=1| instead of \mintinline{cpp}|int i==1|?
        \begin{solution}[0.5in]
          We are assigning {\tt i} the initial value 1, not checking
          to see if it equals 1.
        \end{solution}
      \item Do you think we always need the \mintinline{cpp}|int| in
        front of the \mintinline{cpp}|i=1| in the {\tt for} loop
        initialization?
        \begin{solution}[0.5in]
          No.  We only need it if the variable {\tt i} has not already been declared.
        \end{solution}
      \item Is it better to use a {\tt for} loop when you know how
        many times the loop should execute ({\it counter-controlled}) or
        when you don't know ({\it sentinel-controlled})?
        \begin{solution}[0.5in]
          The {\tt for} loop is naturally a counter-controlled loop
          as a counter is built into it.
        \end{solution}
    \end{enumerate}

\newpage

  \item Rewrite the following {\tt while} loop as a {\tt for} loop that does the same thing.
      \begin{center}
        \begin{tabular}{p{2.75in}p{2.75in}}
          \begin{minipage}{2.75in}
            \begin{minted}[
              frame=lines,
              framesep=2mm,
              bgcolor=gray!15,
              baselinestretch=1.2,
            ]{cpp}
  int cnt = 20;            
  while (cnt >= 10) {
    cnt -= 2;
    cout << cnt << " ";
  }
            \end{minted}
          \end{minipage}
          &
          \begin{minipage}{2.75in}
            \begin{solution}
              \footnotesize
              \begin{minted}[
                frame=lines,
                framesep=2mm,
                bgcolor=gray!15,
                baselinestretch=1.2,
              ]{cpp}
  for (int cnt=18; cnt >= 8; cnt -= 2) {
    cout << cnt << " ";
  }
              \end{minted}
            \end{solution}
          \end{minipage}                  
        \end{tabular}
      \end{center}
      \ifprintanswers\else\par\vskip 10pt\fi
  
  {\bf\large Model 2: Another C++ Code Snippet} \\[-20pt]
  
  \begin{center}
    \small
    \begin{minipage}{5.5in}
      \begin{minted}[
        frame=lines,
        framesep=2mm,
        bgcolor=gray!15,
        baselinestretch=1.2,
        linenos,
        firstnumber=5
      ]{cpp}
  int number;
  int total = 0;
  for (int i = 0; i < 5; i += 1) {
    cout << "Enter a number: ";
    cin >> number;
    total += number;
  }
  cout << "The total is: " << total << endl;
      \end{minted}
    \end{minipage}
  \end{center}

  {\it\large Refer to Model 2 above as your team develops consensus answers
    to the questions below.}
    \par\vskip 5pt

    \item The code for this program is in {\tt
      activity08b.cpp}.  Run it and explain what the program does.
      \begin{solution}[1in]
        \par
        The program prompts the user for five numbers and then prints
        out the sum of those five numbers.
      \end{solution}
      
    \item Explain what each of the indicated lines of code from the model above does.
      \par\vskip 10pt
      \begin{enumerate}[(a)]
        \itemsep 15pt
        \item Line 6:    \hfill \fillin[Initializes the variable {\tt total} to zero][5.25in]
        \item Line 7:    \hfill \fillin[Sets up the {\tt for} loop to repeat five times.][5.25in]
        \item Line 10:   \hfill \fillin[Adds the user-entered number to the accumulated total][5.25in]
      \end{enumerate}
      
    \item An {\it accumulator} is a variable that stores the sum of a group of values.  Which variable 
      in this model is an accumulator?  Check all that apply.
      
      \begin{checkboxes}
        \begin{multicols}{3}
          \choice {\tt number}
          \correctchoice {\tt total}
          \choice {\tt i}
        \end{multicols}
      \end{checkboxes}

    \item Why is the variable {\tt total} initialized to zero in line 2 of the model?
      \begin{solution}[0.5in]
        \par
        Because we need to start the loop with nothing accumulated so that after the loop is
        done, total will contain just the sum of the five numbers entered.
      \end{solution}

    \item Would it be possible to use the same variable as both a counter and an accumulator? Explain.
      \begin{solution}[0.5in]
        \par
        No.  A counter must count how many times the loop executes, so 
        it can not accumulate a sum of a group of values.
      \end{solution}

    \item An accumulator can also store the product of a set of numbers.  How, if at all, would you \key\\[-2.5mm]
      change the following lines of the model to compute $5!$ ($5! = 1\times 2\times 3\times 4\times 5$ is called ``five factorial'').
      
      \begin{enumerate}[(a)]
        \item How would you change \mintinline{cpp}|int total = 0;| on line 6?
          \begin{solution}[0.5in]
            It would become \mintinline{cpp}|int total=1;| to initialize the accumulator to 1 instead of 0.
          \end{solution}
        \item How would you change \mintinline{cpp}|for (int i = 0; i < 5; i += 1)| on line 7?
          \begin{solution}[0.5in]
            This does not need to change.
          \end{solution}
        \item How would you change lines 8-9 in the model?
          \begin{solution}[0.5in]
            They would be deleted since user input is no longer needed.
          \end{solution}
        \item How would you change \mintinline{cpp}|total += number| on line 10 of the model?
          \begin{solution}[0.5in]
            It would become \mintinline{cpp}|total *= i+1| to multiply the accumulator by the counter.
          \end{solution}
      \end{enumerate}
      \par\vskip 20pt
            

  {\bf\large Model 3: Increment / Decrement Operators} \\[-5pt]
  \begin{center}
    \renewcommand{\arraystretch}{1.4}
    \begin{tabular}{|c|c|c|c|}
      \hline
      \rowcolor{orange!20} Initial {\tt x} Value & Statement & Final {\tt y} value & Final {\tt x} Value \\
      \hline
      2 & \mintinline{cpp}|y = x++;| & 2 & 3 \\
      \hline
      2 & \mintinline{cpp}|y = ++x;| & 3 & 3 \\
      \hline
      2 & \mintinline{cpp}|y = x--;| & 2 & 1 \\
      \hline
      2 & \mintinline{cpp}|y = --x;| & 1 & 1 \\
      \hline
    \end{tabular}
  \end{center}
  \par\vskip 10pt
  
  {\it\large Refer to Model 3 above as your team develops consensus answers
    to the questions below.}

    \item Based on the model above, describe what each operator does.
      \par\vskip 20pt
      
      \begin{enumerate}[(a)]
        \itemsep 15pt
        \item \mintinline{cpp}|++| \hspace{20pt} 
          \fillin[Adds one to the variable involved.][5.25in]
        \item \mintinline{cpp}|--| \hspace{20pt}
          \fillin[Subtracts one from the variable involved.][5.25in]
      \end{enumerate}

\newpage

    \item Rewrite the following {\tt for} loop to use an {\it increment operator}.
    
      \begin{center}
        \begin{tabular}{p{2.75in}p{2.75in}}
          \begin{minipage}{2.75in}
            \begin{minted}[
              frame=lines,
              framesep=2mm,
              bgcolor=gray!15,
              baselinestretch=1.2,
            ]{cpp}
  for (int i=5; i<=10; i+=1) {
    cout << i << " ";
  }
            \end{minted}
          \end{minipage}
          &
          \begin{minipage}{2.75in}
            \begin{solution}
              \scriptsize
              \begin{minted}[
                frame=lines,
                framesep=2mm,
                bgcolor=gray!15,
                baselinestretch=1.2,
              ]{cpp}
  for (int i=5; i<=10; i++) {
    cout << i << " ";
  }
              \end{minted}
            \end{solution}
          \end{minipage}                  
        \end{tabular}
      \end{center}


    \item Rewrite the following {\tt while} loop to use an {\it decrement operator}.
    
      \begin{center}
        \begin{tabular}{p{2.75in}p{2.75in}}
          \begin{minipage}{2.75in}
            \begin{minted}[
              frame=lines,
              framesep=2mm,
              bgcolor=gray!15,
              baselinestretch=1.2,
            ]{cpp}
  int counter = 10;            
  while (counter > 0) {
    cout << "text" << endl;
    counter = counter - 1;
  }
            \end{minted}
          \end{minipage}
          &
          \begin{minipage}{2.75in}
            \begin{solution}
              \scriptsize
              \begin{minted}[
                frame=lines,
                framesep=2mm,
                bgcolor=gray!15,
                baselinestretch=1.2,
              ]{cpp}
  int counter = 10;            
  while (counter > 0) {
    cout << "text" << endl;
    counter--;
  }
              \end{minted}
            \end{solution}
          \end{minipage}                  
        \end{tabular}
      \end{center}

    \item Why are increment and decrement operators especially useful for counters?
      \begin{solution}[0.5in]
        Because they are a compact way to add or subtract one to a counter.
      \end{solution}
    
    \item Note that there are two versions of the {\it increment operator}: a {\it pre-increment} version \key\\[-2.5mm]
      (i.e. {\tt ++x}) and a {\it post-increment} version (i.e. {\tt x++}).  The same is true of the decrement operator.
      Use the file {\tt activity08c.cpp} to assist as you complete the missing entries in the table below.
      
      \begin{center}
        \renewcommand{\arraystretch}{1.8}
        \begin{tabular}{|c|c|c|c|}
          \hline
          \rowcolor{orange!20} Initial {\tt x} Value & Expression & Expression Value & Final {\tt x} Value \\
          \hline
          5 & {\tt 3 + 2*(x++)} & \ifprintanswers 13\fi & 6 \\
          \hline
          3 & {\tt 5 - (++x) / 2} & 3 & \ifprintanswers 4\fi  \\
          \hline
          \ifprintanswers 1\fi & {\tt 6 * (--x)} & 0 & 0\\
          \hline
          -3 & \ifprintanswers {\tt 2+(x--)}\fi & -1 & -4\\
          \hline
        \end{tabular}
      \end{center}
      
    \item Based on your work above, what is the difference between the pre- and post- versions of these operators?
      \begin{solution}[0.5in]
        \par
        The pre- version changes the value of the variable before it is used in the expression.  The post- version changes the
        value after it has been used in the expression.
      \end{solution}

  \end{enumerate}  
    
\end{document}
