\model{Two C++ Programs}
  \begin{center}
    \small
    \begin{tabular}{p{3.3in}p{3.9in}}
      \begin{minipage}{3.3in}
        \begin{cpplst}
#include <iostream>
#include <string>

using namespace std;

int main() {
  string name;
  cout << "Enter your name: ";
  cin >> name;
  int x = 0;
  while (x < 10) {
    cout << name << endl;
    x = x + 1;
  }
  cout << "Nice to meet you!" << endl;
}
        \end{cpplst}
      \end{minipage}
      &
      \begin{minipage}{3.9in}
        \begin{cpplst}
#include <iostream>
#include <string>

using namespace std;

int main() {
string name;
  cout << "Enter your name: ";
  cin >> name;
  for (int x = 0; x < 10; x += 1) {
    cout << name << endl;
  }
  cout << "Nice to meet you!" << endl;
}
        \end{cpplst}
      \end{minipage}
    \end{tabular}
  \end{center}

  {\it\large Refer to Model 1 above as your team develops consensus answers
    to the questions below.}
  
  \quest{20 min}
  
  \Q What is the output of each program?  Note that you saw the
    first program in the previous activity and {\tt activity08a.cpp}
    contains the second program.
    \begin{answer}[0.5in]
      The output of the two programs is identical -- the entered
      name is printed ten times.
    \end{answer}
    
  \Q The loop shown on the right is called a {\tt for} loop.
    Identify the code that makes up each part of the loop and the line
    on which it appears.
    \begin{enumerate}
      \itemsep 10pt
      \item The Initialization Statement: \hfill
        \ans[4in]{\cpp{int x = 0} on line 10}

      \item The Test Condition: \hfill
        \ans[4in]{\cpp{x < 10} on line 10}

      \item The Update Statement: \hfill
        \ans[4in]{\cpp{x += 1} on line 10}
    \end{enumerate}
  
  \Q You should have noted above that both programs produce the
    same output.  Which is more concise?
    \begin{answer}[0.5in]
      The {\tt for} loop is more concise.
    \end{answer}

  \Q What output will each of the following code snippets
    produce?  You can use {\tt activity08a.cpp} to check your
    answers.
    \begin{enumerate}
      \item \begin{tabular}{p{3.5in}p{2.5in}}
        \begin{minipage}{3.5in}
          \begin{cpplst}
for (int i = 0; i < 5; i += 1) {
  cout << i << " ";
}
          \end{cpplst}
        \end{minipage}
        &
        \ans[2.5in]{\tt 0 1 2 3 4 }
      \end{tabular}
        
      \item \begin{tabular}{p{3.5in}p{2.5in}}
        \begin{minipage}{3.5in}
          \begin{cpplst}
for (int i = 1; i < 5; i += 1) {
  cout << i << " ";
}
          \end{cpplst}
        \end{minipage}
        &
        \ans[2.5in]{\tt 1 2 3 4}
      \end{tabular}

      \item \begin{tabular}{p{3.5in}p{2.5in}}
        \begin{minipage}{3.5in}
          \begin{cpplst}
for (int i = 2; i <= 6; i += 1) {
  cout << i << " ";
}
          \end{cpplst}
        \end{minipage}
        &
        \ans[2.5in]{\tt 2 3 4 5 6}
      \end{tabular}

      \item \begin{tabular}{p{3.5in}p{2.5in}}
        \begin{minipage}{3.5in}
          \begin{cpplst}
for (int i = 2; i <= 6; i += 2) {
  cout << i << " ";
}
          \end{cpplst}
        \end{minipage}
        &
        \ans[2.5in]{\tt 2 4 6}
      \end{tabular}
    \end{enumerate}
      
  \vskip -40pt

  \Q Complete the missing code in the {\tt for} loops below so
    that they print the indicated \key\\[-2.5mm] output.
    \begin{enumerate}
      \begin{multicols}{2}
        \item Even numbers from 100 to  200, inclusive\par\hspace{0.1in}
          \begin{minipage}{2.6in}
            \begin{answer}[1in]
              \begin{cpplst}
for (int i=100; i<=200; i+=2) {
  cout << i << " ";
}
              \end{cpplst}
            \end{answer}
            \begin{cpplst}
for (                     ) {
  cout << i << " ";
}
            \end{cpplst}
          \end{minipage}

        \item {\tt 5 4 3 2 1 0}\par\hspace{0.1in}
          \begin{minipage}{2.6in}
            \begin{answer}[1in]
              \begin{cpplst}
for (int i=5; i>=0; i-=1) {
  cout << i << " ";
}
              \end{cpplst}
            \end{answer}
            \begin{cpplst}
for (                     ) {
  cout << i << " ";
}
            \end{cpplst}
          \end{minipage}          
      \end{multicols}
    \end{enumerate}
  
  \newpage

  \Q Based on your solutions above, answer the following questions about the {\tt for} loop.
    \begin{enumerate}
      \item Why do we start a for loop with something like
        \cpp{int i=1} instead of \cpp{int i==1}?
        \begin{answer}[0.5in]
          We are assigning {\tt i} the initial value 1, not checking
          to see if it equals 1.
        \end{answer}

      \item Do you think we always need the \cpp{int} in
        front of the \cpp{i=1} in the {\tt for} loop
        initialization?
        \begin{answer}[0.5in]
          No.  We only need it if the variable {\tt i} has not already been declared.
        \end{answer}

      \item Is it better to use a {\tt for} loop when you know how
        many times the loop should execute ({\it counter-controlled}) or
        when you don't know ({\it sentinel-controlled})?
        \begin{answer}[0.5in]
          The {\tt for} loop is naturally a counter-controlled loop
          as a counter is built into it.
        \end{answer}
    \end{enumerate}

  \vskip -30pt

  \Q Rewrite the following {\tt while} loop as a {\tt for} loop that does the same thing.
      \begin{center}
        \begin{tabular}{p{2.75in}p{2.75in}}
          \begin{minipage}{2.75in}
            \begin{cpplst}
int cnt = 20;            
while (cnt >= 10) {
  cnt -= 2;
  cout << cnt << " ";
}
            \end{cpplst}
          \end{minipage}
          &
          \begin{minipage}{2.75in}
            \begin{answer}
              \begin{cpplst}
for (int cnt=18; cnt >= 8; cnt -= 2) {
  cout << cnt << " ";
}
              \end{cpplst}
            \end{answer}
          \end{minipage}                  
        \end{tabular}
      \end{center}