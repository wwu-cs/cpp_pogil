\model{Two C++ Programs} \\
  \begin{center}
    \small
    \begin{tabular}{p{2.5in}p{0.5in}p{2.5in}}
      \begin{minipage}{2.5in}
        \begin{cprlst}[
          frame=lines,
          framesep=2mm,
          bgcolor=gray!15,
          baselinestretch=1.2,
          linenos,
          firstnumber=7
        ]{cpp}
  string name;
  cout << "Enter your name: ";
  cin >> name;
  int x = 0;
  while (x < 10) {
    cout << name << endl;
    x = x + 1;
  }
  cout << "Nice to meet you!" << endl;
        \end{cprlst}
      \end{minipage}
      & &
      \begin{minipage}{2.5in}
        \begin{cprlst}[
          frame=lines,
          framesep=2mm,
          bgcolor=gray!15,
          baselinestretch=1.2,
          linenos,
          firstnumber=7
        ]{cpp}
  string name;
  cout << "Enter your name: ";
  cin >> name;
  for (int x = 0; x < 10; x += 1) {
    cout << name << endl;
  }
  cout << "Nice to meet you!" << endl;
        \end{cprlst}
      \end{minipage}
    \end{tabular}
  \end{center}
  \TPMargin{5pt}
  
  
  {\it\large Refer to Model 1 above as your team develops consensus answers
    to the questions below.}
    \par\vskip 10pt
    
  \begin{enumerate}
    \itemsep 20pt
    
    \Q What is the output of each program?  Note that you saw the
      first program in the previous activity and {\tt activity08a.cpp}
      contains the second program.
      \begin{answer}[0.5in]
        The output of the two programs is identical -- the entered
        name is printed ten times.
      \end{answer}
      
    \Q The loop shown on the right is called a {\tt for} loop.
      Identify the code that makes up each part of the loop and the line
      on which it appears.
      \par\vskip 15pt
      
      \begin{enumerate}[(a)]
        \itemsep 15pt
        \item The Initialization Statement: \hfill
          \ans[4in]{\cpp{int x = 0} on line 10}
        \item The Test Condition: \hfill
          \ans[4in]{\cpp{x < 10} on line 10}
        \item The Update Statement: \hfill
          \ans[4in]{\cpp{x += 1} on line 10}
      \end{enumerate}
    

    \Q You should have noted above that both programs produce the
      same output.  Which is more concise?
      \begin{answer}[0.5in]
        The {\tt for} loop is more concise.
      \end{answer}
      
\newpage

    \Q What output will each of the following code snippets
      produce?  You can use {\tt activity08a.cpp} to check your
      answers.
      \par\vskip 10pt
      \begin{enumerate}[(a)]
        \item \begin{tabular}{p{3in}p{2.5in}}
          \begin{minipage}{3in}
            \begin{cprlst}[
              frame=lines,
              framesep=2mm,
              bgcolor=gray!15,
              baselinestretch=1.2,
            ]{cpp}
  for (int i = 0; i < 5; i += 1) {
    cout << i << " ";
  }
            \end{cprlst}
          \end{minipage}
          &
          \ans[2.5in]{\tt 1 2 3 4 5}
        \end{tabular}
        
        \item \begin{tabular}{p{3in}p{2.5in}}
          \begin{minipage}{3in}
            \begin{cprlst}[
              frame=lines,
              framesep=2mm,
              bgcolor=gray!15,
              baselinestretch=1.2,
            ]{cpp}
  for (int i = 1; i < 5; i += 1) {
    cout << i << " ";
  }
            \end{cprlst}
          \end{minipage}
          &
          \ans[2.5in]{\tt 1 2 3 4}
        \end{tabular}

        \item \begin{tabular}{p{3in}p{2.5in}}
          \begin{minipage}{3in}
            \begin{cprlst}[
              frame=lines,
              framesep=2mm,
              bgcolor=gray!15,
              baselinestretch=1.2,
            ]{cpp}
  for (int i = 2; i <= 6; i += 1) {
    cout << i << " ";
  }
            \end{cprlst}
          \end{minipage}
          &
          \ans[2.5in]{\tt 2 3 4 5 6}
        \end{tabular}

        \item \begin{tabular}{p{3in}p{2.5in}}
          \begin{minipage}{3in}
            \begin{cprlst}[
              frame=lines,
              framesep=2mm,
              bgcolor=gray!15,
              baselinestretch=1.2,
            ]{cpp}
  for (int i = 2; i <= 6; i += 2) {
    cout << i << " ";
  }
            \end{cprlst}
          \end{minipage}
          &
          \ans[2.5in]{\tt 2 4 6}
        \end{tabular}
      \end{enumerate}
      \par\vskip -40pt\null
      
    \Q Complete the missing code in the {\tt for} loops below so
      that they print the indicated output.\key\\[-2.5mm]

      \begin{enumerate}[(a)]
        \begin{multicols}{2}
          \item Even numbers from 100 to  200, inclusive\par
            \begin{minipage}{2.6in}
              % \ifprintanswers
              \begin{cprlst}[
                frame=lines,
                framesep=2mm,
                bgcolor=gray!15,
                baselinestretch=1.5,
              ]{cpp}
  for (int i=100; i<=200; i+=1) {
    cout << i << " ";
  }
              \end{cprlst}
              \else\large
              \begin{cprlst}[
                frame=lines,
                framesep=2mm,
                bgcolor=gray!15,
                baselinestretch=1.5,
              ]{cpp}
  for (                     ) {
    cout << i << " ";
  }
              \end{cprlst}
              \fi
            \end{minipage}
          \item {\tt 5 4 3 2 1 0}\par
            \begin{minipage}{2.6in}
              % \ifprintanswers
              \begin{cprlst}[
                frame=lines,
                framesep=2mm,
                bgcolor=gray!15,
                baselinestretch=1.5,
              ]{cpp}
  for (int i=5; i>=0; i-=1) {
    cout << i << " ";
  }
              \end{cprlst}
              \else\large
              \begin{cprlst}[
                frame=lines,
                framesep=2mm,
                bgcolor=gray!15,
                baselinestretch=1.5,
              ]{cpp}
  for (                     ) {
    cout << i << " ";
  }
              \end{cprlst}
              \fi
            \end{minipage}          
        \end{multicols}
      \end{enumerate}
      
  \Q Based on your solutions above, answer the following questions about the {\tt for} loop.
    \begin{enumerate}[(a)]
      \item Why do we start a for loop with something like
        \cpp{int i=1} instead of \cpp{int i==1}?
        \begin{answer}[0.5in]
          We are assigning {\tt i} the initial value 1, not checking
          to see if it equals 1.
        \end{answer}
      \item Do you think we always need the \cpp{int} in
        front of the \cpp{i=1} in the {\tt for} loop
        initialization?
        \begin{answer}[0.5in]
          No.  We only need it if the variable {\tt i} has not already been declared.
        \end{answer}
      \item Is it better to use a {\tt for} loop when you know how
        many times the loop should execute ({\it counter-controlled}) or
        when you don't know ({\it sentinel-controlled})?
        \begin{answer}[0.5in]
          The {\tt for} loop is naturally a counter-controlled loop
          as a counter is built into it.
        \end{answer}
    \end{enumerate}

\newpage

  \Q Rewrite the following {\tt while} loop as a {\tt for} loop that does the same thing.
      \begin{center}
        \begin{tabular}{p{2.75in}p{2.75in}}
          \begin{minipage}{2.75in}
            \begin{cprlst}[
              frame=lines,
              framesep=2mm,
              bgcolor=gray!15,
              baselinestretch=1.2,
            ]{cpp}
  int cnt = 20;            
  while (cnt >= 10) {
    cnt -= 2;
    cout << cnt << " ";
  }
            \end{cprlst}
          \end{minipage}
          &
          \begin{minipage}{2.75in}
            \begin{answer}
              \footnotesize
              \begin{cprlst}[
                frame=lines,
                framesep=2mm,
                bgcolor=gray!15,
                baselinestretch=1.2,
              ]{cpp}
  for (int cnt=18; cnt >= 8; cnt -= 2) {
    cout << cnt << " ";
  }
              \end{cprlst}
            \end{answer}
          \end{minipage}                  
        \end{tabular}
      \end{center}
      % \ifprintanswers\else\par\vskip 10pt\fi