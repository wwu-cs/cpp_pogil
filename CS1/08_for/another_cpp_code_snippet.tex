{\bf\large Model 2: Another C++ Code Snippet} \\[-20pt]
  
  \begin{center}
    \small
    \begin{minipage}{5.5in}
      \begin{minted}[
        frame=lines,
        framesep=2mm,
        bgcolor=gray!15,
        baselinestretch=1.2,
        linenos,
        firstnumber=5
      ]{cpp}
  int number;
  int total = 0;
  for (int i = 0; i < 5; i += 1) {
    cout << "Enter a number: ";
    cin >> number;
    total += number;
  }
  cout << "The total is: " << total << endl;
      \end{minted}
    \end{minipage}
  \end{center}

  {\it\large Refer to Model 2 above as your team develops consensus answers
    to the questions below.}
    \par\vskip 5pt

    \item The code for this program is in {\tt
      activity08b.cpp}.  Run it and explain what the program does.
      \begin{solution}[1in]
        \par
        The program prompts the user for five numbers and then prints
        out the sum of those five numbers.
      \end{solution}
      
    \item Explain what each of the indicated lines of code from the model above does.
      \par\vskip 10pt
      \begin{enumerate}[(a)]
        \itemsep 15pt
        \item Line 6:    \hfill \fillin[Initializes the variable {\tt total} to zero][5.25in]
        \item Line 7:    \hfill \fillin[Sets up the {\tt for} loop to repeat five times.][5.25in]
        \item Line 10:   \hfill \fillin[Adds the user-entered number to the accumulated total][5.25in]
      \end{enumerate}
      
    \item An {\it accumulator} is a variable that stores the sum of a group of values.  Which variable 
      in this model is an accumulator?  Check all that apply.
      
      \begin{checkboxes}
        \begin{multicols}{3}
          \choice {\tt number}
          \correctchoice {\tt total}
          \choice {\tt i}
        \end{multicols}
      \end{checkboxes}

    \item Why is the variable {\tt total} initialized to zero in line 2 of the model?
      \begin{solution}[0.5in]
        \par
        Because we need to start the loop with nothing accumulated so that after the loop is
        done, total will contain just the sum of the five numbers entered.
      \end{solution}

    \item Would it be possible to use the same variable as both a counter and an accumulator? Explain.
      \begin{solution}[0.5in]
        \par
        No.  A counter must count how many times the loop executes, so 
        it can not accumulate a sum of a group of values.
      \end{solution}

    \item An accumulator can also store the product of a set of numbers.  How, if at all, would you \key\\[-2.5mm]
      change the following lines of the model to compute $5!$ ($5! = 1\times 2\times 3\times 4\times 5$ is called ``five factorial'').
      
      \begin{enumerate}[(a)]
        \item How would you change \mintinline{cpp}|int total = 0;| on line 6?
          \begin{solution}[0.5in]
            It would become \mintinline{cpp}|int total=1;| to initialize the accumulator to 1 instead of 0.
          \end{solution}
        \item How would you change \mintinline{cpp}|for (int i = 0; i < 5; i += 1)| on line 7?
          \begin{solution}[0.5in]
            This does not need to change.
          \end{solution}
        \item How would you change lines 8-9 in the model?
          \begin{solution}[0.5in]
            They would be deleted since user input is no longer needed.
          \end{solution}
        \item How would you change \mintinline{cpp}|total += number| on line 10 of the model?
          \begin{solution}[0.5in]
            It would become \mintinline{cpp}|total *= i+1| to multiply the accumulator by the counter.
          \end{solution}
      \end{enumerate}
      \par\vskip 20pt