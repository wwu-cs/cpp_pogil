\model{Type Sizes}
  Primitive data types in C++ have predefined sizes, which determines 
  the range of values that can be stored in each data type.  Several examples
  are given below.
  
  \begin{center}
    \renewcommand{\arraystretch}{1.8}
    \begin{tabular}{|l|c|c|c|c|}
      \hline
      \cellcolor{orange!20} type & {\tt bool} & {\tt char} & {\tt int} & {\tt double} \\
      \hline
      \cellcolor{orange!20} size & 1 byte & 1 byte & 4 bytes & 8 bytes \\
      \hline
    \end{tabular}
  \end{center}
  
  {\it\large Refer to Model 1 above as your team develops consensus answers
    to the questions below.}

  \par\vskip 10pt

  \quest{15 min}

  \Q A single {\bf bit} has only two values (0 or 1), whereas two bits can store four values (00, 01, 10, and 11).
    \begin{enumerate}
      \item How many different values can be stored in three bits, and what are they?
        \begin{answer}[0.75in]
          8 different values: 000, 001, 010, 011, 100, 101, 110, and 111
        \end{answer}

      \item What is the smallest binary number in your list in part (a) and what is its base 10 value?
        \begin{answer}[0.75in]
          The smallest number is 000 and its base 10 value is 0.
        \end{answer}

      \item What is the largest binary number in your list in part (a) and what is its base 10 value?
        \begin{answer}[0.75in]
          The smallest number is 111 and its base 10 value is 7.
        \end{answer}
    \end{enumerate}

  \vskip -20pt
    
  \Q A {\bf byte} is equivalent to 8 bits.  How many values can be stored in a single byte, and what is the largest
    possible value?  You may wish to use Google as a calculator to solve this problem.
    \begin{answer}[0.75in]
      A byte can hold $2^8 = 256$ values and the largest is 255.
    \end{answer}

  \newpage

  \Q Complete the following table to determine how many values can be stored in a different number of bytes.  The first
    one is done for you.
    \begin{center}
      \renewcommand{\arraystretch}{2}
      \begin{tabular}{|c|c|c|c|}
        \hline
        \rowcolor{orange!20} \# bytes & \# bits & number of values & largest unsigned number possible \\
        \hline
        1 byte & 8 & $2^8 = 256$ & 255 \\
        \hline
        2 bytes & \ans[0.5in]{16} & \ans[1.3in]{$2^{16}=65,536$} & \ans[1.3in]{65,535}\\
        \hline
        4 bytes & \ans[0.5in]{32} & \ans[1.3in]{$2^{32}=4,294,967,296$} & \ans[1.3in]{4,294,967,295}\\
        \hline
        8 bytes & \ans[0.5in]{64} & \ans[1.3in]{$2^{64}$} & \ans[1.3in]{$2^{64}-1$}\\
        \hline
        $x$ bytes & \ans[0.5in]{$8x$} & \ans[1.3in]{$2^{8x}$} & \ans[1.3in]{$2^{8x}-1$}\\
        \hline
      \end{tabular}        
    \end{center}
    
  \Q What is the largest value that a {\tt unsigned short int} variable (of size 2 bytes) can hold?
    \begin{answer}[0.5in]
      An {\tt unsigned short int} can hold $2^16 = 65,536$ values, the largest is 65,535.
    \end{answer}
  
  \Q What happens when you write a C++ assignment statement to set a {\tt unsigned short \key\\[-2.5mm] int} variable equal to one more than this value?
    Try it out using the file {\tt activity04a.cpp}.  Why?
    \begin{answer}[1in]
      \par
      The value wraps around to 0.  That is, the number 65535, represented as 1111111111111111 becomes 0000000000000000
      when we add one to it.
    \end{answer}