
  {\bf\large Model 3: A C++ Program} \\[-10pt]
  \begin{center}
    \begin{minipage}{5.5in}
      \begin{minted}[
        frame=lines,
        framesep=2mm,
        bgcolor=gray!15,
        baselinestretch=1.2,
        linenos
      ]{cpp}
#include <iostream>
using namespace std;

int main() {
  // ----------------------------------------- declare variables
  double floatOne,floatTwo;
  int integer = 35;
  // ----------------------------------------- part I
  floatOne = integer;
  cout << "original number: " << integer << endl;
  cout << "after conversion: " << floatOne << endl << endl;
  // ----------------------------------------- part II
  floatOne = 3.9;
  floatTwo = static_cast<int>(floatOne);
  cout << "second number: " << floatOne << endl;
  cout << "after conversion: " << floatTwo << endl;
}      
      \end{minted}
    \end{minipage}
  \end{center}
  \TPMargin{5pt}
  
  {\it\large Refer to Model 3 above as your group develops consensus answers
    to the questions below.}
    \par\vskip 10pt
    
      \item This program can be found in {\tt activity02c.cpp}.  Run it and then answer the following questions.
        \par\vskip 20pt
        \begin{enumerate}[(a)]
          \itemsep 15pt
          \item What is the printed value of the variable {\tt integer} in part I?  \hfill \fillin[35][1.5in]
          \item What is the printed value of the variable {\tt floatOne} in part I?  \hfill \fillin[35][1.5in]
          \item What is the printed value of the variable {\tt floatOne} in part II? \hfill \fillin[3.9][1.5in]
          \item What is the printed value of the variable {\tt floatTwo} in part II? \hfill \fillin[3][1.5in]
        \end{enumerate}
                
      \item Lines 9 and 14 are both examples of {\it type casting}.  What is a type cast?
        \begin{solution}[0.5in]
          It is when a variable of one type is treated as if it were a variable of another type.
        \end{solution}
        
      \item In the previous model we saw the term {\it implicit type casting}.  Another type of type casting is {\it explicit
        type casting}.  Which statement (line 9 or 14) is an example of an explicit type cast? 
        \begin{solution}[0.5in]
          Line 14 is an explicit type cast because we specifically asked the computer to treat {\tt floatOne} as an integer.
        \end{solution}
        
\newpage

    \item What is the difference between an implicit type cast and an explicit type cast?
      \begin{solution}[1in]
        \par
        An implicit type cast happens automatically, while an explicit type cast is asked for.
      \end{solution}
      
    \item What is the C++ syntax for an explicit type cast?  Give an example other than the one in the model.
      \begin{solution}[1in]
        \par
        The syntax is \mintinline{cpp}|static_cast<newType>(variable)|.  Examples will vary.
      \end{solution}
      
    \item Write a snippet of code that will output the string WWU without using the ``W'' or\key\\[-2.5mm] ``U'' characters.
      Hint: use explicit type casting and the ASCII table seen in model 2.\\
      Feel free to test your code in the file {\tt activity04c.cpp}.
      \begin{solution}[1in]
        \par
  \begin{center}
    \begin{minipage}{5.5in}
      \begin{minted}[
        frame=lines,
        framesep=2mm,
        bgcolor=gray!15,
        baselinestretch=1.2,
        linenos
      ]{cpp}
  cout << static_cast<char>(87);
  cout << static_cast<char>(87);
  cout << static_cast<char>(85);
      \end{minted}
    \end{minipage}
  \end{center}
      \end{solution}