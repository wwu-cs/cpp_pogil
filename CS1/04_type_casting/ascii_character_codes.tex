\model{ASCII Character Codes} \\
  The American Standard Code for Information Interchange is a system that assigns a numeric value to various characters.  A
  selection of assignments from that system is shown below.
  
  \begin{center}
    \footnotesize\renewcommand{\arraystretch}{1.2}
    \begin{tabular}{|c|c||c|c||c|c||c|c||c|c||c|c|}
      \hline
      \rowcolor{orange!20} Code & Symbol & Code & Symbol & Code & Symbol & Code & Symbol & Code & Symbol & Code & Symbol \\
      \hline
        32 & (space) & 48 & 0 & 64 & @ & 80 & P & 96  & ` & 112 & p \\
        33 & !       & 49 & 1 & 65 & A & 81 & Q & 97  & a & 113 & q \\
        34 & "       & 50 & 2 & 66 & B & 82 & R & 98  & b & 114 & r \\
        35 & \#      & 51 & 3 & 67 & C & 83 & S & 99  & c & 115 & s \\
        36 & \$      & 52 & 4 & 68 & D & 84 & T & 100 & d & 116 & t \\
        37 & \%      & 53 & 5 & 69 & E & 85 & U & 101 & e & 117 & u \\
        38 & \&      & 54 & 6 & 70 & F & 86 & V & 102 & f & 118 & v \\
        39 & '       & 55 & 7 & 71 & G & 87 & W & 103 & g & 119 & w \\
        40 & (       & 56 & 8 & 72 & H & 88 & X & 104 & h & 120 & x \\
        41 & )       & 57 & 9 & 73 & I & 89 & Y & 105 & i & 121 & y \\
        42 & *       & 58 & : & 74 & J & 90 & Z & 106 & j & 122 & z \\
        43 & +       & 59 & ; & 75 & K & 91 & [ & 107 & k & 123 & \{ \\
        44 & ,       & 60 & \textless & 76 & L & 92 & $\backslash$ & 108 & l & 124 & $\vert$ \\
        45 & -       & 61 & = & 77 & M & 93 & ] & 109 & m & 125 & \} \\
        46 & .       & 62 & \textgreater & 78 & N & 94 & \^{} & 110 & b & 126 & \textasciitilde \\
        47 & /       & 63 & ? & 79 & O & 95 & \_ & 110 & o & 127 & \\
        \hline
    \end{tabular}
  \end{center}

  {\it\large Refer to Model 2 above as your team develops consensus answers
    to the questions below.}
  \par\vskip 10pt

  \quest{15 min}

  \Q Determine the numeric value associated with each character below in the ASCII system.
    \begin{enumerate}
      \itemsep 10pt
      \begin{multicols}{2}
        \item capital A   \hspace{7pt}  \ans[1.5in]{65}
        \item lowercase f \hspace{1pt}  \ans[1.5in]{102}
        \item capital Z   \hspace{8pt}  \ans[1.5in]{90}
        \item number 0    \hspace{34pt} \ans[1.5in]{48}
        \item open square bracket       \ans[1.35in]{91}
        \item plus sign   \hspace{38pt} \ans[1.5in]{43}
      \end{multicols}
    \end{enumerate}
    
  \Q Based on your work in model 1, how many distinct characters can be stored in a C++ {\tt char} variable?
    \begin{answer}[0.5in]
      A {\tt char} variable is 1 byte, so it can hold 256 different characters.
    \end{answer}
  
  \newpage
  
  \Q In the file {\tt activity04b.cpp} you will find two character variables and a statement that outputs
    the ``sum'' of the two variables.
    \par\vskip 10pt
    \begin{enumerate}
      \item What is the output for each pair of variable values?
        \begin{enumerate}
          \begin{multicols}{2}
            \item \cpp{char charOne = '(';}\\\cpp{char charTwo = ')';}\\[10pt]
              \ans{ 81, sum = Q }\par\vskip 20pt
              
            \item \cpp{char charOne = '!';}\\\cpp{char charTwo = ':';}\\[10pt]
              \ans[2in]{ 91, sum = [ }\par\vskip 20pt

            \item \cpp{char charOne = '1';}\\\cpp{char charTwo = '4';}\\[10pt]
              \ans[2in]{ 101, sum = e }\par\vskip 20pt

            \item \cpp{char charOne = '\%';}\\\cpp{char charTwo = '=';}\\[10pt]
              \ans[2in]{ sum = b }\par\vskip 20pt
          \end{multicols}
        \end{enumerate}
        
      \item Explain how C++ calculated these answers.  Refer to the ASCII table given in the \key\\[-2.5mm] model. 
        \begin{answer}[1in]
          \par
          C++ computes the answers by adding the ASCII code numbers together.  The character that is output is the character
          that has an ASCII code that matches the sum of the two initial characters' ASCII codes.
        \end{answer}
    \end{enumerate}
    
  \Q An {\it implicit type cast} happens when C++ treats a variable of one type as if it were a variable of another type.
    This happened in the program above.
    \begin{enumerate}
      \item How can you tell that an implicit type cast happen in the program above?
        \begin{answer}[0.5in]
          \par
          Because the ``characters'' were added together as if they were integer values.
        \end{answer}
        
      \item Do you think the type of the variables {\tt charOne} and {\tt charTwo} actually change?
        \begin{answer}[0.5in]
          \par
          No, types in C++ can not be changed.  You can verify that these variables are still characters 
          outputting their value, which will be printed as a character, not a number.
        \end{answer}
    \end{enumerate}