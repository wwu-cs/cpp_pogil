\model{A Program with Functions and Branches}
  \begin{center}
    \begin{minipage}{3.5in}
      \scriptsize
      \begin{cpplst}
#include <iostream>
using namespace std;

// first function      
void printSum(int num1, int num2) {
  cout << num1 << " + " << num2 << " = " << (num1 + num2) << endl;
}
// second function
void printDifference(int num1, int num2) {
  cout << num1 << " - " << num2 << " = " << (num1 - num2) << endl;
}
// main program
int main() {
  // define variables
  int firstNumber, secondNumber;
  char operation;
  // collect user input
  cout << "Enter a number between 1 and 10: ";
  cin >> firstNumber;
  cout << "Enter another number between 1 and 10: ";
  cin >> secondNumber;
  cout << "Enter a '+' to add or a '-' to subtract: ";
  cin >> operation;
  // decide which function to call
  if (operation == '+') {
    printSum(firstNumber, secondNumber);
  } else if (operation == '-') {
    printDifference(firstNumber, secondNumber);
  } else {
    cout << "Invalid Operation" << endl;
  }  
}
      \end{cpplst}
    \end{minipage}  
  \end{center}
      
  {\it\large Refer to Model 3 above as your team develops consensus answers
    to the questions below.}

  \quest{25 min}

  \Q What is the first line of code that is executed in this
    program?\hfill \ans[2in]{Line 18}
    
  \Q Use the file {\tt activity11c.cpp} to execute this program
    with the following inputs and record the results.
    \begin{center}
      \renewcommand{\arraystretch}{1.75}
      \begin{tabular}{|c|c|c|c|p{2in}|}
        \hline
        \rowcolor{orange!20} Data Set & First Number & Second Number & Operation & Result \\
        \hline
        1 & 2 & 6 & $+$ & \ans[2in]{$2+6=8$} \\
        \hline
        2 & 3 & 8 & $-$ & \ans[2in]{$3-8=-5$} \\
        \hline
        3 & 34 & 23 & $+$ & \ans[2in]{$34 + 23 = 57$} \\
        \hline
        4 & 4 & 5 & $/$ & \ans[2in]{Invalid Operation} \\
        \hline
      \end{tabular}
    \end{center}
    
  \Q What problems do you see when you entered data set 3?
    \begin{answer}[0.5in]
      The numbers are not in the range from 1-10, but no error was
      reported.
    \end{answer}

  \Q Write a function named {\tt checkRange} that takes two integer 
    values as parameters, check to make sure they are both between 1 and 10,
    and  prints out a warning if they are not.  For example, calling
    \cpp{checkRange(34,23)} might result in the
    following output.
    \begin{center}
      \tt
      WARNING! At least one of the numbers you entered was out of
      range!
    \end{center}
    \begin{answer}[1.5in]
      \begin{center}
        \fs
        \begin{minipage}{6in}
          \begin{cpplst}
void checkRange(int num1, int num2) {
  if ( (num1 < 0 || num1 > 10) || (num2 < 0 || num2 > 10) ) {
    cout << "WARNING! At least one of the numbers you entered was out of range!" 
         << endl;
  }
}       
          \end{cpplst}
        \end{minipage}
      \end{center}
    \end{answer}

  \vskip -10pt
      
  \Q Modify the two functions in the original model to
    produce the following sample \key\\[-2.5mm] output. Do not change the
    {\tt main} program.  Only change the functions on lines 5-7 
    and 9-11 of the original model.
    \begin{center}
      \fbox{
        \begin{minipage}{5in}
          {\bf Sample Output:} 
          \hrule\vskip 5pt\tt
          Enter a number between 1 and 10: 56\\
          Enter another number between 1 and 10: 4\\
          Enter a '+' to add or a '-' to subtract: +\\
          56 + 4 = 60\\
          WARNING! At least one of the numbers you entered was out of range!
        \end{minipage}
      }
    \end{center}
    \begin{answer}[2in]
      \fs
      \begin{center}
        \begin{minipage}{4.5in}
          \begin{cpplst}
// first function      
void printSum(int num1, int num2) {
  cout << num1 << " + " << num2 << " = " << (num1+num2) << endl;
  checkRange(num1,num2);
}
// second function
void printDifference(int num1, int num2) {
  cout << num1 << " - " << num2 << " = " << (num1-num2) << endl;
  checkRange(num1,num2);
}
          \end{cpplst}
        \end{minipage}
      \end{center}
    \end{answer}

  \newpage
    
  \Q Write a function to draw a frog.  Then call this function
    from the {\tt main} program in a loop to create the output shown.
    \begin{center}
      \begin{minipage}{3in}
        \begin{cpplst}
Frog 1...
          @..@
         (----)
        ( >__< )
        ^^ ~~ ^^
Frog 2...
          @..@
         (----)
        ( >__< )
        ^^ ~~ ^^ 
Frog 3...
          @..@
         (----)
        ( >__< )
        ^^ ~~ ^^
Frog 4...
          @..@
         (----)
        ( >__< )
        ^^ ~~ ^^
        \end{cpplst}
      \end{minipage}
      \begin{minipage}{3in}
        \begin{answer}[4in]
          \scriptsize
          \begin{cpplst}
void function frog() {
  cout << "          @..@"   << endl;
  cout << "         (----)"  << endl;
  cout << "        ( >__< )" << endl;
  cout << "        ^^ ~~ ^^" << endl;
}
int main() {
  for(int i=1; i<=4; i++) {
    cout << "Frog " << i << "..." << endl;
    frog();
  }
}
          \end{cpplst}
        \end{answer}
      \end{minipage}
    \end{center}
     