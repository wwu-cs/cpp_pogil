{\bf\large Model 3: A Program with Functions and Branches} \\[-20pt]

  \begin{center}
    \begin{minipage}{3.5in}
      \scriptsize
      \begin{minted}[
        frame=lines,
        framesep=2mm,
        bgcolor=gray!15,
        baselinestretch=1.2,
        linenos,
        firstnumber=4
      ]{cpp}
// first function      
void printSum(int num1, int num2) {
  cout << num1 << " + " << num2 << " = " << (num1+num2) << endl;
}
// second function
void printDifference(int num1, int num2) {
  cout << num1 << " - " << num2 << " = " << (num1-num2) << endl;
}
// main program
int main() {
  // define variables
  int firstNumber,secondNumber;
  char operation;
  // collect user input
  cout << "Enter a number between 1 and 10: ";
  cin >> firstNumber;
  cout << "Enter another number between 1 and 10: ";
  cin >> secondNumber;
  cout << "Enter a '+' to add or a '-' to subtract: ";
  cin >> operation;
  // decide which function to call
  if (operation == '+') {
    printSum(firstNumber,secondNumber);
  } else if (operation == '-') {
    printDifference(firstNumber,secondNumber);
  } else {
    cout << "Invalid Operation" << endl;
  }  
}
      \end{minted}
    \end{minipage}  
  \end{center}
      
  {\it\large Refer to Model 3 above as your team develops consensus answers
    to the questions below.}

    \item What is the first line of code that is executed in this
      program?\hfill \fillin[Line 15][2in]
      
    \item Use the file {\tt activity11c.cpp} to execute this program
      with the following inputs and record the results.
      
      \begin{center}
        \renewcommand{\arraystretch}{1.75}
        \begin{tabular}{|c|c|c|c|p{2in}|}
          \hline
          \rowcolor{orange!20} Data Set & First Number & Second Number & Operation & Result \\
          \hline
          1 & 2 & 6 & $+$ & \ifprintanswers $2+6=8$\fi \\
          \hline
          2 & 3 & 8 & $-$ & \ifprintanswers $3-8=-5$\fi \\
          \hline
          3 & 34 & 23 & $+$ & \ifprintanswers $34 + 23 = 57$\fi \\
          \hline
          4 & 4 & 5 & $/$ & \ifprintanswers Invalid Operation\fi \\
          \hline
        \end{tabular}
      \end{center}
      
    \item What problems do you see when you entered data set 3?
      \begin{solution}[0.5in]
        The numbers are not in the range from 1-10, but no error was
        reported.
      \end{solution}
      
\newpage

    \item Write a function named {\tt checkRange} that takes two integer 
      values as parameters, check to make sure they are both between 1 and 10,
      and  prints out a warning if they are not.  For example, calling
      \mintinline{cpp}|checkRange(34,23)| might result in the
      following output.
      \begin{center}
        \tt
        WARNING! At least one of the numbers you entered was out of
        range!
      \end{center}
      \begin{solution}[1.5in]
        \scriptsize\vskip -35pt\null
        \begin{center}
          \begin{minipage}{4.5in}
            \begin{minted}[
              frame=lines,
              framesep=2mm,
              bgcolor=gray!15,
              baselinestretch=1.2,
            ]{cpp}
void checkRange(int num1, int num2) {
  if ( (num1 < 0 || num1 > 10) || (num2 < 0 || num2 > 10) ) {
    cout << "WARNING! At least one of the numbers you entered was out of range!" 
         << endl;
  }
}       
            \end{minted}
          \end{minipage}
        \end{center}\vskip -20pt\null
      \end{solution}
      \par\vskip -40pt\null
      
    \item Modify the two functions in the original model to
      produce the following sample output.\key\\[-2.5mm]  Do not change the
      {\tt main} program.  Only change the functions on lines 5-7 
      and 9-11 of the original model.
    
      \begin{center}
        \fbox{
          \begin{minipage}{5in}
            {\bf Sample Output:} 
            \hrule\vskip 5pt\tt
            Enter a number between 1 and 10: 56\\
            Enter another number between 1 and 10: 4\\
            Enter a '+' to add or a '-' to subtract: +\\
            56 + 4 = 60\\
            WARNING! At least one of the numbers you entered was out of range!
          \end{minipage}
        }
      \end{center}
      \begin{solution}[2in]
        \scriptsize\vskip -35pt\null
        \begin{center}
          \begin{minipage}{4.5in}
            \begin{minted}[
              frame=lines,
              framesep=2mm,
              bgcolor=gray!15,
              baselinestretch=1.2,
            ]{cpp}
// first function      
void printSum(int num1, int num2) {
  cout << num1 << " + " << num2 << " = " << (num1+num2) << endl;
  checkRange(num1,num2);
}
// second function
void printDifference(int num1, int num2) {
  cout << num1 << " - " << num2 << " = " << (num1-num2) << endl;
  checkRange(num1,num2);
}
            \end{minted}
          \end{minipage}
        \end{center}\vskip -20pt\null
      \end{solution}
    
    \item Write a function to draw a frog.  Then call this function
      from the\\ {\tt main} program in a loop to create the output shown.
      \TPMargin{0pt}
      \par  % avoid warning of textblock not in vertical mode
      \begin{textblock*}{1.6in}[0,0](4.9in,-40pt)
        \small
        \textblockcolor{white}
        \begin{minted}[xleftmargin=-10pt]{cpp}
Frog 1...
          @..@
         (----)
        ( >__< )
        ^^ ~~ ^^
Frog 2...
          @..@
         (----)
        ( >__< )
        ^^ ~~ ^^ 
Frog 3...
          @..@
         (----)
        ( >__< )
        ^^ ~~ ^^
Frog 4...
          @..@
         (----)
        ( >__< )
        ^^ ~~ ^^
        \end{minted}
      \end{textblock*}
      \begin{minipage}{4in}
      \begin{solution}[2in]
        \scriptsize
        \begin{minted}[
          frame=lines,
          framesep=2mm,
          bgcolor=gray!15,
          baselinestretch=1.2,
        ]{cpp}
void function frog() {
  cout << "          @..@"   << endl;
  cout << "         (----)"  << endl;
  cout << "        ( >__< )" << endl;
  cout << "        ^^ ~~ ^^" << endl;
}
int main() {
  for(int i=1; i<=4; i++) {
    cout << "Frog " << i << "..." << endl;
    frog();
  }
}
            \end{minted}
        \end{solution}
      \end{minipage}
     