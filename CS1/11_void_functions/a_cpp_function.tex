 {\bf\large Model 2: A C++ Function} \\[-20pt]
  \begin{center}
    \begin{minipage}{3.5in}
      \begin{minted}[
        frame=lines,
        framesep=2mm,
        bgcolor=gray!15,
        baselinestretch=1.2,
        linenos,
        firstnumber=6
      ]{cpp}
void printArea(double radius) {
  double area = 3.14159 * pow(radius,2);
  cout << fixed << setprecision(2);
  cout << "The area of a circle with radius "
       << radius << " is " << area << endl;
}
      \end{minted}
    \end{minipage}
  \end{center}


  {\it\large Refer to Model 2 above as your team develops consensus answers
    to the questions below.}

    \item What does this function do?
      \ifprintanswers\vskip -20pt\null\fi
      \begin{solution}[0.35in]
        This function computes and prints out the area of a circle of
        a given radius.
      \end{solution}
      \ifprintanswers\vskip -35pt\null\fi
      
    \item Copy down the function header below and underline the name
      of the function.
      \ifprintanswers\vskip -20pt\null\fi
      \begin{solution}[0.35in]
        Header: \mintinline{cpp}|void printArea(double radius)|, name: {\tt printArea}
      \end{solution}
      \ifprintanswers\vskip -35pt\null\fi

    \item Other than using a different name, how is this function header 
      different from the header in Model 1?
      \ifprintanswers\vskip -20pt\null\fi
      \begin{solution}[0.35in]
        This function also has a variable listed inside the parentheses.
      \end{solution}
      \ifprintanswers\vskip -35pt\null\fi
      
    \item A variable defined in a function header is called a 
      {\it parameter}.  What is the name and purpose of the parameter in this
      function?
      \ifprintanswers\vskip -20pt\null\fi
      \begin{solution}[0.35in]
        The parameter is named {\tt radius} and it holds the radius of
        the circle.
      \end{solution}

\newpage

    \item Determine the output produced by each of the following
      calls to this function.  You may find it helpful to use 
      the file {\tt activity11b.cpp}.
      
      \begin{enumerate}[(a)]
        \begin{multicols}{2}
          \item {\tt printArea(3); }\par
            \begin{minipage}{2.75in}
              \begin{solution}[1in]
                \par
                The area of a circle with radius 3.00 is 28.67
              \end{solution}
            \end{minipage}
          \item {\tt printArea(4.5); }\par
            \begin{minipage}{2.75in}
              \begin{solution}[1in]
                \par
                The area of a circle with radius 4.50 is 63.62
              \end{solution}
            \end{minipage}
        \end{multicols}
      \end{enumerate}

   \item Modify the {\tt main} function in the file {\tt activity11b.cpp}
     to prompt the user for a radius and then call the function {\tt
     printArea} with that radius.
      \begin{solution}[1.25in]
        \scriptsize\vskip -35pt\null
        \begin{center}
          \begin{minipage}{2.5in}
            \begin{minted}[
              frame=lines,
              framesep=2mm,
              bgcolor=gray!15,
              baselinestretch=1.2,
            ]{cpp}
int main() {            
  double radius;
  cout << "Enter circle radius: ";
  printArea(radius);
}
            \end{minted}
          \end{minipage}
        \end{center}\vskip -20pt\null
      \end{solution}
      \par\vskip -40pt\null

    \item An {\it argument} is a value or variable that is passed into
      the function.\key
      
      \begin{enumerate}[(a)]
        \itemsep 15pt
        \item What are the arguments in the two function calls in problem 8? \hfill
          \fillin[3 and 4.5][2in]
        \item What is the arguments in your solution to problem 9? \hfill
          \fillin[{\tt radius}][2in]
        \item What happens to these arguments inside the function?
          \begin{solution}[0.5in]
            \par
            The arguments are placed in the function parameter (the
            variable {\tt radius}) to be used in the body of the
            function.
          \end{solution}
        \item Do variable arguments have to have the same name as 
          the corresponding parameter?
          \hfill
          \fillin[No][0.5in]
      \end{enumerate}
      
    \item Write a function that calculates and prints out
      the diameter of a circle with a given radius.
      \begin{solution}[1.25in]
        \scriptsize\vskip -35pt\null
        \begin{center}
          \begin{minipage}{2.75in}
            \begin{minted}[
              frame=lines,
              framesep=2mm,
              bgcolor=gray!15,
              baselinestretch=1.2,
            ]{cpp}
void printDiameter(double radius) {
  double diameter = 2*radius;
  cout << fixed << setprecision(2);
  cout << "The diameter of a circle with radius "
       << radius << " is " << area << endl;
}       
            \end{minted}
          \end{minipage}
        \end{center}\vskip -20pt\null
      \end{solution}
      
    \item Add this function to the program in {\tt activity11b.cpp}
      and add a function call for the same user-entered radius you
      used in problem \#9.  Did your function work as expected?