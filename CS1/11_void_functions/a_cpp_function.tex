\model{A C++ Function} \\
  \begin{center}
    \begin{minipage}{4in}
      \begin{cpplst}
void printArea(double radius) {
  double area = 3.14159 * pow(radius,2);
  cout << fixed << setprecision(2);
  cout << "The area of a circle with radius "
       << radius << " is " << area << endl;
}
      \end{cpplst}
    \end{minipage}
  \end{center}

  {\it\large Refer to Model 2 above as your team develops consensus answers
    to the questions below.}

  \quest{15 min}

  \Q What does this function do?
    \begin{answer}[0.35in]
      This function computes and prints out the area of a circle of
      a given radius.
    \end{answer}
    
  \Q Copy down the function header below and underline the name
    of the function.
    \begin{answer}[0.35in]
      Header: \cpp{void printArea(double radius)}, name: {\tt printArea}
    \end{answer}

  \Q Other than using a different name, how is this function header 
    different from the header in Model 1?
    \begin{answer}[0.35in]
      This function also has a variable listed inside the parentheses.
    \end{answer}
    
  \Q A variable defined in a function header is called a 
    {\it parameter}.  What is the name and purpose of the parameter in this
    function?
    \begin{answer}[0.35in]
      The parameter is named {\tt radius} and it holds the radius of
      the circle.
    \end{answer}

  \Q Determine the output produced by each of the following
    calls to this function.  You may find it helpful to use 
    the file {\tt activity11b.cpp}.
    \begin{enumerate}
      \begin{multicols}{2}
        \item {\tt printArea(3); }\par
          \begin{minipage}{2.75in}
            \begin{answer}[1in]
              \par
              The area of a circle with radius 3.00 is 28.67
            \end{answer}
          \end{minipage}

        \item {\tt printArea(4.5); }\par
          \begin{minipage}{2.75in}
            \begin{answer}[1in]
              \par
              The area of a circle with radius 4.50 is 63.62
            \end{answer}
          \end{minipage}
      \end{multicols}
    \end{enumerate}

   \Q Modify the {\tt main} function in the file {\tt activity11b.cpp}
     to prompt the user for a radius and then call the function {\tt
     printArea} with that radius.
      \begin{answer}[1.25in]
        \begin{center}
          \begin{minipage}{2.5in}
            \begin{cpplst}
int main() {            
  double radius;
  cout << "Enter circle radius: ";
  printArea(radius);
}
            \end{cpplst}
          \end{minipage}
        \end{center}
      \end{answer}

  \vskip -10pt

  \Q An {\it argument} is a value or variable that is passed into
    the function.\key\\[-2.5mm]
    \begin{enumerate}
      \itemsep 10pt
      \item What are the arguments in the two function calls in problem 8? \hfill
        \ans[2in]{3 and 4.5}

      \item What is the argument in your solution to problem 9? \hfill
        \ans[2in]{\tt radius}

      \item What happens to these arguments inside the function?
        \begin{answer}[0.5in]
          \par
          The arguments are placed in the function parameter (the
          variable {\tt radius}) to be used in the body of the
          function.
        \end{answer}
        
      \item Do variable arguments have to have the same name as 
        the corresponding parameter?
        \hfill
        \ans[0.5in]{No}
    \end{enumerate}
    
  \Q Write a function that calculates and prints out
    the diameter of a circle with a given radius.
    \begin{answer}[1.25in]
      \footnotesize
      \begin{center}
        \begin{minipage}{2.75in}
          \begin{cpplst}
void printDiameter(double radius) {
  double diameter = 2*radius;
  cout << fixed << setprecision(2);
  cout << "The diameter of a circle with radius "
       << radius << " is " << area << endl;
}       
          \end{cpplst}
        \end{minipage}
      \end{center}\vskip -20pt\null
    \end{answer}
      
  \Q Add this function to the program in {\tt activity11b.cpp}
    and add a function call for the same user-entered radius you
    used in problem \#9.  Did your function work as expected?
    \begin{answer}[0.5in]
      Yes, the function worked as expected.
    \end{answer}