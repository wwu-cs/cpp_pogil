\model{A Third C++ Program} \\
  \begin{center}
    \begin{minipage}{5.5in}
      \begin{cpplst}
#include <iostream>
using namespace std;

int main() {
  int grade;
  cout << "Enter your grade: ";
  cin >> grade;
  if (grade >= 90) {
    cout << "Very Good!" << endl;
  } else {
    if (grade >= 60) {
      cout << "Satisfactory." << endl;
    } else {
      cout << "Poor." << endl;
    }
  }
}
      \end{cpplst}
    \end{minipage}
  \end{center}
  
  {\it\large Refer to Model 3 above as your group develops consensus answers
    to the questions below.}

  \quest{20 min}

  \Q Circle the {\tt if/else} statement that is {\it nested} inside of 
    another {\tt if/else} statement.
    \begin{answer}[0.4in]
      The {\tt if/else} statement from lines 11-14 is nested inside the one from lines 8-16.
    \end{answer}

  \Q This code can be found in the file {\tt activity06c.cpp}.  Give three different grade
    values that could be used to test different parts of the program.  Indicate what part of 
    the program the value is testing.
    \par\vskip 10pt
    \begin{center}
      \renewcommand{\arraystretch}{2.5}
      \begin{tabular}{|c|c|p{4in}|}
        \hline
        \rowcolor{orange!20} Test & {\tt grade} Value & Part of Program Tested \\
        \hline
        1 & \ans[0.5in] {95} & \ans[2in] {Tests lines 8-9} \\
        \hline
        2 & \ans[0.5in] {82} & \ans[2in] {Tests lines 10-12} \\
        \hline
        3 & \ans[0.5in] {55} & \ans[2in] {Tests lines 10, 11, 13-14} \\
        \hline
      \end{tabular}
    \end{center}
    
  \Q Run the program and enter try your tests.  Does everything work as expected?
    \begin{answer}[0.5in]
      Yes, the program does appear to work as expected.
    \end{answer}
    
  \Q Consider the code snippet below.
    \begin{center}
      \begin{minipage}{3.5in}
        \begin{cpplst}[
  if (grade >= 90) {
    cout << "Very Good!" << endl;
  } else if (grade >= 60) {
    cout << "Satisfactory." << endl;
  } else {
    cout << "Poor." << endl;
  }
}
        \end{cpplst}
      \end{minipage}
    \end{center}
        
    \begin{enumerate}
      \item Replace lines 8-18 in model 3 with this code. How does the output change?
        \begin{answer}[0.5in]
          The output does not change.
        \end{answer}
        
      \item Which method of solving this problem contains simpler syntax and indentation -- the 
        one in the original model, or the one above?  Explain.
        \begin{answer}[0.5in]
          The new one has simpler syntax and does not require nesting.
        \end{answer}
        
      \item You can use as many {\tt else/if} statements as you need.  Suppose you wanted to add the comment ``Good!'' for grades
        that are between 80 and 90.  Write the code for this change.
        \begin{answer}[2in]
          \begin{minipage}{3.5in}
            \begin{cpplst}
  if (grade >= 90) {
    cout << "Very Good!" << endl;
  } else if (grade >= 80) {       // added
    cout << "Good!" << endl;      // added
  } else if (grade >= 60) {
    cout << "Satisfactory." << endl;
  } else {
    cout << "Poor." << endl;
  }
            \end{cpplst}
          \end{minipage}           
        \end{answer}

      \newpage
            
      \item Does it make a difference where you add the additional {\tt else/if} statement? \key\\[-2.5mm]
        Compare adding it at line 10 in the code snippet above vs. at line 12.
        \begin{answer}[1in]
          Yes, it makes a difference.  If we add it at line 12 instead of at line 10, then the ``Good!'' option is never used because
          any grade above 80 will also be above 60 and would fall into the earlier ``Satisfactory.''
          part of the {\tt if/else/if} statement.
        \end{answer}
        
      \item Is the use of the final {\tt else} statement mandatory when creating an {\tt if/else/if} statement?  
        Try it out and supply an example to support your claim.
        \begin{answer}[0.75in]
          Not mandatory.  Dropping lines 12-14 will just print nothing for grades less than 60.
        \end{answer}
        
      \item Make one final change to your program.  Adjust it so that it prints an error message 
        if the grade entered is greater than 100 or less than 0.
        \begin{answer}[2.5in]
          \begin{cpplst}
  if (grade >= 90) {
    cout << "Very Good!" << endl;
  } else if (grade >= 80) {
    cout << "Good!" << endl;
  } else if (grade >= 60) {
    cout << "Satisfactory." << endl;
  } else if (grade < 0 || grade > 100) {
    cout << "Error: Invalid grade." << endl;
  } else {
    cout << "Poor." << endl;
  }
          \end{cpplst}
        \end{answer}
      \end{enumerate}