
  {\bf\large Model 3: A Third C++ Program} \\[-10pt]
  \begin{center}
    \begin{minipage}{5.5in}
      \begin{minted}[
        frame=lines,
        framesep=2mm,
        bgcolor=gray!15,
        baselinestretch=1.2,
        linenos
      ]{cpp}
#include <iostream>
using namespace std;

int main() {
  int grade;
  cout << "Enter your grade: ";
  cin >> grade;
  if (grade >= 90) {
    cout << "Very Good!" << endl;
  } else {
    if (grade >= 60) {
      cout << "Satisfactory." << endl;
    } else {
      cout << "Poor." << endl;
    }
  }
}
      \end{minted}
    \end{minipage}
  \end{center}
  \TPMargin{5pt}
  
  {\it\large Refer to Model 3 above as your group develops consensus answers
    to the questions below.}

      \item Circle the {\tt if/else} statement that is {\it nested} inside of 
        another {\tt if/else} statement.
        \begin{solution}[0.4in]
          The {\tt if/else} statement from lines 11-14 is nested inside the one from lines 8-16.
        \end{solution}

      \item This code can be found in the file {\tt activity06c.cpp}.  Give three different grade
        values that could be used to test different parts of the program.  Indicate what part of 
        the program the value is testing.
        \par\vskip 10pt
        \begin{center}
          \renewcommand{\arraystretch}{2.5}
          \begin{tabular}{|c|c|p{4in}|}
            \hline
            \rowcolor{orange!20} Test & {\tt grade} Value & Part of Program Tested \\
            \hline
            1 & \ifprintanswers 95\fi & \ifprintanswers Tests lines 8-9\fi \\
            \hline
            2 & \ifprintanswers 82\fi & \ifprintanswers Tests lines 10-12\fi \\
            \hline
            3 & \ifprintanswers 55\fi & \ifprintanswers Tests lines 10, 11, 13-14\fi \\
            \hline
          \end{tabular}
        \end{center}
        
      \item Run the program and enter try your tests.  Does everything work as expected?
        \begin{solution}[0.5in]
          Yes, the program does appear to work as expected.
        \end{solution}

\newpage

      \item Consider the code snippet below.
        \begin{center}
          \begin{minipage}{3.5in}
            \begin{minted}[
              frame=lines,
              framesep=2mm,
              bgcolor=gray!15,
              baselinestretch=1.2,
              linenos,
              firstnumber=8
            ]{cpp}
  if (grade >= 90) {
    cout << "Very Good!" << endl;
  } else if (grade >= 60) {
    cout << "Satisfactory." << endl;
  } else {
    cout << "Poor." << endl;
  }
}
            \end{minted}
          \end{minipage}
        \end{center}
        \par\vskip -20pt\null
        
        \begin{enumerate}
          \item Replace lines 8-18 in model 3 with this code. How does the output change?
            \begin{solution}[0.5in]
              The output does not change.
            \end{solution}
            
          \item Which method of solving this problem contains simpler syntax and indentation -- the 
            one in the original model, or the one above?  Explain.
            \begin{solution}[0.5in]
              The new one has simpler syntax and does not require nesting.
            \end{solution}
            
          \item You can use as many {\tt else/if} statements as you need.  Suppose you wanted to add the comment ``Good!'' for grades
            that are between 80 and 90.  Write the code for this change.
            \begin{solution}[1.5in]
              \vskip -30pt\ \null
              \begin{center}
                \scriptsize
                \begin{minipage}{3.5in}
                  \begin{minted}[
                    frame=lines,
                    framesep=2mm,
                    bgcolor=gray!15,
                    baselinestretch=1.2,
                    linenos
                  ]{cpp}
  if (grade >= 90) {
    cout << "Very Good!" << endl;
  } else if (grade >= 80) {       // added
    cout << "Good!" << endl;      // added
  } else if (grade >= 60) {
    cout << "Satisfactory." << endl;
  } else {
    cout << "Poor." << endl;
  }
                  \end{minted}
                \end{minipage}
              \end{center}              
            \end{solution}
            
          \item Does it make a difference where you add the additional {\tt else/if} statement?
            Compare\key\\[-2.5mm] adding it at line 10 in the code snippet above vs. at line 12.
            \begin{solution}[1in]
              Yes, it makes a difference.  If we add it at line 12 instead of at line 10, then the ``Good!'' option is never used because
              any grade above 80 will also be above 60 and would fall into the earlier ``Satisfactory.''
              part of the {\tt if/else/if} statement.
            \end{solution}
            
          \item Is the use of the final {\tt else} statement mandatory when creating an {\tt if/else/if} statement?  
            Try it out and supply an example to support your claim.
            \begin{solution}[0.75in]
              Not mandatory.  Dropping lines 12-14 will just print nothing for grades less than 60.
            \end{solution}
            
          \item Make one final change to your program.  Adjust it so that it prints an error message 
            if the grade entered is greater than 100 or less than 0.
        \end{enumerate}