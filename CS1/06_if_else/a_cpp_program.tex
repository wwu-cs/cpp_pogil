\model{A C++ Program} \\
  \begin{center}
    \small
    \begin{minipage}{5.5in}
      \begin{cprlst}[
        frame=lines,
        framesep=2mm,
        bgcolor=gray!15,
        baselinestretch=1.2,
        linenos
      ]{cpp}
#include <iostream>
using namespace std;

int main() {
  float originalPrice, salePrice;
  cout << "Enter the original cost of the item: ";
  cin >> originalPrice;
  cout << "Enter the sale price: ";
  cin >> salePrice;
  int percentOff = ((originalPrice - salePrice)/originalPrice) * 100;
  cout << "Percent off: " << percentOff << "%" << endl;
  if (percentOff >= 50) {
    cout << "You found a great deal!" << endl;
  }
}      
      \end{cprlst}
    \end{minipage}
  \end{center}
  \TPMargin{5pt}
  
  
  {\it\large Refer to Model 1 above as your team develops consensus answers
    to the questions below.}
    \par\vskip 10pt
    
  \begin{enumerate}
    \itemsep 20pt
    
    \Q You will find this program in {\tt activity06a.cpp}.  Run it with various
      original cost and sale prices and then answer the questions below.
      \begin{enumerate}[(a)]
        \item What do lines 6 and 7 do?%\ifprintanswers\par\vskip -20pt\null\fi
          \begin{answer}[0.4in]
            They prompt for and store the original price in the {\tt originalPrice} variable.
          \end{answer}%\ifprintanswers\par\vskip -20pt\null\fi
        \item What do lines 8 and 9 do?%\ifprintanswers\par\vskip -20pt\null\fi
          \begin{answer}[0.4in]
            They prompt for and store the sale price in the {\tt salePrice} variable.
          \end{answer}%\ifprintanswers\par\vskip -20pt\null\fi
        \item What do lines 10 and 11 do?%\ifprintanswers\par\vskip -20pt\null\fi
          \begin{answer}[0.4in]
            They compute the percent price reduction and print it out.
          \end{answer}%\ifprintanswers\par\vskip -20pt\null\fi
        \item What do lines 12-14 do?%\ifprintanswers\par\vskip -20pt\null\fi
          \begin{answer}[0.4in]
            They print out ``You found a great deal!'' if the savings was
            50\% or more.
          \end{answer}%\ifprintanswers\par\vskip -20pt\null\fi
      \end{enumerate}
      
    \Q Revise the program in model 1 so that right after printing ``You found a great
      deal!'' it prints ``Congratulations!'' if the percent savings was 50\% or more.  Use a
      separate {\tt cout} statement to do this and make note of what you did below.
      \begin{answer}[1in]
        We added the command:\vskip -20pt\null
        \begin{center}
          \begin{minipage}{3.5in}
            \begin{cprlst}[
              frame=lines,
              framesep=2mm,
              bgcolor=gray!15,
              baselinestretch=1.2,
              linenos
            ]{cpp}
    cout << "Congratulations!" << endl;
            \end{cprlst}
          \end{minipage}
        \end{center}\vskip -15pt\null
        right below line 13.
      \end{answer}
      % \ifprintanswers\vskip -30pt\null\fi
      
    \Q Revise the code further so that it prints ``Done!'' when the program is
      complete, no matter what the percent off is.  Again, use a separate {\tt cout}
      statement and describe how the placement of this line of code differs from the
      placement of the code you added above.
      \begin{answer}[1in]
        We added the line of code\vskip -20pt\null
        \begin{center}
          \begin{minipage}{3.5in}
            \begin{cprlst}[
              frame=lines,
              framesep=2mm,
              bgcolor=gray!15,
              baselinestretch=1.2,
              linenos
            ]{cpp}
  cout << "Done!" << endl;
            \end{cprlst}
          \end{minipage}
        \end{center}\vskip -15pt\null
        right below the curly brace on line 14 of the original program (line 15 after or
        modification above.
      \end{answer}
      % \ifprintanswers\vskip -40pt\null\fi

    \item What happens if you remove the open curly brace (\cpp{\{}) from the
      end of line 12 and the close\key\\[-2.5mm] curly brace (\cpp{\}}) from line 14 (15 in
      your modified program)?  Explain why this happens.
      
      \begin{answer}[1in]
        \par
        If we remove these curly braces, the program always prints ``Congratulations!'',
        regardless of what the percent savings is.  This is because the {\tt if} statement
        only  applies to the next statement or block of statements.  Removing the curly brace
        makes it only apply to line 13.
      \end{answer}
      
  