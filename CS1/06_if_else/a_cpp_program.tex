\model{A C++ Program}
  \begin{center}
    \small
    \begin{minipage}{5.5in}
      \begin{cpplst}
#include <iostream>

using namespace std;

int main() {
  float originalPrice, salePrice;
  cout << "Enter the original cost of the item: ";
  cin >> originalPrice;
  cout << "Enter the sale price: ";
  cin >> salePrice;
  int percentOff = ((originalPrice - salePrice)/originalPrice) * 100;
  cout << "Percent off: " << percentOff << "%" << endl;
  if (percentOff >= 50) {
    cout << "You found a great deal!" << endl;
  }
}      
      \end{cpplst}
    \end{minipage}
  \end{center}
  
  
  {\it\large Refer to Model 1 above as your team develops consensus answers
    to the questions below.}

  \quest{15 min}
  
  \Q You will find this program in {\tt activity06a.cpp}.  Run it with various
    original cost and sale prices and then answer the questions below.
    \begin{enumerate}
      \item What do lines 7 and 8 do?
        \begin{answer}[0.4in]
          They prompt for and store the original price in the {\tt originalPrice} variable.
        \end{answer}

      \item What do lines 9 and 10 do?
        \begin{answer}[0.4in]
          They prompt for and store the sale price in the {\tt salePrice} variable.
        \end{answer}

      \item What do lines 11 and 12 do?
        \begin{answer}[0.4in]
          They compute the percent price reduction and print it out.
        \end{answer}
        
      \item What do lines 13 and 14 do?
        \begin{answer}[0.4in]
          They print out ``You found a great deal!'' if the savings was
          50\% or more.
        \end{answer}
    \end{enumerate}
    
  \Q Revise the program in model 1 so that right after printing ``You found a great
    deal!'' it prints ``Congratulations!'' if the percent savings was 50\% or more.  Use a
    separate {\tt cout} statement to do this and make note of what you did below.
    \begin{answer}[1in]
      We added the command:\vskip -20pt\null
      \begin{center}
        \begin{minipage}{3.5in}
          \begin{cpplst}
    cout << "Congratulations!" << endl;
          \end{cpplst}
        \end{minipage}
      \end{center}\vskip -15pt\null
      right below line 14.
    \end{answer}
      
  \Q Revise the code further so that it prints ``Done!'' when the program is
    complete, no matter what the percent off is.  Again, use a separate {\tt cout}
    statement and describe how the placement of this line of code differs from the
    placement of the code you added above.
    \begin{answer}[1in]
      We added the line of code\vskip -20pt\null
      \begin{center}
        \begin{minipage}{3.5in}
          \begin{cpplst}
  cout << "Done!" << endl;
          \end{cpplst}
        \end{minipage}
      \end{center}\vskip -15pt\null
      right below the curly brace on line 15 of the original program (line 16 after or
      modification above.
    \end{answer}

  \Q What happens if you remove the open curly brace (\cpp{\{}) from the
      end of line 13 and \key\\[-2.5mm] the close curly brace (\cpp{\}}) from line 15 (16 in
      your modified program)?  Explain why this happens.
    \begin{answer}[1in]
      \par
      If we remove these curly braces, the program always prints ``Congratulations!'',
      regardless of what the percent savings is.  This is because the {\tt if} statement
      only  applies to the next statement or block of statements.  Removing the curly brace
      makes it only apply to line 14.
    \end{answer}