\model{Another C++ Program} \\
  \begin{center}
    \small
    \begin{minipage}{5.5in}
      \begin{cpplst}
#include <iostream>
using namespace std;

int main() {
  int numCredits = 194;
  double majorGPA = 2.9;
  double overallGPA = 2.1;
  if ( /* missing Boolean expression */ ) {
    cout << "Congratulations!" << endl;
    cout << "You seem to meet the criteria for graduation." << endl;
  } else {
    cout << "Sorry!" << endl;
    cout << "You do not meet all the criteria for graduation." << endl;
  }
}      
      \end{cpplst}
    \end{minipage}
  \end{center}

  {\it\large Refer to Model 2 above as your team develops consensus answers
    to the questions below.}
    
  \quest{15 min}

  \Q In order to graduate from WWU with a Bachelor's degree, students must have
    earned at least 192 credits, have a GPA of at least 2.0 in their major, and have an
    overall GPA of at least 2.0 (among other things).  Which of the following Boolean
    expressions should be used on line 8 of this model to test if a student meets these
    graduation criteria?
    \par\vskip 10pt
    \begin{enumerate}[label=]
      \item \ans[0.2in]{} \cpp{ numCredits >= 192 || majorGPA >= 2.0 || overallGPA >= 2.0}
      \item \ans[0.2in]{} \cpp{ numCredits > 192 && majorGPA > 2.0 && overallGPA > 2.0}
      \item \ans[0.2in]{\checkmark} \cpp{ numCredits > 191 && majorGPA >= 2.0 && overallGPA >= 2.0}
      \item \ans[0.2in]{\checkmark} \cpp{ numCredits >= 192 && majorGPA >= 2.0 && overallGPA >= 2.0}
    \end{enumerate}

  \newpage

  \Q Add the Boolean expression you chose above to the program in {\tt activity06b.cpp}. \key\\[-2.5mm] Create
    test data to check all eight ($2\times 2\times 2$) different combinations for
    the sub-expressions of the Boolean expression.  Then run each of those test cases and verify that 
    the program passes the test. 
    \par\vskip 15pt
    \begin{center}
      \renewcommand{\arraystretch}{1.8}
      \begin{tabular}{|c|c|c|c|c|c|}
        \hline
        \rowcolor{orange!20} Test Case & {\tt numCredits} & {\tt majorGPA} & {\tt overallGPA} & Expected (graduate/don't) & Passed \\
        \hline
          1 & \ans[0.5in]{196} & \ans[0.5in]{3.6} & \ans[0.5in]{3.2} & \ans[1in]{graduate} & \\
        \hline
          2 & \ans[0.5in]{196} & \ans[0.5in]{3.6} & \ans[0.5in]{1.8} & \ans[1in]{don't} & \\
        \hline
          3 & \ans[0.5in]{196} & \ans[0.5in]{1.9} & \ans[0.5in]{3.2} & \ans[1in]{don't} & \\
        \hline
          4 & \ans[0.5in]{196} & \ans[0.5in]{1.7} & \ans[0.5in]{1.9} & \ans[1in] {don't} & \\
        \hline
          5 & \ans[0.5in]{188} & \ans[0.5in]{3.6} & \ans[0.5in]{3.2} & \ans[1in] {don't} & \\
        \hline
          6 & \ans[0.5in]{184} & \ans[0.5in]{3.2} & \ans[0.5in]{1.8} & \ans[1in] {don't} & \\
        \hline
          7 & \ans[0.5in]{90} & \ans[0.5in]{1.9} & \ans[0.5in]{3.2} & \ans[1in] {don't} & \\
        \hline
          8 & \ans[0.5in]{53} & \ans[0.5in]{1.6} & \ans[0.5in]{1.9} & \ans[1in] {don't} & \\
        \hline
      \end{tabular}
    \end{center}
    
  \Q A {\it edge case test} is a test of a natural edges or boundary of the program.  An example of a natural boundary
    in this program is where the {\tt majorGPA} is exactly 2.0, since this is a division line in determining if somebody can
    graduate.  Give at least two other edge cases that could be tested.
    \begin{answer}[1in]
      \par
      We could test where {\tt overallGPA} is exactly 2.0 and where {\tt numCredits} is exactly 192.  Additional boundary
      conditions might include the limits of an integer range for {\tt numCredits}, the division line between positive and
      negative values, etc.
    \end{answer}