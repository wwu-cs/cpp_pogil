\documentclass{exam}
%\documentclass[answers]{exam}
\hbadness=99999
\usepackage[total={6.5in,9in}]{geometry}

\usepackage{enumerate}
\usepackage{amsmath}
\usepackage[table]{xcolor}
\usepackage{graphicx}
\usepackage{tikz}
%\usepackage{pgfplots}
\usepackage{multicol}

% for syntax highlighting
\usepackage{minted}
\usemintedstyle[cpp]{xcode}

% for overlay of output
\usepackage[overlay,showboxes]{textpos}

\pagestyle{plain}

\setlength\columnsep{50pt}
\newcommand{\key}{\hfill
      \raisebox{-.3\height}{\includegraphics[width=0.6in]{figures/key.png}}}

\begin{document}
  \thispagestyle{empty}
  \setlength{\parindent}{0pt}

  \begin{center}
    \Large Activity \#2: Constructors and Overloading \\[5pt]
    \large Recorder's Report\\[20pt]
    \normalsize
    \begin{tabular}{lrp{0.1in}lr}
      Manager:  & \fillin[][2.0in] & & Presenter: & \fillin[][2.0in]\\[15pt]
      Recorder: & \fillin[][2.0in] & & Driver:    & \fillin[][2.0in]\\[15pt]
      Date:     & \fillin[][2.0in] & & Score:     & Satisfactory \hspace{10pt} /
      \hspace{10pt} Not Satisfactory
    \end{tabular}
  \end{center}
  \par\vskip 15pt
  
  Record your group's answers to the key questions (marked with
  \raisebox{-.3\height}{\includegraphics[width=0.5in]{figures/key.png}})
  below.
  \begin{enumerate}[(a)]
    \itemsep 1.75in
    \item Model 1, Question \#4
    \item Model 2, Question \#12
    \item Model 3, Question \#15
  \end{enumerate}

  \clearpage\pagenumbering{arabic} 
  
  \begin{center}
    \Large Activity \#2: Constructors and Overloading \\[5pt]
    \large Activity Guide\\[20pt]
  \end{center}

  \begin{center}
    \fbox{
      \begin{minipage}{5.5in}
        {\bf Learning Objectives:} Students will be able to:
        \begin{itemize}
          \item Content:\\[-20pt]
            \begin{itemize}
              \itemsep 0pt
              \item Explain how constructors are defined and called 
              \item Explain how the implicit parameter {\tt this}
                functions in C++
              \item Explain what it means to overload an operator in
                the context of a class
            \end{itemize}
          \item Process\\[-20pt]
            \begin{itemize}
              \itemsep 0pt
              \item Write constructors for a given class
              \item Write operator methods for a given class\\[-5pt]
            \end{itemize}
        \end{itemize}
      \end{minipage}
      }
  \end{center}
  \par\vskip 10pt
  
  
  {\bf\large Model 1: A C++ Class for a $2\times 2$ Matrix}\\[-10pt]
  \begin{center}
    \small
    \begin{tabular}{p{2.8in}p{0.1in}p{2.8in}}
      \begin{minipage}{2.8in}
        \begin{minted}[
          frame=lines,
          framesep=2mm,
          bgcolor=gray!15,
          baselinestretch=1.2,
          linenos,
          firstnumber=5
        ]{cpp}
class Matrix {
  public:
    Matrix();
    void setValues(int,int,int,int);
    void print();
    int getTrace();
    int getDeterminant();
  private:
    int a,b,c,d; // Entries in | a b |
                 // matrix     | c d |
};
        \end{minted}      
      \end{minipage}
      & &
      \begin{minipage}{2.8in}
        \begin{minted}[
          frame=lines,
          framesep=2mm,
          bgcolor=gray!15,
          baselinestretch=1.2,
          linenos,
          firstnumber=17
        ]{cpp}        
Matrix::Matrix() {
  a = 1;  b = 0;
  c = 0;  d = 1;
}

int main() {
  Matrix A,B;
  A.setValues(2,2,2,2);
  A.print();
  B.print();
}
        \end{minted}      
      \end{minipage}
    \end{tabular}
  \end{center}
  \par\vskip 5pt
  
  {\it\large Refer to Model 1 above as your group develops consensus answers
    to the questions below.}
    \par\vskip 10pt
    
  \begin{enumerate}
    \itemsep 20pt
    
    \item The C++ code snippets above define a class for a $2 \times 2$
      matrix of integers and given an example main program.  Recall
      that a $2\times 2$ matrix is a grid of two rows and two columns.
      Write the matrix that corresponds to the given values for {\tt
      a}, {\tt b}, {\tt c}, and {\tt d} below.  The first one is done
      for you.
      \begin{enumerate}[(a)]
        \itemsep 10pt
        \begin{multicols}{2}
          \item \mintinline{cpp}|a=1, b=2, c=3, d=4|\par
            \ifprintanswers\else\vskip 0.15in\fi\null
            \begin{minipage}{2.75in}
              \[ \begin{vmatrix} 1 & 2 \\ 3 & 4 \end{vmatrix} \]
            \end{minipage}
            \ifprintanswers\else\vskip 0.15in\fi\null
          \item \mintinline{cpp}|a=1, b=1, c=1, d=1|\par
            \begin{minipage}{2.75in}
              \begin{solution}[0.75in]
                \[ \begin{vmatrix} 1 & 1 \\ 1 & 1 \end{vmatrix} \]
              \end{solution}
            \end{minipage}                        
          \item \mintinline{cpp}|a=0, b=1, c=1, d=0|\par
            \begin{minipage}{2.75in}
              \begin{solution}[0.75in]
                \[ \begin{vmatrix} 0 & 1 \\ 1 & 0 \end{vmatrix} \]
              \end{solution}
            \end{minipage}                        
          \item \mintinline{cpp}|a=1, b=0, c=0, d=1|\par
            \begin{minipage}{2.75in}
              \begin{solution}[0.75in]
                \[ \begin{vmatrix} 1 & 0 \\ 0 & 1 \end{vmatrix} \]
              \end{solution}
            \end{minipage}                        
        \end{multicols}
      \end{enumerate}

\newpage

    \item Without running any code, predict the output produced by the
      following statements in the {\tt main} program of this model.
      
      \begin{enumerate}[(a)]
        \itemsep 10pt
        \begin{multicols}{2}
          \item \mintinline{cpp}|A.print();| on line 25\par
            \begin{minipage}{2.75in}
              \begin{solution}[0.75in]
                \[ \begin{vmatrix} 2 & 2 \\ 2 & 2 \end{vmatrix} \]
              \end{solution}
            \end{minipage}                        
          \item \mintinline{cpp}|B.print();| on line 26\par
            \begin{minipage}{2.75in}
              \begin{solution}[0.75in]
                \[ \begin{vmatrix} 1 & 0 \\ 0 & 1 \end{vmatrix} \]
              \end{solution}
            \end{minipage}  
        \end{multicols}
      \end{enumerate}
      
    \item The complete code for this model can be found in 
      {\tt activity02a.cpp}.  Run the code.  Were your predictions
      accurate?
      \begin{solution}[0.5in]
        Answers will vary.
      \end{solution}
      
    \item The method \mintinline{cpp}|B.setValues()| was never called
      in this code.  Explain how the values of the \key\\[-2.5mm] entries
      for matrix {\tt B} were initialized.
      \begin{solution}[1in]
        They were initialied by the method
        \mintinline{cpp}|Matrix::Matrix()| dfeind on lines 17-20.
      \end{solution}
      
      
    \item A {\it constructor} for a class is a method that is called
      automatically when a new object variable of the class is created.
      Perform the following tasks to help determine how a constructor 
      is defined in C++.
      \par\vskip 20pt
      \begin{enumerate}[(a)]
        \itemsep 10pt
        \item Change the name of the method {\tt Matrix()} to {\tt
          matrix()} (with a lower case `m') on lines 7 and 17 of the
          model.  What happens when you compile the program?
          \begin{solution}[0.75in]
            The program fails to compile and gives an error message
            that {\tt matrix} has no type.
          \end{solution}
        \item Add the appropriate function type on lines 7 and 17 so
          that the program compiles.  Run it several times and observe
          how the output different from the original model.
          \begin{solution}[0.75in]
            After adding a {\tt void} to the function {\tt matrix} the
            program compiles, but the contents of matrix {\tt B} are
            not random integers.
          \end{solution}
        \item Based on your observations above, how does C++ determine
          if an object method is a constructor?
          \begin{solution}[0.75in]
            If the name matches the class name exactly.
          \end{solution}
      \end{enumerate}              
      \par\vskip 10pt


  {\bf\large Model 2: More Constructor Options}\\[-10pt]
  \begin{center}
    \small
    \begin{minipage}{4.5in}
      \begin{minted}[
        frame=lines,
        framesep=2mm,
        bgcolor=gray!15,
        baselinestretch=1.2,
        linenos,
        firstnumber=22
      ]{cpp}
int main() {
  Matrix Z(0);       // create a matrix of all zeros
  cout << "Z = ";
  Z.print();
  
  Matrix I;          // create identity matrix [ [1,0], [0,1] ]
  cout << "I = ";
  I.print();
  
  Matrix A(1,2,3,4); // create matrix [ [1,2], [3,4] ]
  cout << "A = ";
  A.print();  
}  
      \end{minted}
    \end{minipage}
  \end{center}
  
  {\it\large Refer to Model 2 above as your group develops consensus answers
    to the questions below.}
    \par\vskip 10pt
    
      \item Based on the model (with comments) above, what output would 
        you expect this program to produce?
        \begin{solution}[0.75in]
          \[ 
            Z = \begin{vmatrix} 0 & 0 \\ 0 & 0 \end{vmatrix}
            \qquad
            I = \begin{vmatrix} 1 & 0 \\ 0 & 1 \end{vmatrix}
            \qquad
            A = \begin{vmatrix} 1 & 2 \\ 3 & 4 \end{vmatrix}
          \]
        \end{solution}
        
      \item The file {\tt activity02b.cpp} contains the same class
        definition and methods as in model 1, but with the {\tt main}
        program above.  Try compiling the program and explain the
        errors you see.
        \begin{solution}[0.75in]
          The errors indicate that there are no matching functions for:
          \begin{itemize}
            \item \mintinline{cpp}|Matrix::Matrix(int)|
            \item \mintinline{cpp}|Matrix::Matrix(int,int,int,int)|
          \end{itemize}
        \end{solution}
        
      \item Thinking back to what you learned about functions in CPTR 141
        (or some other prerequisite class), answer the following
        questions.
        \begin{enumerate}[(a)]
          \item What does it mean to {\it overload} a function name?
            \begin{solution}[0.5in]
              It means to give several different functions the same
              name, but distinguish them by the number and/or type of
              parameters they have.              
            \end{solution}
          \item What is the {\it signature} of a function?
            \begin{solution}[0.5in]
              It is the name and list of parameter types of the function.
            \end{solution}
          \item Based on the error messages you saw above, what are the 
            signatures for the missing constructors in this model?
            \begin{solution}[0.5in]
              They are:
              \begin{itemize}
                \begin{multicols}{2}
                  \item \mintinline{cpp}|Matrix::Matrix(int)|
                  \item \mintinline{cpp}|Matrix::Matrix(int,int,int,int)|
                \end{multicols}
              \end{itemize}
            \end{solution}
        \end{enumerate}
        
        
      \item Suppose we wish to {\it overload} the constructor for this
        class to allow for initializing all matrix entries to a single
        value (as seen on line 23 of this model). We can accomplish
        this by adding the following prototype to the public portion of the
        {\tt Matrix} class.
        \begin{center}
          \mintinline{cpp}|Matrix(int);|
        \end{center}
        Add this prototype to the class in {\tt activity02b.cpp} and
        then define it simliarly to how the original constructor was
        defined on lines 17-20 of the first model.  Hint: comment out
        appropriate lines in the {\tt main} program so that you can test your code.
        \begin{solution}[1.25in]
          \begin{minipage}{3in}
            \begin{minted}[
              frame=lines,
              framesep=2mm,
              bgcolor=gray!15,
              baselinestretch=1.2,
              linenos,
              firstnumber=4
            ]{cpp}
Matrix::Matrix(int x) {
  a = x;
  b = x;
  c = x;
  d = x;
}
            \end{minted}
          \end{minipage}
        \end{solution}
        
      \item Give a method prototype for the constructor needed to
        initialize matrix $A$ in this model.  Where should you put
        this prototype?
        \begin{solution}[0.5in]
          An example prototype is \mintinline{cpp}|{Matrix(int, int,
          int, int);| and it should be added to the public portion of
          the class declaration.
        \end{solution}

      \item Fill in the definition of this constructor below.
        \begin{center}
          \begin{minipage}{4.5in}
            \mintinline{cpp}|Matrix::Matrix(int a, int b, int c, int d) { |\par
            \begin{solution}[1in]
              \mintinline{cpp}|a = a;  // this won't work, should be this->a = a|\\
              \mintinline{cpp}|b = b;  // this won't work, should be this->b = b|\\
              \mintinline{cpp}|c = c;  // this won't work, should be this->c = c|\\
              \mintinline{cpp}|d = d;  // this won't work, should be this->d = d|\\
            \end{solution}
            \mintinline{cpp}|}|
          \end{minipage}
        \end{center}
        \par\vskip -30pt\null
        
      \item Add this constructor to the code in {\tt activity02b.cpp}
        and then compile and run the \key\\[-2.5mm] code (uncommenting lines if
        {\tt main} if needed). Does it work as expected?
        \begin{solution}[0.5in]
          Probably not.  See the comments above and the question below.
        \end{solution}
        
      \item In the example above, the parameter names ({\tt a}, 
        {\tt b}, {\tt c}, and {\tt d}) {\it shadow} the data members
        of the same name.  You can still access the class data members
        using the {\it implicit parameter} {\tt this} to indicate that
        you want the data members of the class.  For example, one line
        of your definition above might be:
        \begin{center}
          \mintinline{cpp}|this->a = a;|
        \end{center}
        Adjust your definition to use {\tt this} so that the
        constructor works as expected.
    
 
\newpage 

  {\bf\large Model 3: Adding Matrices} \\[-10pt]
  \begin{center}
    \small
    \begin{tabular}{p{2.8in}p{0.1in}p{2.8in}}
      \begin{minipage}{2.8in}
        \begin{minted}[
          frame=lines,
          framesep=2mm,
          bgcolor=gray!15,
          baselinestretch=1.2,
          linenos,
          firstnumber=5
        ]{cpp}
class Matrix {
  public:
    Matrix();
    Matrix(int);
    Matrix(int,int,int,int);
    void setValues(int,int,int,int);
    void print();
    int getTrace();
    int getDeterminant();
    Matrix operator+(Matrix);
  private:
    int a,b,c,d; // Entries in | a b |
                 // matrix     | c d |
};
        \end{minted}      
      \end{minipage}
      & &
      \begin{minipage}{2.8in}
        \begin{minted}[
          frame=lines,
          framesep=2mm,
          bgcolor=gray!15,
          baselinestretch=1.2,
          linenos,
          firstnumber=39
        ]{cpp}        
Matrix Matrix::operator+(Matrix rhs) {
  Matrix Sum;
  Sum.a = a + rhs.a;
  Sum.b = b + rhs.b;
  Sum.c = c + rhs.c;
  Sum.d = d + rhs.d;  
  return Sum;
}

int main() {
  Matrix A(2),B(1,2,3,4);
  Matrix C = A + B;
  C.print();
}
        \end{minted}      
      \end{minipage}
    \end{tabular}
  \end{center}
  \par\vskip 5pt
  
  {\it\large Refer to Model 3 above as your group develops consensus answers
    to the questions below.}
    \par\vskip 10pt

    \item Recall that matrices are added together by adding their corresponding
      entries.  So, for example:
      \[
        \begin{vmatrix} 2 & 5 \\ 1 & 3 \end{vmatrix}
        +
        \begin{vmatrix} 1 & 3 \\ 0 & 1 \end{vmatrix}
        =
        \begin{vmatrix} 3 & 8 \\ 1 & 4 \end{vmatrix}
      \]
      \vskip 5pt
      Use this same notation to express the matrix sum computed by
      the {\tt main} program in this model.
      \begin{solution}[0.5in]
      \end{solution}
      \vskip -30pt\null
      
    \item Since we defined the {\tt Matrix} class ourself, we have to tell C++ how
      to add two matrices\key\\[-2.8mm] together with the {\tt +} operator.  This
      is called {\it operator overloading}.  Find the lines in the model above 
      where each of the following is accomplished.
      \par\vskip 15pt
      
      \begin{enumerate}[(a)]
        \itemsep 15pt
        \item Adding a prototype function for the addition operation:
          \hfill\fillin[Line 14][2in]
        \item Defining how two {\tt Matrix} objects are added together:
          \hfill\fillin[Lines 39-46][2in]
        \item Adding together the two top-right entries in the matrices:
          \hfill\fillin[Line 42][2in]
        \item Adding together the two bottom-left entries in the matrices:
          \hfill\fillin[Line 43][2in]
      \end{enumerate}
      
    \item A mathematical definition of matrix multiplication is given below.
      Using {\tt activity02c.cpp}, overload the {\tt *} operator to allow you
      to multiply two {\tt Matrix} objects together using the command 
      \mintinline{cpp}{C = A * B;}.
      \vskip 0pt
      \[
        \begin{vmatrix} a_1 & b_1 \\ c_1 & d_1 \end{vmatrix}
        *
        \begin{vmatrix} a_2 & b_2 \\ c_2 & d_2 \end{vmatrix}
        =
        \begin{vmatrix} 
          a_1*a_2 + b_1 * c_2 & a_1*b_2 + b_1*d_2 \\
          c_1*a_2 + d_1*c_2 & c_1*b_2+d_1*d_2 
        \end{vmatrix}
      \]

  \end{enumerate}
  
  
    
\end{document}
