\model{More Constructor Options}
  \begin{center}
    \small
    \begin{minipage}{4.5in}
      \begin{cpplst}
int main() {
  Matrix Z(0);       // create a matrix of all zeros
  cout << "Z = ";
  Z.print();
  
  Matrix I;          // create identity matrix [ [1,0], [0,1] ]
  cout << "I = ";
  I.print();
  
  Matrix A(1,2,3,4); // create matrix [ [1,2], [3,4] ]
  cout << "A = ";
  A.print();  
}  
      \end{cpplst}
    \end{minipage}
  \end{center}
  
  {\it\large Refer to Model 2 above as your group develops consensus answers
    to the questions below.}

  \quest{15 min}
    
  \Q Based on the model (with comments) above, what output would 
    you expect this program to produce?
    \begin{answer}[0.75in]
      \[ 
        Z = \begin{vmatrix} 0 & 0 \\ 0 & 0 \end{vmatrix}
        \qquad
        I = \begin{vmatrix} 1 & 0 \\ 0 & 1 \end{vmatrix}
        \qquad
        A = \begin{vmatrix} 1 & 2 \\ 3 & 4 \end{vmatrix}
      \]
    \end{answer}
    
  \Q The file {\tt activity21b.cpp} contains the same class
    definition and methods as in model 1, but with the {\tt main}
    program above.  Try compiling the program and explain the
    errors you see.
    \begin{answer}[0.75in]
      \fs
      The errors indicate that there are no matching functions for:
      \begin{itemize}
        \item \cpp{Matrix::Matrix(int)}
        \item \cpp{Matrix::Matrix(int,int,int,int)}
      \end{itemize}
    \end{answer}
    
  \Q Thinking back to what you learned about functions in CPTR 141
    (or some other prerequisite class), answer the following
    questions.
    \begin{enumerate}
      \item What does it mean to {\it overload} a function name?
        \begin{answer}[0.5in]
          It means to give several different functions the same
          name, but distinguish them by the number and/or type of
          parameters they have.              
        \end{answer}

      \item What is the {\it signature} of a function?
        \begin{answer}[0.5in]
          It is the name and list of parameter types of the function.
        \end{answer}

      \item Based on the error messages you saw above, what are the 
        signatures for the missing constructors in this model?
        \begin{answer}[0.5in]
          They are:
          \begin{itemize}
            \begin{multicols}{2}
              \item \cpp{Matrix::Matrix(int)}
              \item \cpp{Matrix::Matrix(int,int,int,int)}
            \end{multicols}
          \end{itemize}
        \end{answer}
    \end{enumerate}
    
  \Q Suppose we wish to {\it overload} the constructor for this
    class to allow for initializing all matrix entries to a single
    value (as seen on line 27 of this model). We can accomplish
    this by adding the following prototype to the public portion of the
    {\tt Matrix} class.
    \begin{center}
      \cpp{Matrix(int);}
    \end{center}
    Add this prototype to the class in {\tt activity21b.cpp} and
    then define it similarly to how the original constructor was
    defined on lines 19-24 of the first model.  Hint: comment out
    appropriate lines in the {\tt main} program so that you can test your code.
    \begin{answer}[1.25in]
      \begin{minipage}{3in}
        \begin{cpplst}
Matrix::Matrix(int x) {
  a = x;
  b = x;
  c = x;
  d = x;
}
        \end{cpplst}
      \end{minipage}
    \end{answer}
        
  \Q Give a method prototype for the constructor needed to
    initialize matrix $A$ in this model.  Where should you put
    this prototype?
    \begin{answer}[0.5in]
      An example prototype is \cpp{Matrix(int, int,
      int, int);} and it should be added to the public portion of
      the class declaration.
    \end{answer}

  \Q Fill in the definition of this constructor below.
    \begin{center}
      \begin{minipage}{4.5in}
        \cpp{Matrix::Matrix(int a, int b, int c, int d);} \par
        \begin{answer}[1in]
          \cpp{a = a;  // this won't work, should be this->a = a}\\
          \cpp{b = b;  // this won't work, should be this->b = b}\\
          \cpp{c = c;  // this won't work, should be this->c = c}\\
          \cpp{d = d;  // this won't work, should be this->d = d}\\
        \end{answer}
      \end{minipage}
    \end{center}

  \newpage
    
  \Q Add this constructor to the code in {\tt activity21b.cpp}
    and then compile and run the \key\\[-2.5mm] code (uncommenting lines if
    {\tt main} if needed). Does it work as expected?
    \begin{answer}[0.5in]
      Probably not.  See the comments above and the question below.
    \end{answer}
    
  \Q In the example above, the parameter names ({\tt a}, 
    {\tt b}, {\tt c}, and {\tt d}) {\it shadow} the data members
    of the same name.  You can still access the class data members
    using the {\it implicit parameter} {\tt this} to indicate that
    you want the data members of the class.  For example, one line
    of your definition above might be:
    \begin{center}
      \cpp{this->a = a;}
    \end{center}
    Adjust your definition to use {\tt this} so that the
    constructor works as expected.