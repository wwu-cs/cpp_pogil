\model{Adding Matrices}
  \begin{center}
    \small
    \begin{tabular}{p{2.8in}p{0.1in}p{2.8in}}
      \begin{minipage}{2.8in}
        \begin{cpplst}
class Matrix {
  public:
    Matrix();
    Matrix(int);
    Matrix(int,int,int,int);
    void setValues(int,int,int,int);
    void print();
    int getTrace();
    int getDeterminant();
    Matrix operator+(Matrix);
  private:
    int a,b,c,d; // Entries in | a b |
                 // matrix     | c d |
};
        \end{cpplst}      
      \end{minipage}
      & &
      \begin{minipage}{2.8in}
        \begin{cpplst}       
Matrix Matrix::operator+(Matrix rhs) {
  Matrix Sum;
  Sum.a = a + rhs.a;
  Sum.b = b + rhs.b;
  Sum.c = c + rhs.c;
  Sum.d = d + rhs.d;  
  return Sum;
}

int main() {
  Matrix A(2),B(1,2,3,4);
  Matrix C = A + B;
  C.print();
}
        \end{cpplst}      
      \end{minipage}
    \end{tabular}
  \end{center}
  
  {\it\large Refer to Model \M above as your group develops consensus answers
    to the questions below.}

  \quest{20 min}

  \Q Recall that matrices are added together by adding their corresponding
    entries.  So, for example:
    \[
      \begin{vmatrix} 2 & 5 \\ 1 & 3 \end{vmatrix}
      +
      \begin{vmatrix} 1 & 3 \\ 0 & 1 \end{vmatrix}
      =
      \begin{vmatrix} 3 & 8 \\ 1 & 4 \end{vmatrix}
    \]
    \vskip 5pt
    Use this same notation to express the matrix sum computed by
    the {\tt main} program in this model.
    \begin{answer}[0.5in]
    \end{answer}
    \vskip -30pt\null
    
  \Q Since we defined the {\tt Matrix} class ourself, we have to tell C++ how
    to add two\key\\[-2.8mm] matrices together with the {\tt +} operator.  This
    is called {\it operator overloading}.  Find the lines in the model above 
    where each of the following is accomplished.
    \begin{enumerate}
      \itemsep 10pt
      \item Adding a prototype function for the addition operation:
        \hfill\ans[2in]{Line 15}

      \item Defining how two {\tt Matrix} objects are added together:
        \hfill\ans[2in]{Lines 43-50}

      \item Adding together the two top-right entries in the matrices:
        \hfill\ans[2in]{Line 46}

      \item Adding together the two bottom-left entries in the matrices:
        \hfill\ans[2in]{Line 47}
    \end{enumerate}

  \newpage
    
  \Q A mathematical definition of matrix multiplication is given below.
    Using {\tt activity21c.cpp}, overload the {\tt *} operator to allow you
    to multiply two {\tt Matrix} objects together using the command 
    \cpp{C = A * B;}.
    \vskip 0pt
    \[
      \begin{vmatrix} a_1 & b_1 \\ c_1 & d_1 \end{vmatrix}
      *
      \begin{vmatrix} a_2 & b_2 \\ c_2 & d_2 \end{vmatrix}
      =
      \begin{vmatrix} 
        a_1*a_2 + b_1 * c_2 & a_1*b_2 + b_1*d_2 \\
        c_1*a_2 + d_1*c_2 & c_1*b_2+d_1*d_2 
      \end{vmatrix}
    \]