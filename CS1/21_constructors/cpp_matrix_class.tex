\model{A C++ Class for a $2\times 2$ Matrix}
  \begin{center}
    \small
    \begin{tabular}{p{2.8in}p{0.1in}p{2.8in}}
      \begin{minipage}{2.8in}
        \begin{cpplst}
class Matrix {
  public:
    Matrix();
    void setValues(int,int,int,int);
    void print();
    int getTrace();
    int getDeterminant();
  private:
    int a,b,c,d; // Entries in | a b |
                 // matrix     | c d |
};
        \end{cpplst}      
      \end{minipage}
      & &
      \begin{minipage}{2.8in}
        \begin{cpplst}      
Matrix::Matrix() {
  a = 1;  b = 0;
  c = 0;  d = 1;
}

int main() {
  Matrix A,B;
  A.setValues(2,2,2,2);
  A.print();
  B.print();
}
        \end{cpplst}      
      \end{minipage}
    \end{tabular}
  \end{center}
  
  {\it\large Refer to Model 1 above as your group develops consensus answers
    to the questions below.}

  \quest{15 min}
    
  \Q The C++ code snippets above define a class for a $2 \times 2$
    matrix of integers and given an example main program.  Recall
    that a $2\times 2$ matrix is a grid of two rows and two columns.
    Write the matrix that corresponds to the given values for {\tt
    a}, {\tt b}, {\tt c}, and {\tt d} below.  The first one is done
    for you.
    \begin{enumerate}
      \itemsep 10pt
      \begin{multicols}{2}
        \item \cpp{a=1, b=2, c=3, d=4}\par
        \begin{minipage}{2.75in}
          \begin{answer}[0.75in]
            \begin{center}
              \begin{vmatrix}
                1 & 2 \\
                3 & 4
              \end{vmatrix}
            \end{center}
          \end{answer}
        \end{minipage}

        \item \cpp{a=1, b=1, c=1, d=1}\par
          \begin{minipage}{2.75in}
            \begin{answer}[0.75in]
              \begin{center}
                \begin{vmatrix} 1 & 1 \\ 1 & 1 \end{vmatrix}
              \end{center}
            \end{answer}
          \end{minipage}   

        \item \cpp{a=0, b=1, c=1, d=0}\par
          \begin{minipage}{2.75in}
            \begin{answer}[0.75in]
              \begin{center}
                \begin{vmatrix} 0 & 1 \\ 1 & 0 \end{vmatrix}
              \end{center}
            \end{answer}
          \end{minipage}  

        \item \cpp{a=1, b=0, c=0, d=1}\par
          \begin{minipage}{2.75in}
            \begin{answer}[0.75in]
              \begin{center}
                \begin{vmatrix} 1 & 0 \\ 0 & 1 \end{vmatrix}
              \end{center}
            \end{answer}
          \end{minipage}                        
      \end{multicols}
    \end{enumerate}

  \Q Without running any code, predict the output produced by the
    following statements in the {\tt main} program of this model.
    \begin{enumerate}
      \itemsep 10pt
      \begin{multicols}{2}
        \item \cpp{A.print();} on line 29\par
          \begin{minipage}{2.75in}
            \begin{answer}[0.5in]
              \begin{center}
                \begin{vmatrix} 2 & 2 \\ 2 & 2 \end{vmatrix}
              \end{center}
            \end{answer}
          \end{minipage}  

        \item \cpp{B.print();} on line 30\par
          \begin{minipage}{2.75in}
            \begin{answer}[0.5in]
              \begin{center}
                \begin{vmatrix} 1 & 0 \\ 0 & 1 \end{vmatrix}
              \end{center}
            \end{answer}
          \end{minipage}  
      \end{multicols}
    \end{enumerate}
    
  \Q The complete code for this model can be found in 
    {\tt activity21a.cpp}.  Run the code.  Were your predictions
    accurate?
    \begin{answer}[0.5in]
      Answers will vary.
    \end{answer}
    
  \Q The method \cpp{B.setValues()} was never called
    in this code.  Explain how the values\key\\[-2.5mm] of the entries
    for matrix {\tt B} were initialized.
    \begin{answer}[1in]
      They were initialized by the method
      \cpp{Matrix::Matrix()} defined on lines 19-24.
    \end{answer}

  \Q A {\it constructor} for a class is a method that is called
    automatically when a new object variable of the class is created.
    Perform the following tasks to help determine how a constructor 
    is defined in C++.
    \begin{enumerate}
      \itemsep 10pt
      \item Change the name of the method {\tt Matrix()} to {\tt
        matrix()} (with a lower case `m') on lines 8 and 19 of the
        model.  What happens when you compile the program?
        \begin{answer}[0.5in]
          The program fails to compile and gives an error message
          that {\tt matrix} has no type.
        \end{answer}

      \item Add the appropriate function type on lines 8 and 19 so
        that the program compiles.  Run it several times and observe
        how the output different from the original model.
        \begin{answer}[0.5in]
          After adding a {\tt void} to the function {\tt matrix} the
          program compiles, but the contents of matrix {\tt B} are
          not random integers.
        \end{answer}

      \item Based on your observations above, how does C++ determine
        if an object method is a constructor?
        \begin{answer}[0.5in]
          If the name matches the class name exactly.
        \end{answer}
    \end{enumerate}