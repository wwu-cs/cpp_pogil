{\bf\large Model 1: A C++ Program} \\[-25pt]
  \begin{center}
    \begin{tabular}{p{3.0in}p{0.05in}p{3.1in}}
      \begin{minipage}{3in}
        \small
        \begin{minted}[
          frame=lines,
          framesep=2mm,
          bgcolor=gray!15,
          baselinestretch=1.2,
          linenos,
          firstnumber=5
        ]{cpp}
double balance = 0;

char getChoice() {
  char userChoice;
  cout << "Balance: $" << balance << endl;
  cout << "(d)eposit, (w)ithdraw, (e)xit: ";
  cin >> userChoice;
  return userChoice;
}

void deposit(double amount) {
  balance += amount;
}

bool withdraw(double amount) {
  bool success = false;
  if(amount < balance) {
    balance -= amount;
    success = true;
  }
  return success;
}

        \end{minted}
      \end{minipage}
      & &
      \begin{minipage}{3.1in}
        \small
        \begin{minted}[
          frame=lines,
          framesep=2mm,
          bgcolor=gray!15,
          baselinestretch=1.2,
          linenos,
          firstnumber=28
        ]{cpp}

int main() {
  char choice;
  double value;
  cout << "Balance Tracker v0.1" << endl
       << endl << fixed << showpoint
       << setprecision(2);
  do {
    choice = getChoice();
    if(choice == 'd') {
      cout << "Enter amount to deposit $";
      cin >> value;
      deposit(value);
    } else if(choice == 'w') {
      cout << "Enter amount to withdraw $";
      cin >> value;
      if(! withdraw(value)) {
        cout << "Insufficient Funds!" << endl;
      }
    }
  } while( choice != 'e' );
  cout << "Goodbye!" << endl;
}
        \end{minted} 
% $        
      \end{minipage}
    \end{tabular} 
  \end{center}  
  
  {\it\large Refer to Model 1 above as your team develops consensus answers
    to the questions below.}
    \par\vskip 10pt
    
  \begin{enumerate}
    \itemsep 20pt

    \item Indicate the line number on which each of the following
      variables is declared.
      \par\vskip 15pt
      
      \begin{enumerate}[(a)]
        \itemsep 15pt
        \begin{multicols}{2}
          \item The variable {\tt amount}: \hfill \fillin[Lines 15, 19][1in]
          \item The variable {\tt balance}: \hfill \fillin[Line 5][1in]
          \item The variable {\tt choice}: \hfill \fillin[Line 29][1in]
          \item The variable {\tt success}: \hfill \fillin[Line 20][1in]
          \item The variable {\tt userChoice}: \hfill \fillin[Line 8][1in]
          \item The variable {\tt value}: \hfill \fillin[Line 30][1in]
        \end{multicols}
      \end{enumerate}
      
\newpage      
      
    \item Categorize each of the variable declarations as one of the
      following:
      \begin{itemize}
        \item[B:] Declared inside of a function body (including {\tt main})
        \item[P:] Declared in a function header as a parameter
        \item[G:] Declared somewhere else
      \end{itemize}
      \par\vskip 15pt
      
      \begin{enumerate}[(a)]
        \itemsep 15pt
        \begin{multicols}{3}
          \item \fillin[P][0.5in] {\tt amount}
          \item \fillin[G][0.5in] {\tt balance}
          \item \fillin[B][0.5in] {\tt choice}
          \item \fillin[B][0.5in] {\tt success}
          \item \fillin[B][0.5in] {\tt userChoice}
          \item \fillin[B][0.5in] {\tt value}
        \end{multicols}
      \end{enumerate}

    \item One quick-and-dirty way to debug a program is to insert {\tt
      cout} commands to check on the value of a variable.  Suppose we
      inserted the given {\tt cout} commands above the indicated line
      numbers.  Place a check mark next to those commands which would
      compile.  You can test using the file {\tt activity14a.cpp}.
      \par\vskip 15pt
      
      \begin{enumerate}[(a)]
        \itemsep 15pt
        \begin{multicols}{2}
          \item \fillin[][0.25in] \mintinline{cpp}|cout << amount;| above line 31
          \item \fillin[$\checkmark$][0.25in] \mintinline{cpp}|cout << amount;| above line 25
          \item \fillin[$\checkmark$][0.25in] \mintinline{cpp}|cout << balance;| above line 31
          \item \fillin[$\checkmark$][0.25in] \mintinline{cpp}|cout << balance;| above line 25
          \item \fillin[$\checkmark$][0.25in] \mintinline{cpp}|cout << choice;| above line 31
          \item \fillin[][0.25in] \mintinline{cpp}|cout << choice;| above line 12
          \item \fillin[][0.25in] \mintinline{cpp}|cout << success;| above line 31
          \item \fillin[$\checkmark$][0.25in] \mintinline{cpp}|cout << success;| above line 12
          \item \fillin[][0.25in] \mintinline{cpp}|cout << userChoice;| above line 31
          \item \fillin[$\checkmark$][0.25in] \mintinline{cpp}|cout << userChoice;| above line 12
        \end{multicols}
      \end{enumerate}
      
    \item Pick one of the {\tt cout} statements above which did not compile
      correctly.  What error did the compiler report?  What does that error mean?
      \ifprintanswers\vskip -20pt\null\fi
      \begin{solution}[0.5in]
        The error is that the variable is ``not declared in this scope.''  That
        means that at this point in the code, the variable is not accessible.
      \end{solution}
      \ifprintanswers\vskip -30pt\null\fi
      
    \item You should have found that exactly one of the variables could be used in a
      {\tt cout} statement on all the given line numbers.  Which one, and why?
      \ifprintanswers\vskip -20pt\null\fi
      \begin{solution}[0.5in]
        The variable {\tt balance} can be referenced anywhere past line 5, inside a
        function or outside a function.  This is because it was declared outside of
        any function.
      \end{solution}
      \ifprintanswers\vskip -30pt\null\fi

    \item A {\it global variable} is declared outside of
      any function, including the {\tt main} program and can be
      accessed anywhere in the program. Why do you think the
      programmer made {\tt balance} a global variable?
      \ifprintanswers\vskip -20pt\null\fi
      \begin{solution}[0.5in]
        Because all parts of the code (functions, main, etc) need to alter this
        variable's value.
      \end{solution}
      \ifprintanswers\vskip -40pt\null\else\vskip -30pt\fi
      
    \item How could you modify the code to accomplish the same tasks if you were not
      allowed to\key\\[-2.5mm] use a global variable?
      \begin{solution}[0.5in]
        We could define {\tt balance} in the {\tt main} program and then pass it by
        reference to the functions that update the balance.
      \end{solution}      

\newpage

    \item The functions below are part of a grade book program.
      Define a global Boolean variable named {\tt debug} and rewrite
      the functions so that when \mintinline{cpp}|debug==true| the
      function prints out the function name when it first starts
      executing and prints out its return values before it returns.
      \vskip -30pt\null
      
      \begin{center}
        \begin{tabular}{p{3in}p{2.8in}}
          \begin{minipage}[t]{3in}
            \small
            \begin{minted}[
              frame=lines,
              framesep=2mm,
              bgcolor=gray!15,
              baselinestretch=1.2,
            ]{cpp}
double minGrade(vector<double> myGrades) {
  double tmpMin = myGrades.at(0);
  for (int i = 1; i < myGrades.size(); i++) {
    if (myGrades.at(i) < tmpMin) {
      tmpMin = myGrades.at(i);
    }
  }
  return tmpMin;
}

double avgGrade(vector<double> myGrades) {
  double sum = 0;
  for (int i = 0; i < myGrades.size(); i++) {
    sum += myGrades.at(i);
  }
  return sum / myGrades.size();
}


            \end{minted}            
          \end{minipage}
          &          
          \begin{solution}[2in]
            \par\vskip -40pt\null\scriptsize
            \begin{center}
              \begin{minipage}{2.5in}
                \begin{minted}[
                  frame=lines,                  
                  framesep=2mm,
                  bgcolor=gray!15,
                  baselinestretch=1.2,
                ]{cpp}
bool debug = true;
double minGrade(vector<double> myGrades) {
  if(debug) cout << "minGrade" << endl;  
  double tmpMin = myGrades.at(0);
  for (int i = 1; i < myGrades.size(); i++) {
    if (myGrades.at(i) < tmpMin) {
      tmpMin = myGrades.at(i);
    }
  }
  if(debug) cout << tmpMin << endl;
  return tmpMin;
}

double avgGrade(vector<double> myGrades) {
  if(debug) cout << "avgGrade" << endl;
  double sum = 0;
  for (int i = 0; i < myGrades.size(); i++) {
    sum += myGrades.at(i);
  }
  if(debug) cout << sum / myGrades.size();
  return sum / myGrades.size();
}
                \end{minted}
                \par\vskip -20pt\null
              \end{minipage}
            \end{center}
          \end{solution}
        \end{tabular}
      \end{center}
      \ifprintanswers\par\vskip -20pt\null\fi