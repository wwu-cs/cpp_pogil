\model{A C++ Program}
  \begin{center}
    \begin{tabular}{c c}
      \begin{minipage}{3.3in}
        \scriptsize
        \begin{cpplst}
double balance = 0;

char getChoice() {
  char userChoice;
  cout << "Balance: \$" << balance << endl;
  cout << "(d)eposit, (w)ithdraw, (e)xit: ";
  cin >> userChoice;
  return userChoice;
}

void deposit(double amount) {
  balance += amount;
}

bool withdraw(double amount) {
  bool success = false;
  if(amount < balance) {
    balance -= amount;
    success = true;
  }
  return success;
}

        \end{cpplst}
      \end{minipage}
      &
      \begin{minipage}{6in}
        \scriptsize
        \begin{cpplst}
int main() {
  char choice;
  double value;
  cout << "Balance Tracker v0.1" << endl
       << endl << fixed << showpoint
       << setprecision(2);
  do {
    choice = getChoice();
    if(choice == 'd') {
      cout << "Enter amount to deposit \$";
      cin >> value;
      deposit(value);
    } else if(choice == 'w') {
      cout << "Enter amount to withdraw \$";
      cin >> value;
      if(! withdraw(value)) {
        cout << "Insufficient Funds!" << endl;
      }
    }
  } while( choice != 'e' );
  cout << "Goodbye!" << endl;
}
        \end{cpplst}
      \end{minipage}
    \end{tabular} 
  \end{center}  
  
  {\it\large Refer to Model 1 above as your team develops consensus answers
    to the questions below.}

  \quest{15 min}

  \Q Indicate the line number on which each of the following
    variables is declared.
    \begin{enumerate}
      \itemsep 10pt
      \begin{multicols}{2}
        \item The variable {\tt amount}: \hspace{0.3in} \ans[1in]{Lines 11L, 15L}
        \item The variable {\tt balance}: \hspace{0.2in} \ans[1in]{Line 1L}
        \item The variable {\tt userChoice}: \hspace{0.01in} \ans[1in]{Line 4L}
        \item The variable {\tt success}: \hfill \ans[1in]{Line 16L}
        \item The variable {\tt choice}: \hspace{0.3in} \ans[1in]{Line 2R}
        \item The variable {\tt value}: \hfill \ans[1in]{Line 3R}
      \end{multicols}
    \end{enumerate}  

  \vskip -20pt
    
  \Q Categorize each of the variable declarations as one of the
    following:
    \begin{itemize}
      \item[B:] Declared inside of a function body (including {\tt main})
      \item[P:] Declared in a function header as a parameter
      \item[G:] Declared somewhere else
    \end{itemize}
    
    \begin{enumerate}
      \itemsep 10pt
      \begin{multicols}{3}
        \item \ans[0.5in]{P} {\tt amount}
        \item \ans[0.5in]{G} {\tt balance}
        \item \ans[0.5in]{B} {\tt choice}
        \item \ans[0.5in]{B} {\tt success}
        \item \ans[0.5in]{B} {\tt userChoice}
        \item \ans[0.5in]{B} {\tt value}
      \end{multicols}
    \end{enumerate}

  \Q One quick-and-dirty way to debug a program is to insert {\tt
    cout} commands to check on the value of a variable.  Suppose we
    inserted the given {\tt cout} commands above the indicated line
    numbers (L - Left, R - Right). Place a check mark next to those commands which would
    compile. You can test using the file {\tt activity14a.cpp}.
    
    \begin{enumerate}
      \itemsep 10pt
      \begin{multicols}{2}
        \item \ans[0.25in]{} \cpp{cout << amount;} above line 11R
        \item \ans[0.25in]{\checkmark} \cpp{cout << amount;} above line 12L
        \item \ans[0.25in]{\checkmark} \cpp{cout << balance;} above line 5L
        \item \ans[0.25in]{\checkmark} \cpp{cout << balance;} above line 4R
        \item \ans[0.25in]{\checkmark} \cpp{cout << choice;} above line 9R
        \item \ans[0.25in]{} \cpp{cout << choice;} above line 4L
        \item \ans[0.25in]{} \cpp{cout << success;} above line 16L
        \item \ans[0.25in]{\checkmark} \cpp{cout << success;} above line 20L
        \item \ans[0.25in]{} \cpp{cout << userChoice;} above line 8R
        \item \ans[0.25in]{\checkmark} \cpp{cout << userChoice;} above line 6L
      \end{multicols}
    \end{enumerate}

  \vskip -20pt
    
  \Q Pick one of the {\tt cout} statements above which did not compile
    correctly.  What error did the compiler report?  What does that error mean?
    \begin{answer}[0.5in]
      The error is that the variable is ``not declared in this scope.''  That
      means that at this point in the code, the variable is not accessible.
    \end{answer}
    
  \Q You should have found that exactly one of the variables could be used in a
    {\tt cout} statement on all the given line numbers.  Which one, and why?
    \begin{answer}[0.5in]
      The variable {\tt balance} can be referenced anywhere past line 1, inside a
      function or outside a function.  This is because it was declared outside of
      any function.
    \end{answer}

  \Q A {\it global variable} is declared outside of
    any function, including the {\tt main} program and can be
    accessed anywhere in the program. Why do you think the
    programmer made {\tt balance} a global variable?
    \begin{answer}[0.5in]
      Because all parts of the code (functions, main, etc) need to alter this
      variable's value.
    \end{answer}

  \vskip -10pt
    
  \Q How could you modify the code to accomplish the same tasks if you were not
    allo-\key\\[-2.5mm]wed to use a global variable?
    \begin{answer}[0.5in]
      We could define {\tt balance} in the {\tt main} program and then pass it by
      reference to the functions that update the balance.
    \end{answer}

  \newpage

  \Q The functions below are part of a grade book program.
    Define a global Boolean variable named {\tt debug} and rewrite
    the functions so that when \cpp{debug==true} the
    function prints out the function name when it first starts
    executing and prints out its return values before it returns.
    \begin{center}
      \begin{minipage}{3.6in}
        \footnotesize
        \begin{cpplst}
double minGrade(vector<double> myGrades) {
  double tmpMin = myGrades.at(0);
  for (int i = 1; i < myGrades.size(); i++) {
    if (myGrades.at(i) < tmpMin) {
      tmpMin = myGrades.at(i);
    }
  }
  return tmpMin;
}

double avgGrade(vector<double> myGrades) {
  double sum = 0;
  for (int i = 0; i < myGrades.size(); i++) {
    sum += myGrades.at(i);
  }
  return sum / myGrades.size();
}
        \end{cpplst}
      \end{minipage}
      \begin{minipage}{3.2in}
        \begin{answer}[3.5in]
          \scriptsize
          \begin{cpplst}
bool debug = true;
double minGrade(vector<double> myGrades) {
  if (debug) cout << "minGrade" << endl;  
  double tmpMin = myGrades.at(0);
  for (int i = 1; i < myGrades.size(); i++) {
    if (myGrades.at(i) < tmpMin) {
      tmpMin = myGrades.at(i);
    }
  }
  if(debug) cout << tmpMin << endl;
  return tmpMin;
}

double avgGrade(vector<double> myGrades) {
  if(debug) cout << "avgGrade" << endl;
  double sum = 0;
  for (int i = 0; i < myGrades.size(); i++) {
    sum += myGrades.at(i);
  }
  if(debug) cout << sum / myGrades.size();
  return sum / myGrades.size();
}
          \end{cpplst}
        \end{answer}
      \end{minipage}
    \end{center}