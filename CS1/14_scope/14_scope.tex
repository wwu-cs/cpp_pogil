\documentclass{exam}
%\documentclass[answers]{exam}
\hbadness=99999
\setlength{\textheight}{9.5in}
\setlength{\textwidth}{6.65in}
\setlength{\topmargin}{-0.75in}
\setlength{\oddsidemargin}{0in}
\setlength{\evensidemargin}{0in}

\usepackage{amsmath}
\usepackage{amssymb}
\usepackage{enumerate}
\usepackage[table]{xcolor}
\usepackage{hhline}
\usepackage{graphicx}
\usepackage{tikz}
%\usepackage{pgfplots}
\usepackage{multicol}
\usepackage{fancyvrb}

% for syntax highlighting
\usepackage{minted}
\usemintedstyle[cpp]{xcode}

% for overlay of output
\usepackage[overlay,showboxes]{textpos}

\pagestyle{plain}

\setlength\columnsep{50pt}
\newcommand{\key}{\hfill
      \raisebox{-.3\height}{\includegraphics[width=0.6in]{figures/key.png}}}

\begin{document}
  \thispagestyle{empty}
  \setlength{\parindent}{0pt}

  \begin{center}
    \Large Activity \#14: Variable Scope \\[5pt]
    \large Recorder's Report\\[20pt]
    \normalsize
    \begin{tabular}{lrp{0.1in}lr}
      Manager:  & \fillin[][2.0in] & & Presenter: & \fillin[][2.0in]\\[15pt]
      Recorder: & \fillin[][2.0in] & & Driver:    & \fillin[][2.0in]\\[15pt]
      Date:     & \fillin[][2.0in] & & Score:     & Satisfactory \hspace{10pt} /
      \hspace{10pt} Not Satisfactory
    \end{tabular}
  \end{center}
  \par\vskip 15pt
  
  Record your team's answers to the key questions (marked with
  \raisebox{-.3\height}{\includegraphics[width=0.5in]{figures/key.png}})
  below.
  \begin{enumerate}[(a)]
    \itemsep 1.75in
    \item Model 1, Question \#7
    \item Model 2, Question \#13
    \item Model 3, Question \#17
  \end{enumerate}

  \clearpage\pagenumbering{arabic} 
  
  \begin{center}
    \Large Activity \#14: Variable Scope \\[5pt]
    \large Activity Guide\\[20pt]
  \end{center}
  \vskip -30pt\null

  \begin{center}
    \fbox{
      \begin{minipage}{5.5in}
        {\bf Learning Objectives:} Students will be able to:
        \begin{itemize}
          \item Content:\\[-20pt]
            \begin{itemize}
              \itemsep 0pt
              \item Explain the difference between global and local variables in C++
              \item Identify the scope of a variable in a C++ code snippet
              \item Recognize variable shadowing and masking in C++ code
            \end{itemize}
          \item Process\\[-20pt]
            \begin{itemize}
              \itemsep 0pt
              \item Write programs that utilize global variables
              \item Correct errors due to variable shadowing and/or masking\\[-5pt]
            \end{itemize}
        \end{itemize}
      \end{minipage}
      }
  \end{center}
  

  {\bf\large Model 1: A C++ Program} \\[-25pt]
  \begin{center}
    \begin{tabular}{p{3.0in}p{0.05in}p{3.1in}}
      \begin{minipage}{3in}
        \small
        \begin{minted}[
          frame=lines,
          framesep=2mm,
          bgcolor=gray!15,
          baselinestretch=1.2,
          linenos,
          firstnumber=5
        ]{cpp}
double balance = 0;

char getChoice() {
  char userChoice;
  cout << "Balance: $" << balance << endl;
  cout << "(d)eposit, (w)ithdraw, (e)xit: ";
  cin >> userChoice;
  return userChoice;
}

void deposit(double amount) {
  balance += amount;
}

bool withdraw(double amount) {
  bool success = false;
  if(amount < balance) {
    balance -= amount;
    success = true;
  }
  return success;
}

        \end{minted}
      \end{minipage}
      & &
      \begin{minipage}{3.1in}
        \small
        \begin{minted}[
          frame=lines,
          framesep=2mm,
          bgcolor=gray!15,
          baselinestretch=1.2,
          linenos,
          firstnumber=28
        ]{cpp}

int main() {
  char choice;
  double value;
  cout << "Balance Tracker v0.1" << endl
       << endl << fixed << showpoint
       << setprecision(2);
  do {
    choice = getChoice();
    if(choice == 'd') {
      cout << "Enter amount to deposit $";
      cin >> value;
      deposit(value);
    } else if(choice == 'w') {
      cout << "Enter amount to withdraw $";
      cin >> value;
      if(! withdraw(value)) {
        cout << "Insufficient Funds!" << endl;
      }
    }
  } while( choice != 'e' );
  cout << "Goodbye!" << endl;
}
        \end{minted} 
% $        
      \end{minipage}
    \end{tabular} 
  \end{center}  
  
  {\it\large Refer to Model 1 above as your team develops consensus answers
    to the questions below.}
    \par\vskip 10pt
    
  \begin{enumerate}
    \itemsep 20pt

    \item Indicate the line number on which each of the following
      variables is declared.
      \par\vskip 15pt
      
      \begin{enumerate}[(a)]
        \itemsep 15pt
        \begin{multicols}{2}
          \item The variable {\tt amount}: \hfill \fillin[Lines 15, 19][1in]
          \item The variable {\tt balance}: \hfill \fillin[Line 5][1in]
          \item The variable {\tt choice}: \hfill \fillin[Line 29][1in]
          \item The variable {\tt success}: \hfill \fillin[Line 20][1in]
          \item The variable {\tt userChoice}: \hfill \fillin[Line 8][1in]
          \item The variable {\tt value}: \hfill \fillin[Line 30][1in]
        \end{multicols}
      \end{enumerate}
      
\newpage      
      
    \item Categorize each of the variable declarations as one of the
      following:
      \begin{itemize}
        \item[B:] Declared inside of a function body (including {\tt main})
        \item[P:] Declared in a function header as a parameter
        \item[G:] Declared somewhere else
      \end{itemize}
      \par\vskip 15pt
      
      \begin{enumerate}[(a)]
        \itemsep 15pt
        \begin{multicols}{3}
          \item \fillin[P][0.5in] {\tt amount}
          \item \fillin[G][0.5in] {\tt balance}
          \item \fillin[B][0.5in] {\tt choice}
          \item \fillin[B][0.5in] {\tt success}
          \item \fillin[B][0.5in] {\tt userChoice}
          \item \fillin[B][0.5in] {\tt value}
        \end{multicols}
      \end{enumerate}

    \item One quick-and-dirty way to debug a program is to insert {\tt
      cout} commands to check on the value of a variable.  Suppose we
      inserted the given {\tt cout} commands above the indicated line
      numbers.  Place a check mark next to those commands which would
      compile.  You can test using the file {\tt activity14a.cpp}.
      \par\vskip 15pt
      
      \begin{enumerate}[(a)]
        \itemsep 15pt
        \begin{multicols}{2}
          \item \fillin[][0.25in] \mintinline{cpp}|cout << amount;| above line 31
          \item \fillin[$\checkmark$][0.25in] \mintinline{cpp}|cout << amount;| above line 25
          \item \fillin[$\checkmark$][0.25in] \mintinline{cpp}|cout << balance;| above line 31
          \item \fillin[$\checkmark$][0.25in] \mintinline{cpp}|cout << balance;| above line 25
          \item \fillin[$\checkmark$][0.25in] \mintinline{cpp}|cout << choice;| above line 31
          \item \fillin[][0.25in] \mintinline{cpp}|cout << choice;| above line 12
          \item \fillin[][0.25in] \mintinline{cpp}|cout << success;| above line 31
          \item \fillin[$\checkmark$][0.25in] \mintinline{cpp}|cout << success;| above line 12
          \item \fillin[][0.25in] \mintinline{cpp}|cout << userChoice;| above line 31
          \item \fillin[$\checkmark$][0.25in] \mintinline{cpp}|cout << userChoice;| above line 12
        \end{multicols}
      \end{enumerate}
      
    \item Pick one of the {\tt cout} statements above which did not compile
      correctly.  What error did the compiler report?  What does that error mean?
      \ifprintanswers\vskip -20pt\null\fi
      \begin{solution}[0.5in]
        The error is that the variable is ``not declared in this scope.''  That
        means that at this point in the code, the variable is not accessible.
      \end{solution}
      \ifprintanswers\vskip -30pt\null\fi
      
    \item You should have found that exactly one of the variables could be used in a
      {\tt cout} statement on all the given line numbers.  Which one, and why?
      \ifprintanswers\vskip -20pt\null\fi
      \begin{solution}[0.5in]
        The variable {\tt balance} can be referenced anywhere past line 5, inside a
        function or outside a function.  This is because it was declared outside of
        any function.
      \end{solution}
      \ifprintanswers\vskip -30pt\null\fi

    \item A {\it global variable} is declared outside of
      any function, including the {\tt main} program and can be
      accessed anywhere in the program. Why do you think the
      programmer made {\tt balance} a global variable?
      \ifprintanswers\vskip -20pt\null\fi
      \begin{solution}[0.5in]
        Because all parts of the code (functions, main, etc) need to alter this
        variable's value.
      \end{solution}
      \ifprintanswers\vskip -40pt\null\else\vskip -30pt\fi
      
    \item How could you modify the code to accomplish the same tasks if you were not
      allowed to\key\\[-2.5mm] use a global variable?
      \begin{solution}[0.5in]
        We could define {\tt balance} in the {\tt main} program and then pass it by
        reference to the functions that update the balance.
      \end{solution}      

\newpage

    \item The functions below are part of a grade book program.
      Define a global Boolean variable named {\tt debug} and rewrite
      the functions so that when \mintinline{cpp}|debug==true| the
      function prints out the function name when it first starts
      executing and prints out its return values before it returns.
      \vskip -30pt\null
      
      \begin{center}
        \begin{tabular}{p{3in}p{2.8in}}
          \begin{minipage}[t]{3in}
            \small
            \begin{minted}[
              frame=lines,
              framesep=2mm,
              bgcolor=gray!15,
              baselinestretch=1.2,
            ]{cpp}
double minGrade(vector<double> myGrades) {
  double tmpMin = myGrades.at(0);
  for (int i = 1; i < myGrades.size(); i++) {
    if (myGrades.at(i) < tmpMin) {
      tmpMin = myGrades.at(i);
    }
  }
  return tmpMin;
}

double avgGrade(vector<double> myGrades) {
  double sum = 0;
  for (int i = 0; i < myGrades.size(); i++) {
    sum += myGrades.at(i);
  }
  return sum / myGrades.size();
}


            \end{minted}            
          \end{minipage}
          &          
          \begin{solution}[2in]
            \par\vskip -40pt\null\scriptsize
            \begin{center}
              \begin{minipage}{2.5in}
                \begin{minted}[
                  frame=lines,                  
                  framesep=2mm,
                  bgcolor=gray!15,
                  baselinestretch=1.2,
                ]{cpp}
bool debug = true;
double minGrade(vector<double> myGrades) {
  if(debug) cout << "minGrade" << endl;  
  double tmpMin = myGrades.at(0);
  for (int i = 1; i < myGrades.size(); i++) {
    if (myGrades.at(i) < tmpMin) {
      tmpMin = myGrades.at(i);
    }
  }
  if(debug) cout << tmpMin << endl;
  return tmpMin;
}

double avgGrade(vector<double> myGrades) {
  if(debug) cout << "avgGrade" << endl;
  double sum = 0;
  for (int i = 0; i < myGrades.size(); i++) {
    sum += myGrades.at(i);
  }
  if(debug) cout << sum / myGrades.size();
  return sum / myGrades.size();
}
                \end{minted}
                \par\vskip -20pt\null
              \end{minipage}
            \end{center}
          \end{solution}
        \end{tabular}
      \end{center}
      \ifprintanswers\par\vskip -20pt\null\fi
      
  {\bf\large Model 2: A C++ Compiler Errors} \\[-25pt]
  \begin{center}
    \begin{tabular}{p{2in}p{0.25in}p{3.4in}}
      \begin{minipage}{2in}
        \begin{minted}[
          frame=lines,
          framesep=2mm,
          bgcolor=gray!15,
          baselinestretch=1.2,
          linenos,
          firstnumber=3
        ]{cpp}
int main() {
  int varOne = 5;
  if(true) {
    int varTwo = 7;
    cout << varOne << endl;
  }
  cout << varTwo << endl;
}
        \end{minted}
      \end{minipage}
      & &
      \begin{minipage}{3.4in}
        {\bf Compiler Error:}
        \scriptsize
        \hrule\vskip 5pt
        \begin{minted}[
          frame=none,
          framesep=2mm,
          bgcolor=white,
          baselinestretch=1,
        ]{text}
model2.cpp: In function ‘int main()’:

model2.cpp:9:11: error: ‘varTwo’ was not declared in this scope
   cout << varTwo << endl;
           ^~~~~~
model2.cpp:9:11: note: suggested alternative: ‘varOne’
   cout << varTwo << endl;
           ^~~~~~
           varOne   
        \end{minted}
      \end{minipage}
    \end{tabular}
  \end{center}
  {\it\large Refer to Model 2 above as your team develops consensus answers
    to the questions below.}
  \par\vskip -20pt\null

  
  \item In your own words, describe what the compiler error shown is all about.
    \ifprintanswers\vskip -20pt\null\fi
    \begin{solution}[0.5in]
      The variable {\tt varTwo} does  not exist when the {\tt cout} command is given on line 7.
    \end{solution}
    \ifprintanswers\vskip -30pt\null\fi
  
  \item On what lines are the following variables defined in the code above?
    \par\vskip 15pt
    
    \begin{enumerate}[(a)]
      \begin{multicols}{2}
        \item The variable {\tt varOne} \hfill \fillin[Line 4][1.25in]
        \item The variable {\tt varTwo} \hfill \fillin[Line 6][1.25in]
      \end{multicols}
    \end{enumerate}
    
  \item The error message above is triggered by line 9, which comes
    after the variable {\tt varTwo} has been declared.  Why do you
    think the variable {\tt varTwo} is not available on line 9?
    \begin{solution}[0.5in]
      The variable {\tt varTwo} is not visible at line 9 because it
      was declared inside the \mintinline{cpp}|if| block.  So it is
      no longer defined after line 8 when that block ends.
    \end{solution}

\newpage

  \item The {\it scope} of a variable is the area of the program in
    which the variable is valid. Variables that have a limited scope
    are called {\it local variables} (unlike global variables, which
    can be accessed anywhere).  In the code in this model:
    \begin{itemize}
      \item The scope of the variable {\tt varOne} is from line 4 to line 10.
      \item The scope of the variable {\tt varTwo} is from line 6 to line 8.
    \end{itemize}
    Based on this information, describe how to find the scope of a
    local variable.
    \begin{solution}[0.5in]
      The scope of a local variable extends from the line on which it
      is defined to the end of the block (code in curly-braces) in
      which it was defined.
    \end{solution}
    \vskip -40pt\null
    
  \item The program below defines six different variables different
    variables.  Answer the following \key\\[-2.5mm] questions about this code, which can be
    found in {\tt activity14b.cpp}.
    
    \begin{center}
      \begin{tabular}{p{2.9in}p{2.9in}}
        \begin{minipage}{2.9in}
          \small
          \begin{minted}[
            frame=lines,
            framesep=2mm,
            bgcolor=gray!15,
            baselinestretch=1.2,
            linenos,
            firstnumber=4
          ]{cpp}
int a = 0;

void myFunc(int b = 4) {
  int c = 5;
  cout << "Value: " << /* OUTPUT 1 */;
}

int main() {
  int d = 5;
  if (d > 2) {
    int e = 3;
    if (e < 10) {
      int f = 7;
      cout << "Value: " << /* OUTPUT 2 */;
    }
    cout << "Value: " << /* OUTPUT 3 */;
  }
  cout << "Value: " << /* OUTPUT 4 */;
}
          \end{minted}
        \end{minipage}
        &
        \begin{minipage}{2.9in}
          \begin{enumerate}[(a)]
            \item What is the scope of each variable?\par\vskip 15pt
              \begin{enumerate}[i.]
                \itemsep 15pt
                \item \mintinline{cpp}|int a| -- Lines \fillin[4][0.5in] to \fillin[22][0.5in]
                \item \mintinline{cpp}|int b| -- Lines \fillin[6][0.5in] to \fillin[9][0.5in]
                \item \mintinline{cpp}|int c| -- Lines \fillin[7][0.5in] to \fillin[9][0.5in]
                \item \mintinline{cpp}|int d| -- Lines \fillin[12][0.5in] to \fillin[22][0.5in]
                \item \mintinline{cpp}|int e| -- Lines \fillin[14][0.5in] to \fillin[20][0.5in]
                \item \mintinline{cpp}|int f| -- Lines \fillin[16][0.5in] to \fillin[18][0.5in]
              \end{enumerate}
              \vskip 5pt\null
            \item Identify all global variables.
              \begin{solution}[0.5in]
                Only {\tt a} is a global variable.
              \end{solution}
          \end{enumerate}
        \end{minipage}
      \end{tabular}
    \end{center}
    
    \begin{enumerate}[(a)]
      \setcounter{enumii}{2}
      
      \item What variables could be used in place of each \mintinline{cpp}|/* OUTPUT */|
        comment without causing a compiler error?  List all variables that would work.
        \par\vskip 15pt
        \begin{enumerate}[i.]
          \itemsep 15pt
          \begin{multicols}{2}
            \item \mintinline{cpp}|/* OUTPUT 1 */| \hfill \fillin[\tt a, b, c][1.5in]
            \item \mintinline{cpp}|/* OUTPUT 2 */| \hfill \fillin[\tt a, d, e, f][1.5in]
            \item \mintinline{cpp}|/* OUTPUT 3 */| \hfill \fillin[\tt a, d, e][1.5in]
            \item \mintinline{cpp}|/* OUTPUT 4 */| \hfill \fillin[\tt a, d][1.5in]
          \end{multicols}
        \end{enumerate}
    \end{enumerate}
    
    \item Why is it important to be aware of the scope of a variable?
      \begin{solution}[0.5in]
        Because you can only access that variable within its scope.  So a programmer needs to
        know what that scope will be.
      \end{solution}
    

\newpage


  {\bf\large Model 3: A C++ Program} \\[-20pt]
  \begin{center}
    \begin{minipage}{3.5in}
      \begin{minted}[
        frame=lines,
        framesep=2mm,
        bgcolor=gray!15,
        baselinestretch=1.2,
        linenos,
        firstnumber=5
      ]{cpp}
string str = "Global Scope";

void myFunc() {
  cout << str << endl;
  string str = "myFunc Scope";
  cout << str << endl;
}

int main() {
  string str = "Main Scope";
  if (true) {
    cout << str << endl;
    string str = "If Scope";
    myFunc();
    cout << str << endl;
  }
  cout << str << endl;
}                       
      \end{minted}
    \end{minipage}
  \end{center}
  \TPMargin{5pt}
  \begin{textblock*}{1.9in}[0,0](4.5in,-2.5in)
    \textblockcolor{white}
    \begin{minipage}{1.75in}
      {\bf Output:}
      \hrule
        \begin{minted}[
          frame=none,
          framesep=2mm,
          bgcolor=white,
          baselinestretch=1,
        ]{text}
Main Scope
Global Scope
myFunc Scope
If Scope
Main Scope
      \end{minted}
    \end{minipage}
  \end{textblock*}
                                                
      
  {\it\large Refer to Model 3 above as your team develops consensus answers
    to the questions below.}
    \ifprintanswers\vskip -20pt\null\fi          

    \item In the program above, the {\tt str} identifier is used for several different
      variables.  Determine the line number of the \mintinline{cpp}|cout << str << endl;|
      command that produces each line of output.
      \par\vskip 15pt
      
      \begin{enumerate}[(a)]
        \itemsep 15pt
        \begin{multicols}{2}
          \item The first {\tt Main Scope}  \hfill \fillin[Line 16][1in]
          \item {\tt Global Scope}          \hfill \fillin[Line 8][1in]
          \item {\tt myFunc Scope}          \hfill \fillin[Line 10][1in]
          \item {\tt If Scope}              \hfill \fillin[Line 19][1in]
          \item The second {\tt Main Scope} \hfill \fillin[Line 21][1in]
        \end{multicols}
      \end{enumerate}
      
    \item When a variable declared within one scope blocks access to a variable of the same
      name declared in an outer scope (that contains the first), we say that the outer variable
      has been {\it shadowed} or that the name of the variable has been {\it masked}.  
      Give two instances of variable shadowing in the code above.
      \begin{solution}[1in]
        \begin{itemize}
          \item The global variable on line 1 is shadowed by the variable defined on line 5
            within the function.
          \item The local variable on line 10 is shadowed by the variable defined on line 13
            in the \mintinline{cpp}|if| block.
        \end{itemize}
      \end{solution}
      \par\vskip -40pt\null
      
    \item Suppose that we changed the function header on line 7 to \mintinline{cpp}|void myFunc(string str)| 
      and\key\\[-2.5mm] the function call on line 18 to \mintinline{cpp}|myFunc("Parameter Scope")|.
      How would the output change?
      \begin{solution}[0.5in]
        The ``Global Scope'' line would now say ``Parameter Scope''.
      \end{solution}

  \end{enumerate}  
    
\end{document}
