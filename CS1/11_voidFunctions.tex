\documentclass{exam}
%\documentclass[answers]{exam}
\hbadness=99999
\setlength{\textheight}{9.5in}
\setlength{\textwidth}{6.5in}
\setlength{\topmargin}{-0.75in}
\setlength{\oddsidemargin}{0in}
\setlength{\evensidemargin}{0in}

\usepackage{amsmath}
\usepackage{amssymb}
\usepackage{enumerate}
\usepackage[table]{xcolor}
\usepackage{hhline}
\usepackage{graphicx}
\usepackage{tikz}
%\usepackage{pgfplots}
\usepackage{multicol}
\usepackage{fancyvrb}

% for syntax highlighting
\usepackage{minted}
\usemintedstyle[cpp]{xcode}

% for overlay of output
\usepackage[overlay,showboxes]{textpos}

\pagestyle{plain}

\setlength\columnsep{50pt}
\newcommand{\key}{\hfill
      \raisebox{-.3\height}{\includegraphics[width=0.6in]{figures/key.png}}}

\begin{document}
  \thispagestyle{empty}
  \setlength{\parindent}{0pt}

  \begin{center}
    \Large Activity \#11: Void Functions \\[5pt]
    \large Recorder's Report\\[20pt]
    \normalsize
    \begin{tabular}{lrp{0.1in}lr}
      Manager:  & \fillin[][2.0in] & & Presenter: & \fillin[][2.0in]\\[15pt]
      Recorder: & \fillin[][2.0in] & & Driver:    & \fillin[][2.0in]\\[15pt]
      Date:     & \fillin[][2.0in] & & Score:     & Satisfactory \hspace{10pt} /
      \hspace{10pt} Not Satisfactory
    \end{tabular}
  \end{center}
  \par\vskip 15pt
  
  Record your team's answers to the key questions (marked with
  \raisebox{-.3\height}{\includegraphics[width=0.5in]{figures/key.png}})
  below.
  \begin{enumerate}[(a)]
    \itemsep 1.75in
    \item Model 1, Question \#3.c
    \item Model 2, Question \#10
    \item Model 3, Question \#17
  \end{enumerate}

  \clearpage\pagenumbering{arabic} 
  
  \begin{center}
    \Large Activity \#11: Void Functions \\[5pt]
    \large Activity Guide\\[20pt]
  \end{center}

  \begin{center}
    \fbox{
      \begin{minipage}{5.5in}
        {\bf Learning Objectives:} Students will be able to:
        \begin{itemize}
          \item Content:\\[-20pt]
            \begin{itemize}
              \itemsep 0pt
              \item Explain the concept and purpose of a function
              \item Recognize a function definition, header, and call in a program
              \item Combine function calls with looping and branching statements
              \item Develop tests for programs which use functions
            \end{itemize}
          \item Process\\[-20pt]
            \begin{itemize}
              \itemsep 0pt
              \item Write code that includes function definitions and function calls 
              \item Write programs using functions together with looping and branching statements\\[-5pt]
            \end{itemize}
        \end{itemize}
      \end{minipage}
      }
  \end{center}
  \par\vskip 10pt
  

  {\bf\large Model 1: A C++ Program} \\[-15pt]
  \begin{center}
    \begin{minipage}{4in}
      \begin{minted}[
        frame=lines,
        framesep=2mm,
        bgcolor=gray!15,
        baselinestretch=1.2,
        linenos,
        firstnumber=4
      ]{cpp}
void printMessage() {
  cout << "Welcome to C++." << endl;
  cout << "Learn the power of functions!" << endl;
}

int main() { 
  cout << "Hello Programmer!" << endl;  
  printMessage();     // function call
}
      \end{minted}
    \end{minipage}
  \end{center}
  \TPMargin{5pt}
  \begin{textblock*}{1.9in}[0,0](4.5in,-1.25in)
    \textblockcolor{white}
    \begin{minipage}{1.75in}
      {\bf Output:} 
      \hrule\vskip 5pt
      Hello Programmer!\\
      Welcome to C++.\\
      Learn the power of functions!
    \end{minipage}
  \end{textblock*}
  
  
  {\it\large Refer to Model 1 above as your team develops consensus answers
    to the questions below.}
    \par\vskip 10pt
    
  \begin{enumerate}
    \itemsep 20pt
    
    \item A {\it function} is a segment of code that encapsulates a
      single task. That is, it combines multiple lines of code into a
      single command.
      \par\vskip 20pt
      
      \begin{enumerate}[(a)]
        \itemsep 15pt
        \item Which lines of code belong to the function {\tt printMessage}? \hfill
          \fillin[Lines 4-7][2in]
        \item What other function is defined in this code? \hfill
          \fillin[{\tt main} on lines 9-12][2in]
        \item What does the function {\tt printMessage} do?
          \begin{solution}[0.4in]
            It prints out the two lines ``Welcome to C++'' and ``Learn
            the power of functions!''
          \end{solution}
      \end{enumerate}
      
    \item A {\it function definition} is the segment of code that
      tells the program what to do with the function is executed.  It
      contains a {\it function header} (the first line of the
      definition) and a {\it function body} (the commands that make up
      the function).
      \par\vskip 15pt
      
      \begin{enumerate}[(a)]
        \itemsep 15pt
        \item On what line is the function header for {\tt printMessage}? \hfill
          \fillin[Line 4][2in]
        \item On what lines is the function body for {\tt printMessage}?\hfill
          \fillin[Lines 5-6][2in]
      \end{enumerate}
      \vfill
      
\newpage      
    
    \item A {\it function call} is a command that executes the
      function.  Use the code found in {\tt activity11a.cpp} to assist
      you in answering this question.
      \par\vskip 15pt
      \begin{enumerate}[(a)]
        \itemsep 15pt
        \item On what line is the function {\tt printMessage} called? \hfill
          \fillin[Line 11][2in]
        \item What happens if you swap lines 10 and 11 in the model above?
          \begin{solution}[0.75in]
            The output would be:\par
            \begin{center}
              \begin{minipage}{3in}
                Welcome to C++\\
                Learn the power of functions!\\
                Hello Programmer!
              \end{minipage}
            \end{center}
          \end{solution}
          \par\vskip -30pt\null
        \item What line of code could you add to make the program
          print the last two lines\key\\[-2.5mm] from the model output twice?  
          Where would you add the code?
          \begin{solution}[1in]
            You could add another function call {\tt printMessage();}
            after line 11.
          \end{solution}
      \end{enumerate}
      
  
  {\bf\large Model 2: A C++ Function} \\[-20pt]
  \begin{center}
    \begin{minipage}{3.5in}
      \begin{minted}[
        frame=lines,
        framesep=2mm,
        bgcolor=gray!15,
        baselinestretch=1.2,
        linenos,
        firstnumber=6
      ]{cpp}
void printArea(double radius) {
  double area = 3.14159 * pow(radius,2);
  cout << fixed << setprecision(2);
  cout << "The area of a circle with radius "
       << radius << " is " << area << endl;
}
      \end{minted}
    \end{minipage}
  \end{center}


  {\it\large Refer to Model 2 above as your team develops consensus answers
    to the questions below.}

    \item What does this function do?
      \ifprintanswers\vskip -20pt\null\fi
      \begin{solution}[0.35in]
        This function computes and prints out the area of a circle of
        a given radius.
      \end{solution}
      \ifprintanswers\vskip -35pt\null\fi
      
    \item Copy down the function header below and underline the name
      of the function.
      \ifprintanswers\vskip -20pt\null\fi
      \begin{solution}[0.35in]
        Header: \mintinline{cpp}|void printArea(double radius)|, name: {\tt printArea}
      \end{solution}
      \ifprintanswers\vskip -35pt\null\fi

    \item Other than using a different name, how is this function header 
      different from the header in Model 1?
      \ifprintanswers\vskip -20pt\null\fi
      \begin{solution}[0.35in]
        This function also has a variable listed inside the parentheses.
      \end{solution}
      \ifprintanswers\vskip -35pt\null\fi
      
    \item A variable defined in a function header is called a 
      {\it parameter}.  What is the name and purpose of the parameter in this
      function?
      \ifprintanswers\vskip -20pt\null\fi
      \begin{solution}[0.35in]
        The parameter is named {\tt radius} and it holds the radius of
        the circle.
      \end{solution}

\newpage

    \item Determine the output produced by each of the following
      calls to this function.  You may find it helpful to use 
      the file {\tt activity11b.cpp}.
      
      \begin{enumerate}[(a)]
        \begin{multicols}{2}
          \item {\tt printArea(3); }\par
            \begin{minipage}{2.75in}
              \begin{solution}[1in]
                \par
                The area of a circle with radius 3.00 is 28.67
              \end{solution}
            \end{minipage}
          \item {\tt printArea(4.5); }\par
            \begin{minipage}{2.75in}
              \begin{solution}[1in]
                \par
                The area of a circle with radius 4.50 is 63.62
              \end{solution}
            \end{minipage}
        \end{multicols}
      \end{enumerate}

   \item Modify the {\tt main} function in the file {\tt activity11b.cpp}
     to prompt the user for a radius and then call the function {\tt
     printArea} with that radius.
      \begin{solution}[1.25in]
        \scriptsize\vskip -35pt\null
        \begin{center}
          \begin{minipage}{2.5in}
            \begin{minted}[
              frame=lines,
              framesep=2mm,
              bgcolor=gray!15,
              baselinestretch=1.2,
            ]{cpp}
int main() {            
  double radius;
  cout << "Enter circle radius: ";
  printArea(radius);
}
            \end{minted}
          \end{minipage}
        \end{center}\vskip -20pt\null
      \end{solution}
      \par\vskip -40pt\null

    \item An {\it argument} is a value or variable that is passed into
      the function.\key
      
      \begin{enumerate}[(a)]
        \itemsep 15pt
        \item What are the arguments in the two function calls in problem 8? \hfill
          \fillin[3 and 4.5][2in]
        \item What is the arguments in your solution to problem 9? \hfill
          \fillin[{\tt radius}][2in]
        \item What happens to these arguments inside the function?
          \begin{solution}[0.5in]
            \par
            The arguments are placed in the function parameter (the
            variable {\tt radius}) to be used in the body of the
            function.
          \end{solution}
        \item Do variable arguments have to have the same name as 
          the corresponding parameter?
          \hfill
          \fillin[No][0.5in]
      \end{enumerate}
      
    \item Write a function that calculates and prints out
      the diameter of a circle with a given radius.
      \begin{solution}[1.25in]
        \scriptsize\vskip -35pt\null
        \begin{center}
          \begin{minipage}{2.75in}
            \begin{minted}[
              frame=lines,
              framesep=2mm,
              bgcolor=gray!15,
              baselinestretch=1.2,
            ]{cpp}
void printDiameter(double radius) {
  double diameter = 2*radius;
  cout << fixed << setprecision(2);
  cout << "The diameter of a circle with radius "
       << radius << " is " << area << endl;
}       
            \end{minted}
          \end{minipage}
        \end{center}\vskip -20pt\null
      \end{solution}
      
    \item Add this function to the program in {\tt activity11b.cpp}
      and add a function call for the same user-entered radius you
      used in problem \#9.  Did your function work as expected?
      
\newpage


  {\bf\large Model 3: A Program with Functions and Branches} \\[-20pt]

  \begin{center}
    \begin{minipage}{3.5in}
      \scriptsize
      \begin{minted}[
        frame=lines,
        framesep=2mm,
        bgcolor=gray!15,
        baselinestretch=1.2,
        linenos,
        firstnumber=4
      ]{cpp}
// first function      
void printSum(int num1, int num2) {
  cout << num1 << " + " << num2 << " = " << (num1+num2) << endl;
}
// second function
void printDifference(int num1, int num2) {
  cout << num1 << " - " << num2 << " = " << (num1-num2) << endl;
}
// main program
int main() {
  // define variables
  int firstNumber,secondNumber;
  char operation;
  // collect user input
  cout << "Enter a number between 1 and 10: ";
  cin >> firstNumber;
  cout << "Enter another number between 1 and 10: ";
  cin >> secondNumber;
  cout << "Enter a '+' to add or a '-' to subtract: ";
  cin >> operation;
  // decide which function to call
  if (operation == '+') {
    printSum(firstNumber,secondNumber);
  } else if (operation == '-') {
    printDifference(firstNumber,secondNumber);
  } else {
    cout << "Invalid Operation" << endl;
  }  
}
      \end{minted}
    \end{minipage}  
  \end{center}
      
  {\it\large Refer to Model 3 above as your team develops consensus answers
    to the questions below.}

    \item What is the first line of code that is executed in this
      program?\hfill \fillin[Line 15][2in]
      
    \item Use the file {\tt activity11c.cpp} to execute this program
      with the following inputs and record the results.
      
      \begin{center}
        \renewcommand{\arraystretch}{1.75}
        \begin{tabular}{|c|c|c|c|p{2in}|}
          \hline
          \rowcolor{orange!20} Data Set & First Number & Second Number & Operation & Result \\
          \hline
          1 & 2 & 6 & $+$ & \ifprintanswers $2+6=8$\fi \\
          \hline
          2 & 3 & 8 & $-$ & \ifprintanswers $3-8=-5$\fi \\
          \hline
          3 & 34 & 23 & $+$ & \ifprintanswers $34 + 23 = 57$\fi \\
          \hline
          4 & 4 & 5 & $/$ & \ifprintanswers Invalid Operation\fi \\
          \hline
        \end{tabular}
      \end{center}
      
    \item What problems do you see when you entered data set 3?
      \begin{solution}[0.5in]
        The numbers are not in the range from 1-10, but no error was
        reported.
      \end{solution}
      
\newpage

    \item Write a function named {\tt checkRange} that takes two integer 
      values as parameters, check to make sure they are both between 1 and 10,
      and  prints out a warning if they are not.  For example, calling
      \mintinline{cpp}|checkRange(34,23)| might result in the
      following output.
      \begin{center}
        \tt
        WARNING! At least one of the numbers you entered was out of
        range!
      \end{center}
      \begin{solution}[1.5in]
        \scriptsize\vskip -35pt\null
        \begin{center}
          \begin{minipage}{4.5in}
            \begin{minted}[
              frame=lines,
              framesep=2mm,
              bgcolor=gray!15,
              baselinestretch=1.2,
            ]{cpp}
void checkRange(int num1, int num2) {
  if ( (num1 < 0 || num1 > 10) || (num2 < 0 || num2 > 10) ) {
    cout << "WARNING! At least one of the numbers you entered was out of range!" 
         << endl;
  }
}       
            \end{minted}
          \end{minipage}
        \end{center}\vskip -20pt\null
      \end{solution}
      \par\vskip -40pt\null
      
    \item Modify the two functions in the original model to
      produce the following sample output.\key\\[-2.5mm]  Do not change the
      {\tt main} program.  Only change the functions on lines 5-7 
      and 9-11 of the original model.
    
      \begin{center}
        \fbox{
          \begin{minipage}{5in}
            {\bf Sample Output:} 
            \hrule\vskip 5pt\tt
            Enter a number between 1 and 10: 56\\
            Enter another number between 1 and 10: 4\\
            Enter a '+' to add or a '-' to subtract: +\\
            56 + 4 = 60\\
            WARNING! At least one of the numbers you entered was out of range!
          \end{minipage}
        }
      \end{center}
      \begin{solution}[2in]
        \scriptsize\vskip -35pt\null
        \begin{center}
          \begin{minipage}{4.5in}
            \begin{minted}[
              frame=lines,
              framesep=2mm,
              bgcolor=gray!15,
              baselinestretch=1.2,
            ]{cpp}
// first function      
void printSum(int num1, int num2) {
  cout << num1 << " + " << num2 << " = " << (num1+num2) << endl;
  checkRange(num1,num2);
}
// second function
void printDifference(int num1, int num2) {
  cout << num1 << " - " << num2 << " = " << (num1-num2) << endl;
  checkRange(num1,num2);
}
            \end{minted}
          \end{minipage}
        \end{center}\vskip -20pt\null
      \end{solution}
    
    \item Write a function to draw a frog.  Then call this function
      from the\\ {\tt main} program in a loop to create the output shown.
      \TPMargin{0pt}
      \par  % avoid warning of textblock not in vertical mode
      \begin{textblock*}{1.6in}[0,0](4.9in,-40pt)
        \small
        \textblockcolor{white}
        \begin{minted}[xleftmargin=-10pt]{cpp}
Frog 1...
          @..@
         (----)
        ( >__< )
        ^^ ~~ ^^
Frog 2...
          @..@
         (----)
        ( >__< )
        ^^ ~~ ^^ 
Frog 3...
          @..@
         (----)
        ( >__< )
        ^^ ~~ ^^
Frog 4...
          @..@
         (----)
        ( >__< )
        ^^ ~~ ^^
        \end{minted}
      \end{textblock*}
      \begin{minipage}{4in}
      \begin{solution}[2in]
        \scriptsize
        \begin{minted}[
          frame=lines,
          framesep=2mm,
          bgcolor=gray!15,
          baselinestretch=1.2,
        ]{cpp}
void function frog() {
  cout << "          @..@"   << endl;
  cout << "         (----)"  << endl;
  cout << "        ( >__< )" << endl;
  cout << "        ^^ ~~ ^^" << endl;
}
int main() {
  for(int i=1; i<=4; i++) {
    cout << "Frog " << i << "..." << endl;
    frog();
  }
}
            \end{minted}
        \end{solution}
      \end{minipage}
     

  \end{enumerate}  
    
\end{document}
