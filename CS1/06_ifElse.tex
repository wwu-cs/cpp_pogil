\documentclass{exam}
%\documentclass[answers]{exam}
\hbadness=99999
\setlength{\textheight}{9.5in}
\setlength{\textwidth}{6.5in}
\setlength{\topmargin}{-0.75in}
\setlength{\oddsidemargin}{0in}
\setlength{\evensidemargin}{0in}

\usepackage{amsmath}
%\usepackage{amsfonts}
\usepackage{amssymb}
\usepackage{enumerate}
\usepackage[table]{xcolor}
\usepackage{graphicx}
\usepackage{tikz}
%\usepackage{pgfplots}
\usepackage{multicol}

% for syntax highlighting
\usepackage{minted}
\usemintedstyle[cpp]{xcode}

% for overlay of output
\usepackage[overlay,showboxes]{textpos}

\pagestyle{plain}

\setlength\columnsep{50pt}
\newcommand{\key}{\hfill
      \raisebox{-.3\height}{\includegraphics[width=0.6in]{figures/key.png}}}

\begin{document}
  \thispagestyle{empty}
  \setlength{\parindent}{0pt}

  \begin{center}
    \Large Activity \#6: If/Else Statements \\[5pt]
    \large Recorder's Report\\[20pt]
    \normalsize
    \begin{tabular}{lrp{0.1in}lr}
      Manager:  & \fillin[][2.0in] & & Presenter: & \fillin[][2.0in]\\[15pt]
      Recorder: & \fillin[][2.0in] & & Driver:    & \fillin[][2.0in]\\[15pt]
      Date:     & \fillin[][2.0in] & & Score:     & Satisfactory \hspace{10pt} /
      \hspace{10pt} Not Satisfactory
    \end{tabular}
  \end{center}
  \par\vskip 15pt
  
  Record your team's answers to the key questions (marked with
  \raisebox{-.3\height}{\includegraphics[width=0.5in]{figures/key.png}})
  below.
  \begin{enumerate}[(a)]
    \itemsep 1.75in
    \item Model 1, Question \#4
    \item Model 2, Question \#6
    \item Model 3, Question \#11.d
  \end{enumerate}

  \clearpage\pagenumbering{arabic} 
  
  \begin{center}
    \Large Activity \#6: If/Else Statements \\[5pt]
    \large Activity Guide\\[20pt]
  \end{center}

  \begin{center}
    \fbox{
      \begin{minipage}{5.5in}
        {\bf Learning Objectives:} Students will be able to:
        \begin{itemize}
          \item Content:\\[-20pt]
            \begin{itemize}
              \itemsep 0pt
              \item Implement the C++ version of an {\tt if/else} statement
              \item Generate testing data for programs that include {\tt if/else} statements
              \item Use conditional operators with strings and numeric values.
              \item Explain the purpose of a nested {\tt if/else} statement.
              \item Explain how to test code that uses an {\tt if/else/if} structure
            \end{itemize}
          \item Process\\[-20pt]
            \begin{itemize}
              \itemsep 0pt
              \item Write code that includes {\tt if}, {\tt if/else}, and {\tt if/else/if} statements. \\[-5pt]
            \end{itemize}
        \end{itemize}
      \end{minipage}
      }
  \end{center}
  \par\vskip 10pt
  
  
  {\bf\large Model 1: A C++ Program} \\[-15pt]
  \begin{center}
    \small
    \begin{minipage}{5.5in}
      \begin{minted}[
        frame=lines,
        framesep=2mm,
        bgcolor=gray!15,
        baselinestretch=1.2,
        linenos
      ]{cpp}
#include <iostream>
using namespace std;

int main() {
  float originalPrice, salePrice;
  cout << "Enter the original cost of the item: ";
  cin >> originalPrice;
  cout << "Enter the sale price: ";
  cin >> salePrice;
  int percentOff = ((originalPrice - salePrice)/originalPrice) * 100;
  cout << "Percent off: " << percentOff << "%" << endl;
  if (percentOff >= 50) {
    cout << "You found a great deal!" << endl;
  }
}      
      \end{minted}
    \end{minipage}
  \end{center}
  \TPMargin{5pt}
  
  
  {\it\large Refer to Model 1 above as your team develops consensus answers
    to the questions below.}
    \par\vskip 10pt
    
  \begin{enumerate}
    \itemsep 20pt
    
    \item You will find this program in {\tt activity06a.cpp}.  Run it with various
      original cost and sale prices and then answer the questions below.
      \begin{enumerate}[(a)]
        \item What do lines 6 and 7 do?\ifprintanswers\par\vskip -20pt\null\fi
          \begin{solution}[0.4in]
            They prompt for and store the original price in the {\tt originalPrice} variable.
          \end{solution}\ifprintanswers\par\vskip -20pt\null\fi
        \item What do lines 8 and 9 do?\ifprintanswers\par\vskip -20pt\null\fi
          \begin{solution}[0.4in]
            They prompt for and store the sale price in the {\tt salePrice} variable.
          \end{solution}\ifprintanswers\par\vskip -20pt\null\fi
        \item What do lines 10 and 11 do?\ifprintanswers\par\vskip -20pt\null\fi
          \begin{solution}[0.4in]
            They compute the percent price reduction and print it out.
          \end{solution}\ifprintanswers\par\vskip -20pt\null\fi
        \item What do lines 12-14 do?\ifprintanswers\par\vskip -20pt\null\fi
          \begin{solution}[0.4in]
            They print out ``You found a great deal!'' if the savings was
            50\% or more.
          \end{solution}\ifprintanswers\par\vskip -20pt\null\fi
      \end{enumerate}
      
    \item Revise the program in model 1 so that right after printing ``You found a great
      deal!'' it prints ``Congratulations!'' if the percent savings was 50\% or more.  Use a
      separate {\tt cout} statement to do this and make note of what you did below.
      \begin{solution}[1in]
        We added the command:\vskip -20pt\null
        \begin{center}
          \begin{minipage}{3.5in}
            \begin{minted}[
              frame=lines,
              framesep=2mm,
              bgcolor=gray!15,
              baselinestretch=1.2,
              linenos
            ]{cpp}
    cout << "Congratulations!" << endl;
            \end{minted}
          \end{minipage}
        \end{center}\vskip -15pt\null
        right below line 13.
      \end{solution}
      \ifprintanswers\vskip -30pt\null\fi
      
    \item Revise the code further so that it prints ``Done!'' when the program is
      complete, no matter what the percent off is.  Again, use a separate {\tt cout}
      statement and describe how the placement of this line of code differs from the
      placement of the code you added above.
      \begin{solution}[1in]
        We added the line of code\vskip -20pt\null
        \begin{center}
          \begin{minipage}{3.5in}
            \begin{minted}[
              frame=lines,
              framesep=2mm,
              bgcolor=gray!15,
              baselinestretch=1.2,
              linenos
            ]{cpp}
  cout << "Done!" << endl;
            \end{minted}
          \end{minipage}
        \end{center}\vskip -15pt\null
        right below the curly brace on line 14 of the original program (line 15 after or
        modification above.
      \end{solution}
      \ifprintanswers\vskip -40pt\null\fi

    \item What happens if you remove the open curly brace (\mintinline{cpp}|{|) from the
      end of line 12 and the close\key\\[-2.5mm] curly brace (\mintinline{cpp}|}|) from line 14 (15 in
      your modified program)?  Explain why this happens.
      
      \begin{solution}[1in]
        \par
        If we remove these curly braces, the program always prints ``Congratulations!'',
        regardless of what the percent savings is.  This is because the {\tt if} statement
        only  applies to the next statement or block of statements.  Removing the curly brace
        makes it only apply to line 13.
      \end{solution}
      
  
  {\bf\large Model 2: Another C++ Program} \\
  \ifprintanswers\vskip -40pt\null\fi
  \begin{center}
    \small
    \begin{minipage}{5.5in}
      \begin{minted}[
        frame=lines,
        framesep=2mm,
        bgcolor=gray!15,
        baselinestretch=1.2,
        linenos
      ]{cpp}
#include <iostream>
using namespace std;

int main() {
  int numCredits = 194;
  double majorGPA = 2.9;
  double overallGPA = 2.1;
  if ( /* missing Boolean expression */ ) {
    cout << "Congratulations!" << endl;
    cout << "You seem to meet the criteria for graduation." << endl;
  } else {
    cout << "Sorry!" << endl;
    cout << "You do not meet all the criteria for graduation." << endl;
  }
}      
      \end{minted}
    \end{minipage}
  \end{center}
  \TPMargin{5pt}
  \ifprintanswers\vskip -20pt\null\fi
  

  {\it\large Refer to Model 2 above as your team develops consensus answers
    to the questions below.}
    \par\vskip 10pt

    \item In order to graduate from WWU with a Bachelor's degree, students must have
      earned at least 192 credits, have a GPA of at least 2.0 in their major, and have an
      overall GPA of at least 2.0 (among other things).  Which of the following Boolean
      expressions should be used on line 8 of this model to test if a student meets these
      graduation criteria?
      \par\vskip 10pt
      \begin{checkboxes}
        \choice \mintinline{cpp}{numCredits >= 192 || majorGPA >= 2.0 || overallGPA >= 2.0}
        \choice \mintinline{cpp}{numCredits > 192 && majorGPA > 2.0 && overallGPA > 2.0}
        \correctchoice \mintinline{cpp}{numCredits > 191 && majorGPA >= 2.0 && overallGPA >= 2.0}
        \correctchoice \mintinline{cpp}{numCredits >= 192 && majorGPA >= 2.0 && overallGPA >= 2.0}
      \end{checkboxes}
      \par\vskip -30pt\null

    \item Add the Boolean expression you chose above to the program in {\tt activity06b.cpp}. Create \key\\[-2.5mm]
      test data to check all eight ($2\times 2\times 2$) different combinations for
      the sub-expressions of the Boolean expression.  Then run each of those test cases and verify that 
      the program passes the test. 
      \par\vskip 15pt
      \begin{center}
        \renewcommand{\arraystretch}{1.8}
        \begin{tabular}{|c|c|c|c|c|c|}
          \hline
            \rowcolor{orange!20} Test Case & {\tt numCredits} & {\tt majorGPA} & {\tt overallGPA} & Expected (graduate/don't) & Passed \\
          \hline
            1 & \ifprintanswers 196\fi & \ifprintanswers 3.6\fi & \ifprintanswers 3.2\fi & \ifprintanswers graduate\fi & \\
          \hline
            2 & \ifprintanswers 196\fi & \ifprintanswers 3.6\fi & \ifprintanswers 1.8\fi & \ifprintanswers don't\fi & \\
          \hline
            3 & \ifprintanswers 196\fi & \ifprintanswers 1.9\fi & \ifprintanswers 3.2\fi & \ifprintanswers don't\fi & \\
          \hline
            4 & \ifprintanswers 196\fi & \ifprintanswers 1.7\fi & \ifprintanswers 1.9\fi & \ifprintanswers don't\fi & \\
          \hline
            5 & \ifprintanswers 188\fi & \ifprintanswers 3.6\fi & \ifprintanswers 3.2\fi & \ifprintanswers don't\fi & \\
          \hline
            6 & \ifprintanswers 184\fi & \ifprintanswers 3.2\fi & \ifprintanswers 1.8\fi & \ifprintanswers don't\fi & \\
          \hline
            7 & \ifprintanswers 190\fi & \ifprintanswers 1.9\fi & \ifprintanswers 3.2\fi & \ifprintanswers don't\fi & \\
          \hline
            8 & \ifprintanswers  53\fi & \ifprintanswers 1.6\fi & \ifprintanswers 1.9\fi & \ifprintanswers don't\fi & \\
          \hline
        \end{tabular}
      \end{center}
      
    \item A {\it edge case test} is a test of a natural edges or boundary of the program.  An example of a natural boundary
      in this program is where the {\tt majorGPA} is exactly 2.0, since this is a division line in determining if somebody can
      graduate.  Give at least two other edge cases that could be tested.
      \begin{solution}[1in]
        \par
        We could test where {\tt overallGPA} is exactly 2.0 and where {\tt numCredits} is exactly 192.  Additional boundary
        conditions might include the limits of an integer range for {\tt numCredits}, the division line between positive and
        negative values, etc.
      \end{solution}
      
\newpage

  {\bf\large Model 3: A Third C++ Program} \\[-10pt]
  \begin{center}
    \begin{minipage}{5.5in}
      \begin{minted}[
        frame=lines,
        framesep=2mm,
        bgcolor=gray!15,
        baselinestretch=1.2,
        linenos
      ]{cpp}
#include <iostream>
using namespace std;

int main() {
  int grade;
  cout << "Enter your grade: ";
  cin >> grade;
  if (grade >= 90) {
    cout << "Very Good!" << endl;
  } else {
    if (grade >= 60) {
      cout << "Satisfactory." << endl;
    } else {
      cout << "Poor." << endl;
    }
  }
}
      \end{minted}
    \end{minipage}
  \end{center}
  \TPMargin{5pt}
  
  {\it\large Refer to Model 3 above as your group develops consensus answers
    to the questions below.}

      \item Circle the {\tt if/else} statement that is {\it nested} inside of 
        another {\tt if/else} statement.
        \begin{solution}[0.4in]
          The {\tt if/else} statement from lines 11-14 is nested inside the one from lines 8-16.
        \end{solution}

      \item This code can be found in the file {\tt activity06c.cpp}.  Give three different grade
        values that could be used to test different parts of the program.  Indicate what part of 
        the program the value is testing.
        \par\vskip 10pt
        \begin{center}
          \renewcommand{\arraystretch}{2.5}
          \begin{tabular}{|c|c|p{4in}|}
            \hline
            \rowcolor{orange!20} Test & {\tt grade} Value & Part of Program Tested \\
            \hline
            1 & \ifprintanswers 95\fi & \ifprintanswers Tests lines 8-9\fi \\
            \hline
            2 & \ifprintanswers 82\fi & \ifprintanswers Tests lines 10-12\fi \\
            \hline
            3 & \ifprintanswers 55\fi & \ifprintanswers Tests lines 10, 11, 13-14\fi \\
            \hline
          \end{tabular}
        \end{center}
        
      \item Run the program and enter try your tests.  Does everything work as expected?
        \begin{solution}[0.5in]
          Yes, the program does appear to work as expected.
        \end{solution}

\newpage

      \item Consider the code snippet below.
        \begin{center}
          \begin{minipage}{3.5in}
            \begin{minted}[
              frame=lines,
              framesep=2mm,
              bgcolor=gray!15,
              baselinestretch=1.2,
              linenos,
              firstnumber=8
            ]{cpp}
  if (grade >= 90) {
    cout << "Very Good!" << endl;
  } else if (grade >= 60) {
    cout << "Satisfactory." << endl;
  } else {
    cout << "Poor." << endl;
  }
}
            \end{minted}
          \end{minipage}
        \end{center}
        \par\vskip -20pt\null
        
        \begin{enumerate}
          \item Replace lines 8-18 in model 3 with this code. How does the output change?
            \begin{solution}[0.5in]
              The output does not change.
            \end{solution}
            
          \item Which method of solving this problem contains simpler syntax and indentation -- the 
            one in the original model, or the one above?  Explain.
            \begin{solution}[0.5in]
              The new one has simpler syntax and does not require nesting.
            \end{solution}
            
          \item You can use as many {\tt else/if} statements as you need.  Suppose you wanted to add the comment ``Good!'' for grades
            that are between 80 and 90.  Write the code for this change.
            \begin{solution}[1.5in]
              \vskip -30pt\ \null
              \begin{center}
                \scriptsize
                \begin{minipage}{3.5in}
                  \begin{minted}[
                    frame=lines,
                    framesep=2mm,
                    bgcolor=gray!15,
                    baselinestretch=1.2,
                    linenos
                  ]{cpp}
  if (grade >= 90) {
    cout << "Very Good!" << endl;
  } else if (grade >= 80) {       // added
    cout << "Good!" << endl;      // added
  } else if (grade >= 60) {
    cout << "Satisfactory." << endl;
  } else {
    cout << "Poor." << endl;
  }
                  \end{minted}
                \end{minipage}
              \end{center}              
            \end{solution}
            
          \item Does it make a difference where you add the additional {\tt else/if} statement?
            Compare\key\\[-2.5mm] adding it at line 10 in the code snippet above vs. at line 12.
            \begin{solution}[1in]
              Yes, it makes a difference.  If we add it at line 12 instead of at line 10, then the ``Good!'' option is never used because
              any grade above 80 will also be above 60 and would fall into the earlier ``Satisfactory.''
              part of the {\tt if/else/if} statement.
            \end{solution}
            
          \item Is the use of the final {\tt else} statement mandatory when creating an {\tt if/else/if} statement?  
            Try it out and supply an example to support your claim.
            \begin{solution}[0.75in]
              Not mandatory.  Dropping lines 12-14 will just print nothing for grades less than 60.
            \end{solution}
            
          \item Make one final change to your program.  Adjust it so that it prints an error message 
            if the grade entered is greater than 100 or less than 0.
        \end{enumerate}        
        
  \end{enumerate}  
    
\end{document}
