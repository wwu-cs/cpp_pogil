\documentclass{exam}
%\documentclass[answers]{exam}
\hbadness=99999
\setlength{\textheight}{9.5in}
\setlength{\textwidth}{6.5in}
\setlength{\topmargin}{-0.75in}
\setlength{\oddsidemargin}{0in}
\setlength{\evensidemargin}{0in}

\usepackage{amsmath}
%\usepackage{amsfonts}
\usepackage{amssymb}
\usepackage{enumerate}
\usepackage[table]{xcolor}
\usepackage{graphicx}
\usepackage{tikz}
%\usepackage{pgfplots}
\usepackage{multicol}
\usepackage{fancyvrb}

% for syntax highlighting
\usepackage{minted}
\usemintedstyle[cpp]{xcode}

% for overlay of output
\usepackage[overlay,showboxes]{textpos}

\pagestyle{plain}

\setlength\columnsep{50pt}
\newcommand{\key}{\hfill
      \raisebox{-.3\height}{\includegraphics[width=0.6in]{figures/key.png}}}

\begin{document}
  \thispagestyle{empty}
  \setlength{\parindent}{0pt}

  \begin{center}
    \Large Activity \#9: Nested Loops \\[5pt]
    \large Recorder's Report\\[20pt]
    \normalsize
    \begin{tabular}{lrp{0.1in}lr}
      Manager:  & \fillin[][2.0in] & & Presenter: & \fillin[][2.0in]\\[15pt]
      Recorder: & \fillin[][2.0in] & & Driver:    & \fillin[][2.0in]\\[15pt]
      Date:     & \fillin[][2.0in] & & Score:     & Satisfactory \hspace{10pt} /
      \hspace{10pt} Not Satisfactory
    \end{tabular}
  \end{center}
  \par\vskip 15pt
  
  Record your team's answers to the key questions (marked with
  \raisebox{-.3\height}{\includegraphics[width=0.5in]{figures/key.png}})
  below.
  \begin{enumerate}[(a)]
    \itemsep 1.75in
    \item Model 1, Question \#5
    \item Model 2, Question \#8 (describe your solution)
    \item Model 3, Question \#13 (write out your code)
  \end{enumerate}

  \clearpage\pagenumbering{arabic} 
  
  \begin{center}
    \Large Activity \#9: Nested Loops \\[5pt]
    \large Activity Guide\\[20pt]
  \end{center}

  \begin{center}
    \fbox{
      \begin{minipage}{5.5in}
        {\bf Learning Objectives:} Students will be able to:
        \begin{itemize}
          \item Content:\\[-20pt]
            \begin{itemize}
              \itemsep 0pt
              \item Read and write nested {\tt for} loops.
              \item Identify inner and outer loops.
            \end{itemize}
          \item Process\\[-20pt]
            \begin{itemize}
              \itemsep 0pt
              \item Write code that uses nested {\tt for} loops. \\[-5pt]
            \end{itemize}
        \end{itemize}
      \end{minipage}
      }
  \end{center}
  \par\vskip 10pt
  

  {\bf\large Model 1: A C++ Programs} \\[-15pt]
  \begin{center}
    \begin{minipage}{2.5in}
      \begin{minted}[
        frame=lines,
        framesep=2mm,
        bgcolor=gray!15,
        baselinestretch=1.2,
        linenos,
        firstnumber=6
      ]{cpp}
  string name;
  cout << "What is your name? ";
  cin >> name;
  for (int i=0; i<5; i++) {
    for (int j=0; j<3; j++) {
      cout << name << " ";
    }
    cout << endl;
  }
      \end{minted}
    \end{minipage}
  \end{center}
  \TPMargin{5pt}
  
  
  {\it\large Refer to Model 1 above as your team develops consensus answers
    to the questions below.}
    \par\vskip 10pt
    
  \begin{enumerate}
    \itemsep 20pt
    
    \item What does this program display?  Try to determine this
      without running it first, then check your work against
      the output of the program in {\tt activity09a.cpp}.
      \begin{solution}[0.75in]
        \par
        The program prints out a 3 x 5 grid of the person's name.
      \end{solution}
      
    \item Answer the following questions regarding the      
      \mintinline{cpp}|for| loops in this code snippet.
      \par\vskip 20pt
      
      \begin{enumerate}[(a)]
        \itemsep 15pt
        \item How many \mintinline{cpp}|for| loops are in this code?: \hfill
          \fillin[There are two][2.5in]
        \item Does one loop completely finish before the next begins?  \hfill
          \fillin[No.  One loop is inside the other][2.5in]
        \item What do you call this arrangement of loops?  \hfill
          \fillin[These are nested loops][2.5in]
      \end{enumerate}
    
    \item How many times does each {\tt cout} statement in this model
      execute?
      \par\vskip 20pt
      
      \begin{enumerate}[(a)]
        \itemsep 15pt
        \item The {\tt cout} on line 7 of the model:\hfill
          \fillin[Executes once][2.5in]
        \item The {\tt cout} on line 11 of the model: \hfill
          \fillin[Executes $5\times 3 = 15$ times][2.5in]
        \item The {\tt cout} on line 13 of the model: \hfill
          \fillin[Executes five times][2.5in]
      \end{enumerate}
      
\newpage

    \item A loop that appears within another loop is called a {\it
      nested loop}.  We can refer to the two loops as the {\it outer
      loop} and the {\it inner loop}.
      \par\vskip 20pt
      
      \begin{enumerate}[(a)]
        \itemsep 15pt
        \item On what line number in the model does the outer loop start?\hfill
          \fillin[Line 4][2in]
        \item On what line number in the model does the inner loop start? \hfill
          \fillin[Line 5][2in]
        \item On what line number does the inner loop in the model end?\hfill
          \fillin[Line 7][2in]
        \item On what line number does the outer loop in the model end?\hfill
          \fillin[Line 9][2in]
      \end{enumerate}
      \par\vskip -35pt\null
                
    \item What does each of the loops in the model do?\key\\[-2.5mm]
      \begin{enumerate}[(a)]
        \item What does the inner loop do?
          \begin{solution}[0.5in]
            The inner loop prints out a name 3 times with a space
            after each one.
          \end{solution}
        \item What does the outer loop do?
          \begin{solution}[0.5in]
            The outer loop repeats the inner loop 5 times and prints
            an {\tt endl} after each one.
          \end{solution}
      \end{enumerate}
      \par\vskip 10pt

  
  {\bf\large Model 2: Output from Several Nested Loops} \\[-10pt]

% define first output as verbatim box
\newbox\verbboxOne
\setbox\verbboxOne=\vbox{\hsize=1.4in
\begin{Verbatim}
* * * * *
* * * * *
* * * * *
* * * * *
* * * * *
* * * * *
* * * * *
\end{Verbatim}
}

% define second output as verbatim box
\newbox\verbboxTwo
\setbox\verbboxTwo=\vbox{\hsize=1.4in
\begin{Verbatim}
1 1 1 1 1
2 2 2 2 2
3 3 3 3 3
4 4 4 4 4
5 5 5 5 5
6 6 6 6 6
7 7 7 7 7
\end{Verbatim}
}

% define third output as verbatim box
\newbox\verbboxThree
\setbox\verbboxThree=\vbox{\hsize=1.4in
\begin{Verbatim}
1 2 3 4 5
1 2 3 4 5
1 2 3 4 5
1 2 3 4 5
1 2 3 4 5
1 2 3 4 5
1 2 3 4 5
\end{Verbatim}
}


  \begin{center}
    \small
    \begin{tabular}{p{1.5in}p{0.25in}p{1.5in}p{0.25in}p{1.5in}}
      \fbox{\begin{minipage}{1.4in}
        \centering\box\verbboxOne
      \end{minipage}}
      & &
      \fbox{\begin{minipage}{1.4in}
        \centering\box\verbboxTwo
      \end{minipage}}      
      & &
      \fbox{\begin{minipage}{1.4in}
        \centering\box\verbboxThree
      \end{minipage}}
      \\
      \centering (a) & & \centering (b) & & \centering (c) \\
    \end{tabular}
  \end{center}

  {\it\large Refer to Model 2 above as your team develops consensus answers
    to the questions below.}

    \item There are several ways to write a C++ program that outputs
      the contents of box (a) above.  Briefly describe how it could be
      done using the following methods.
      \par\vskip 15pt
      
      \begin{enumerate}[(a)]
        \item Using a set of {\tt cout} statements without any loops.
          \begin{solution}[0.75in]
            \par
            Create seven identical {\tt cout} statements,
            each of which prints {\tt * * * * *} and an {\tt endl}.
          \end{solution}
        \item Using a single \mintinline{cpp}|for| loop (without any nesting).
          \begin{solution}[0.75in]
            \par
            The \mintinline{cpp}|for| loop would execute seven times
            and each time {\tt cout} the five {\tt *}'s and an {\tt endl}.
          \end{solution}
      \end{enumerate}
      
    \item In this exercise we will create the output in box (a) using nested loops.    
      \par\vskip 10pt
      
      \begin{enumerate}[(a)]
        \item Surround the {\tt cout} statement below by a \mintinline{cpp}|for|
          so that five {\tt *} characters are printed on a single line.
          \ifprintanswers
            \begin{solution}
              \scriptsize\vskip -35pt\null
              \begin{center}
                \begin{minipage}{2.5in}
                  \begin{minted}[
                    frame=lines,
                    framesep=2mm,
                    bgcolor=gray!15,
                    baselinestretch=1.2,
                  ]{cpp}
for (int i=0; i<5; i++) {
  cout << "* ";
}
                  \end{minted}
                \end{minipage}
              \end{center}\vskip -20pt\null
            \end{solution}
          \else
            \par\vskip 10pt\null
            \begin{center}
              \mintinline{cpp}{cout << "* ";}
            \end{center}
            \par\vskip 10pt\null
          \fi
        \item Now build on the block of code above by wrapping the
          loop you created in another \mintinline{cpp}|for| loop that
          prints seven of these lines, with an {\tt endl} between each one.
          \begin{solution}[1.25in]
            \scriptsize\vskip -35pt\null
            \begin{center}
              \begin{minipage}{2.5in}
                \begin{minted}[
                  frame=lines,
                  framesep=2mm,
                  bgcolor=gray!15,
                  baselinestretch=1.2,
                ]{cpp}
for (int j=0; j<7; j++) {                
  for (int i=0; i<5; i++) {
    cout << "* ";
  }
  cout << endl;
}
                \end{minted}
              \end{minipage}
            \end{center}\vskip -20pt\null
          \end{solution}
      \end{enumerate}
      
    \item Suppose that variables \mintinline{cpp}|int rows| and
      \mintinline{cpp}|int cols| contain the number of rows and 
      columns\key\\[-2.5mm] of {\tt *}'s you wish to print.  How
      would you modify your code above to print out that number of
      rows and columns?  Test your solution in {\tt activity09b.cpp}.
      
      \begin{solution}[1.25in]
        \scriptsize\vskip -35pt\null
        \begin{center}
          \begin{minipage}{2.5in}
            \begin{minted}[
              frame=lines,
              framesep=2mm,
              bgcolor=gray!15,
              baselinestretch=1.2,
            ]{cpp}
for (int j=0; j<rows; j++) {
  for (int i=0; i<cols; i++) {
    cout << "* ";
  }
  cout << endl;
}
            \end{minted}
          \end{minipage}
        \end{center}\vskip -20pt\null
      \end{solution}

    \item How would you modify your code in the previous question to
      produce the output seen in box (b), assuming that
      \mintinline{cpp}|int rows = 7| and \mintinline{cpp}|int cols = 4|?

      \begin{solution}[1.25in]
        \scriptsize\vskip -35pt\null
        \begin{center}
          \begin{minipage}{2.5in}
            \begin{minted}[
              frame=lines,
              framesep=2mm,
              bgcolor=gray!15,
              baselinestretch=1.2,
            ]{cpp}
for (int j=0; j<rows; j++) {
  for (int i=0; i<cols; i++) {
    cout << j;
  }
  cout << endl;
}
            \end{minted}
          \end{minipage}
        \end{center}\vskip -20pt\null
      \end{solution}

    \item How would you modify your code so that the output
      matches the output in box (c) instead of (b)?
      \begin{solution}[1.25in]
        \scriptsize\vskip -35pt\null
        \begin{center}
          \begin{minipage}{2.5in}
            \begin{minted}[
              frame=lines,
              framesep=2mm,
              bgcolor=gray!15,
              baselinestretch=1.2,
            ]{cpp}
for (int j=0; j<rows; j++) {
  for (int i=0; i<cols; i++) {
    cout << i;
  }
  cout << endl;
}
            \end{minted}
          \end{minipage}
        \end{center}\vskip -20pt\null
      \end{solution}

\newpage

  {\bf\large Model 3: More Nested Loop Output} \\[-5pt]
  
% define fourth output as verbatim box
\newbox\verbboxFour
\setbox\verbboxFour=\vbox{\hsize=1.4in
\begin{Verbatim}
1
1 2
1 2 3
1 2 3 4
1 2 3 4 5
\end{Verbatim}
}

% define fifth output as verbatim box
\newbox\verbboxFive
\setbox\verbboxFive=\vbox{\hsize=1.4in
\begin{Verbatim}
1 2 3 4 5
1 2 3 4
1 2 3
1 2 
1
\end{Verbatim}
}

% define third output as verbatim box
\newbox\verbboxSix
\setbox\verbboxSix=\vbox{\hsize=1.4in
\begin{Verbatim}
1
1 2
1 2 3
1 2
1
\end{Verbatim}
}


  \begin{center}
    \small
    \begin{tabular}{p{1.5in}p{0.25in}p{1.5in}p{0.25in}p{1.5in}}
      \fbox{\begin{minipage}{1.4in}
        \centering\box\verbboxFour
      \end{minipage}}
      & &
      \fbox{\begin{minipage}{1.4in}
        \centering\box\verbboxFive
      \end{minipage}}      
      & &
      \fbox{\begin{minipage}{1.4in}
        \centering\box\verbboxSix
      \end{minipage}}
      \\
      \centering (a) & & \centering (b) & & \centering (c) \\
    \end{tabular}
  \end{center}
  
  {\it\large Refer to Model 3 above as your team develops consensus answers
    to the questions below.}

    \item How are the output samples in this model different from
      those in model 2?  Note: Write a single general statement that
      summarizes the difference for all output boxes.
      
      \begin{solution}[1in]
        \par
        The number of columns in a row (i.e. its length) depends on
        which row we are in.
      \end{solution}
      
    \item The code below will produce one of these three outputs.
      Determine which one and justify your answer.
      
      \begin{center}
        \begin{tabular}{p{2.5in}p{2.5in}}
          \begin{minipage}{2.5in}
            \begin{minted}[
              frame=lines,
              framesep=2mm,
              bgcolor=gray!15,
              baselinestretch=1.2,
            ]{cpp}
int numRows = 5;
for (int j=numRows; j>0; j--) {
  for (int i=1; i<=j; i++) {
    cout << i << " ";
  }
  cout << endl;
}
            \end{minted}
          \end{minipage}
          &
          \begin{minipage}{2.5in}
            \begin{solution}[1in]
              \par
              It will produce the output in box (b).  We can tell
              because it is counting down from 5 instead of counting
              up to 5 or counting up and then down.
            \end{solution}
          \end{minipage}
        \end{tabular}
      \end{center}
      \par\vskip -20pt\null

    \item The code from the previous problem is in
      {\tt activity09c.cpp}.  Rewrite it so that it produces \key\\[-2.5mm] the output
      in the other of box (a) or (b).
      
      \begin{solution}[1.5in]
        \scriptsize\vskip -35pt\null
        \begin{center}
          \begin{minipage}{2.5in}
            \begin{minted}[
              frame=lines,
              framesep=2mm,
              bgcolor=gray!15,
              baselinestretch=1.2,
            ]{cpp}
int numRows = 5;
for (int j=1; j<=numRows; j++) {
  for (int i=1; i<=j; i++) {
    cout << i << " ";
  }
  cout << endl;
}
            \end{minted}
          \end{minipage}
        \end{center}\vskip -20pt\null
      \end{solution}
      
    \item Add some initialization code that prompts the user to enter
      the number of rows in the triangle that your program creates.
      Test your solution to verify that it works.

\newpage

    \item Use nested \mintinline{cpp}|for| loops to produce the output
      in box (c) above.
      
      \begin{solution}[2.5in]
        \scriptsize\vskip -35pt\null
        \begin{center}
          \begin{minipage}{3.5in}
            \begin{minted}[
              frame=lines,
              framesep=2mm,
              bgcolor=gray!15,
              baselinestretch=1.2,
            ]{cpp}
  int numRows = 5, length;
  for (int i = 1; i <= numRows; i++) {
    if ((numRows % 2 == 0 && i <= numRows / 2) ||
        (numRows % 2 == 1 && i <= numRows / 2 + 1) ) {
      length = i;
    }
    else {
      length = numRows - i + 1;
    }
    for (int j = 1; j <= length; j++) {
      cout << j << " ";
    }
    cout << endl;
  }
            \end{minted}
          \end{minipage}
        \end{center}\vskip -20pt\null
      \end{solution}

    \item Write a program that prompts the user for information on
      three students.  For each student, collect the student name and
      three quiz grades.  Then display the name and quiz average
      (formatted to two decimal places).  Sample output is shown below.
    
      \begin{center}
        \begin{tabular}{p{2.6in}p{3in}}
          \begin{minipage}{2.6in}
            \small
            \begin{minted}[
              frame=lines,
              framesep=2mm,
              bgcolor=gray!15,
              baselinestretch=1.2,
            ]{html}
Enter name of student 1: Mary
Enter score 1: 78
Enter score 2: 90
Enter score 3: 91
Name: Mary
Average: 86.33

Enter name of student 2: Kevin
Enter score 1: 90
Enter score 2: 77
Enter score 3: 85
Name: Kevin
Average: 84.00

Enter name of student 3: Jose
Enter score 1: 79
Enter score 2: 83
Enter score 3: 92
Name: Jose
Average: 84.67
            \end{minted}
          \end{minipage}
          &
          \begin{minipage}{3in}
            \begin{solution}
              \scriptsize
              \begin{minted}[
                frame=lines,
                framesep=2mm,
                bgcolor=gray!15,
                baselinestretch=1.2,
              ]{cpp}
string name;
double score,sum;
cout << showpoint << fixed << setprecision(2);
for (int i=1; i<=3; i++) {
  cout << "Enter name of student " << i << ": ";
  cin >> name;
  sum = 0;
  for (int j=1; j<=3; j++) {
    cout << "Enter score " << j << ": ";
    cin >> score;
    sum += score;
  }
  cout << "Name: " << name << endl;
  cout << "Average: " << (sum / 3) << endl << endl;
  }
}  
              \end{minted}
            \end{solution}
          \end{minipage}
        \end{tabular}
      \end{center}

  \end{enumerate}  
    
\end{document}
