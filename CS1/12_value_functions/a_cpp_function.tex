\model{A C++ Function}
  \begin{center}
    \begin{minipage}{4.5in}
      \begin{cpplst}
void getHypotenuse(double a, double b) {
  double square = pow(a,2) + pow(b,2);
  double squareRoot = sqrt(square);
  cout << "The hypotenuse length is " << squareRoot << endl;
}
      \end{cpplst}
    \end{minipage}
  \end{center}


  {\it\large Refer to Model 2 above as your team develops consensus answers
    to the questions below.}

  \quest{20 min}

  \Q What does this function do?
    \begin{answer}[0.35in]
      It solves for the length of the hypotenuse of a right triangle given
      the length of the two legs and prints it out.
    \end{answer}
    
  \Q Is this a {\it void function} or a {\it value-returning} function?
    \hfill\ans[2in]{void}

  \Q Suppose you wanted to turn this into a value-returning function
    that returns the hypotenuse length.
    \begin{enumerate}
      \itemsep 5pt
      \item How would you change line 1 in the model? \hfill
        \ans[3in]{\scriptsize\cpp{double getHypotenuse(double a, double b)}}

      \item What command would you add to the end? \hfill
        \ans[3in]{\cpp{return squareRoot;}}

      \item What line would you remove and why? \hfill
        \ans[3in]{Line 4 -- function doesn't produce output}
    \end{enumerate}
  
  \vskip -10pt
    
  \Q What is the difference between using {\tt cout} in a function and using
    {\tt return} in a function?
    \begin{answer}[0.5in]
      Using {\tt cout} prints values out to the screen while using {\tt return}
      sends the value back to the caller, where it can decide what to do with it.
    \end{answer}

  \vskip -10pt
  
  \Q Below is the framework for a program that drills students on addition
    problems.\key\\[-2.5mm]  In particular, the program should do the following:
    \begin{itemize}
      \item Display five addition problems, one at a time, and let the student answer each
      \item Print the correct answer if the user enters an incorrect answer
      \item Print a congratulatory message if the student's answer is correct
      \item Keep track of the number of problems the student answers correctly
      \item Print a special message if the user gets all five problems right
    \end{itemize}
    Fill in the missing code to make this program function
    as described.  The code is in {\tt activity12b.cpp}.
    \begin{center}
      \begin{tabular}{p{3.2in}p{0.1in}p{3.2in}}
        \begin{minipage}{3.2in}
          \begin{cpplst}
// Print celebratory rocket

/* ANSWER A */ printRocket() {
  cout << "Blast-off!" << endl;
  cout << "  ^"    << endl;
  cout << " /*\\"  << endl;
  cout << "/***\\" << endl;
  cout << "|***|"  << endl;
  cout << "|***|"  << endl;
  cout << "|#|#|"  << endl;
  /* ANSWER B */
}

// give problem and check answer
bool giveProblem(/* ANSWER C */) {
  int studentAns;
  int correctAns = num1+num2;
  cout << num1 << "+" << num2 << "=";
  cin >> studentAns;
  /* ANSWER D */
}
            \end{cpplst}
          \end{minipage}
          & &
          \begin{minipage}{3.2in}
            \begin{cpplst}        
int main() {
  srand(time(0));  // seed random num
  int numCorrect = 0;
  for(/* ANSWER E */) {
    int num1 = rand() % 10 + 1;
    int num2 = rand() % 10 + 1;
    if (/* ANSWER F */) {
      cout << "Correct!" << endl;
      numCorrect++;
    } else {
      cout << "Incorrect! It is "
           << (num1 + num2) << endl;
    }
  }
  if( /* ANSWER G */ ) {
    /* ANSWER H */
  } else {
    cout << "You got " << numCorrect
         << " correct." << endl;
  }
}
            \end{cpplst}
          \end{minipage}
        \end{tabular}
      \end{center}
      
      \begin{enumerate}
        \begin{multicols}{2}
          \item Line 3: The type of {\tt printRocket}:\par
            \begin{minipage}{2.75in}
              \begin{answer}[0.55in]
                {\tt void}
              \end{answer}
            \end{minipage}

          \item Line 11: A {\tt return} statement:\par
            \begin{minipage}{2.75in}
              \begin{answer}[0.55in]
                {\tt return} or empty
              \end{answer}
            \end{minipage}

          \item Line 15: Parameters for {\tt giveProblem}:\par
            \begin{minipage}{2.75in}
              \begin{answer}[0.55in]
                \cpp{int num1, int num2}
              \end{answer}
            \end{minipage}

          \item Line 20: A {\tt return} statement:\par
            \begin{minipage}{2.75in}
              \begin{answer}[0.55in]
                \small
                \cpp{return studentAns==correctAns}
              \end{answer}
            \end{minipage}

          \item Line 4: Setup for {\tt for} loop:\par
            \begin{minipage}{2.75in}
              \begin{answer}[0.55in]
                \cpp{int i=0; i<5; i++}
              \end{answer}
            \end{minipage}

          \item Line 7: Function call in {\tt if} condition:\par
            \begin{minipage}{2.75in}
              \begin{answer}[0.55in]
                \cpp{giveProblem(num1,num2)}
              \end{answer}
            \end{minipage}

          \item Line 15: Condition for {\tt if} statement:\par
            \begin{minipage}{2.75in}
              \begin{answer}[0.55in]
                \cpp{numCorrect==5}
              \end{answer}
            \end{minipage}

          \item Line 16: Command if all 5 correct:\par
            \begin{minipage}{2.75in}
              \begin{answer}[0.55in]
                \cpp{printRocket();}
              \end{answer}
            \end{minipage}                      
        \end{multicols}
      \end{enumerate}