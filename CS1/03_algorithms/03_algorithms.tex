\documentclass{exam}
%\documentclass[answers]{exam}
\hbadness=99999
\setlength{\textheight}{9.5in}
\setlength{\textwidth}{6.5in}
\setlength{\topmargin}{-0.75in}
\setlength{\oddsidemargin}{0in}
\setlength{\evensidemargin}{0in}

\usepackage{amsmath}
%\usepackage{amsfonts}
\usepackage{amssymb}
\usepackage{enumerate}
\usepackage[table]{xcolor}
\usepackage{graphicx}
\usepackage{tikz}
%\usepackage{pgfplots}
\usepackage{multicol}
\usepackage[
  colorlinks=true,
  pdftex,
  pdfauthor={Jonathan Duncan},
]{hyperref}

% for syntax highlighting
\usepackage{minted}
\usemintedstyle[cpp]{xcode}

% for overlay of output
\usepackage[overlay,showboxes]{textpos}

\pagestyle{plain}

\setlength\columnsep{50pt}
\newcommand{\key}{\hfill
      \raisebox{-.3\height}{\includegraphics[width=0.6in]{figures/key.png}}}

\begin{document}
  \thispagestyle{empty}
  \setlength{\parindent}{0pt}

  \begin{center}
    \Large Activity \#3: Algorithmic Thinking \\[5pt]
    \large Recorder's Report\\[20pt]
    \normalsize
    \begin{tabular}{lrp{0.1in}lr}
      Manager:  & \fillin[][2.0in] & & Presenter: & \fillin[][2.0in]\\[15pt]
      Recorder: & \fillin[][2.0in] & & Driver:    & \fillin[][2.0in]\\[15pt]
      Date:     & \fillin[][2.0in] & & Score:     & Satisfactory \hspace{10pt} /
      \hspace{10pt} Not Satisfactory
    \end{tabular}
  \end{center}
  \par\vskip 15pt
  
  Record your team's answers to the key questions (marked with
  \raisebox{-.3\height}{\includegraphics[width=0.5in]{figures/key.png}})
  below.
  \begin{enumerate}[(a)]
    \item Model 1, Question \#4
      \par\vskip 1.5in
    \item Model 2, Question \#5
      \par\vskip 3.5in
    \item Model 3, Question \#16
  \end{enumerate}

  \clearpage\pagenumbering{arabic} 
  
  \begin{center}
    \Large Activity \#3: Algorithmic Thinking \\[5pt]
    \large Activity Guide\\[20pt]
  \end{center}

  \begin{center}
    \fbox{
      \begin{minipage}{5.5in}
        {\bf Learning Objectives:} Students will be able to:
        \begin{itemize}
          \item Content:\\[-20pt]
            \begin{itemize}
              \itemsep 0pt
              \item Explain to other students how to duplicate a model
              \item Understand the need for clarity when describing an algorithm.
              \item Analyze the efficiency of oral and written algorithms.
            \end{itemize}
          \item Process\\[-20pt]
            \begin{itemize}
              \itemsep 0pt
              \item Create oral algorithms
              \item Create a written algorithm
            \end{itemize}
        \end{itemize}
      \end{minipage}
      }
  \end{center}
  \par\vskip 10pt
  
  
  {\bf\large Model 1: Google Directions from KRH to Walmart} \\
  \begin{center}
    \fcolorbox{black}{orange!20}{
    \begin{minipage}{3.5in}
      \begin{enumerate}
        \itemsep -2pt
        \item Head south towards SW 4th Street
        \item Turn left onto SW 4th Street
        \item Travel 499 feet
        \item Turn right onto S. College Avenue
        \item Travel 0.7 miles
        \item Turn left onto Lamperti Lane
        \item Travel 0.2 miles
        \item Continue onto SE Lamperti Street
        \item Travel 469 feet
        \item Turn right into Walmart Parking Lot
        \item Destination will be on the right
      \end{enumerate}
    \end{minipage}
    }
  \end{center}
  \par\vskip 10pt
  
  {\it\large Refer to Model 1 above as your team develops consensus answers
    to the questions below.}
    \par\vskip 10pt
    
  \begin{enumerate}
    \itemsep 20pt
    
    \item Give at least one assumption that is being made in the
      directions above.
      \begin{solution}[0.75in]
        We are assuming that the individual is driving a car.
      \end{solution}
      \vfill

    \item List at least two additional instructions that could be
      added to improve on these directions.  Indicate where in the list
      they should be added.
      \begin{solution}[1in]
        \begin{itemize}
          \item Before 1: Get into your car
          \item Between items 3 and 4: Stop at the intersection until
            the light turns green.
          \item Answers will vary{\ldots} 
        \end{itemize}
      \end{solution}
      \vfill
      
\newpage      
      
    \item An {\it algorithm} is a series of instructions that can be
      repeated over and over with the same results. Based on your
      answers above, give two important things to consider when 
      describing an algorithm.      
      \begin{solution}[1in]
        \begin{itemize}
          \item The context that is assumed
          \item The completeness of the list of instructions
          \item The preciseness of each instruction
          \item etc{\ldots} 
        \end{itemize}
      \end{solution}
      
      
    \item Think about the process of brushing your teeth.  Write down
      step-by-step instructions\key\\[-3.5mm] (e.g. an algorithm) that
      would allow somebody who has never brushed their teeth before\\
      to successfully accomplish this task.
      \begin{solution}[2.5in]
        Answers will vary.
      \end{solution}
      \vfill
      
  
  {\bf\large Model 2: An Object Built Using LEGOs} \\[-10pt]
    \begin{center}
      \fcolorbox{black}{orange!20}{
        \begin{minipage}{5in}
          \vskip 5pt
          {\bf Important Instructions}
          \hrule\vskip 5pt
          This model is an object built with LEGO-like blocks. In our
	  online environment, we will use the following online LEGO 
	  environment.
	  \begin{center}
  	    \href{https://www.mecabricks.com/en/workshop}{https://www.mecabricks.com/en/workshop}
	  \end{center}
	  It is important that you follow the rules below precisely.
          \begin{enumerate}
            \item Your {\bf driver} should show the video
	    ``MecaBricks.mp4'' to the entire group.  This can be found
	    in Teams.
            \item Your {\bf manager} should watch the video
	    ``ManagerOnly.mp4'' {\bf by themselves}.  Nobody
	    else should watch this video.  
	    \item Based on the object the manager sees in this video,
	    he or she should give instructions to the driver as to
	    how to build the object.
            \item Your {\bf team recorder} should write down all
	    instructions that are given by the manager.
          \end{enumerate}
          Limit yourself to 5 minutes to complete this task.\vskip -4pt\null
        \end{minipage}
      }
    \end{center}

  {\it\large Refer to Model 2 above as your team develops consensus answers
    to the questions below.}
    \par\vskip 10pt
    
    \item Give yourself a five minute timmer and then begin constructing
      your object.  Remember\key\\[-3.5mm] that the {\bf team manager} is
      the only one who can give instructions and the {\bf team driver}\\
      and {\bf presenter} may not see the original model.  Work until
      your five minute timer runs out.
      \vfill
      
    \item How many total instructions were given during your build
      time? \hfill \fillin[][2in]
      
    \item Looking through your list of instructions.  Identify several
      that were useful and copy them below.
      \begin{solution}[0.75in]
        Answers will vary.
      \end{solution}
      
    \item Looking through your list of instructions again, identify
      several that weren't helpful and list them.
      \begin{solution}[0.75in]
        Answers will vary.
      \end{solution}
      
    \item Describe the types of instructions that work best in
      defining an algorithm.
      \begin{solution}[1in]
        Answers will vary.
      \end{solution}
      
    \item How was this activity related to computer programming?  How
      is it different?
      \begin{solution}[1in]
        Answers will vary.
      \end{solution}
      
\newpage

  {\bf\large Model 3: Your Own Object Built Using LEGOs} \\[-10pt]
    \begin{center}
      \fcolorbox{black}{orange!20}{
        \begin{minipage}{5in}
          \vskip 5pt
          {\bf Important Instructions}
          \hrule\vskip 5pt
	  Work together to construct a simple object using the LEGO 
	  building program.  Once you are finished, take a screen shot
	  of it to show later.
        \end{minipage}
      }
    \end{center}
    \par\vskip 10pt
  
  {\it\large Refer to Model 3 above as your team develops consensus answers
    to the questions below.}
    
      \item Develop a written algorithm (list of instructions) for
        building your object.
	\par\vskip 1.5in
        
      \item Post your algorithm and a list of the number of each size
        of brick used in the general Teams channel.
        
      \item Pick one of the other team's algorithm to construct their
        object. Reply to their instructions with a screen shot of your
	object.
        
      \item How well did the other group do at building your object?      
        \begin{solution}[0.75in]
          Answers vary.
        \end{solution}
        
      \item Suppose the object they built doesn't match your original.
        Describe a scenario in which it would be:
        \begin{enumerate}[(a)]
          \item the fault of the other team.
            \begin{solution}[0.75in]
              The the other team did not follow the list of
              instructions correctly, it would be their fault.
            \end{solution}
          \item the fault of your team.
            \begin{solution}[0.75in]
              If we did provided an incomplete or imprecise algorithm,
              it would be our fault.
            \end{solution}
        \end{enumerate}
	
	\newpage
            
      \item Suppose a program you write does not do what you intended
        it to do.  Describe a scenario\key\\[-2.5mm] in which it would be:
        \begin{enumerate}[(a)]
          \item the computer's fault.
            \begin{solution}[0.75in]
              If there is a bug in the program that is running your
              code or possibly faulty hardware.
            \end{solution}
          \item your fault as the programmer.
            \begin{solution}[0.75in]
              If the code we gave was inaccurate.
            \end{solution}
        \end{enumerate}
              
      \item Which of the scenarios in the previous problem do you
        think is most likely?
        \begin{solution}[0.5in]
          It is most likely the programmers fault.
        \end{solution}        
        
  \end{enumerate}  
    
\end{document}
