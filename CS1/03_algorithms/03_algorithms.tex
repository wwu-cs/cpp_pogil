% Source: https://github.com/ChrisMayfield/cspogil/tree/master/CS0/Act01
% File: "algorithms.tex"
% Access: 04-15-2022

% comment out for student version
% \ifdefined\Student\relax\else\def\Teacher{}\fi

\documentclass[12pt]{article}

\title{Activity \#3: Algorithmic Thinking}
\author{Chris Mayfield}
\newcommand{\activityeditor}{Preston Carman}
\newcommand{\activitysource}{\url{https://github.com/ChrisMayfield/cspogil/tree/master/CS0/Act01}}
\date{Spring 2022}

\input{../../cspogil.sty}
\usepackage{textpos}

\begin{document}

\begin{center}
  \Large Activity \#3: Algorithmic Thinking \\[5pt]
  \large Recorder's Report\\[20pt]
  \normalsize
  \begin{tabular}{lrp{0.1in}lr}
      Manager:  & \ans{} &  & Reader: & \ans{}            \\[15pt]
      Recorder: & \ans{} &  & Driver: & \ans{}            \\[15pt]
      Date:     & \ans{} &  & Score:  & Satisfactory \hspace{10pt} /
      \hspace{10pt} Not Satisfactory
  \end{tabular}
\end{center}
\par\vskip 15pt

Record your team's answers to the key questions (marked with
\raisebox{-.3\height}{\includegraphics[width=0.5in]{../figures/key.png}})
below.
\begin{enumerate}[label=(\alph*)]
  \itemsep 1.25in
  \item Model 1, Question \#4
  \item Model 2, Question \#5
  \item Model 3, Question \#16
\end{enumerate}

\newpage
\maketitle

In this course, you will work in teams of 3--4 students to learn new concepts.
This activity will introduce you to the process of analyzing an algorithm complexity.

%\rolenames

\guide{
  \item Explain to other students how to duplicate a model
  \item Understand the need for clarity when describing an algorithm.
  \item Analyze the efficiency of oral and written algorithms.

}{
  \item Create oral algorithms
  \item Create a written algorithm
}{
No additional notes.
}


  {\bf\large Model 1: Google Directions from KRH to Walmart} \\
  \begin{center}
    \fcolorbox{black}{orange!20}{
    \begin{minipage}{3.5in}
      \begin{enumerate}
        \itemsep -2pt
        \item Head south towards SW 4th Street
        \item Turn left onto SW 4th Street
        \item Travel 499 feet
        \item Turn right onto S. College Avenue
        \item Travel 0.7 miles
        \item Turn left onto Lamperti Lane
        \item Travel 0.2 miles
        \item Continue onto SE Lamperti Street
        \item Travel 469 feet
        \item Turn right into Walmart Parking Lot
        \item Destination will be on the right
      \end{enumerate}
    \end{minipage}
    }
  \end{center}
  \par\vskip 10pt
  
  {\it\large Refer to Model 1 above as your team develops consensus answers
    to the questions below.}
    \par\vskip 10pt
    
  \begin{enumerate}
    \itemsep 20pt
    
    \Q Give at least one assumption that is being made in the
      directions above.
      \begin{answer}[0.75in]
        We are assuming that the individual is driving a car.
      \end{answer}
      \vfill

    \Q List at least two additional instructions that could be
      added to improve on these directions.  Indicate where in the list
      they should be added.
      \begin{answer}[1in]
        \begin{itemize}
          \item Before 1: Get into your car
          \item Between items 3 and 4: Stop at the intersection until
            the light turns green.
          \item Answers will vary{\ldots} 
        \end{itemize}
      \end{answer}
      \vfill
      
\newpage      
      
    \Q An {\it algorithm} is a series of instructions that can be
      repeated over and over with the same results. Based on your
      answers above, give two important things to consider when 
      describing an algorithm.      
      \begin{answer}[1in]
        \begin{itemize}
          \item The context that is assumed
          \item The completeness of the list of instructions
          \item The preciseness of each instruction
          \item etc{\ldots} 
        \end{itemize}
      \end{answer}
      
      
    \Q Think about the process of brushing your teeth.  Write down
      step-by-step instructions\key\\[-3.5mm] (e.g. an algorithm) that
      would allow somebody who has never brushed their teeth before\\
      to successfully accomplish this task.
      \begin{answer}[2.5in]
        Answers will vary.
      \end{answer}
      \vfill
      
\newpage

\model{An Object Built Using LEGOs} \\
  \begin{center}
    \fcolorbox{black}{orange!20}{
      \begin{minipage}{5in}
        {\bf Important Instructions}
        \hrule\vskip 5pt
        This model is an object built with LEGO-like blocks. In our
        online environment, we will use the following online LEGO 
        environment.
        \begin{center}
          \href{https://www.mecabricks.com/en/workshop}{https://www.mecabricks.com/en/workshop}
        \end{center}
        It is important that you follow the rules below precisely.
        \begin{enumerate}
          \item Your {\bf driver} should show the video
            ``MecaBricks.mp4'' to the entire group.  This can be found
            in Teams.
          \item Your {\bf manager} should watch the video
            ``ManagerOnly.mp4'' {\bf by themselves}.  Nobody
            else should watch this video.  
          \item Based on the object the manager sees in this video,
            he or she should give instructions to the driver as to
            how to build the object.
          \item Your {\bf team recorder} should write down all
            instructions that are given by the manager.
        \end{enumerate}
        Limit yourself to 5 minutes to complete this task.
      \end{minipage}
    }
  \end{center}
  
{\it\large Refer to Model 2 above as your team develops consensus answers
  to the questions below.}
  \par\vskip 10pt
  
  \Q Give yourself a five minute timmer and then begin constructing
    your object.  Remem\key\\[-3.5mm]ber that the {\bf team manager} is
    the only one who can give instructions and the {\bf team driver}
    and {\bf presenter} may not see the original model.  Work until
    your five minute timer runs out.
    % \begin{answer}[0.3in]
    %   Answers will vary.
    % \end{answer}
    

  \Q How many total instructions were given during your build
    time?
    \begin{answer}[0.75in]
      Answers will vary.
    \end{answer}
    
  \Q Looking through your list of instructions.  Identify several
    that were useful and copy them below.
    \begin{answer}[0.75in]
      Answers will vary.
    \end{answer}
    
  \Q Looking through your list of instructions again, identify
    several that weren't helpful and list them.
    \begin{answer}[0.75in]
      Answers will vary.
    \end{answer}
    
  \Q Describe the types of instructions that work best in
    defining an algorithm.
    \begin{answer}[1in]
      Answers will vary.
    \end{answer}
    
  \Q How was this activity related to computer programming?  How
    is it different?
    \begin{answer}[1in]
      Answers will vary.
    \end{answer}
\newpage
\model{Your Own Object Built Using LEGOs} \\
    \begin{center}
      \fcolorbox{black}{orange!20}{
        \begin{minipage}{5in}
          \vskip 5pt
          {\bf Important Instructions}
          \hrule\vskip 5pt
            Work together to construct a simple object using the LEGO 
            building program.  Once you are finished, take a screen shot
            of it to show later.
        \end{minipage}
      }
    \end{center}
    \par\vskip 10pt
  
  {\it\large Refer to Model 3 above as your team develops consensus answers
    to the questions below.}
    
      \Q Develop a written algorithm (list of instructions) for
        building your object.
        \begin{answer}[1.5in]
          Answers will vary.
        \end{answer}
        
      \Q Post your algorithm and a list of the number of each size
        of brick used in the general Teams channel.
        
      \Q Pick one of the other team's algorithm to construct their
        object. Reply to their instructions with a screen shot of your
	object.
        
      \Q How well did the other group do at building your object?      
        \begin{answer}[0.75in]
          Answers vary.
        \end{answer}
        
      \Q Suppose the object they built doesn't match your original.
        Describe a scenario in which it would be:
        \begin{enumerate}
          \item the fault of the other team.
            \begin{answer}[0.75in]
              The the other team did not follow the list of
              instructions correctly, it would be their fault.
            \end{answer}
          \item the fault of your team.
            \begin{answer}[0.75in]
              If we did provided an incomplete or imprecise algorithm,
              it would be our fault.
            \end{answer}
        \end{enumerate}
            
      \Q Suppose a program you write does not do what you intended
        it to do.  Describe a\key\\[-2.5mm] scenario in which it would be:
        \begin{enumerate}
          \item the computer's fault.
            \begin{answer}[0.75in]
              If there is a bug in the program that is running your
              code or possibly faulty hardware.
            \end{answer}
          \item your fault as the programmer.
            \begin{answer}[0.75in]
              If the code we gave was inaccurate.
            \end{answer}
        \end{enumerate}
              
      \Q Which of the scenarios in the previous problem do you
        think is most likely?
        \begin{answer}[0.5in]
          It is most likely the programmers fault.
        \end{answer}        
    
\end{document}
