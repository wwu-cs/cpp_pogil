\model{A C++ Program and Input File} \\
  \begin{center}
    \begin{tabular}{p{3in}p{0.25in}p{1.5in}}
      \begin{minipage}{3in}
        \centering {\bf C++ Code}\vskip -15pt\null
        \small
        \begin{minted}[
          frame=lines,
          framesep=2mm,
          bgcolor=gray!15,
          baselinestretch=1,
          linenos
        ]{cpp}
#include <iostream>
#include <fstream>
#include <string>
using namespace std;

int main() {
  string sport;
  ifstream fin;
  int count = 1;

  fin.open("sports.txt");
  if (!fin.is_open()) {     
    cout << "Could not open file" << endl;
    return 1;
  }

  while (fin >> sport) {
    sport.at(0) = toupper(sport.at(0));
    cout << "Sport " << count 
         << ": " << sport << endl;
    count++;
  }
  fin.close();  
}
        \end{minted}
      \end{minipage}
      & &
      \begin{minipage}{1.5in}
        \centering {\bf sports.txt File}\vskip -15pt\null
        \small
        \begin{minted}[
          frame=lines,
          framesep=2mm,
          bgcolor=gray!15,
          linenos
        ]{bash}
basketball
baseball
football
volleyball
tennis
golf
lacrosse
soccer
badminton
bowling
fortnight
skiing
diving
ice hockey
biking
rugby
swimming
sailing
rowing
skateboarding
        \end{minted}
      \end{minipage}
    \end{tabular}
  \end{center}  
  
  {\it\large Refer to Model 1 above as your team develops consensus answers
    to the questions below.}
    \par\vskip 10pt
    
  \begin{enumerate}
    \itemsep 20pt
 
    \Q This code can be found in {\tt activity17a.cpp}.  Run it and then
      describe what the program does.
      % \ifprintanswers\vskip -20pt\null\fi
      \begin{answer}[0.75in]
        The program reads in one line from the {\tt sports.txt} file at a time, changes the first
        letter of the sport to be upper case, and then prints out the sport name with a label of
        ``Sport n:'' where $n$ is the line number on which the sport was found.
      \end{answer}
      % \ifprintanswers\vskip -30pt\null\fi

    \Q On line 11 of the C++ code, what does the string \cpp{"sports.txt"} represent?
      % \ifprintanswers\vskip -20pt\null\fi
      \begin{answer}[0.5in]
        This is the name of the file to be read.
      \end{answer}

\newpage

    \Q Without discussing its relative merits as a sport, remove line 11 from the text file 
      \cpp{"sports.txt"|, save your change, and rerun the program. What happened?
      \ifprintanswers\vskip -20pt\null\fi
      \begin{answer}[0.5in]
        The ``sport'' of {\it fortnight} is no longer printed out.
      \end{answer}
      \ifprintanswers\vskip -35pt\null\fi

    \Q C++ can get input from a text file in much the same way it does from the
      keyboard.  To do this it uses {\it file input streams}.  Answer the questions below
      about the use of file input streams in this model.
      \par\vskip 15pt
      
      \begin{enumerate}[(a)]
        \itemsep 12pt
        \item What is the name of the library that includes these streams?
          \hfill \ans[1.5in]{\tt fstream}
        \item What is the type of variable we use for a file input stream?
          \hfill \ans[1.5in]{\tt ifstream}
        \item The file input stream {\tt fin} is used on line 11.  What does this line do?
          % \ifprintanswers\vskip -20pt\null\fi
          \begin{answer}[0.4in]
            It opens the file that we will be reading from (gives the name, etc).
          \end{answer}
        \item The file input stream {\tt fin} is used again on line 12.  What does this line do?
          % \ifprintanswers\vskip -20pt\null\fi
          \begin{answer}[0.4in]
            It opens the file that we will be reading from (gives the name, etc).
          \end{answer}
        \item On what line does the program actually read input from the file?
          \hfill \ans[1.5in]{line 17}
        \item On what line is the file closed (indicating we are done reading)?
          \hfill \ans[1.5in]{line 13}
      \end{enumerate}          
      
    \Q How is the sport named on line 14 of the text file printed by this program?
      % \ifprintanswers\vskip -20pt\null\fi
      \begin{answer}[0.5in]
        On two lines instead of one (i.e. ``Ice'' and ``Hockey'' instead of ``Ice
        hockey'').
      \end{answer}
      % \ifprintanswers\vskip -35pt\null\fi
    
    \Q Below is a slight modification of this program (only lines 17-22 are shown).  Make this
      modification in the file {\tt activity17a.cpp} and record how the output changed.
      \par\vskip -30pt\null
      
      \begin{center}
        \begin{tabular}{p{2.7in}p{3.1in}}
          \begin{minipage}{2.7in}
            \small
            \begin{minted}[
              frame=lines,
              framesep=2mm,
              bgcolor=gray!15,
              baselinestretch=1,
              linenos,
              firstnumber=17
            ]{cpp}
  while (getline(fin,sport)) {
    sport.at(0) = toupper(sport.at(0));
    cout << "Sport " << count 
         << ": " << sport << endl;
    count++;
  }
            \end{minted}
          \end{minipage}
          &
          \begin{minipage}{3.1in}
            \begin{answer}[1in]
              \par
              The sport is printed as ``Ice hockey'' (on one line).
            \end{answer}
          \end{minipage}
        \end{tabular}
      \end{center}
      % \ifprintanswers\vskip -35pt\null\fi
      
    \Q What is the difference between the command \cpp{fin >> sport} and the
      command \cpp{getline(fin,sport)}?
      % \ifprintanswers\vskip -20pt\null\fi
      \begin{answer}[0.5in]
        The first reads up to the first white space (space or newline), the second reads the
        entire line (up to a newline).
      \end{answer}
      % \ifprintanswers\vskip -60pt\else\vskip -40pt\fi\null
      
    \Q Rewrite the program so that it opens a user-entered filename and prints out the
      sports\key\\[-2.5mm] found inside. Test it with the file \cpp{"alternatives.txt"}.
      % \ifprintanswers\vskip -20pt\null\fi
      \begin{answer}[0.5in]
        Add a variable \cpp{string filename} and change line 11 to
        \cpp{fin.open(filename)}.
      \end{answer}