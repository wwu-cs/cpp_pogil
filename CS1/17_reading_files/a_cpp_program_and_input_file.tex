\model{A C++ Program and Input File}
  \begin{center}
    \begin{tabular}{p{3.5in}p{0.25in}p{1.5in}}
      \begin{minipage}{3.5in}
        \centering {\bf C++ Code}
        \small
        \begin{cpplst}
#include <iostream>
#include <fstream>
#include <string>

using namespace std;

int main() {
  string sport;
  ifstream fin;
  int count = 1;

  fin.open("sports.txt");
  if (!fin.is_open()) {     
    cout << "Could not open file" << endl;
    return 1;
  }

  while (fin >> sport) {
    sport.at(0) = toupper(sport.at(0));
    cout << "Sport " << count 
         << ": " << sport << endl;
    count++;
  }
  fin.close();  
}
        \end{cpplst}
      \end{minipage}
      & &
      \begin{minipage}{1.5in}
        \centering {\bf sports.txt File}
        \small
        \begin{verbatim}
basketball
baseball
football
volleyball
tennis
golf
lacrosse
soccer
badminton
bowling
fortnight
skiing
diving
ice hockey
biking
rugby
swimming
sailing
rowing
skateboarding
        \end{verbatim}
      \end{minipage}
    \end{tabular}
  \end{center}  
  
  {\it\large Refer to Model 1 above as your team develops consensus answers
    to the questions below.}

  \quest{15 min}
 
  \Q This code can be found in {\tt activity17a.cpp}.  Run it and then
    describe what the program does.
    \begin{answer}[0.75in]
      The program reads in one line from the {\tt sports.txt} file at a time, changes the first
      letter of the sport to be upper case, and then prints out the sport name with a label of
      ``Sport n:'' where $n$ is the line number on which the sport was found.
    \end{answer}

  \Q On line 12 of the C++ code, what does the string \cpp{"sports.txt"} represent?
    \begin{answer}[0.5in]
      This is the name of the file to be read.
    \end{answer}

  \Q Without discussing its relative merits as a sport, remove line 11 from the text file 
    \cpp{"sports.txt"}, save your change, and rerun the program. What happened?
    \begin{answer}[0.5in]
      The ``sport'' of {\it fortnight} is no longer printed out.
    \end{answer}

  \Q C++ can get input from a text file in much the same way it does from the
    keyboard. To do this it uses {\it file input streams}. Answer the questions below
    about the use of file input streams in this model.
    \begin{enumerate}
      \itemsep 10pt
      \item What is the name of the library that includes these streams?
        \hfill \ans[1.5in]{\tt fstream}

      \item What is the type of variable we use for a file input stream?
        \hfill \ans[1.5in]{\tt ifstream}

      \item The file input stream {\tt fin} is used on line 12.  What does this line do?
        \begin{answer}[0.4in]
          It opens the file that we will be reading from (gives the name, etc).
        \end{answer}

      \item The file input stream {\tt fin} is used again on line 13.  What does this line do?
        \begin{answer}[0.4in]
          It checks whether the file is open or not.
        \end{answer}

      \item On what line does the program actually read input from the file?
        \hfill \ans[1.5in]{line 18}

      \item On what line is the file closed (indicating we are done reading)?
        \hfill \ans[1.5in]{line 24}
    \end{enumerate}          
    
  \Q How is the sport named on line 14 of the text file printed by this program?
    \begin{answer}[0.5in]
      On two lines instead of one (i.e. ``Ice'' and ``Hockey'' instead of ``Ice
      hockey'').
    \end{answer}
  
  \Q Below is a slight modification of this program (only lines 18-23 are shown).  Make this
    modification in the file {\tt activity17a.cpp} and record how the output changed.
    \begin{center}
      \begin{tabular}{p{3.2in}p{3.1in}}
        \begin{minipage}{3.2in}
          \small
          \begin{cpplst}
while (getline(fin,sport)) {
  sport.at(0) = toupper(sport.at(0));
  cout << "Sport " << count 
       << ": " << sport << endl;
  count++;
}
          \end{cpplst}
        \end{minipage}
        &
        \begin{minipage}{3.1in}
          \begin{answer}[1in]
            \par
            The sport is printed as ``Ice hockey'' (on one line).
          \end{answer}
        \end{minipage}
      \end{tabular}
    \end{center}
      
  \Q What is the difference between the command \cpp{fin >> sport} and the
    command\\ \cpp{getline(fin, sport)}?
    \begin{answer}[0.5in]
      The first reads up to the first white space (space or newline), the second reads the
      entire line (up to a newline).
    \end{answer}
    
  \Q Rewrite the program so that it opens a user-entered filename and prints out the\key\\[-2.5mm]
    sports found inside. Test it with the file \cpp{"alternatives.txt"}.
    \begin{answer}[0.5in]
      Add a variable \cpp{string filename} and change line 11 to
      \cpp{fin.open(filename)}.
    \end{answer}