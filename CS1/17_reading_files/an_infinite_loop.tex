\model{An Infinite Loop}
  \begin{center}
    \begin{minipage}{4in}
      \begin{cpplst}
#include <iostream>
#include <fstream>

using namespace std;

int main() {
  ifstream fin;
  int number;

  fin.open("numbers.txt");
  do {
    fin >> number;
    cout << "Number: " << number << endl;
  } while (number != 6 && ! fin.eof());
  fin.close();
}
      \end{cpplst}
    \end{minipage}
    \begin{minipage}{1.5in}
      \centering {\bf numbers.txt File}
      \small
      \begin{verbatim}
2
four
6
      \end{verbatim}
    \end{minipage}
  \end{center}
  
  {\it\large Refer to Model 2 above as your team develops consensus answers
    to the questions below.}

  \quest{20 min}

  \Q This code can be found in {\tt activity17b.cpp}.  What happens when you run it?  Hint: To
    stop a program from executing, click on the terminal and press {\tt Ctrl+C}.
    \begin{answer}[0.4in]
      The program prints out 2 and then an infinite number of 0's.
    \end{answer}
    
  \Q What change to \cpp{"numbers.txt"} causes the program to
    print out 2, 4, and 6 and then exit?
    \begin{answer}[0.4in]
      Turn the ``four'' into a ``4'' on line 2 of the text file.
    \end{answer}
    
  \Q For a file input stream {\tt fin}, the function {\tt fin.eof()} returns true if
    we've reached the {\it end-of-file} and false otherwise.  Answer these questions
    to help you determine why the program runs in an infinite loop.
    \begin{enumerate}
      \itemsep 10pt
      \item What does the code on line 12 do? 
        \hfill\ans[3.25in]{reads a line from the file into {\tt number}}

      \item What type is the variable {\tt number}?
        \hfill\ans[3.25in]{it is an integer}

      \item What type of data is the {\tt four} in the file?
        \hfill\ans[3.25in]{it is a string}

      \item When will the loop end?
        \hfill\ans[3.25in]{when we've read a 6 or reached the eof}

      \item If C++ tries to put something from an input stream into a
        variable and fails, the data being read is left on the input
        stream for later use. Why do you think the program goes into an infinite loop?
        \begin{answer}[0.4in]
          When C++ tries to put a string into an integer, it can't do it.  So it just keeps
          trying infinitely, never making it to the 6 or reaching the eof.
        \end{answer}
    \end{enumerate}

  \vskip -30pt
    
  \Q Modify the code in {\tt activity17b.cpp} so that it reads a list of 
    numbers from the keyboard and prints them out until the number 6 is entered.  Record the
    changes you make.
    \begin{answer}[1in]
      \begin{itemize}
        \item Remove lines 7, 10, and 15
        \item Change line 12 to \cpp{cin >> number;} and line 14 to
          \cpp{while( number != 6);}.
      \end{itemize}
    \end{answer}

  \Q Now test your program by entering {\tt 2}, {\tt four}, and {\tt 6} from the keyboard.
    What happens?
    \begin{answer}[0.5in]
      We get the same infinite loop that we saw with the file input.
    \end{answer}

  \vskip -10pt
  
  \Q The code below should be similar to what you wrote in problem 12, with
    some\key\\[-2.5mm]  additions.
    \begin{center}
      \begin{tabular}{p{3.3in}p{2.9in}}
        \begin{minipage}{3.3in}
          \small
          \begin{cpplst}
do {
  cin >> number;
  if (cin.fail()) {
    cin.clear();
    cin.ignore(100,'\n');
  } else {
    cout << "Number: " << number << endl;
  }
} while (number != 6);
          \end{cpplst}
        \end{minipage}
        &
        \begin{minipage}{2.9in}
          \begin{enumerate}
            \item Update your code and run
              it again with inputs {\tt 2}, {\tt four}, and 
              {\tt 6}.  What happens?
              \begin{answer}[0.5in]
                It ignores the ``{\tt four}''
              \end{answer}  

            \item What do you think {\tt cin.fail()} does?
              \begin{answer}[0.5in]
                It checks to see if there was an error
              \end{answer}
          \end{enumerate}
        \end{minipage}
      \end{tabular}
    \end{center}

  \newpage
          
  \Q Write a C++ program to print the sum of the numbers read
    from the text file {\tt numbersTwo.txt} shown below. Any
    non-numeric data in the text file (such as the ``{\tt four}''
    in our model) should be ignored.     
    \begin{center}
      \begin{tabular}{p{1in}p{4.5in}}
        \begin{minipage}{1in}
          \centering\small{\tt numbersTwo.txt}
          \begin{verbatim}
5
seven
3
12
15
ten
1
6
          \end{verbatim}
        \end{minipage}
        &
        \begin{minipage}{4.5in}
          \begin{answer}[2.5in]
            \hspace{0.5in}
            \begin{minipage}{3in}
              \scriptsize
              \begin{cpplst}
ifstream fin;
int number, sum;
fin.open("numbers.txt");
do {
  fin >> number;
  if(fin.fail()) {
    fin.clear();
    fin.ignore(100,'\n');
  } else {
    sum += number;
  }
} while (! fin.eof());
fin.close();
cout << "The sum is: " << sum << endl;
              \end{cpplst}
            \end{minipage}
          \end{answer}
        \end{minipage}
      \end{tabular}
    \end{center}