\model{An Infinite Loop} \\
  \begin{center}
    \begin{minipage}{3.5in}
      \begin{minted}[
        frame=lines,
        framesep=2mm,
        bgcolor=gray!15,
        baselinestretch=1.2,
        linenos,
        firstnumber=6
      ]{cpp}
ifstream fin;
int number;

fin.open("numbers.txt");
do {
  fin >> number;
  cout << "Number: " << number << endl;
} while (number != 6 && ! fin.eof());
fin.close();
      \end{minted}
    \end{minipage}
  \end{center}  
  \TPMargin{5pt}
  \begin{textblock*}{1.15in}[0,0](4.5in,-1.75in)
    \textblockcolor{white}
    \begin{minipage}{1in}
      \mintinline{cpp}|"numbers.txt"|\vskip 5pt
      \hrule\vskip 5pt
      \tt
      2\\
      four\\
      6
    \end{minipage}
  \end{textblock*}  
  \par\vskip 10pt
  
  {\it\large Refer to Model 2 above as your team develops consensus answers
    to the questions below.}
    \par\vskip -20pt\null

  \Q This code can be found in {\tt activity17b.cpp}.  What happens when you run it?  Hint: To
    stop a program from executing, click on the terminal and press {\tt Ctrl+C}.
    % \ifprintanswers\vskip -20pt\null\fi
    \begin{answer}[0.4in]
      The program prints out 2 and then an infinite number of 0's.
    \end{answer}
    % \ifprintanswers\vskip -35pt\null\fi
    
  \Q What change to \mintinline{cpp}|"numbers.txt"| causes the program to
    print out 2, 4, and 6 and then exit?
    % \ifprintanswers\vskip -20pt\null\fi
    \begin{answer}[0.4in]
      Turn the ``four'' into a ``4'' on line 2 of the text file.
    \end{answer}
    % \ifprintanswers\vskip -35pt\null\fi
    
  \Q For a file input stream {\tt fin}, the function {\tt fin.eof()} returns true if
    we've reached the {\it end-of-file} and false otherwise.  Answer these questions
    to help you determine why the program runs in an infinite loop.
    \par\vskip 15pt
    
    \begin{enumerate}[(a)]
      \itemsep 15pt
      \item What does the code on line 11 do? 
        \hfill\ans[3.25in]{reads a line from the file into {\tt number}}
      \item What type is the variable {\tt number}?
        \hfill\ans[3.25in]{it is an integer}
      \item What type of data is the {\tt four} in the file?
        \hfill\ans[3.25in]{it is a string}\par\vskip 15pt
      \item When will the loop end?
        \hfill\ans[3.25in]{when we've read a 6 or reached the eof}
      \item If C++ tries to put something from an input stream into a
        variable and fails, the data being read is left on the input
        stream for later use. Why do you think the program goes into an infinite loop?
        % \ifprintanswers\vskip -15pt\null\fi
        \begin{answer}[0.4in]
          When C++ tries to put a string into an integer, it can't do it.  So it just keeps
          trying infinitely, never making it to the 6 or reaching the eof.
        \end{answer}
        % \ifprintanswers\vskip -35pt\null\fi
    \end{enumerate}
    
  \Q Modify the code in {\tt activity17b.cpp} so that it reads a list of 
    numbers from the keyboard and prints them out until the number 6 is entered.  Record the
    changes you make.
    % \ifprintanswers\vskip -20pt\null\fi
    \begin{answer}[1in]
      \begin{itemize}
        \itemsep 0pt
        \item Remove lines 6, 9, and 14
        \item Change line 11 to ``\mintinline{cpp}|cin >> number;|'' and line 13 to
          ``\mintinline{cpp}|} while( number != 6);|''.
      \end{itemize}
    \end{answer}

\newpage

  \Q Now test your program by entering {\tt 2}, {\tt four}, and {\tt 6} from the keyboard.
    What happens?
    % \ifprintanswers\vskip -20pt\null\fi
    \begin{answer}[0.5in]
      We get the same infinite loop that we saw with the file input.
    \end{answer}
    \par\vskip -40pt\null
  
    \Q The code below should be similar to what you wrote in problem 12, with
      some additions. \key\\[-2.5mm] 
      \vskip -30pt\null
      \begin{center}
        \begin{tabular}{p{2.9in}p{2.9in}}
          \begin{minipage}{2.9in}
            \small
            \begin{minted}[
              frame=lines,
              framesep=2mm,
              bgcolor=gray!15,
              baselinestretch=1.2,
              linenos,
              firstnumber=8
            ]{cpp}
do {
  cin >> number;
  if( cin.fail() ) {
    cin.clear();
    cin.ignore(100,'\n');
  } else {
    cout << "Number: " << number << endl;
  }
} while (number != 6);  
            \end{minted}
          \end{minipage}
          &
          \begin{minipage}{2.9in}
            \begin{enumerate}[(a)]
              \item Update your code and run
                it again with inputs {\tt 2}, {\tt four}, and 
                {\tt 6}.  What happens?
                % \ifprintanswers\vskip -20pt\null\fi
                \begin{answer}[0.5in]
                  It ignores the ``{\tt four}''
                \end{answer}
                % \ifprintanswers\vskip -25pt\null\fi             
              \item What do you think {\tt cin.fail()} does?
                % \ifprintanswers\vskip -20pt\null\fi
                \begin{answer}[0.5in]
                  It checks to see if there was an error
                \end{answer}
                % \ifprintanswers\vskip -25pt\null\fi
            \end{enumerate}
          \end{minipage}
        \end{tabular}
      \end{center}
      % \ifprintanswers\vskip -25pt\null\fi
            

     \Q Write a C++ program to print the sum of the numbers read
       from the text file {\tt numbersTwo.txt} shown below. Any
       non-numeric data in the text file (such as the ``{\tt four}''
       in our model) should be ignored. 
       \par\vskip -30pt\null
       
       \begin{center}
         \begin{tabular}{p{1in}p{4.5in}}
           \begin{minipage}{1in}
            \centering\small{\tt numbersTwo.txt}\vskip -15pt\null
            \begin{minted}[
              frame=lines,
              framesep=2mm,
              bgcolor=gray!15,
              baselinestretch=1.2
            ]{html}
5
seven
3
12
15
ten
1
6
             \end{minted}
           \end{minipage}
           &
           \begin{minipage}{4.5in}
             \begin{answer}[2.5in]
               \hspace{0.5in}
               \begin{minipage}{3in}
                 % \ifprintanswers\vskip -15pt\null\fi
                 \scriptsize
                 \begin{minted}[
                   frame=lines,
                   framesep=2mm,
                   bgcolor=gray!15,
                   baselinestretch=1.2,
                   linenos
                 ]{cpp}
ifstream fin;
int number, sum;
fin.open("numbers.txt");
do {
  fin >> number;
  if(fin.fail()) {
    fin.clear();
    fin.ignore(100,'\n');
  } else {
    sum += number;
  }
} while (! fin.eof());
fin.close();
cout << "The sum is: " << sum << endl;
                 \end{minted}
                 % \ifprintanswers\vskip -20pt\null\fi
               \end{minipage}
             \end{answer}
           \end{minipage}
         \end{tabular}
       \end{center}
       \par\vskip 20pt