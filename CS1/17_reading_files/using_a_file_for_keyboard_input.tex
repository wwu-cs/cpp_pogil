\model{Using a File for Keyboard Input}
  \begin{center}
    \begin{tabular}{p{2.8in}p{0.1in}p{2.7in}}
      \begin{minipage}{2.8in}
        \small
        \begin{cpplst}   
#include <iostream>
#include <string>

using namespace std;

int main() {
  int number;
  string name;
  cout << "Enter Number: ";
  cin >> number;
  cout << "Enter Name: ";
  cin >> name;
  for (int i = 0; i < number; i++) {
    cout << name << endl;
  }
}
        \end{cpplst}
      \end{minipage}
      & &
      \begin{minipage}{2.7in}
        \centering\small \tt input.txt
        \begin{verbatim}
5
Duncan
        \end{verbatim}
        \tt terminal commands
        \begin{verbatim}
g++ activity17c.cpp -o test.o
./test.o < input.txt
        \end{verbatim}        
      \end{minipage}
    \end{tabular} 
  \end{center}  
      
  {\it\large Refer to Model 3 above as your team develops consensus answers
    to the questions below.}

  \quest{15 min}

  \Q The file {\tt activity17c.cpp} contains the code shown on the left-hand side of the
    model.  Run this program as you normally would and describe what it does.
    \begin{answer}[0.75in]
      The program prompts for a number and name and then prints out the name the given number
      of times.
    \end{answer}
    
  \Q Now open a terminal in this same folder (right click on the folder name and pick {\it
    Open in Terminal}) and type in the commands shown in the model, pressing {\tt Enter} after each line.
    What happens?
    \begin{answer}[0.75in]
      It prints out the name ``Duncan'' five times.
    \end{answer}

  \Q Modify the text file {\tt input.txt} so that when you run the second terminal command
    in the model it prints out the name of one of your group members 15 times.  What did your
    text file look like?
    \begin{answer}[0.5in]
      Answers will vary, but the first line should be {\tt 15}.
    \end{answer}
    
  \Q When you run a command {\tt program < textfile} in the terminal, the contents of\key\\[-2.5mm] the
    text file are {\it redirected} to the {\it standard input} when the program is run.
    That is, the computer behaves as if you had typed the contents of the text file in at the
    keyboard while the program was running.
    \begin{enumerate}
      \item How does the program output differ when the input is
        redirected from a file instead of typed in?
        \begin{answer}[0.75in]
          The input from the file is not shown in the output,
          including the places where a user would press {\tt Enter}
          and create a new line.
        \end{answer}

      \item Have you seen this type of output before?
        \begin{answer}[0.75in]
          Yes, this is the type of output that Submitty gives us
          because it tests homework by redirecting input from a file.
        \end{answer}
    \end{enumerate}

  \vskip -30pt
    
  \Q Pick a previous homework assignment that requires user
    input.  Write a text file with example input, compile the
    assignment, and run it with input redirected from your file.
    Record the assignment you chose and the input file you created
    below.
    \begin{answer}[1in]
      Answers will vary.
    \end{answer}