% Source: https://www.dropbox.com/sh/2fx6pg4ydpu9t7x/AAAdJfzvLjeym1gJwKrIWwhBa?preview=Python+Activity+14+Reading+from+Files+-+POGIL.docx
% File: "reading from files.pdf"
% Access: 04-15-2022

% comment out for student version
%\ifdefined\Student\relax\else\def\Teacher{}\fi

\documentclass[12pt]{article}

\title{Activity \#17: Reading from Files}
\author{Lisa Olivieri}
\newcommand{\activityeditor}{Preston Carman}
\newcommand{\activitysource}{\url{https://www.dropbox.com/sh/2fx6pg4ydpu9t7x/AAAdJfzvLjeym1gJwKrIWwhBa?preview=Python+Activity+14+Reading+from+Files+-+POGIL.docx}}
\date{Spring 2022}

\input{../../cspogil.sty}

\begin{document}

\begin{center}
  \Large Activity \#17: Reading from Files \\[5pt]
  \large Recorder's Report\\[20pt]
  \normalsize
  \begin{tabular}{lrp{0.1in}lr}
      Manager:  & \ans{} &  & Reader: & \ans{}            \\[15pt]
      Recorder: & \ans{} &  & Driver: & \ans{}            \\[15pt]
      Date:     & \ans{} &  & Score:  & Satisfactory \hspace{10pt} /
      \hspace{10pt} Not Satisfactory
  \end{tabular}
\end{center}
\par\vskip 15pt

Record your team's answers to the key questions (marked with
\raisebox{-.3\height}{\includegraphics[width=0.5in]{../figures/key.png}})
below.
\begin{enumerate}[label=(\alph*)]
  \itemsep 1.25in
  \item Model 1, Question \#8
  \item Model 2, Question \#14
  \item Model 3, Question \#19
\end{enumerate}

\newpage
\maketitle

In this course, you will work in teams of 3--4 students to learn new concepts.
This activity will introduce you to the process of analyzing an algorithm complexity.

%\rolenames

\guide{
  \item Explain how to open a text file for reading
  \item Explain how to handle errors with input streams
  \item Explain how to redirect a file from the command line to use as input

}{
  \item Write code that opens, reads from, and closes a file
  \item Write code that correctly handles input stream errors \\[-5pt]
}{
No additional notes.
}

  \model{An Infinite Loop}
  \begin{center}
    \begin{minipage}{4in}
      \begin{cpplst}
#include <iostream>
#include <fstream>

using namespace std;

int main() {
  ifstream fin;
  int number;

  fin.open("numbers.txt");
  do {
    fin >> number;
    cout << "Number: " << number << endl;
  } while (number != 6 && ! fin.eof());
  fin.close();
}
      \end{cpplst}
    \end{minipage}
    \begin{minipage}{1.5in}
      \centering {\bf numbers.txt File}
      \small
      \begin{verbatim}
2
four
6
      \end{verbatim}
    \end{minipage}
  \end{center}
  
  {\it\large Refer to Model 2 above as your team develops consensus answers
    to the questions below.}

  \quest{20 min}

  \Q This code can be found in {\tt activity17b.cpp}.  What happens when you run it?  Hint: To
    stop a program from executing, click on the terminal and press {\tt Ctrl+C}.
    \begin{answer}[0.4in]
      The program prints out 2 and then an infinite number of 0's.
    \end{answer}
    
  \Q What change to \cpp{"numbers.txt"} causes the program to
    print out 2, 4, and 6 and then exit?
    \begin{answer}[0.4in]
      Turn the ``four'' into a ``4'' on line 2 of the text file.
    \end{answer}
    
  \Q For a file input stream {\tt fin}, the function {\tt fin.eof()} returns true if
    we've reached the {\it end-of-file} and false otherwise.  Answer these questions
    to help you determine why the program runs in an infinite loop.
    \begin{enumerate}
      \itemsep 10pt
      \item What does the code on line 12 do? 
        \hfill\ans[3.25in]{reads a line from the file into {\tt number}}

      \item What type is the variable {\tt number}?
        \hfill\ans[3.25in]{it is an integer}

      \item What type of data is the {\tt four} in the file?
        \hfill\ans[3.25in]{it is a string}

      \item When will the loop end?
        \hfill\ans[3.25in]{when we've read a 6 or reached the eof}

      \item If C++ tries to put something from an input stream into a
        variable and fails, the data being read is left on the input
        stream for later use. Why do you think the program goes into an infinite loop?
        \begin{answer}[0.4in]
          When C++ tries to put a string into an integer, it can't do it.  So it just keeps
          trying infinitely, never making it to the 6 or reaching the eof.
        \end{answer}
    \end{enumerate}

  \vskip -30pt
    
  \Q Modify the code in {\tt activity17b.cpp} so that it reads a list of 
    numbers from the keyboard and prints them out until the number 6 is entered.  Record the
    changes you make.
    \begin{answer}[1in]
      \begin{itemize}
        \item Remove lines 7, 10, and 15
        \item Change line 12 to \cpp{cin >> number;} and line 14 to
          \cpp{while( number != 6);}.
      \end{itemize}
    \end{answer}

  \Q Now test your program by entering {\tt 2}, {\tt four}, and {\tt 6} from the keyboard.
    What happens?
    \begin{answer}[0.5in]
      We get the same infinite loop that we saw with the file input.
    \end{answer}

  \vskip -10pt
  
  \Q The code below should be similar to what you wrote in problem 12, with
    some\key\\[-2.5mm]  additions.
    \begin{center}
      \begin{tabular}{p{3.3in}p{2.9in}}
        \begin{minipage}{3.3in}
          \small
          \begin{cpplst}
do {
  cin >> number;
  if (cin.fail()) {
    cin.clear();
    cin.ignore(100,'\n');
  } else {
    cout << "Number: " << number << endl;
  }
} while (number != 6);
          \end{cpplst}
        \end{minipage}
        &
        \begin{minipage}{2.9in}
          \begin{enumerate}
            \item Update your code and run
              it again with inputs {\tt 2}, {\tt four}, and 
              {\tt 6}.  What happens?
              \begin{answer}[0.5in]
                It ignores the ``{\tt four}''
              \end{answer}  

            \item What do you think {\tt cin.fail()} does?
              \begin{answer}[0.5in]
                It checks to see if there was an error
              \end{answer}
          \end{enumerate}
        \end{minipage}
      \end{tabular}
    \end{center}

  \newpage
          
  \Q Write a C++ program to print the sum of the numbers read
    from the text file {\tt numbersTwo.txt} shown below. Any
    non-numeric data in the text file (such as the ``{\tt four}''
    in our model) should be ignored.     
    \begin{center}
      \begin{tabular}{p{1in}p{4.5in}}
        \begin{minipage}{1in}
          \centering\small{\tt numbersTwo.txt}
          \begin{verbatim}
5
seven
3
12
15
ten
1
6
          \end{verbatim}
        \end{minipage}
        &
        \begin{minipage}{4.5in}
          \begin{answer}[2.5in]
            \hspace{0.5in}
            \begin{minipage}{3in}
              \scriptsize
              \begin{cpplst}
ifstream fin;
int number, sum;
fin.open("numbers.txt");
do {
  fin >> number;
  if(fin.fail()) {
    fin.clear();
    fin.ignore(100,'\n');
  } else {
    sum += number;
  }
} while (! fin.eof());
fin.close();
cout << "The sum is: " << sum << endl;
              \end{cpplst}
            \end{minipage}
          \end{answer}
        \end{minipage}
      \end{tabular}
    \end{center}
  \newpage
  \model{A C++ Program and Input File} \\
  \begin{center}
    \begin{tabular}{p{3in}p{0.25in}p{1.5in}}
      \begin{minipage}{3in}
        \centering {\bf C++ Code}\vskip -15pt\null
        \small
        \begin{minted}[
          frame=lines,
          framesep=2mm,
          bgcolor=gray!15,
          baselinestretch=1,
          linenos
        ]{cpp}
#include <iostream>
#include <fstream>
#include <string>
using namespace std;

int main() {
  string sport;
  ifstream fin;
  int count = 1;

  fin.open("sports.txt");
  if (!fin.is_open()) {     
    cout << "Could not open file" << endl;
    return 1;
  }

  while (fin >> sport) {
    sport.at(0) = toupper(sport.at(0));
    cout << "Sport " << count 
         << ": " << sport << endl;
    count++;
  }
  fin.close();  
}
        \end{minted}
      \end{minipage}
      & &
      \begin{minipage}{1.5in}
        \centering {\bf sports.txt File}\vskip -15pt\null
        \small
        \begin{minted}[
          frame=lines,
          framesep=2mm,
          bgcolor=gray!15,
          linenos
        ]{bash}
basketball
baseball
football
volleyball
tennis
golf
lacrosse
soccer
badminton
bowling
fortnight
skiing
diving
ice hockey
biking
rugby
swimming
sailing
rowing
skateboarding
        \end{minted}
      \end{minipage}
    \end{tabular}
  \end{center}  
  
  {\it\large Refer to Model 1 above as your team develops consensus answers
    to the questions below.}
    \par\vskip 10pt
    
  \begin{enumerate}
    \itemsep 20pt
 
    \Q This code can be found in {\tt activity17a.cpp}.  Run it and then
      describe what the program does.
      % \ifprintanswers\vskip -20pt\null\fi
      \begin{answer}[0.75in]
        The program reads in one line from the {\tt sports.txt} file at a time, changes the first
        letter of the sport to be upper case, and then prints out the sport name with a label of
        ``Sport n:'' where $n$ is the line number on which the sport was found.
      \end{answer}
      % \ifprintanswers\vskip -30pt\null\fi

    \Q On line 11 of the C++ code, what does the string \cpp{"sports.txt"} represent?
      % \ifprintanswers\vskip -20pt\null\fi
      \begin{answer}[0.5in]
        This is the name of the file to be read.
      \end{answer}

\newpage

    \Q Without discussing its relative merits as a sport, remove line 11 from the text file 
      \cpp{"sports.txt"|, save your change, and rerun the program. What happened?
      \ifprintanswers\vskip -20pt\null\fi
      \begin{answer}[0.5in]
        The ``sport'' of {\it fortnight} is no longer printed out.
      \end{answer}
      \ifprintanswers\vskip -35pt\null\fi

    \Q C++ can get input from a text file in much the same way it does from the
      keyboard.  To do this it uses {\it file input streams}.  Answer the questions below
      about the use of file input streams in this model.
      \par\vskip 15pt
      
      \begin{enumerate}[(a)]
        \itemsep 12pt
        \item What is the name of the library that includes these streams?
          \hfill \ans[1.5in]{\tt fstream}
        \item What is the type of variable we use for a file input stream?
          \hfill \ans[1.5in]{\tt ifstream}
        \item The file input stream {\tt fin} is used on line 11.  What does this line do?
          % \ifprintanswers\vskip -20pt\null\fi
          \begin{answer}[0.4in]
            It opens the file that we will be reading from (gives the name, etc).
          \end{answer}
        \item The file input stream {\tt fin} is used again on line 12.  What does this line do?
          % \ifprintanswers\vskip -20pt\null\fi
          \begin{answer}[0.4in]
            It opens the file that we will be reading from (gives the name, etc).
          \end{answer}
        \item On what line does the program actually read input from the file?
          \hfill \ans[1.5in]{line 17}
        \item On what line is the file closed (indicating we are done reading)?
          \hfill \ans[1.5in]{line 13}
      \end{enumerate}          
      
    \Q How is the sport named on line 14 of the text file printed by this program?
      % \ifprintanswers\vskip -20pt\null\fi
      \begin{answer}[0.5in]
        On two lines instead of one (i.e. ``Ice'' and ``Hockey'' instead of ``Ice
        hockey'').
      \end{answer}
      % \ifprintanswers\vskip -35pt\null\fi
    
    \Q Below is a slight modification of this program (only lines 17-22 are shown).  Make this
      modification in the file {\tt activity17a.cpp} and record how the output changed.
      \par\vskip -30pt\null
      
      \begin{center}
        \begin{tabular}{p{2.7in}p{3.1in}}
          \begin{minipage}{2.7in}
            \small
            \begin{minted}[
              frame=lines,
              framesep=2mm,
              bgcolor=gray!15,
              baselinestretch=1,
              linenos,
              firstnumber=17
            ]{cpp}
  while (getline(fin,sport)) {
    sport.at(0) = toupper(sport.at(0));
    cout << "Sport " << count 
         << ": " << sport << endl;
    count++;
  }
            \end{minted}
          \end{minipage}
          &
          \begin{minipage}{3.1in}
            \begin{answer}[1in]
              \par
              The sport is printed as ``Ice hockey'' (on one line).
            \end{answer}
          \end{minipage}
        \end{tabular}
      \end{center}
      % \ifprintanswers\vskip -35pt\null\fi
      
    \Q What is the difference between the command \cpp{fin >> sport} and the
      command \cpp{getline(fin,sport)}?
      % \ifprintanswers\vskip -20pt\null\fi
      \begin{answer}[0.5in]
        The first reads up to the first white space (space or newline), the second reads the
        entire line (up to a newline).
      \end{answer}
      % \ifprintanswers\vskip -60pt\else\vskip -40pt\fi\null
      
    \Q Rewrite the program so that it opens a user-entered filename and prints out the
      sports\key\\[-2.5mm] found inside. Test it with the file \cpp{"alternatives.txt"}.
      % \ifprintanswers\vskip -20pt\null\fi
      \begin{answer}[0.5in]
        Add a variable \cpp{string filename} and change line 11 to
        \cpp{fin.open(filename)}.
      \end{answer}
  \newpage
  {\bf\large Model 3: Using a File for Keyboard Input} \\[-20pt]
  \begin{center}
    \begin{tabular}{p{2.8in}p{0.1in}p{2.7in}}
      \begin{minipage}{2.8in}
        \small
        \begin{minted}[
          frame=lines,
          framesep=2mm,
          bgcolor=gray!15,
          baselinestretch=1.2,
          linenos,
          firstnumber=5
        ]{cpp}      
int number;
string name;
cout << "Enter Number: ";
cin >> number;
cout << "Enter Name: ";
cin >> name;
for (int i = 0; i < number; i++) {
  cout << name << endl;
}
        \end{minted}
      \end{minipage}
      & &
      \begin{minipage}{2.7in}
        \centering\small \tt input.txt\vskip -5pt
        \begin{minted}[
          frame=lines,
          framesep=2mm,
          bgcolor=gray!15,
          baselinestretch=1.2,
        ]{bash}
5
Duncan
        \end{minted}
        \tt terminal commands
        \begin{minted}[
          frame=lines,
          framesep=2mm,
          bgcolor=gray!15,
          baselinestretch=1.2,
        ]{bash}
g++ activity17c.cpp -o test.o
./test.o < input.txt
        \end{minted}        
      \end{minipage}
    \end{tabular} 
  \end{center}  
  \par\vskip 10pt
                                                
      
  {\it\large Refer to Model 3 above as your team develops consensus answers
    to the questions below.}
    \ifprintanswers\vskip -20pt\null\fi          

\newpage

    \item The file {\tt activity17c.cpp} contains the code shown on the left-hand side of the
      model.  Run this program as you normally would and describe what it does.
      \ifprintanswers\vskip -20pt\null\fi
      \begin{solution}[0.75in]
        The program prompts for a number and name and then prints out the name the given number
        of times.
      \end{solution}
      \ifprintanswers\vskip -35pt\null\fi
      
    \item Now open a terminal in this same folder (right click on the folder name and pick {\it
      Open in Terminal}) and type in the commands shown in the model, pressing {\tt Enter} after each line.
      What happens?
      \ifprintanswers\vskip -20pt\null\fi
      \begin{solution}[0.75in]
        It prints out the name ``Duncan'' five times.
      \end{solution}
      \ifprintanswers\vskip -35pt\null\fi

    \item Modify the text file {\tt input.txt} so that when you run the second terminal command
      in the model it prints out the name of one of your group members 15 times.  What did your
      text file look like?
      \ifprintanswers\vskip -20pt\null\fi
      \begin{solution}[0.75in]
        Answers will vary, but the first line should be {\tt 15}.
      \end{solution}
      \ifprintanswers\vskip -45pt\null\fi
      
    \item When you run a command {\tt program < textfile} in the terminal, the contents of the
      text\key\\[-2.5mm] file are {\it redirected} to the {\it standard input} when the program is run.
      That is, the computer behaves as if you had typed the contents of the text file in at the
      keyboard while the program was running.
      
      \begin{enumerate}[(a)]
        \item How does the program output differ when the input is
          redirected from a file instead of typed in?
          \ifprintanswers\vskip -20pt\null\fi
          \begin{solution}[0.75in]
            The input from the file is not shown in the output,
            including the places where a user would press {\tt Enter}
            and create a new line.
          \end{solution}
        \item Have you seen this type of output before?
          \ifprintanswers\vskip -20pt\null\fi
          \begin{solution}[0.75in]
            Yes, this is the type of output that Submitty gives us
            because it tests homework by redirecting input from a file.
          \end{solution}
      \end{enumerate}
      
    \item Pick a previous homework assignment that requires user
      input.  Write a text file with example input, compile the
      assignment, and run it with input redirected from your file.
      Record the assignment you chose and the input file you created
      below.
      \begin{solution}[1in]
        Answers will vary.
      \end{solution}
  
\end{document}
