 {\bf\large Model 2: Output from Several Nested Loops} \\[-10pt]

% define first output as verbatim box
\newbox\verbboxOne
\setbox\verbboxOne=\vbox{\hsize=1.4in
\begin{Verbatim}
* * * * *
* * * * *
* * * * *
* * * * *
* * * * *
* * * * *
* * * * *
\end{Verbatim}
}

% define second output as verbatim box
\newbox\verbboxTwo
\setbox\verbboxTwo=\vbox{\hsize=1.4in
\begin{Verbatim}
1 1 1 1 1
2 2 2 2 2
3 3 3 3 3
4 4 4 4 4
5 5 5 5 5
6 6 6 6 6
7 7 7 7 7
\end{Verbatim}
}

% define third output as verbatim box
\newbox\verbboxThree
\setbox\verbboxThree=\vbox{\hsize=1.4in
\begin{Verbatim}
1 2 3 4 5
1 2 3 4 5
1 2 3 4 5
1 2 3 4 5
1 2 3 4 5
1 2 3 4 5
1 2 3 4 5
\end{Verbatim}
}


  \begin{center}
    \small
    \begin{tabular}{p{1.5in}p{0.25in}p{1.5in}p{0.25in}p{1.5in}}
      \fbox{\begin{minipage}{1.4in}
        \centering\box\verbboxOne
      \end{minipage}}
      & &
      \fbox{\begin{minipage}{1.4in}
        \centering\box\verbboxTwo
      \end{minipage}}      
      & &
      \fbox{\begin{minipage}{1.4in}
        \centering\box\verbboxThree
      \end{minipage}}
      \\
      \centering (a) & & \centering (b) & & \centering (c) \\
    \end{tabular}
  \end{center}

  {\it\large Refer to Model 2 above as your team develops consensus answers
    to the questions below.}

    \item There are several ways to write a C++ program that outputs
      the contents of box (a) above.  Briefly describe how it could be
      done using the following methods.
      \par\vskip 15pt
      
      \begin{enumerate}[(a)]
        \item Using a set of {\tt cout} statements without any loops.
          \begin{solution}[0.75in]
            \par
            Create seven identical {\tt cout} statements,
            each of which prints {\tt * * * * *} and an {\tt endl}.
          \end{solution}
        \item Using a single \mintinline{cpp}|for| loop (without any nesting).
          \begin{solution}[0.75in]
            \par
            The \mintinline{cpp}|for| loop would execute seven times
            and each time {\tt cout} the five {\tt *}'s and an {\tt endl}.
          \end{solution}
      \end{enumerate}
      
    \item In this exercise we will create the output in box (a) using nested loops.    
      \par\vskip 10pt
      
      \begin{enumerate}[(a)]
        \item Surround the {\tt cout} statement below by a \mintinline{cpp}|for|
          so that five {\tt *} characters are printed on a single line.
          \ifprintanswers
            \begin{solution}
              \scriptsize\vskip -35pt\null
              \begin{center}
                \begin{minipage}{2.5in}
                  \begin{minted}[
                    frame=lines,
                    framesep=2mm,
                    bgcolor=gray!15,
                    baselinestretch=1.2,
                  ]{cpp}
for (int i=0; i<5; i++) {
  cout << "* ";
}
                  \end{minted}
                \end{minipage}
              \end{center}\vskip -20pt\null
            \end{solution}
          \else
            \par\vskip 10pt\null
            \begin{center}
              \mintinline{cpp}{cout << "* ";}
            \end{center}
            \par\vskip 10pt\null
          \fi
        \item Now build on the block of code above by wrapping the
          loop you created in another \mintinline{cpp}|for| loop that
          prints seven of these lines, with an {\tt endl} between each one.
          \begin{solution}[1.25in]
            \scriptsize\vskip -35pt\null
            \begin{center}
              \begin{minipage}{2.5in}
                \begin{minted}[
                  frame=lines,
                  framesep=2mm,
                  bgcolor=gray!15,
                  baselinestretch=1.2,
                ]{cpp}
for (int j=0; j<7; j++) {                
  for (int i=0; i<5; i++) {
    cout << "* ";
  }
  cout << endl;
}
                \end{minted}
              \end{minipage}
            \end{center}\vskip -20pt\null
          \end{solution}
      \end{enumerate}
      
    \item Suppose that variables \mintinline{cpp}|int rows| and
      \mintinline{cpp}|int cols| contain the number of rows and 
      columns\key\\[-2.5mm] of {\tt *}'s you wish to print.  How
      would you modify your code above to print out that number of
      rows and columns?  Test your solution in {\tt activity09b.cpp}.
      
      \begin{solution}[1.25in]
        \scriptsize\vskip -35pt\null
        \begin{center}
          \begin{minipage}{2.5in}
            \begin{minted}[
              frame=lines,
              framesep=2mm,
              bgcolor=gray!15,
              baselinestretch=1.2,
            ]{cpp}
for (int j=0; j<rows; j++) {
  for (int i=0; i<cols; i++) {
    cout << "* ";
  }
  cout << endl;
}
            \end{minted}
          \end{minipage}
        \end{center}\vskip -20pt\null
      \end{solution}

    \item How would you modify your code in the previous question to
      produce the output seen in box (b), assuming that
      \mintinline{cpp}|int rows = 7| and \mintinline{cpp}|int cols = 4|?

      \begin{solution}[1.25in]
        \scriptsize\vskip -35pt\null
        \begin{center}
          \begin{minipage}{2.5in}
            \begin{minted}[
              frame=lines,
              framesep=2mm,
              bgcolor=gray!15,
              baselinestretch=1.2,
            ]{cpp}
for (int j=0; j<rows; j++) {
  for (int i=0; i<cols; i++) {
    cout << j;
  }
  cout << endl;
}
            \end{minted}
          \end{minipage}
        \end{center}\vskip -20pt\null
      \end{solution}

    \item How would you modify your code so that the output
      matches the output in box (c) instead of (b)?
      \begin{solution}[1.25in]
        \scriptsize\vskip -35pt\null
        \begin{center}
          \begin{minipage}{2.5in}
            \begin{minted}[
              frame=lines,
              framesep=2mm,
              bgcolor=gray!15,
              baselinestretch=1.2,
            ]{cpp}
for (int j=0; j<rows; j++) {
  for (int i=0; i<cols; i++) {
    cout << i;
  }
  cout << endl;
}
            \end{minted}
          \end{minipage}
        \end{center}\vskip -20pt\null
      \end{solution}