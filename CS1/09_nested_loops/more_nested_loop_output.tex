\model{More Nested Loop Output} \\
  
% define fourth output as verbatim box
\newbox\verbboxFour
\setbox\verbboxFour=\vbox{\hsize=1.4in
\begin{Verbatim}
1
1 2
1 2 3
1 2 3 4
1 2 3 4 5
\end{Verbatim}
}

% define fifth output as verbatim box
\newbox\verbboxFive
\setbox\verbboxFive=\vbox{\hsize=1.4in
\begin{Verbatim}
1 2 3 4 5
1 2 3 4
1 2 3
1 2 
1
\end{Verbatim}
}

% define third output as verbatim box
\newbox\verbboxSix
\setbox\verbboxSix=\vbox{\hsize=1.4in
\begin{Verbatim}
1
1 2
1 2 3
1 2
1
\end{Verbatim}
}


  \begin{center}
    \small
    \begin{tabular}{p{1.5in}p{0.25in}p{1.5in}p{0.25in}p{1.5in}}
      \fbox{\begin{minipage}{1.4in}
        \centering\box\verbboxFour
      \end{minipage}}
      & &
      \fbox{\begin{minipage}{1.4in}
        \centering\box\verbboxFive
      \end{minipage}}      
      & &
      \fbox{\begin{minipage}{1.4in}
        \centering\box\verbboxSix
      \end{minipage}}
      \\
      \centering (a) & & \centering (b) & & \centering (c) \\
    \end{tabular}
  \end{center}
  
  {\it\large Refer to Model 3 above as your team develops consensus answers
    to the questions below.}

    \Q How are the output samples in this model different from
      those in model 2?  Note: Write a single general statement that
      summarizes the difference for all output boxes.
      
      \begin{answer}[1in]
        \par
        The number of columns in a row (i.e. its length) depends on
        which row we are in.
      \end{answer}
      
    \Q The code below will produce one of these three outputs.
      Determine which one and justify your answer.
      
      \begin{center}
        \begin{tabular}{p{2.5in}p{2.5in}}
          \begin{minipage}{2.5in}
            \begin{cprlst}[
              frame=lines,
              framesep=2mm,
              bgcolor=gray!15,
              baselinestretch=1.2,
            ]{cpp}
int numRows = 5;
for (int j=numRows; j>0; j--) {
  for (int i=1; i<=j; i++) {
    cout << i << " ";
  }
  cout << endl;
}
            \end{cprlst}
          \end{minipage}
          &
          \begin{minipage}{2.5in}
            \begin{answer}[1in]
              \par
              It will produce the output in box (b).  We can tell
              because it is counting down from 5 instead of counting
              up to 5 or counting up and then down.
            \end{answer}
          \end{minipage}
        \end{tabular}
      \end{center}
      \par\vskip -20pt\null

    \Q The code from the previous problem is in
      {\tt activity09c.cpp}.  Rewrite it so that it produces \key\\[-2.5mm] the output
      in the other of box (a) or (b).
      
      \begin{answer}[1.5in]
        \scriptsize\vskip -35pt\null
        \begin{center}
          \begin{minipage}{2.5in}
            \begin{cprlst}[
              frame=lines,
              framesep=2mm,
              bgcolor=gray!15,
              baselinestretch=1.2,
            ]{cpp}
int numRows = 5;
for (int j=1; j<=numRows; j++) {
  for (int i=1; i<=j; i++) {
    cout << i << " ";
  }
  cout << endl;
}
            \end{cprlst}
          \end{minipage}
        \end{center}\vskip -20pt\null
      \end{answer}
      
    \Q Add some initialization code that prompts the user to enter
      the number of rows in the triangle that your program creates.
      Test your solution to verify that it works.

\newpage

    \Q Use nested \cpp{for} loops to produce the output
      in box (c) above.
      
      \begin{answer}[2.5in]
        \scriptsize\vskip -35pt\null
        \begin{center}
          \begin{minipage}{3.5in}
            \begin{cprlst}[
              frame=lines,
              framesep=2mm,
              bgcolor=gray!15,
              baselinestretch=1.2,
            ]{cpp}
  int numRows = 5, length;
  for (int i = 1; i <= numRows; i++) {
    if ((numRows % 2 == 0 && i <= numRows / 2) ||
        (numRows % 2 == 1 && i <= numRows / 2 + 1) ) {
      length = i;
    }
    else {
      length = numRows - i + 1;
    }
    for (int j = 1; j <= length; j++) {
      cout << j << " ";
    }
    cout << endl;
  }
            \end{cprlst}
          \end{minipage}
        \end{center}\vskip -20pt\null
      \end{answer}

    \Q Write a program that prompts the user for information on
      three students.  For each student, collect the student name and
      three quiz grades.  Then display the name and quiz average
      (formatted to two decimal places).  Sample output is shown below.
    
      \begin{center}
        \begin{tabular}{p{2.6in}p{3in}}
          \begin{minipage}{2.6in}
            \small
            \begin{cprlst}[
              frame=lines,
              framesep=2mm,
              bgcolor=gray!15,
              baselinestretch=1.2,
            ]{html}
Enter name of student 1: Mary
Enter score 1: 78
Enter score 2: 90
Enter score 3: 91
Name: Mary
Average: 86.33

Enter name of student 2: Kevin
Enter score 1: 90
Enter score 2: 77
Enter score 3: 85
Name: Kevin
Average: 84.00

Enter name of student 3: Jose
Enter score 1: 79
Enter score 2: 83
Enter score 3: 92
Name: Jose
Average: 84.67
            \end{cprlst}
          \end{minipage}
          &
          \begin{minipage}{3in}
            \begin{answer}
              \scriptsize
              \begin{cprlst}[
                frame=lines,
                framesep=2mm,
                bgcolor=gray!15,
                baselinestretch=1.2,
              ]{cpp}
string name;
double score,sum;
cout << showpoint << fixed << setprecision(2);
for (int i=1; i<=3; i++) {
  cout << "Enter name of student " << i << ": ";
  cin >> name;
  sum = 0;
  for (int j=1; j<=3; j++) {
    cout << "Enter score " << j << ": ";
    cin >> score;
    sum += score;
  }
  cout << "Name: " << name << endl;
  cout << "Average: " << (sum / 3) << endl << endl;
  }
}  
              \end{cprlst}
            \end{answer}
          \end{minipage}
        \end{tabular}
      \end{center}  
    