\model{A Configuration Function} \\
  \begin{center}
    \begin{minipage}{4in}
      \scriptsize
      \begin{cprls}[
        frame=lines,
        framesep=2mm,
        bgcolor=gray!15,
        baselinestretch=1.2,
        linenos,
        firstnumber=38
      ]{cpp}
void getSetup(int &rows, int &cols, char &charOne, char &charTwo) {
  ifstream configIn;
  configIn.open("gameBoard.cfg");
  if (configIn.is_open()) {
    configIn >> rows;
    configIn >> cols;
    configIn >> charOne;
    configIn >> charTwo;
  } else {
    input("Enter Number of Rows: ",rows);
    input("Enter Number of Columns: ",cols);
    input("Enter Player One Symbol: ",charOne);
    input("Enter Player Two Symbol: ",charTwo);
  }
}
      \end{cprls}
    \end{minipage}
  \end{center}  
  \TPMargin{5pt}
  \begin{textblock*}{1.25in}[0,0](4.75in,-1.8in)
    \textblockcolor{white}
    \begin{minipage}{1.1in}
      \cpp{"gameBoard.cfg"}\vskip 5pt
      \hrule\vskip 5pt
      \tt
      6\\
      5\\
      *\\
      \#
    \end{minipage}
  \end{textblock*}  
  
  {\it\large Refer to Model 2 above as your team develops consensus answers
    to the questions below.}
    \par\vskip -20pt\null
    
\newpage

  \Q Without running the program, determine what the function {\tt getSetup} does.
    % \ifprintanswers\vskip -20pt\null\fi
    \begin{answer}[0.4in]
      It reads {\tt rows}, {\tt columns}, {\tt charOne}, and
      {\tt charTwo} from a configuration file, if it exists.
      If the file does not exists, it calls the {\tt input} function
      to get them.
    \end{answer}
    % \ifprintanswers\vskip -35pt\null\fi
    
  \Q Suppose that this function were called with the contents of
    {\tt gameBoard.cfg} as shown.  To what value would each of the
    following variables be set?
    \par\vskip 15pt
    
    \begin{enumerate}[(a)]
      \itemsep 15pt
      \begin{multicols}{2}
        \item {\tt rows} = \hfill\ans[1.5in]{6}
        \item {\tt cols} = \hfill\ans[1.5in]{5}
        \item {\tt charOne} = \hfill\ans[1.5in]{*}
        \item {\tt charTwo} = \hfill\ans[1.5in]{\#}
      \end{multicols}      
    \end{enumerate}
    
  \Q What is missing from the function with regard to the {\tt
    configIn} stream?
    % \ifprintanswers\vskip -20pt\null\fi
    \begin{answer}[0.4in]
      The stream is never closed
    \end{answer}
    % \ifprintanswers\vskip -35pt\null\fi
  
  \Q This function is part of the program in {\tt activity18b.cpp}.
    Run it and record what it does.
    % \ifprintanswers\vskip -20pt\null\fi
    \begin{answer}[0.4in]
      It prints out a checkerboard that is 6 x 5 tiles the {\tt *}
      character and {\tt \#} characters.
    \end{answer}
    % \ifprintanswers\vskip -35pt\null\fi
    
  \Q Now delete the file {\tt gameBoard.cfg} and run the program
    again.  What does it do differently?
    % \ifprintanswers\vskip -20pt\null\fi
    \begin{answer}[0.4in]
      It prompts the user for the information that was previously
      stored in the file.
    \end{answer}
    % \ifprintanswers\vskip -35pt\null\fi
    
  \Q A {\it configuration file} is a file that saves the setup for
    a program so that it does not have to be entered every time the
    program runs.  Recreate the file {\tt gameBoard.cfg} and adjust its
    contents so that an 8 by 10 board is printed using characters 
    {\tt \$} and {\tt @}.  What should appear on each line of {\tt gameBoard.cfg}?
    \par\vskip 15pt
    
    \begin{enumerate}[(a)]
      \itemsep 15pt
      \begin{multicols}{2}
        \item Line 1: \hfill\ans[1.5in]{8}
        \item Line 2: \hfill\ans[1.5in]{10}
        \item Line 3: \hfill\ans[1.5in]{\$}
        \item Line 4: \hfill\ans[1.5in]{@}
      \end{multicols}
    \end{enumerate}

  \Q Instead of changing the contents of the configuration file by
    hand, it makes sense to have the program save them.  Answer the
    questions below to help you think through adding this feature
    to the function.
    \par\vskip 15pt
    
    \begin{enumerate}[(a)]
      \itemsep 15pt
      \item What type of variable will we need to declare in the function body?
        \hfill\ans[1.5in]{\tt ofstream}
      \item Where in the function should we write to the file?
        \hfill\ans[1.5in]{else clause}
      \item Should we call {\tt .open()} with or without {\tt ofstream::app}?
        \hfill\ans[1.5in]{without}     
    \end{enumerate}
    \par\vskip -30pt\null
    
  \Q Modify the function so that it saves the configuration
    file.  Indicate what you add.\key\\[-2.5mm]
    % \ifprintanswers\vskip -20pt\null\fi
    \begin{answer}[0.5in]
      Add \cpp{ofstream fout} after line 7, and after line
      18 open the file, print out the variables, and then close the file.
    \end{answer}
    % \ifprintanswers\vskip -35pt\null\fi