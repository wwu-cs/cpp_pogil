\documentclass{exam}
%\documentclass[answers]{exam}
\hbadness=99999
\setlength{\textheight}{9.5in}
\setlength{\textwidth}{6.5in}
\setlength{\topmargin}{-0.75in}
\setlength{\oddsidemargin}{0in}
\setlength{\evensidemargin}{0in}

\usepackage{amsmath}
\usepackage{amssymb}
\usepackage{enumerate}
\usepackage[table]{xcolor}
\usepackage{hhline}
\usepackage{graphicx}
\usepackage{tikz}
%\usepackage{pgfplots}
\usepackage{multicol}
\usepackage{fancyvrb}

% for syntax highlighting
\usepackage{minted}
\usemintedstyle[cpp]{xcode}

% for overlay of output
\usepackage[overlay,showboxes]{textpos}

\pagestyle{plain}

\setlength\columnsep{50pt}
\newcommand{\key}{\hfill
      \raisebox{-.3\height}{\includegraphics[width=0.6in]{figures/key.png}}}

\begin{document}
  \thispagestyle{empty}
  \setlength{\parindent}{0pt}

  \begin{center}
    \Large Activity \#18: Writing to Files \\[5pt]
    \large Recorder's Report\\[20pt]
    \normalsize
    \begin{tabular}{lrp{0.1in}lr}
      Manager:  & \fillin[][2.0in] & & Presenter: & \fillin[][2.0in]\\[15pt]
      Recorder: & \fillin[][2.0in] & & Driver:    & \fillin[][2.0in]\\[15pt]
      Date:     & \fillin[][2.0in] & & Score:     & Satisfactory \hspace{10pt} /
      \hspace{10pt} Not Satisfactory
    \end{tabular}
  \end{center}
  \par\vskip 15pt
  
  Record your team's answers to the key questions (marked with
  \raisebox{-.3\height}{\includegraphics[width=0.5in]{figures/key.png}})
  below.
  \begin{enumerate}[(a)]
    \itemsep 1.75in
    \item Model 1, Question \#4
    \item Model 2, Question \#13
    \item Model 3, Question \#17
  \end{enumerate}

  \clearpage\pagenumbering{arabic} 
  
  \begin{center}
    \Large Activity \#18: Writing to Files \\[5pt]
    \large Activity Guide\\[20pt]
  \end{center}
  \vskip -30pt\null

  \begin{center}
    \fbox{
      \begin{minipage}{5.5in}
        {\bf Learning Objectives:} Students will be able to:
        \begin{itemize}
          \item Content:\\[-20pt]
            \begin{itemize}
              \itemsep 0pt
              \item Explain how to open a text file for writing
              \item Explain the difference between writing and appending to a file
              \item Explain how files can be used to save program configuration information
            \end{itemize}
          \item Process\\[-20pt]
            \begin{itemize}
              \itemsep 0pt
              \item Write code that opens, writes to, and closes a file 
              \item Redirect the output of a program to a text file\\[-5pt]
            \end{itemize}
        \end{itemize}
      \end{minipage}
      }
  \end{center}
  \par\vskip 10pt
  

  {\bf\large Model 1: A C++ Main Function} \\[-15pt]
  \begin{center}
    \begin{minipage}{4.25in}
      \small
      \begin{minted}[
        frame=lines,
        framesep=2mm,
        bgcolor=gray!15,
        baselinestretch=1,
        linenos,
        firstnumber=8
      ]{cpp}
int main() {
  ofstream fout;
  fout.open("studentInfo.txt", ofstream::app);

  string firstName = input("Enter first name: ");
  string lastName = input("Enter last name: ");
  string studentID = input("Enter student ID: ");

  fout << "Name: " << firstName << " " << lastName << endl;
  fout << "Student ID: " << studentID << endl;
  fout << endl;
  fout.close();

  cout << "Done! Data is saved in: studentInfo.txt" << endl;  
}        
      \end{minted}
    \end{minipage}
  \end{center}  
  
  {\it\large Refer to Model 1 above as your team develops consensus answers
    to the questions below.}
    \par\vskip 10pt
    
  \begin{enumerate}
    \itemsep 20pt
 
    \item This code can be found in {\tt activity18a.cpp}.  Run it and
      record what output appears on the screen.
      \ifprintanswers\vskip -20pt\null\fi
      \begin{solution}[1in]
        The program says:
        \begin{itemize}
          \itemsep 0pt
          \item Enter first name:
          \item Enter last name:
          \item Enter student ID:
          \item Done! Data is saved in: studentInfo.txt
        \end{itemize}
      \end{solution}
      \ifprintanswers\vskip -30pt\null\fi

    \item Locate the file {\tt studentInfo.txt} and examine its contents.
      \par\vskip 15pt
      
      \begin{enumerate}[(a)]
        \itemsep 15pt
        \item What is in this file?
          \hfill \fillin[the student info we entered][3in]
        \item Did this file exist before the program ran?
          \hfill \fillin[no, it was created by the program][3in]
        \item On which line of code was the file created?
          \hfill \fillin[line 10, by the {\tt .open()} command][3in]
        \item Which lines generated the file contents?
          \hfill \fillin[lines 16 through 18][3in]
        \item Why wasn't the message on line 21 in the file?
          \hfill \fillin[because it was sent to {\tt cout} not {\tt fout}][3in]
      \end{enumerate}
    
\newpage

    \item Run the program again using different input.  Then open the
      {\tt studentInfo.txt} file.  What is in the file?  Is the data
      from the first time you ran it still there?
      \begin{solution}[0.5in]
        The file has original student's info as well as the
        new one from our second execution.
      \end{solution}
      \ifprintanswers\vskip -40pt\else\vskip -20pt\fi\null

    \item Remove the {\tt ofstream::app} argument in the {\tt .open()}
      function call on line 10 and run the\key\\[-2.5mm] program again with different input.
      \par\vskip 15pt
      
      \begin{enumerate}[(a)]
        \item What happened to the contents of {\tt studentInfo.txt}?
          \begin{solution}[0.5in]
            The contents of {\tt studentInfo.txt} were replaced with the
            new input.
          \end{solution}
        \item What did the {\tt ofstream::app} argument do?
          \begin{solution}[0.5in]
            The argument made the program {\it append} to the
            file instead of {\it overwrite} the file.
          \end{solution}
      \end{enumerate}
            
      
    \item You should have observed that lines 16-18 send output to the
      the file {\tt studentInfo.txt}.  How does C++ know to send the
      output to that file?
      \begin{solution}[0.5in]
        The {\tt fout} variable was opened with that file name.
      \end{solution}
      
    \item Rewrite the program so the user can keep entering names
      until they type in {\tt DONE} for the first name.
      \begin{solution}[1.75in]
        Wrap lines 12-18 in a {\tt do-while} loop with condition 
        \mintinline{cpp}|firstName != "DONE"|.
      \end{solution}


  {\bf\large Model 2: A Configuration Function} \\[-20pt]
  \begin{center}
    \begin{minipage}{4in}
      \scriptsize
      \begin{minted}[
        frame=lines,
        framesep=2mm,
        bgcolor=gray!15,
        baselinestretch=1.2,
        linenos,
        firstnumber=38
      ]{cpp}
void getSetup(int &rows, int &cols, char &charOne, char &charTwo) {
  ifstream configIn;
  configIn.open("gameBoard.cfg");
  if (configIn.is_open()) {
    configIn >> rows;
    configIn >> cols;
    configIn >> charOne;
    configIn >> charTwo;
  } else {
    input("Enter Number of Rows: ",rows);
    input("Enter Number of Columns: ",cols);
    input("Enter Player One Symbol: ",charOne);
    input("Enter Player Two Symbol: ",charTwo);
  }
}
      \end{minted}
    \end{minipage}
  \end{center}  
  \TPMargin{5pt}
  \begin{textblock*}{1.25in}[0,0](4.75in,-1.8in)
    \textblockcolor{white}
    \begin{minipage}{1.1in}
      \mintinline{cpp}|"gameBoard.cfg"|\vskip 5pt
      \hrule\vskip 5pt
      \tt
      6\\
      5\\
      *\\
      \#
    \end{minipage}
  \end{textblock*}  
  
  {\it\large Refer to Model 2 above as your team develops consensus answers
    to the questions below.}
    \par\vskip -20pt\null
    
\newpage

  \item Without running the program, determine what the function {\tt getSetup} does.
    \ifprintanswers\vskip -20pt\null\fi
    \begin{solution}[0.4in]
      It reads {\tt rows}, {\tt columns}, {\tt charOne}, and
      {\tt charTwo} from a configuration file, if it exists.
      If the file does not exists, it calls the {\tt input} function
      to get them.
    \end{solution}
    \ifprintanswers\vskip -35pt\null\fi
    
  \item Suppose that this function were called with the contents of
    {\tt gameBoard.cfg} as shown.  To what value would each of the
    following variables be set?
    \par\vskip 15pt
    
    \begin{enumerate}[(a)]
      \itemsep 15pt
      \begin{multicols}{2}
        \item {\tt rows} = \hfill\fillin[6][1.5in]
        \item {\tt cols} = \hfill\fillin[5][1.5in]
        \item {\tt charOne} = \hfill\fillin[*][1.5in]
        \item {\tt charTwo} = \hfill\fillin[\#][1.5in]
      \end{multicols}      
    \end{enumerate}
    
  \item What is missing from the function with regard to the {\tt
    configIn} stream?
    \ifprintanswers\vskip -20pt\null\fi
    \begin{solution}[0.4in]
      The stream is never closed
    \end{solution}
    \ifprintanswers\vskip -35pt\null\fi
  
  \item This function is part of the program in {\tt activity18b.cpp}.
    Run it and record what it does.
    \ifprintanswers\vskip -20pt\null\fi
    \begin{solution}[0.4in]
      It prints out a checkerboard that is 6 x 5 tiles the {\tt *}
      character and {\tt \#} characters.
    \end{solution}
    \ifprintanswers\vskip -35pt\null\fi
    
  \item Now delete the file {\tt gameBoard.cfg} and run the program
    again.  What does it do differently?
    \ifprintanswers\vskip -20pt\null\fi
    \begin{solution}[0.4in]
      It prompts the user for the information that was previously
      stored in the file.
    \end{solution}
    \ifprintanswers\vskip -35pt\null\fi
    
  \item A {\it configuration file} is a file that saves the setup for
    a program so that it does not have to be entered every time the
    program runs.  Recreate the file {\tt gameBoard.cfg} and adjust its
    contents so that an 8 by 10 board is printed using characters 
    {\tt \$} and {\tt @}.  What should appear on each line of {\tt gameBoard.cfg}?
    \par\vskip 15pt
    
    \begin{enumerate}[(a)]
      \itemsep 15pt
      \begin{multicols}{2}
        \item Line 1: \hfill\fillin[8][1.5in]
        \item Line 2: \hfill\fillin[10][1.5in]
        \item Line 3: \hfill\fillin[\$][1.5in]
        \item Line 4: \hfill\fillin[@][1.5in]
      \end{multicols}
    \end{enumerate}
    
  \item Instead of changing the contents of the configuration file by
    hand, it makes sense to have the program save them.  Answer the
    questions below to help you think through adding this feature
    to the function.
    \par\vskip 15pt
    
    \begin{enumerate}[(a)]
      \itemsep 15pt
      \item What type of variable will we need to declare in the function body?
        \hfill\fillin[\tt ofstream][1.5in]
      \item Where in the function should we write to the file?
        \hfill\fillin[else clause][1.5in]
      \item Should we call {\tt .open()} with or without {\tt ofstream::app}?
        \hfill\fillin[without][1.5in]        
    \end{enumerate}
    \par\vskip -30pt\null
    
  \item Modify the function so that it saves the configuration
    file.  Indicate what you add.\key\\[-2.5mm]
    \ifprintanswers\vskip -20pt\null\fi
    \begin{solution}[0.5in]
      Add \mintinline{cpp}|ofstream fout| after line 7, and after line
      18 open the file, print out the variables, and then close the file.
    \end{solution}
    \ifprintanswers\vskip -35pt\null\fi

\newpage

  {\bf\large Model 3: Redirecting Output to a File} \\[-20pt]
  \begin{center}
    \begin{tabular}{p{2.8in}p{0.1in}p{2.7in}}
      \begin{minipage}{2.8in}
        \small
        \begin{minted}[
          frame=lines,
          framesep=2mm,
          bgcolor=gray!15,
          baselinestretch=1.2,
          linenos,
          firstnumber=5
        ]{cpp}      
for (int i = 1; i <= 5; i++) {
  cout << "Number: " << i << endl;
}
cerr << "Done!" << endl;
        \end{minted}
      \end{minipage}
      & &
      \begin{minipage}{2.7in}
        \centering\tt terminal commands\vskip -5pt
        \begin{minted}[
          frame=lines,
          framesep=2mm,
          bgcolor=gray!15,
          baselinestretch=1.2,
        ]{bash}
g++ activity18c.cpp -o test.o
./test.o > output.txt 2> error.txt
        \end{minted}        
      \end{minipage}
    \end{tabular} 
  \end{center}  
  \par\vskip 10pt
                                                
      
  {\it\large Refer to Model 3 above as your team develops consensus answers
    to the questions below.}
    \ifprintanswers\vskip -20pt\null\fi          

    \item The file {\tt activity17c.cpp} contains the code shown on the left-hand side of the
      model.  Run this program as you normally would and describe its
      output.
      \ifprintanswers\vskip -20pt\null\fi
      \begin{solution}[0.75in]
        The program outputs ``Number: 1'' through ``Number: 5'' and
        then outputs ``Done!''.
      \end{solution}
      \ifprintanswers\vskip -35pt\null\fi
      
    \item Now open a terminal in this same folder (right click on the folder name and pick {\it
      Open in Terminal}) and type in the commands shown in the model, pressing {\tt Enter} after each line.
      \par\vskip 15pt
      
      \begin{enumerate}[(a)]
        \itemsep 15pt
        \item What is output to the screen? \hfill\fillin[nothing][2.5in]
        \item What new files now appear in the folder?
          \hfill\fillin[{\tt output.txt} and {\tt error.txt}][2.5in]
        \item Which file contains the output from {\tt cout}?
          \hfill\fillin[{\tt output.txt}][2.5in]
        \item Which file contains the output from {\tt cerr}?
          \hfill\fillin[{\tt error.txt}][2.5in]
      \end{enumerate}

    \item Recall that the command {\tt ./program.o < input.txt} {\it
      redirects} the contents of the {\tt input.txt} file to the
      program as if it were entered from the keyboard.  The command
      above redirects the program output to various files.
      \par\vskip 15pt
      
      \begin{enumerate}
        \itemsep 15pt
        \item What part of the second terminal command redirects the
          output set to {\tt cout} to a file?
          \begin{solution}[0.5in]
            The part {\tt > output.txt}
          \end{solution}
        \item What par tof the command redirects the output sent to
          {\tt cerr} to a file?
          \begin{solution}[0.5in]
            The part {\tt 2> error.txt}
          \end{solution}
      \end{enumerate}
      \par\vskip -25pt\null
            
    \item Pick a previous homework assignment that produces output.
      Open a terminal in that folder, \key\\[-2.5mm] compile the program, and save
      the output to a file.  Recall that we have not used {\tt cerr} 
      previously, so you only need to direct the output from {\tt cout}.
      What command did you use?
      \begin{solution}[0.5in]
        {\tt ./program.o > output.txt}
      \end{solution}

  \end{enumerate}
    
\end{document}
