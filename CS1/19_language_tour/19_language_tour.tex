% Source: https://drive.google.com/drive/folders/1JeCcwPkeQ1e5LSFeBPQm2Lo_3tSUscSw
% File: ".pdf"
% Access: 04-15-2022

% comment out for student version
%\ifdefined\Student\relax\else\def\Teacher{}\fi

\documentclass[12pt]{article}

\title{Activity \#19: A Programming Language Tour}
\author{Clif Kussmaul}
\newcommand{\activityeditor}{Preston Carman}
\newcommand{\activitysource}{\url{https://drive.google.com/drive/folders/1JeCcwPkeQ1e5LSFeBPQm2Lo_3tSUscSw}}
\date{Spring 2022}

\input{../../cspogil.sty}

\begin{document}

  \begin{center}
    \maketitle
    \rolenames
  \end{center}

  \keyquestions{
    \item Model 1, Question \#4
    \item Model 2, Question \#6
    \item Model 3, Question \#10
    \item Model 4, Question \#16
    \item Model 5, Question \#23
  }

  \newpage
  \maketitle

  In this course, you will work in teams of 3--4 students to learn new concepts.
  This activity will introduce you to several different programming languages by having you implement the same algorithm in each language.

  \guide{
    \item Observe key differences between several programming languages
  }{
    \item Write code to solve a problem in several different programming languages\\[-5pt]
  }{
  No additional notes.
  }

  \model{Computing the Value of $\pi$}
  \[
    \frac{\pi}{4} 
       \, = \, \frac{1}{1} \, -\, \frac{1}{3} \, +\, \frac{1}{5} \, -\, \frac{1}{7} \, +\, \cdots
  \]\vskip 5pt
  
  {\it\large Refer to Model 1 above as your team develops consensus answers
    to the questions below.}

  \quest{10 min}
 
  \Q How does the {\bf denominator} (bottom) change from term to
    term in this series of fractions?
    \begin{answer}[0.4in]
      It goes up by two, hitting all the odd numbers.
    \end{answer}

  \Q How does the {\bf sign} (plus or minus) change from term to term?
    \begin{answer}[0.4in]
      It alternates between positive and negative.
    \end{answer}
    
  \Q Fill in the missing values (white cells) in the table below to estimate
    $\pi$ with 2, 3, 4, and 5 terms.
    \begin{center}
      \renewcommand{\arraystretch}{2}
      \begin{tabular}{|l*{7}{|p{0.5in}}|}
        \hline
        \rowcolor{orange!20} & 
          \vskip -12pt\centering$\displaystyle +\frac{1}{1}$ & 
          \vskip -12pt\centering$\displaystyle -\frac{1}{3}$ &
          \vskip -12pt\centering$\displaystyle +\frac{1}{5}$ &
          \vskip -12pt\centering$\displaystyle -\frac{1}{7}$ &
          \vskip -12pt\centering$\displaystyle +\frac{1}{9}$ &
          Approx. of $\frac{\pi}{4}$ & 
          Approx. of $\pi$ \\
        \hline
          a. 1 term & 
          \centering $+1.0$ & 
          \cellcolor{gray!40} &
          \cellcolor{gray!40} &
          \cellcolor{gray!40} &
          \cellcolor{gray!40} &
          \centering 1.0 &
          \hfill 4.0\hfill\null \\
        \hline
          b. 2 terms &
          \centering $+1.0$ &
          \centering $-0.333$ &
          \cellcolor{gray!40} &
          \cellcolor{gray!40} &
          \cellcolor{gray!40} &
          \ans[0.5in]{0.667} &
          \ans[0.5in]{2.668} \\
        \hline
          c. 3 terms &
          \centering $+1.0$ &
          \centering $-0.333$ &
          \ans[0.5in]{$+0.2$} &
          \cellcolor{gray!40} &
          \cellcolor{gray!40} &
          \ans[0.5in]{0.867} &
          \ans[0.5in]{3.468} \\
        \hline
          d. 4 terms & 
          \centering $+1.0$ &
          \centering $-0.333$ &
          \ans[0.5in]{$+0.2$} &
          \ans[0.5in]{$-0.143$} &
          \cellcolor{gray!40} &
          \ans[0.5in]{0.724} &
          \ans[0.5in]{2.896} \\
        \hline
          e. 5 terms &
          \centering $+1.0$ &
          \centering $-0.333$ &
          \ans[0.5in]{$+0.2$} &
          \ans[0.5in]{$-0.143$} &
          \ans[0.5in]{$+0.111$} &
          \ans[0.5in]{0.835} &
          \ans[0.5in]{3.34} \\
        \hline
      \end{tabular}
    \end{center}
    
  \Q The infinite series above can be rewritten in several ways.
    Fill in the empty nume-\key\\[-2.5mm]rators in the series below to provide an
    equivalent series.
    {\small
      \[
        \pi = 
          \enskip \frac{\ans[0.2in]{4}}{1} \enskip +
          \enskip \frac{\ans[0.2in]{-4}}{3} \enskip +
          \enskip \frac{\ans[0.2in]{4}}{5} \enskip +
          \enskip \frac{\ans[0.2in]{-4}}{7} \enskip +
          \enskip \frac{\ans[0.2in]{4}}{9} \enskip + \enskip \cdots
      \]
    }
  \newpage
  \model{A Python Function to Estimate $\pi$} \\
  \begin{center}
    \begin{minipage}{4in}
      \scriptsize
      \begin{cprlst}[
        frame=lines,
        framesep=2mm,
        bgcolor=gray!15,
        baselinestretch=1.2,
        linenos
      ]{python}
# Python function to estimate pi using a given number of terms
def estimatePi(terms):
  if ( terms <= 0 ): return(0.0)  # check for bad inputs
  numerator = -1.0
  pi        = 0.0
  for i in range(0,terms,1):      # loop from 0 to terms-1
    numerator   *= -1
    denominator  = 2*i + 1
    pi          += numerator/denominator
  return(4*pi)
    
# Python function to test the estimatePi() function above
def testEstimatePi():
  vals = [ -1, 0, 1, 2, 10, 100 ]
  for i in vals: print i, estimatePi(i)
  
testEstimatePi()                 # Run the test function
      \end{cprlst}
    \end{minipage}
  \end{center}  
  
  {\it\large Refer to Model 2 above as your team develops consensus answers
    to the questions below.}
    \par\vskip -20pt\null
    
  \Q The {\it Python} programming language was designed by Guido
    van Rossum to be clear and concise.  Python has many specialized
    modules such as: {\it AstroPy} for astronomy, {\it PyGame} for video
    games, {\it SciPy} for scientific computing, and {\it PsychoPy} for
    behavioral science experiments.  Answer the following questions
    based on the Python code above.
    \par\vskip 15pt
    
    \begin{enumerate}[(a)]
      \itemsep 15pt
      \item What character marks the start of a comment?
        \hfill\ans[1.5in]{\tt \#}
      \item What character assigns a value to a variable?
        \hfill\ans[1.5in]{\tt =}
      \item What keyword starts an {\it if} statement?
        \hfill\ans[1.5in]{{\tt if}}
      \item What keyword starts a loop?
        \hfill\ans[1.5in]{{\tt for}}
      \item How are the values that the loop counter goes through given?
        \hfill\ans[1.5in]{{\tt range(0,terms,1)}}      
      \item What keyword specifies the results of a function?
        \hfill\ans[1.5in]{{\tt return}}             
      \item On what line is output printed?
        \hfill\ans[1.5in]{Line 15 ({\tt print})}      
      \item What marks the start and end of a list of values?
        \hfill\ans[1.5in]{Square brackets}       
      \item What marks the start of the body of block of code (if,
        loop, function, etc)?
        \hfill\ans[1.5in]{Colon ({\tt :})}        
      \item What marks the end of the body of block of code (if,
        loop, function, etc)?
        \hfill\ans[1.5in]{Ending indentation}        
    \end{enumerate}
    \vskip -35pt\null
      
  \Q How alike are Python and C++?  Give at least two similarities
    and two differences.\key\\[-2.5mm]
    \begin{answer}[0.5in]
      Answers will vary but examples include:
      \begin{itemize}
        \item Similarities: both use {\tt if} and {\tt for} for branches and loops
        \item Differences: no semicolons or curly braces, no variable types
      \end{itemize}
    \end{answer}
    
\newpage

  \Q An online Python interpreter is available at
    \url{https://repl.it/languages/python}.  Copy the code in {\tt activity19.py}
    into this interpreter and run it.  What is the estimated value of $\pi$
    with 100 terms?
    \begin{answer}[0.5in]
      It is 3.13159290356
    \end{answer}
    
  \Q Now make the following changes to your program, testing those
    changes in the interpreter.  Record how you implement each change.
    
    \begin{enumerate}[(a)]
      \item Add 1000 to the list of test values.
        \begin{answer}[1in]
          % Change line 14 to \python{vals = [ -1. 0. 1, 2, 10, 100, 1000 ]}
        \end{answer}
      \item Set {\tt numerator = 4.0} on line 4 and
        adjust the code so that it still returns the correct value
        of $\pi$.
        \begin{answer}[1in]
          \begin{itemize}
            \itemsep 0pt
            % \item Change line 4 to \python{  numerator = 4.0}
            \item Move line 7 to after line 9
            % \item Change line 10 to \python{  return(pi)}
          \end{itemize}
        \end{answer}
      \item Change the loop starting on line 6 so that it steps
        through odd integers only (1, 3, 5, etc) and adjust the body
        of the loop so that the function still returns the correct value.
        \begin{answer}[1in]
          \begin{itemize}
            \itemsep 0pt
            % \item Change line 6 to \python{  for i in range(1,terms+1,2):}
            % \item Change line 8 to \python{    denominator = i}
          \end{itemize}
        \end{answer}
    \end{enumerate} 
  \newpage
  \model{A JavaScript Function to Estimate $\pi$} \\
  \begin{center}
    \begin{minipage}{4in}
      \scriptsize
      \begin{cprlst}[
        frame=lines,
        framesep=2mm,
        bgcolor=gray!15,
        baselinestretch=1.2,
        linenos
      ]{javascript}
/* JavaScript function to estimate pi using a given number of terms */
function estimatePi(terms) {
  if (terms <= 0) { return 0.0; }      // check for bad inputs
  var numerator = -1.0;
  var pi        = 0.0;
  for ( var i = 0; i < terms; i++ ) {  // loop from 0 to terms-1
    numerator   *= -1;
    denominator  = 2*i + 1;
    pi          += numerator/denominator;
  }
  return 4*pi;
}

/* JavaScript function to test the estimatePi() function above */
function testEstimatePi() {
  var vals = [ -1, 0, 1, 2, 10, 100 ];
  for (var i = 0; i < vals.length; i++ ) {
    console.log( vals[i] + " " + estimatePi(vals[i]) );
  }
}

testEstimatePi();                      // Run the test function
      \end{cprlst}
    \end{minipage}
  \end{center}    
      
  {\it\large Refer to Model 3 above as your team develops consensus answers
    to the questions below.}

\newpage

    \Q The {\it JavaScript} language runs in all major web browsers
    and is primarily used for client-side computation.  Despite their
    names, JavaScript and Java are {\bf not} closely related.  Answer
    the following questions based on the JavaScript code above.
    \par\vskip 15pt
    
    \begin{enumerate}[(a)]
      \itemsep 15pt
      \item What character marks a comment?
        \hfill\ans[1.5in]{{\tt //} or {\tt /* */}}
      \item What keyword specifies the results of a function?
        \hfill\ans[1.5in]{{\tt return}}             
      \item What keyword declares a variable?
        \hfill\ans[1.5in]{\tt var]}     
      \item What command prints output?
        \hfill\ans[1.5in]{\tt console.log}       
      \item What marks the start and end of a list of values?
        \hfill\ans[1.5in]{Square brackets}
      \item What marks a block of code (body of if, loop, function, etc)?
        \hfill\ans[1.5in]{Curly braces}        
    \end{enumerate}
    \vskip -35pt\null
      
  \Q How alike are JavaScript and C++?  Give at least two similarities
    and two differences.\key\\[-2.5mm]
    % \ifprintanswers\vskip -20pt\null\fi
    \begin{answer}[0.5in]
      Answers will vary but may include:
      \begin{itemize}
        \itemsep 0pt
        \item Similarities: both use curly braces and semicolons
        \item Differences: no variable types, functions declared with {\tt function} keyword
      \end{itemize}
    \end{answer}
    % \ifprintanswers\vskip -35pt\null\fi
    
  \Q Give one way in which JavaScript is more similar to Python
    than it is to C++.
    % \ifprintanswers\vskip -20pt\null\fi
    \begin{answer}[0.5in]
      Variables do not have a static type.
    \end{answer}
    % \ifprintanswers\vskip -35pt\null\fi

  \Q An online JavaScript interpreter is available at
    \url{https://repl.it/languages/javascript}.  Copy the code in {\tt activity19.js}
    into this interpreter and find the estimated value of
    $\pi$ with 10 terms.
    % \ifprintanswers\vskip -20pt\null\fi
    \begin{answer}[0.5in]
      It is 3.0418396189294032
    \end{answer}
    % \ifprintanswers\vskip -35pt\null\fi
    
  \Q Now make the following changes to your program, testing those
    changes in the interpreter.  Record how you implement each change.
    
    \begin{enumerate}[(a)]
      \item Add 1000 to the list of test values.
        \begin{answer}[1in]
          % Change line 16 to \javascript{var vals = [ -1, 0, 1, 2, 10, 100, 1000];}
        \end{answer}
      \item Set {\tt numerator = 4.0} on line 4 and
        adjust the code so that it still returns the correct value
        of $\pi$.
        \begin{answer}[1in]
          \begin{itemize}
            \itemsep 0pt
            % \item Change line 4 to \javascript{var numerator = 4.0;}
            \item Move line 7 to after line 9
          \end{itemize}
        \end{answer}
        
\newpage

      \item Change the loop starting on line 6 so that it steps
        through odd integers only (1, 3, 5, etc) and adjust the body
        of the loop so that the function still returns the correct value.
        % \ifprintanswers\vskip -10pt\null\fi
        \begin{answer}[1in]
          \begin{itemize}
            \itemsep 0pt
            % \item Change line 6 to \javascript{for ( var i = 1; i <= terms; i++) {}
            % \item Change line 8 to \javascript{ denominator = i;}
            % \item Change line 11 to \javascript{ return pi;}
          \end{itemize}
        \end{answer}
        % \ifprintanswers\vskip -25pt\null\fi
    \end{enumerate}
    
  \Q Indentation is required in Python, but not in C++ or
    JavaScript (or most other languages).  Thus, the function
    {\tt testEstimatePi()} could be written as:
    
    \begin{center}
      \begin{minipage}{4in}
        \scriptsize
        \begin{cprlst}[
          frame=lines,
          framesep=2mm,
          bgcolor=gray!15,
          baselinestretch=1.2
        ]{javascript}
function testEstimatePi() {var vals=[-1,0,1,2,10,100]; for (var
i=0;i<vals.length;i++){console.log(vals[i]+" "+estimatePi(vals[i]));}
        \end{cprlst}
      \end{minipage}
    \end{center}
    Explain why it is still a good idea to indent code in these
    languages.
    % \ifprintanswers\vskip -20pt\null\fi
    \begin{answer}[0.5in]
      It makes the code much more readable.
    \end{answer}
  \newpage
  \model{An R Function to Estimate $\pi$} \\
  \begin{center}
    \begin{minipage}{4in}
      \scriptsize
      \begin{cprlst}[
        frame=lines,
        framesep=2mm,
        bgcolor=gray!15,
        baselinestretch=1.2,
        linenos
      ]{R}
# R function to estimate pi using a given number of terms
estimatePi <- function(terms) {
  if ( terms <= 0 ) { return(0.0) }      # check for bad inputs
  -1.0 -> numerator
   0.0 -> pi
  for ( i in seq(0,terms-1,1) ) {        # loop from 0 to terms-1
    numerator   <- -numerator
    denominator <- 2*i + 1
    pi          <- pi + numerator / denominator
  }
  return(4*pi)
}

# R function to test the estimatePi() function above
testEstimatePi <- function() {
  vals <- c( -1, 0, 1, 2, 10, 100 )
  for ( i in vals ) { cat(i, estimatePi(i), "\n") }
}

testEstimatePi()                         # Run the test function
      \end{cprlst}
    \end{minipage}
  \end{center}    
      
  {\it\large Refer to Model 4 above as your team develops consensus answers
    to the questions below.}

  \item The {\it R} language was designed by Ihaka and Robert
    Gentlemen for statistical computing and graphics, based on the
    language {\it S}, originally developed by John Chambers at Bell
    Labs.  Answer the following questions based on the R code above.
    \par\vskip 15pt
    
    \begin{enumerate}[(a)]
      \itemsep 15pt
      \item What character marks a comment?
        \hfill\ans[1.5in]{\tt \#}
      \item What assigns a value to a variable (look for two options)?
        \hfill\ans[1.5in]{{\tt ->} or {\tt <-}}             
      \item What keyword starts an {\it if} statement?
        \hfill\ans[1.5in]{\tt if}      
      \item What keyword starts a loop?
        \hfill\ans[1.5in]{\tt for}
        
\newpage

      \item What keyword starts a function definition?
        \hfill\ans[1.5in]{\tt function}
      \item What keyword specifies the results of a function?
        \hfill\ans[1.5in]{\tt return}
      \item What command prints output?
        \hfill\ans[1.5in]{\tt cat}
      \item What marks a block of code (body of if, loop, function, etc)?
        \hfill\ans[1.5in]{Curly Braces}
    \end{enumerate}
    \vskip -35pt\null
      
  \Q Why might R's creators have chosen these unusual assignment
    operators?  What confusion \key\\[-2.5mm] might it avoid?
    % \ifprintanswers\vskip -20pt\null\fi
    \begin{answer}[0.5in]
      It always points towards the variable, avoiding the confusion
      about which side of the {\tt =} the variable can appear on.
    \end{answer}
    % \ifprintanswers\vskip -35pt\null\fi
    
  \Q Besides the assignment operator discussed above, give one difference between R
    code and each of Python, JavaScript, and C++ code.
    % \ifprintanswers\vskip -20pt\null\fi
    \begin{answer}[0.5in]
      Answers will vary but may include:
      \begin{itemize}
        \itemsep 0pt
        % \item the use of \r{seq(0,terms-1,1)} in the {\tt for} loop
        % \item The use of \r{c(-1,0,1,2,10,100)} to create an array (vector actually)      
      \end{itemize}
    \end{answer}
    % \ifprintanswers\vskip -35pt\null\fi

  \Q An online R interpreter is available at
    \url{https://repl.it/languages/rlang}.  Copy the code in {\tt activity19.R}
    into this interpreter and run it.  What is the estimated value of
    $\pi$ with 2 terms?
    % \ifprintanswers\vskip -20pt\null\fi
    \begin{answer}[0.5in]
      It is 2.666667
    \end{answer}
    % \ifprintanswers\vskip -35pt\null\fi
    
  \Q Now make the following changes to your program, testing those
    changes in the interpreter.  Record how you implement each change.
    
    \begin{enumerate}[(a)]
      \item Add 1000 to the list of test values.
        % \ifprintanswers\vskip -20pt\null\fi
        \begin{answer}[1in]
          % Change line 16 to \r{vals <- c( -1, 0, 1, 2, 10, 100, 1000)}
        \end{answer}
        % \ifprintanswers\vskip -15pt\null\fi
      \item Set {\tt numerator = 4.0} on line 4 and
        adjust the code so that it still returns the correct value
        of $\pi$.
        % \ifprintanswers\vskip -20pt\null\fi
        \begin{answer}[1in]
          \begin{itemize}
            \itemsep 0pt
            % \item Change line 4 to \r{4.0 -> numerator}
            \item Move line 7 to below line 9
          \end{itemize}
        \end{answer}
        % \ifprintanswers\vskip -15pt\null\fi
      \item Change the loop starting on line 6 so that it steps
        through odd integers only (1, 3, 5, etc) and adjust the body
        of the loop so that the function still returns the correct value.
        % \ifprintanswers\vskip -20pt\null\fi
        \begin{answer}[1in]
          \begin{itemize}
            \itemsep 0pt
            % \item Change line 6 to \r{for ( i in seq(1,terms,2) ) {}
            % \item Change line 8 to \r{denominator <- i}
            % \item Change line 11 to \r{return(pi)}
          \end{itemize}
        \end{answer}
        % \ifprintanswers\vskip -15pt\null\fi
    \end{enumerate}
  \newpage
   {\bf\large Model 5: Recursive Programs to Estimate $\pi$} \\[-15pt]
  \begin{center}
    \begin{tabular}{p{2.7in}p{0.2in}p{2.7in}}
      \begin{minipage}{2.7in}
        \scriptsize A Recursive Python Program
        \begin{minted}[
          frame=lines,
          framesep=2mm,
          bgcolor=gray!15,
          baselinestretch=1.2,
          linenos
        ]{Python}
# Python function to estimate pi
def estimatePi(terms):
  if ( terms <= 0 ): return 0.0
  val = 4.0 / (2*terms - 1)
  if (terms % 2 == 0): val *= -1
  return val + estimatePi(terms-1)
  
# Function to test estimatePi
def testEstimatePi():
  vals = [ -1, 0, 1, 2, 10, 100 ]
  for i in vals: print i, estimatePi(i)
  
testEstimatePi()
        \end{minted}
        \par\vskip 40pt\null
      \end{minipage}
      & &
      \begin{minipage}{2.7in}
        \scriptsize A Scheme / LISP Program
        \begin{minted}[
          frame=lines,
          framesep=2mm,
          bgcolor=gray!15,
          baselinestretch=1.2,
          linenos
        ]{scheme}
; Scheme function to estimate pi
(define estimatePi
  (lambda (terms)
    (if (<= terms 0)
      0
      (+ (estimatePi (- terms 1))
        (/ (if (odd? terms) +4 -4)
          (+ terms terms -1)
          )))))
        
; Function to test estimatePi
(define testEstimatePi
  (lambda ()
    (define vals `(-1 0 1 2 10 100))
    (map estimatePi vals)   
  ))
  
(testEstimatePi)
        \end{minted}
      \end{minipage}
    \end{tabular}
  \end{center}    
      
  {\it\large Refer to Model 5 above as your team develops consensus answers
    to the questions below.}

  \item The Python code above estimates $\pi$ using the same series,
    but with a different approach.  Answer the following questions based
    on the {\it Python} code above.
    \par\vskip 15pt
    
    \begin{enumerate}[(a)]
      \itemsep 15pt
      \item What variable(s) are defined in the {\tt testEstimatePi()} function?
        \hfill\fillin[{\tt vals} and {\tt i}][1.5in]
      \item What function is called by {\tt testEstimatePi()}?
        \hfill\fillin[{\tt estimatePi()}][1.5in]             
      \item What variable(s) are defined in the {\tt estimatePi()} function?
        \hfill\fillin[{\tt terms} and {\tt val}][1.5in]
      \item What function is called by {\tt estimatePi()}?
        \hfill\fillin[\tt estimatePi()][1.5in]       
      \item When {\tt estimatePi(0)} is called, what value is returned?
        \hfill\fillin[\tt 0.0][1.5in]
      \item When {\tt estimatePi(1)} is called, what function call is made on line 6?
        \hfill\fillin[\tt estimatePi(0)][1.5in]
      \item When {\tt estimatePi(1)} is called, what value is returned?
        \hfill\fillin[\tt 4.0/1 + 0.0 = 4.0][1.5in]
      \item When {\tt estimatePi(2)} is called, what function call is made on line 6?
        \hfill\fillin[\tt estimatePi(1)][1.5in]
      \item When {\tt estimatePi(2)} is called, what value is returned?
        \hfill\fillin[\tt -4.0/3 + 4.0 = 2.6667][1.5in]
    \end{enumerate}
      
  \item A function that calls itself is called a {\it recursive}
    function.  While possibly confusing at first, recursive functions
    can be both powerful and flexible.  Which function(s) in the {\it
    Python} code above is recursive?
    \ifprintanswers\vskip -20pt\null\fi
    \begin{solution}[0.5in]
      The function {\tt estimatePi()}
    \end{solution}
    
\newpage

  \item If recursive function {\it always} calls itself it will never
    end, and eventually the computer will run out of memory.  Situations
    when a recursive function {\it does} call itself are called {\it
    recursive cases} and those when it {\it does not} are called
    {\it base cases}.  What are the base and recursive cases
    for the function {\tt estimatePi()}?
    \ifprintanswers\vskip -20pt\null\fi
    \begin{solution}[0.5in]
      The base case is when {\tt terms = 0}, everything else is a recursive case.
    \end{solution}
    \ifprintanswers\vskip -35pt\null\fi

  \item The {\it Scheme} language is a dialect of the {\it LISP}
    (short for LISt Processing) language, which\key\\[-2.5mm] was developed in the
    1970's by Guy Steele and Gerald Sussman for artificial intelligence.
    Ideally, in Scheme programs the value of a variable never changes,
    and functions always return something useful.  Thus, Scheme is
    called a {\it functional language}.  Scheme also uses {\it
    recursion} (functions that call themselves) instead of loops.
    Answer the following questions based on the {\it Scheme} code above.
    \par\vskip 15pt
    
    \begin{enumerate}[(a)]
      \itemsep 15pt
      \item What marks a comment in {\it Scheme}?
        \hfill\fillin[\tt ;][1.5in] 
      \item What keyword assigns a value to a name?
        \hfill\fillin[\tt define][1.5in]             
      \item What keyword defines a function?
        \hfill\fillin[\tt lambda][1.5in]
      \item What expression subtracts 1 from {\tt terms}?
        \hfill\fillin[\tt (- terms 1)][1.5in]       
      \item What expression checks whether {\tt terms} is odd?
        \hfill\fillin[\tt (odd? terms) ][1.5in]
      \item If that expression evaluates to true, what value is returned?
        \hfill\fillin[\tt +4][1.5in]
      \item If that expression evaluates to false, what value is returned?
        \hfill\fillin[\tt -4][1.5in]
      \item When {\tt estimatePi(2)} is called, what function call is made?
        \hfill\fillin[\tt estimatePi(1)][1.5in]
    \end{enumerate}
    
  \item You'll notice that Scheme uses parentheses quite a bit.  Some
    people jokingly claim that ``LISP'' stands for ``Lost In Stupid
    Parentheses''. How many pairs of parentheses are used in
    defining each of the functions {\tt estimatePi} and {\tt testEstimatePi}?
    \ifprintanswers\vskip -20pt\null\fi
    \begin{solution}[0.75in]
      There are 11 pairs of parentheses in {\tt estimatePi} and 6 in {\tt testEstimatePi}.
    \end{solution}
    \ifprintanswers\vskip -35pt\null\fi

  \item An online {\tt Scheme} interpreter is available at
    \url{https://repl.it/languages/scheme}.  Copy the Scheme code
    from {\tt activity19.scm} into this interpreter and run it. Then determine the 
    approximation of $\pi$ with 10000 terms by adding 10000 to the
    list of test values.
    \ifprintanswers\vskip -20pt\null\fi
    \begin{solution}[1in]
      Change line 14 to \mintinline{scheme}|(define vals `(-1 0 1 2 10 100 10000))|
      and we get the estimate 3.131592035585537 with 10000 terms.
    \end{solution}
    \ifprintanswers\vskip -35pt\null\fi
    
\end{document}
