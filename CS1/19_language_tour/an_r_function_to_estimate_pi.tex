\model{An R Function to Estimate $\pi$}
  \begin{center}
    \begin{minipage}{4in}
      \scriptsize
      \begin{pythlst}
# Function to estimate pi using a given number of terms
estimatePi <- function(terms) {
  if (terms <= 0) {
    return(0.0)  # check for bad inputs
  }
  
  numerator <- -1.0
  pi <- 0.0
  
  for (i in seq(0, terms - 1, 1)) {  # loop from 0 to terms-1
    numerator <- -numerator
    denominator <- 2 * i + 1
    pi <- pi + numerator / denominator
  }
  
  return(4 * pi)
}

# Function to test the estimatePi() function
testEstimatePi <- function() {
  vals <- c(-1, 0, 1, 2, 10, 100)
  for (i in vals) {
    cat(i, estimatePi(i), "\n")
  }
}

testEstimatePi()  # Run the test function
      \end{pythlst}
    \end{minipage}
  \end{center}    
      
  {\it\large Refer to Model 4 above as your team develops consensus answers
    to the questions below.}

  \quest{10 min}

  \Q The {\it R} language was designed by Ihaka and Robert
    Gentlemen for statistical computing and graphics, based on the
    language {\it S}, originally developed by John Chambers at Bell
    Labs.  Answer the following questions based on the R code above.  
    \begin{enumerate}
      \itemsep 10pt
      \item What character marks a comment?
        \hfill\ans[1.5in]{\tt \#}

      \item What assigns a value to a variable (look for two options)?
        \hfill\ans[1.5in]{{\tt ->} or {\tt <-}}

      \item What keyword starts an {\it if} statement?
        \hfill\ans[1.5in]{\tt if}

      \item What keyword starts a loop?
        \hfill\ans[1.5in]{\tt for}

      \item What keyword starts a function definition?
        \hfill\ans[1.5in]{\tt function}

      \item What keyword specifies the results of a function?
        \hfill\ans[1.5in]{\tt return}

      \item What command prints output?
        \hfill\ans[1.5in]{\tt cat}

      \item What marks a block of code (body of if, loop, function, etc)?
        \hfill\ans[1.5in]{Curly Braces}
    \end{enumerate}
      
  \Q Why might R's creators have chosen these unusual assignment
    operators?  What\key\\[-2.5mm] confusion might it avoid?
    \begin{answer}[0.5in]
      It always points towards the variable, avoiding the confusion
      about which side of the {\tt =} the variable can appear on.
    \end{answer}
    
  \Q Besides the assignment operator discussed above, give one difference between R
    code and each of Python, JavaScript, and C++ code.
    \begin{answer}[0.5in]
      Answers will vary but may include:
      \begin{itemize}
        \itemsep 0pt
        \item the use of \python{seq(0,terms-1,1)} in the {\tt for} loop
        \item The use of \python{c(-1,0,1,2,10,100)} to create an array (vector actually)      
      \end{itemize}
    \end{answer}

  \Q An online R interpreter is available at
    \url{https://www.programiz.com/r/online-compiler/}.  Copy the code in {\tt activity19.R}
    into this interpreter and run it.  What is the estimated value of
    $\pi$ with 2 terms?
    \begin{answer}[0.5in]
      It is 2.666667
    \end{answer}
    
  \Q Now make the following changes to your program, testing those
    changes in the interpreter.  Record how you implement each change.
    \begin{enumerate}
      \item Add 1000 to the list of test values.
        \begin{answer}[1in]
          Change line 21 to \python{vals <- c( -1, 0, 1, 2, 10, 100, 1000)}
        \end{answer}

      \item Set {\tt numerator = 4.0} on line 7 and
        adjust the code so that it still returns the correct value
        of $\pi$.
        \begin{answer}[1in]
          \begin{itemize}
            \itemsep 0pt
            \item Change line 7 to \python{4.0 -> numerator}
            \item Change line 16 to \python{return(pi)}
          \end{itemize}
        \end{answer}

      \item Change the loop starting on line 10 so that it steps
        through odd integers only (1, 3, 5, etc) and adjust the body
        of the loop so that the function still returns the correct value.
        \begin{answer}[0.5in]
          \begin{itemize}
            \itemsep 0pt
            \item Change line 10 to \python{for ( i in seq(1, 2 * terms, 2))}
            \item Change line 12 to \python{denominator <- i}
            \item Change line 16 to \python{return(pi)}
          \end{itemize}
        \end{answer}
    \end{enumerate}