\model{An R Function to Estimate $\pi$} \\
  \begin{center}
    \begin{minipage}{4in}
      \scriptsize
      \begin{cprlst}[
        frame=lines,
        framesep=2mm,
        bgcolor=gray!15,
        baselinestretch=1.2,
        linenos
      ]{R}
# R function to estimate pi using a given number of terms
estimatePi <- function(terms) {
  if ( terms <= 0 ) { return(0.0) }      # check for bad inputs
  -1.0 -> numerator
   0.0 -> pi
  for ( i in seq(0,terms-1,1) ) {        # loop from 0 to terms-1
    numerator   <- -numerator
    denominator <- 2*i + 1
    pi          <- pi + numerator / denominator
  }
  return(4*pi)
}

# R function to test the estimatePi() function above
testEstimatePi <- function() {
  vals <- c( -1, 0, 1, 2, 10, 100 )
  for ( i in vals ) { cat(i, estimatePi(i), "\n") }
}

testEstimatePi()                         # Run the test function
      \end{cprlst}
    \end{minipage}
  \end{center}    
      
  {\it\large Refer to Model 4 above as your team develops consensus answers
    to the questions below.}

  \item The {\it R} language was designed by Ihaka and Robert
    Gentlemen for statistical computing and graphics, based on the
    language {\it S}, originally developed by John Chambers at Bell
    Labs.  Answer the following questions based on the R code above.
    \par\vskip 15pt
    
    \begin{enumerate}[(a)]
      \itemsep 15pt
      \item What character marks a comment?
        \hfill\ans[1.5in]{\tt \#}
      \item What assigns a value to a variable (look for two options)?
        \hfill\ans[1.5in]{{\tt ->} or {\tt <-}}             
      \item What keyword starts an {\it if} statement?
        \hfill\ans[1.5in]{\tt if}      
      \item What keyword starts a loop?
        \hfill\ans[1.5in]{\tt for}
        
\newpage

      \item What keyword starts a function definition?
        \hfill\ans[1.5in]{\tt function}
      \item What keyword specifies the results of a function?
        \hfill\ans[1.5in]{\tt return}
      \item What command prints output?
        \hfill\ans[1.5in]{\tt cat}
      \item What marks a block of code (body of if, loop, function, etc)?
        \hfill\ans[1.5in]{Curly Braces}
    \end{enumerate}
    \vskip -35pt\null
      
  \Q Why might R's creators have chosen these unusual assignment
    operators?  What confusion \key\\[-2.5mm] might it avoid?
    % \ifprintanswers\vskip -20pt\null\fi
    \begin{answer}[0.5in]
      It always points towards the variable, avoiding the confusion
      about which side of the {\tt =} the variable can appear on.
    \end{answer}
    % \ifprintanswers\vskip -35pt\null\fi
    
  \Q Besides the assignment operator discussed above, give one difference between R
    code and each of Python, JavaScript, and C++ code.
    % \ifprintanswers\vskip -20pt\null\fi
    \begin{answer}[0.5in]
      Answers will vary but may include:
      \begin{itemize}
        \itemsep 0pt
        % \item the use of \r{seq(0,terms-1,1)} in the {\tt for} loop
        % \item The use of \r{c(-1,0,1,2,10,100)} to create an array (vector actually)      
      \end{itemize}
    \end{answer}
    % \ifprintanswers\vskip -35pt\null\fi

  \Q An online R interpreter is available at
    \url{https://repl.it/languages/rlang}.  Copy the code in {\tt activity19.R}
    into this interpreter and run it.  What is the estimated value of
    $\pi$ with 2 terms?
    % \ifprintanswers\vskip -20pt\null\fi
    \begin{answer}[0.5in]
      It is 2.666667
    \end{answer}
    % \ifprintanswers\vskip -35pt\null\fi
    
  \Q Now make the following changes to your program, testing those
    changes in the interpreter.  Record how you implement each change.
    
    \begin{enumerate}[(a)]
      \item Add 1000 to the list of test values.
        % \ifprintanswers\vskip -20pt\null\fi
        \begin{answer}[1in]
          % Change line 16 to \r{vals <- c( -1, 0, 1, 2, 10, 100, 1000)}
        \end{answer}
        % \ifprintanswers\vskip -15pt\null\fi
      \item Set {\tt numerator = 4.0} on line 4 and
        adjust the code so that it still returns the correct value
        of $\pi$.
        % \ifprintanswers\vskip -20pt\null\fi
        \begin{answer}[1in]
          \begin{itemize}
            \itemsep 0pt
            % \item Change line 4 to \r{4.0 -> numerator}
            \item Move line 7 to below line 9
          \end{itemize}
        \end{answer}
        % \ifprintanswers\vskip -15pt\null\fi
      \item Change the loop starting on line 6 so that it steps
        through odd integers only (1, 3, 5, etc) and adjust the body
        of the loop so that the function still returns the correct value.
        % \ifprintanswers\vskip -20pt\null\fi
        \begin{answer}[1in]
          \begin{itemize}
            \itemsep 0pt
            % \item Change line 6 to \r{for ( i in seq(1,terms,2) ) {}
            % \item Change line 8 to \r{denominator <- i}
            % \item Change line 11 to \r{return(pi)}
          \end{itemize}
        \end{answer}
        % \ifprintanswers\vskip -15pt\null\fi
    \end{enumerate}