\model{Computing the Value of $\pi$}
  \[
    \frac{\pi}{4} 
       \, = \, \frac{1}{1} \, -\, \frac{1}{3} \, +\, \frac{1}{5} \, -\, \frac{1}{7} \, +\, \cdots
  \]\vskip 5pt
  
  {\it\large Refer to Model 1 above as your team develops consensus answers
    to the questions below.}

  \quest{10 min}
 
  \Q How does the {\bf denominator} (bottom) change from term to
    term in this series of fractions?
    \begin{answer}[0.4in]
      It goes up by two, hitting all the odd numbers.
    \end{answer}

  \Q How does the {\bf sign} (plus or minus) change from term to term?
    \begin{answer}[0.4in]
      It alternates between positive and negative.
    \end{answer}
    
  \Q Fill in the missing values (white cells) in the table below to estimate
    $\pi$ with 2, 3, 4, and 5 terms.
    \begin{center}
      \renewcommand{\arraystretch}{2}
      \begin{tabular}{|l*{7}{|p{0.5in}}|}
        \hline
        \rowcolor{orange!20} & 
          \vskip -12pt\centering$\displaystyle +\frac{1}{1}$ & 
          \vskip -12pt\centering$\displaystyle -\frac{1}{3}$ &
          \vskip -12pt\centering$\displaystyle +\frac{1}{5}$ &
          \vskip -12pt\centering$\displaystyle -\frac{1}{7}$ &
          \vskip -12pt\centering$\displaystyle +\frac{1}{9}$ &
          Approx. of $\frac{\pi}{4}$ & 
          Approx. of $\pi$ \\
        \hline
          a. 1 term & 
          \centering $+1.0$ & 
          \cellcolor{gray!40} &
          \cellcolor{gray!40} &
          \cellcolor{gray!40} &
          \cellcolor{gray!40} &
          \centering 1.0 &
          \hfill 4.0\hfill\null \\
        \hline
          b. 2 terms &
          \centering $+1.0$ &
          \centering $-0.333$ &
          \cellcolor{gray!40} &
          \cellcolor{gray!40} &
          \cellcolor{gray!40} &
          \ans[0.5in]{0.667} &
          \ans[0.5in]{2.668} \\
        \hline
          c. 3 terms &
          \centering $+1.0$ &
          \centering $-0.333$ &
          \ans[0.5in]{$+0.2$} &
          \cellcolor{gray!40} &
          \cellcolor{gray!40} &
          \ans[0.5in]{0.867} &
          \ans[0.5in]{3.468} \\
        \hline
          d. 4 terms & 
          \centering $+1.0$ &
          \centering $-0.333$ &
          \ans[0.5in]{$+0.2$} &
          \ans[0.5in]{$-0.143$} &
          \cellcolor{gray!40} &
          \ans[0.5in]{0.724} &
          \ans[0.5in]{2.896} \\
        \hline
          e. 5 terms &
          \centering $+1.0$ &
          \centering $-0.333$ &
          \ans[0.5in]{$+0.2$} &
          \ans[0.5in]{$-0.143$} &
          \ans[0.5in]{$+0.111$} &
          \ans[0.5in]{0.835} &
          \ans[0.5in]{3.34} \\
        \hline
      \end{tabular}
    \end{center}
    
  \Q The infinite series above can be rewritten in several ways.
    Fill in the empty nume-\key\\[-2.5mm]rators in the series below to provide an
    equivalent series.
    {\small
      \[
        \pi = 
          \enskip \frac{\ans[0.2in]{4}}{1} \enskip +
          \enskip \frac{\ans[0.2in]{-4}}{3} \enskip +
          \enskip \frac{\ans[0.2in]{4}}{5} \enskip +
          \enskip \frac{\ans[0.2in]{-4}}{7} \enskip +
          \enskip \frac{\ans[0.2in]{4}}{9} \enskip + \enskip \cdots
      \]
    }