{\bf\large Model 1: Computing the Value of $\pi$} \\
  \[
    \frac{\pi}{4} 
       \, = \, \frac{1}{1} \, -\, \frac{1}{3} \, +\, \frac{1}{5} \, -\, \frac{1}{7} \, +\, \cdots
  \]\vskip 5pt
  
  {\it\large Refer to Model 1 above as your team develops consensus answers
    to the questions below.}
    \par\vskip 10pt
    
  \begin{enumerate}
    \itemsep 20pt
 
    \item How does the {\bf denominator} (bottom) change from term to
      term in this series of fractions?
      \ifprintanswers\vskip -20pt\null\fi
      \begin{solution}[0.4in]
        It goes up by two, hitting all the odd numbers.
      \end{solution}
      \ifprintanswers\vskip -30pt\null\fi

    \item How does the {\bf sign} (plus or minus) change from term to term?
      \ifprintanswers\vskip -20pt\null\fi
      \begin{solution}[0.4in]
        It alternates between positive and negative.
      \end{solution}
      \ifprintanswers\vskip -30pt\null\fi
      
    \item Fill in the missing values (white cells) in the table below to estimate
      $\pi$ with 2, 3, 4, and 5 terms.
      \vskip -20pt\null
      \begin{center}
        \renewcommand{\arraystretch}{2}
        \begin{tabular}{|l*{7}{|p{0.5in}}|}
          \hline
          \rowcolor{orange!20} & 
            \vskip -12pt\centering$\displaystyle +\frac{1}{1}$ & 
            \vskip -12pt\centering$\displaystyle -\frac{1}{3}$ &
            \vskip -12pt\centering$\displaystyle +\frac{1}{5}$ &
            \vskip -12pt\centering$\displaystyle -\frac{1}{7}$ &
            \vskip -12pt\centering$\displaystyle +\frac{1}{9}$ &
            Approx. of $\frac{\pi}{4}$ & 
            Approx. of $\pi$ \\
          \hline
            a. 1 term & 
            \centering $+1.0$ & 
            \cellcolor{gray!40} &
            \cellcolor{gray!40} &
            \cellcolor{gray!40} &
            \cellcolor{gray!40} &
            \centering 1.0 &
            \hfill 4.0\hfill\null \\
          \hline
            b. 2 terms &
            \centering $+1.0$ &
            \centering $-0.333$ &
            \cellcolor{gray!40} &
            \cellcolor{gray!40} &
            \cellcolor{gray!40} &
            \ifprintanswers\centering 0.667\fi &
            \ifprintanswers\hfill 2.668\hfill\null\fi \\
          \hline
            c. 3 terms &
            \centering $+1.0$ &
            \centering $-0.333$ &
            \ifprintanswers\centering $+0.2$\fi &
            \cellcolor{gray!40} &
            \cellcolor{gray!40} &
            \ifprintanswers\centering 0.867\fi &
            \ifprintanswers\hfill 3.468\hfill\null\fi \\
          \hline
            d. 4 terms & 
            \centering $+1.0$ &
            \centering $-0.333$ &
            \ifprintanswers\centering $+0.2$\fi &
            \ifprintanswers\centering $-0.143$\fi &
            \cellcolor{gray!40} &
            \ifprintanswers\centering 0.724\fi &
            \ifprintanswers\hfill 2.896\hfill\null\fi \\
          \hline
            e. 5 terms &
            \centering $+1.0$ &
            \centering $-0.333$ &
            \ifprintanswers\centering $+0.2$\fi &
            \ifprintanswers\centering $-0.143$\fi &
            \ifprintanswers\centering $+0.111$\fi &
            \ifprintanswers\centering 0.835\fi &
            \ifprintanswers\hfill 3.34\hfill\null\fi \\
          \hline
        \end{tabular}
      \end{center}
      \vskip -35pt\null

    \item The infinite series above can be rewritten in several ways.
      Fill in the empty numerators in\key\\[-2.5mm] the series below to provide an
      equivalent series.
      
      {\Large
        \[
          \pi = 
          \ifprintanswers
            \enskip \frac{4}{1} \enskip +
            \enskip \frac{-4}{3} \enskip +
            \enskip \frac{4}{5} \enskip +
            \enskip \frac{-4}{7} \enskip +
            \enskip \frac{4}{9} \enskip + \enskip \cdots
          \else
            \enskip \frac{\enskip\null\enskip}{1} \enskip + 
            \enskip \frac{\enskip\null\enskip}{3} \enskip + 
            \enskip \frac{\enskip\null\enskip}{5} \enskip + 
            \enskip \frac{\enskip\null\enskip}{7} \enskip + 
            \enskip \frac{\enskip\null\enskip}{9} \enskip + \enskip \cdots
          \fi
        \]
      }