\model{A Python Function to Estimate $\pi$} \\
  \begin{center}
    \begin{minipage}{4in}
      \scriptsize
      \begin{cprlst}[
        frame=lines,
        framesep=2mm,
        bgcolor=gray!15,
        baselinestretch=1.2,
        linenos
      ]{python}
# Python function to estimate pi using a given number of terms
def estimatePi(terms):
  if ( terms <= 0 ): return(0.0)  # check for bad inputs
  numerator = -1.0
  pi        = 0.0
  for i in range(0,terms,1):      # loop from 0 to terms-1
    numerator   *= -1
    denominator  = 2*i + 1
    pi          += numerator/denominator
  return(4*pi)
    
# Python function to test the estimatePi() function above
def testEstimatePi():
  vals = [ -1, 0, 1, 2, 10, 100 ]
  for i in vals: print i, estimatePi(i)
  
testEstimatePi()                 # Run the test function
      \end{cprlst}
    \end{minipage}
  \end{center}  
  
  {\it\large Refer to Model 2 above as your team develops consensus answers
    to the questions below.}
    \par\vskip -20pt\null
    
  \Q The {\it Python} programming language was designed by Guido
    van Rossum to be clear and concise.  Python has many specialized
    modules such as: {\it AstroPy} for astronomy, {\it PyGame} for video
    games, {\it SciPy} for scientific computing, and {\it PsychoPy} for
    behavioral science experiments.  Answer the following questions
    based on the Python code above.
    \par\vskip 15pt
    
    \begin{enumerate}[(a)]
      \itemsep 15pt
      \item What character marks the start of a comment?
        \hfill\ans[1.5in]{\tt \#}
      \item What character assigns a value to a variable?
        \hfill\ans[1.5in]{\tt =}
      \item What keyword starts an {\it if} statement?
        \hfill\ans[1.5in]{{\tt if}}
      \item What keyword starts a loop?
        \hfill\ans[1.5in]{{\tt for}}
      \item How are the values that the loop counter goes through given?
        \hfill\ans[1.5in]{{\tt range(0,terms,1)}}      
      \item What keyword specifies the results of a function?
        \hfill\ans[1.5in]{{\tt return}}             
      \item On what line is output printed?
        \hfill\ans[1.5in]{Line 15 ({\tt print})}      
      \item What marks the start and end of a list of values?
        \hfill\ans[1.5in]{Square brackets}       
      \item What marks the start of the body of block of code (if,
        loop, function, etc)?
        \hfill\ans[1.5in]{Colon ({\tt :})}        
      \item What marks the end of the body of block of code (if,
        loop, function, etc)?
        \hfill\ans[1.5in]{Ending indentation}        
    \end{enumerate}
    \vskip -35pt\null
      
  \Q How alike are Python and C++?  Give at least two similarities
    and two differences.\key\\[-2.5mm]
    \begin{answer}[0.5in]
      Answers will vary but examples include:
      \begin{itemize}
        \item Similarities: both use {\tt if} and {\tt for} for branches and loops
        \item Differences: no semicolons or curly braces, no variable types
      \end{itemize}
    \end{answer}
    
\newpage

  \Q An online Python interpreter is available at
    \url{https://repl.it/languages/python}.  Copy the code in {\tt activity19.py}
    into this interpreter and run it.  What is the estimated value of $\pi$
    with 100 terms?
    \begin{answer}[0.5in]
      It is 3.13159290356
    \end{answer}
    
  \Q Now make the following changes to your program, testing those
    changes in the interpreter.  Record how you implement each change.
    
    \begin{enumerate}[(a)]
      \item Add 1000 to the list of test values.
        \begin{answer}[1in]
          % Change line 14 to \python{vals = [ -1. 0. 1, 2, 10, 100, 1000 ]}
        \end{answer}
      \item Set {\tt numerator = 4.0} on line 4 and
        adjust the code so that it still returns the correct value
        of $\pi$.
        \begin{answer}[1in]
          \begin{itemize}
            \itemsep 0pt
            % \item Change line 4 to \python{  numerator = 4.0}
            \item Move line 7 to after line 9
            % \item Change line 10 to \python{  return(pi)}
          \end{itemize}
        \end{answer}
      \item Change the loop starting on line 6 so that it steps
        through odd integers only (1, 3, 5, etc) and adjust the body
        of the loop so that the function still returns the correct value.
        \begin{answer}[1in]
          \begin{itemize}
            \itemsep 0pt
            % \item Change line 6 to \python{  for i in range(1,terms+1,2):}
            % \item Change line 8 to \python{    denominator = i}
          \end{itemize}
        \end{answer}
    \end{enumerate} 