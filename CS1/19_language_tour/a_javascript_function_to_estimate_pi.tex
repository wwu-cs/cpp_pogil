\model{A JavaScript Function to Estimate $\pi$} \\
  \begin{center}
    \begin{minipage}{4in}
      \scriptsize
      \begin{cprlst}[
        frame=lines,
        framesep=2mm,
        bgcolor=gray!15,
        baselinestretch=1.2,
        linenos
      ]{javascript}
/* JavaScript function to estimate pi using a given number of terms */
function estimatePi(terms) {
  if (terms <= 0) { return 0.0; }      // check for bad inputs
  var numerator = -1.0;
  var pi        = 0.0;
  for ( var i = 0; i < terms; i++ ) {  // loop from 0 to terms-1
    numerator   *= -1;
    denominator  = 2*i + 1;
    pi          += numerator/denominator;
  }
  return 4*pi;
}

/* JavaScript function to test the estimatePi() function above */
function testEstimatePi() {
  var vals = [ -1, 0, 1, 2, 10, 100 ];
  for (var i = 0; i < vals.length; i++ ) {
    console.log( vals[i] + " " + estimatePi(vals[i]) );
  }
}

testEstimatePi();                      // Run the test function
      \end{cprlst}
    \end{minipage}
  \end{center}    
      
  {\it\large Refer to Model 3 above as your team develops consensus answers
    to the questions below.}

\newpage

    \Q The {\it JavaScript} language runs in all major web browsers
    and is primarily used for client-side computation.  Despite their
    names, JavaScript and Java are {\bf not} closely related.  Answer
    the following questions based on the JavaScript code above.
    \par\vskip 15pt
    
    \begin{enumerate}[(a)]
      \itemsep 15pt
      \item What character marks a comment?
        \hfill\ans[1.5in]{{\tt //} or {\tt /* */}}
      \item What keyword specifies the results of a function?
        \hfill\ans[1.5in]{{\tt return}}             
      \item What keyword declares a variable?
        \hfill\ans[1.5in]{\tt var]}     
      \item What command prints output?
        \hfill\ans[1.5in]{\tt console.log}       
      \item What marks the start and end of a list of values?
        \hfill\ans[1.5in]{Square brackets}
      \item What marks a block of code (body of if, loop, function, etc)?
        \hfill\ans[1.5in]{Curly braces}        
    \end{enumerate}
    \vskip -35pt\null
      
  \Q How alike are JavaScript and C++?  Give at least two similarities
    and two differences.\key\\[-2.5mm]
    % \ifprintanswers\vskip -20pt\null\fi
    \begin{answer}[0.5in]
      Answers will vary but may include:
      \begin{itemize}
        \itemsep 0pt
        \item Similarities: both use curly braces and semicolons
        \item Differences: no variable types, functions declared with {\tt function} keyword
      \end{itemize}
    \end{answer}
    % \ifprintanswers\vskip -35pt\null\fi
    
  \Q Give one way in which JavaScript is more similar to Python
    than it is to C++.
    % \ifprintanswers\vskip -20pt\null\fi
    \begin{answer}[0.5in]
      Variables do not have a static type.
    \end{answer}
    % \ifprintanswers\vskip -35pt\null\fi

  \Q An online JavaScript interpreter is available at
    \url{https://repl.it/languages/javascript}.  Copy the code in {\tt activity19.js}
    into this interpreter and find the estimated value of
    $\pi$ with 10 terms.
    % \ifprintanswers\vskip -20pt\null\fi
    \begin{answer}[0.5in]
      It is 3.0418396189294032
    \end{answer}
    % \ifprintanswers\vskip -35pt\null\fi
    
  \Q Now make the following changes to your program, testing those
    changes in the interpreter.  Record how you implement each change.
    
    \begin{enumerate}[(a)]
      \item Add 1000 to the list of test values.
        \begin{answer}[1in]
          % Change line 16 to \javascript{var vals = [ -1, 0, 1, 2, 10, 100, 1000];}
        \end{answer}
      \item Set {\tt numerator = 4.0} on line 4 and
        adjust the code so that it still returns the correct value
        of $\pi$.
        \begin{answer}[1in]
          \begin{itemize}
            \itemsep 0pt
            % \item Change line 4 to \javascript{var numerator = 4.0;}
            \item Move line 7 to after line 9
          \end{itemize}
        \end{answer}
        
\newpage

      \item Change the loop starting on line 6 so that it steps
        through odd integers only (1, 3, 5, etc) and adjust the body
        of the loop so that the function still returns the correct value.
        % \ifprintanswers\vskip -10pt\null\fi
        \begin{answer}[1in]
          \begin{itemize}
            \itemsep 0pt
            % \item Change line 6 to \javascript{for ( var i = 1; i <= terms; i++) {}
            % \item Change line 8 to \javascript{ denominator = i;}
            % \item Change line 11 to \javascript{ return pi;}
          \end{itemize}
        \end{answer}
        % \ifprintanswers\vskip -25pt\null\fi
    \end{enumerate}
    
  \Q Indentation is required in Python, but not in C++ or
    JavaScript (or most other languages).  Thus, the function
    {\tt testEstimatePi()} could be written as:
    
    \begin{center}
      \begin{minipage}{4in}
        \scriptsize
        \begin{cprlst}[
          frame=lines,
          framesep=2mm,
          bgcolor=gray!15,
          baselinestretch=1.2
        ]{javascript}
function testEstimatePi() {var vals=[-1,0,1,2,10,100]; for (var
i=0;i<vals.length;i++){console.log(vals[i]+" "+estimatePi(vals[i]));}
        \end{cprlst}
      \end{minipage}
    \end{center}
    Explain why it is still a good idea to indent code in these
    languages.
    % \ifprintanswers\vskip -20pt\null\fi
    \begin{answer}[0.5in]
      It makes the code much more readable.
    \end{answer}