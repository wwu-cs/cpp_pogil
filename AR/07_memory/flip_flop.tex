\model{Flip-Flop}

  Recall that the output from a NOR gate is 1 only when both inputs are 0.
  If either input is 1 then the output is 0.\\

  {\it\large Refer to Model 2 above as your team develops consensus answers
  to the questions below.}

  \quest{10 min}

  \Q Complete Figure 3 by showing both switches open and highlight (in red if you can) any portion of the circuit with high voltage (assuming both inputs to the left NOR are both 0).
    \vspace{10pt}
    \begin{center}
      \ifdefined\Teacher
        \includegraphics[width=0.5\textwidth]{figures/flipflop_fig3_solution.png}
      \else
        \includegraphics[width=0.5\textwidth]{figures/flipflop_fig3.png}
      \fi
      \par\vspace{5pt}
      {\small Figure 3}
    \end{center}

  \Q With the circuit in the state shown in Figure 3, imagine that Switch 1 is closed. Complete Figure 4 highlighting any portion of the circuit with high voltage.
    \vspace{10pt}
    \begin{center}
      \ifdefined\Teacher
        \includegraphics[width=0.5\textwidth]{figures/flipflop_fig4_solution.png}
      \else
        \includegraphics[width=0.5\textwidth]{figures/flipflop_fig4.png}
      \fi
      \par\vspace{5pt}
      {\small Figure 4}
    \end{center}

  \Q With the circuit in the state shown in Figure 4, imagine that Switch 1 is opened. Complete Figure 5 highlighting any portion of the circuit with high voltage.
    \vspace{10pt}
    \begin{center}
      \ifdefined\Teacher
        \includegraphics[width=0.5\textwidth]{figures/flipflop_fig5_solution.png}
      \else
        \includegraphics[width=0.5\textwidth]{figures/flipflop_fig5.png}
      \fi
      \par\vspace{5pt}
      {\small Figure 5}
    \end{center}

  \Q Both Figures 3 and 5 have both switches open. How are they different?\key\\[-2.5mm]
    \begin{answer}[0.75in]
      The light is off in Figure 3 and on in Figure 5.
    \end{answer}

  \Q With the circuit in the state shown in Figure 5 (Switch 1 is open), imagine that Switch 2 is closed. Complete Figure 6 highlighting any portion of the circuit with high voltage.
    \vspace{10pt}
    \begin{center}
      \ifdefined\Teacher
        \includegraphics[width=0.5\textwidth]{figures/flipflop_fig6_solution.png}
      \else
        \includegraphics[width=0.5\textwidth]{figures/flipflop_fig6.png}
      \fi
      \par\vspace{5pt}
      {\small Figure 6}
    \end{center}

  \Q With the circuit in the state shown in Figure 6, imagine that Switch 2 is opened. Complete Figure 7 highlighting any portion of the circuit with high voltage.
    \vspace{10pt}
    \begin{center}
      \ifdefined\Teacher
        \includegraphics[width=0.5\textwidth]{figures/flipflop_fig7_solution.png}
      \else
        \includegraphics[width=0.5\textwidth]{figures/flipflop_fig7.png}
      \fi
      \par\vspace{5pt}
      {\small Figure 7}
    \end{center}