\model{Wrap-Around}

  \quest{10 min}

  \Q Consider an odometer with only five digits (left of the decimal).
    \vspace{10pt}
    \begin{center}
      \includegraphics[width=0.35\textwidth]{figures/odometer.png}
    \end{center}

    \begin{enumerate}
      \item If it reads 99999 and you add one mile, what will it show?
        \begin{answer}[0.5in]
          00000
        \end{answer}

      \item If you have a limited number of digits, is there a way to avoid this wrap-around?
        \begin{answer}[0.5in]
          No, it will always happen eventually.
        \end{answer}
    \end{enumerate}

  \vspace{-20pt}

  \Q Consider a binary value with eight bits.\key\\[-2.5mm]
    \begin{enumerate}
      \item How many unique values could you represent (the count, not a formula)?
        \begin{answer}[0.3in]
          256
        \end{answer}

      \item If you chose to represent only non-negative integers, what would be the range of unsigned integers (converted to decimal)?
        \begin{answer}[0.3in]
          0 to 255
        \end{answer}

      \item If you added one to the maximum value, what would you get?
        \begin{answer}[0.3in]
          0
        \end{answer}

      \item If you subtracted one from the minimum value, what would you get?
        \begin{answer}[0.3in]
          255
        \end{answer}

      \item If you used one bit for a sign (indicating positive or negative), how many bits would be left for the number?
        \begin{answer}[0.3in]
          Seven (7)
        \end{answer}

      \item How many values could be represented using the remaining bits?
        \begin{answer}[0.3in]
          128 values
        \end{answer}

      \item What is the range of unsigned integers you could represent in the remaining bits?
        \begin{answer}[0.3in]
          0 to 127
        \end{answer}

      \item What would be the range of signed integers you could represent with a single sign bit and seven value bits?
        \begin{answer}[0.3in]
          -127 to +127
        \end{answer}

      \item How many unique values does that represent?
        \begin{answer}[0.3in]
          255 values
        \end{answer}

      \item Are there any missing values? If so, where did they go?
        \begin{answer}[0.3in]
          We have a representation for both +0 and -0, and we should not need both.
        \end{answer}
    \end{enumerate}

  Next, we will consider another approach to representing negative numbers that allows us to store the maximum number of values and allows
  adding negative numbers without any special work.