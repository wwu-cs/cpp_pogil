% Source: CPTR 280 Computer Organization and Assembly Language Fall 2021
% File: "06 Subtraction (key).pdf"
% Author: James Foster, pogil@jgfoster.net

% comment out for student version
% \ifdefined\Student\relax\else\def\Teacher{}\fi

\documentclass[12pt]{article}

\title{Activity 6: Subtraction}
\author{James Foster}
\newcommand{\activityeditor}{James Foster}
\newcommand{\activitysource}{\url{pogil@jgfoster.net}}
\date{Fall 2021}

\input{../../cspogil.sty}
\usepackage{graphicx}
\usepackage{tabularx}

\begin{document}

  \begin{center}
    \maketitle
    \rolenames
  \end{center}

  \keyquestions{
  \item Model 1, Question \#2
  \item Model 2, Question \#7
  \item Model 3, Question \#12
  \item Model 3, Question \#16
  \item Model 4, Question \#21, a--c
  }

  \newpage
  \maketitle

  We said earlier that most math operations can be simplified to addition. In this lesson we will build on
  earlier exercises dealing with binary and with digital logic gates to understand how computers do subtraction.
  
  \guides{
    \item Use nine's complement to subtract without borrowing; and,
    \item Represent negative numbers in binary using two's complement.
   }{
    \item Reflect on how what how the team could work and learn more effectively.
   }{
    No additional notes
   }{
    \item odometer \href{https://search.creativecommons.org/photos/57c70ba8-18f2-482a-87ab-285606463446}{link here}
    \item clock: \url{https://pngio.com/images/png-a1553797.html}
   }

  \model{Subtraction without Borrowing}

  \quest{10 min}

  \Q Complete the problem below. With subtraction, what complexity arises analogous to (but more difficult than) the carry step in addition?
    \vspace{10pt}
    \begin{center}
      \begin{tabular}{|c|c|c|c|}
        \hline
        & \ans[0.3in]{1} & \ans[0.3in]{14} & \\
        \hline
        & 2 & 5 & 3 \\
        \hline
        - & 1 & 7 & 6 \\
        \hline
        & & \ans[0.3in]{7} & \ans[0.3in]{7} \\
        \hline
      \end{tabular}
    \end{center}
    \begin{answer}[0.75in]
      Borrowing.
    \end{answer}

  \Q Given the expression $x - y$ (where $y$ is no more than three digits), what decimal va-\key\\[-2.5mm]lue for $x$ will guarantee that you never have to borrow?
    \begin{answer}[0.25in]
      All nines (999)
    \end{answer}

  \Q To avoid borrowing, we can use algebra to rewrite $x - y$ as follows:
    \vspace{5pt}
    \begin{center}
      \begin{align*}
        x - y &= x - y + 0 \\
              &= x - y + (1000 - 1000) \\
              &= x - y + (999 + 1 - 1000) \\
              &= x + (999 - y) + 1 - 1000 \\
              &= (999 - y) + x + 1 - 1000
      \end{align*}
    \end{center}

    Complete the problems in the table below and confirm that the result is the same.

    \vspace{10pt}
    \begin{center}
      \begin{tabular}{|c|c|c|c|c|}
        \hline
        & & 9 & 9 & 9 \\
        \hline
        - & & 1 & 7 & 6 \\
        \hline
        & \ans[0.3in]{} & \ans[0.3in]{8} & \ans[0.3in]{2} & \ans[0.3in]{3} \\
        \hline
        + & & 2 & 5 & 3 \\
        \hline
        & \ans[0.3in]{1} & \ans[0.3in]{0} & \ans[0.3in]{7} & \ans[0.3in]{6} \\
        \hline
        + & & & & 1 \\
        \hline
        & \ans[0.3in]{1} & \ans[0.3in]{0} & \ans[0.3in]{7} & \ans[0.3in]{7} \\
        \hline
        - & 1 & 0 & 0 & 0 \\
        \hline
        & & & \ans[0.3in]{7} & \ans[0.3in]{7} \\
        \hline
      \end{tabular}
    \end{center}

  \vspace{10pt}
  The result of subtracting a number from nines gives us a nines' complement. Note that while the word \textit{compliment} means to say something nice,
  the word \textit{complement} means ``a thing that completes.'' Adding a number and its nines' complement gives us all (or just) nines.

  \Q Of course, in a computer with digital logic we will be using binary instead of decimal. Given the expression $x - y$ (where $y$ is no more than 8 digits), what binary value for $x$ will guarantee that you never have to borrow?
    \begin{answer}[0.5in]
      1111 1111
    \end{answer}

  \Q Complete the binary subtraction problem below.
    \vspace{10pt}
    \begin{center}
      \begin{tabular}{|c|c|c|}
        \hline
        & 1 & 1 \\
        \hline
        - & 1 & 0 \\
        \hline
        & \ans[0.3in]{0} & \ans[0.3in]{1} \\
        \hline
      \end{tabular}
    \end{center}

    \begin{enumerate}
      \item What pattern do you notice when you compare the bits in rows 2 and 3 column by column (the subtrahend and the difference)?
        \begin{answer}[0.5in]
          They are inverse (opposite).
        \end{answer}

      \item With respect to each column, which truth table takes one input (from line 2) and gives one output (line 3)?
        \begin{answer}[0.5in]
          NOT
        \end{answer}
    \end{enumerate}
  \newpage
  \model{Wrap-Around}

  \quest{10 min}

  \Q Consider an odometer with only five digits (left of the decimal).
    \vspace{10pt}
    \begin{center}
      \includegraphics[width=0.35\textwidth]{figures/odometer.png}
    \end{center}

    \begin{enumerate}
      \item If it reads 99999 and you add one mile, what will it show?
        \begin{answer}[0.5in]
          00000
        \end{answer}

      \item If you have a limited number of digits, is there a way to avoid this wrap-around?
        \begin{answer}[0.5in]
          No, it will always happen eventually.
        \end{answer}
    \end{enumerate}

  \vspace{-20pt}

  \Q Consider a binary value with eight bits.\key\\[-2.5mm]
    \begin{enumerate}
      \item How many unique values could you represent (the count, not a formula)?
        \begin{answer}[0.3in]
          256
        \end{answer}

      \item If you chose to represent only non-negative integers, what would be the range of unsigned integers (converted to decimal)?
        \begin{answer}[0.3in]
          0 to 255
        \end{answer}

      \item If you added one to the maximum value, what would you get?
        \begin{answer}[0.3in]
          0
        \end{answer}

      \item If you subtracted one from the minimum value, what would you get?
        \begin{answer}[0.3in]
          255
        \end{answer}

      \item If you used one bit for a sign (indicating positive or negative), how many bits would be left for the number?
        \begin{answer}[0.3in]
          Seven (7)
        \end{answer}

      \item How many values could be represented using the remaining bits?
        \begin{answer}[0.3in]
          128 values
        \end{answer}

      \item What is the range of unsigned integers you could represent in the remaining bits?
        \begin{answer}[0.3in]
          0 to 127
        \end{answer}

      \item What would be the range of signed integers you could represent with a single sign bit and seven value bits?
        \begin{answer}[0.3in]
          -127 to +127
        \end{answer}

      \item How many unique values does that represent?
        \begin{answer}[0.3in]
          255 values
        \end{answer}

      \item Are there any missing values? If so, where did they go?
        \begin{answer}[0.3in]
          We have a representation for both +0 and -0, and we should not need both.
        \end{answer}
    \end{enumerate}

  Next, we will consider another approach to representing negative numbers that allows us to store the maximum number of values and allows
  adding negative numbers without any special work.
  \newpage
  \model{Negative Numbers}

  \quest{15 min}

  \Q Consider a traditional analog clock face.
    \vspace{10pt}
    \begin{center}
      \includegraphics[width=0.25\textwidth]{figures/clock.png}
    \end{center}

    \begin{enumerate}
      \item How many unique hours can it represent?
        \hfill\ans{12}

      \item What is the range of hours it represents?
        \hfill\ans{1 to 12}

      \item If we think of 3:00 as ``three hours after noon'' (or ``+3''), what hour position on the clock is represented by ``three hours before noon'' (``-3'')?
        \begin{answer}[0.5in]
          9
        \end{answer}

      \item What range of integers would support the same number of negative clock-face ``hours'' as nonnegative clock-face ``hours''?
        \begin{answer}[0.5in]
          -6 to +5
        \end{answer}
    \end{enumerate}

  \vspace{-20pt}

  \Q Complete the following table to map the typical clock face hours (line 1) to the hours before and after noon (line 2).
    (Hint: start at each end and work to the middle.)
    \vspace{10pt}
    \begin{center}
      \begin{tabular}{|c|c|c|c|c|c|c|c|c|c|c|c|}
        \hline
        12 & 1 & 2 & 3 & 4 & 5 & 6 & 7 & 8 & 9 & 10 & 11 \\
        \hline
        \ans[0.3in]{0} & \ans[0.3in]{1} & \ans[0.3in]{2} & \ans[0.3in]{3} & \ans[0.3in]{4} & \ans[0.3in]{5} & \ans[0.3in]{-6} & \ans[0.3in]{-5} & \ans[0.3in]{-4} & \ans[0.3in]{-3} & \ans[0.3in]{-2} & \ans[0.3in]{-1} \\
        \hline
      \end{tabular}
    \end{center}

  \Q With this relabeled clock, answer the following questions:
    \begin{enumerate}
      \item If we start at 2 and add 2, what do we get?
        \hfill\ans{+4}

      \item If we start at 2 and subtract 1, what do we get?
        \hfill\ans{+1}

      \item If we start at 2 and subtract 2, what do we get? (Hint: not 12)
        \hfill\ans[1.5in]{0}

      \item If we start at 2 and subtract 5, what do we get?
        \hfill\ans{-3}

      \item If we start at -2 and subtract 2, what do we get?
        \hfill\ans{-4}

      \item If we start at -2 and add 6, what do we get?
        \hfill\ans{+4}

      \item So far this maps very nicely to ``normal'' arithmetic. But if we start at 5 and add 3, what do we get? (Hint: not 8!)
        \begin{answer}[0.5in]
          -4
        \end{answer}

      \item If we start at -5 and subtract 4, what do we get? (Hint: not -9!)
        \hfill\ans[1.5in]{+3}
    \end{enumerate}

  \Q Using a similar approach, fill in the following table to map selected unsigned values [0, 255] to signed values [-128, 127]. This is called a two's complement.
    \vspace{10pt}
    \begin{center}
      \begin{tabular}{|c|c|c|c|c|c|c|c|c|c|c|c|}
        \hline
        0 & 1 & 2 & \ldots & 126 & 127 & 128 & 129 & 130 & \ldots & 254 & 255 \\
        \hline
        \ans[0.2in]{0} & \ans[0.2in]{1} & \ans[0.2in]{2} & \ldots & \ans[0.25in]{126} & \ans[0.25in]{127} & \ans[0.3in]{-128} & \ans[0.3in]{-127} & \ans[0.3in]{-126} & \ldots & \ans[0.2in]{-2} & \ans[0.2in]{-1} \\
        \hline
      \end{tabular}
    \end{center}

  \Q Convert some of these binary values to unsigned and signed decimal:\key\\[-2.5mm]
    \vspace{10pt}
    \begin{center}
      \begin{tabular}{|c|c|c|}
        \hline
        \textbf{Binary} & \textbf{Unsigned Decimal} & \textbf{Signed Decimal} \\
        \hline
        0000 0000 & \ans[0.4in]{0} & \ans[0.4in]{0} \\
        \hline
        0000 0001 & \ans[0.4in]{1} & \ans[0.4in]{1} \\
        \hline
        0111 1110 & \ans[0.4in]{126} & \ans[0.4in]{126} \\
        \hline
        0111 1111 & \ans[0.4in]{127} & \ans[0.4in]{127} \\
        \hline
        1000 0000 & \ans[0.4in]{128} & \ans[0.4in]{-128} \\
        \hline
        1000 0001 & \ans[0.4in]{129} & \ans[0.4in]{-127} \\
        \hline
        1111 1110 & \ans[0.4in]{254} & \ans[0.4in]{-2} \\
        \hline
        1111 1111 & \ans[0.4in]{255} & \ans[0.4in]{-1} \\
        \hline
      \end{tabular}
    \end{center}

  \Q How does the high-order (left-most) bit correlate with the sign?
    \hfill\ans[1.5in]{0 $\Rightarrow$ positive; 1 $\Rightarrow$ negative}

  \newpage

  \Q Perform some addition of signed binary numbers and note how the sign works:
  \vspace{10pt}
  \begin{center}
    \begin{tabular}{|c|c|c|c|}
      \hline
      11 & 0000 1011 & -118 & 1000 1010 \\
      + 5 & + 0000 0101 & + 44 & + 0010 1100 \\
      \hline
      \ans[1in]{16} & \ans[1in]{0001 0000} & \ans[1in]{-74} & \ans[1in]{1011 0110} \\
      \hline
      106 & 0110 1010 & -22 & 1110 1010 \\
      + -63 & + 1100 0001 & + -86 & +1010 1010 \\
      \hline
      \ans[1in]{43} & \ans[1in]{1 0010 1011} & \ans[1in]{-204} & \ans[1in]{1 1001 0100} \\
      \hline
    \end{tabular}
  \end{center}
  
  This suggests that another way to do subtraction is just to add the negative value. So how do we get a negative? Compare a few examples and
  note that in each case, if you added the values you would get zeros in the right-most (low order) 8 bits (which is, of course, correct).

  \vspace{10pt}
  \begin{center}
    \begin{tabular}{ccc}
      1 = 0000 0001 & & -1 = 1111 1111 \\
      2 = 0000 0010 & & -2 = 1111 1110 \\
      126 = 0111 1110 & & -126 = 1000 0010 \\
      127 = 0111 1111 & & -127 = 1000 0001 \\
    \end{tabular}
  \end{center}

  \Q For each positive value above (1, 2, 126, and 127), write out its one's complement or inverse (replace 0 with 1, 1 with 0).
    \vspace{10pt}
    \begin{center}
      \begin{tabular}{cccc}
        \ans[1in]{1111 1110} & \ans[1in]{1111 1101} & \ans[1in]{1000 0001} & \ans[1in]{1000 0000} \\
      \end{tabular}
    \end{center}

  \vspace{-20pt}

  \Q How does the inverse compare with the negative value? What would you add\key\\[-2.5mm] to the inverse to get the negative? Learn this useful rule for computing the negative of a number!
    \begin{answer}[0.5in]
      The inverse is one less than the negative. To get a negative, invert and add 1.
    \end{answer}
  \newpage
  \model{Encoding in Binary}

  While we said that there isn't anything special about decimal (base ten), there is something special about binary:
  it is the simplest number system possible and forms the foundation of all digital computers. Anything that can be mapped to
  discrete values can be encoded in binary; the only requirement is that everyone who uses the code must understand the mapping.\\

  {\it\large Refer to Model 4 above as your team develops consensus answers
  to the questions below.}

  \quest{15 min}

  \Q ``Hang a lantern aloft in the belfry arch / Of the North Church tower as a special light, --- / One, if by land, and two. if by sea \ldots''
    Map Paul Revere's code to binary.
    \vspace{10pt}
    \begin{center}
      \begin{tabular}{|c|l|}
        \hline
        \textbf{Code} & \textbf{Meaning} \\
        \hline
        00 & \ans[1in]{``no invasion''} \\
        \hline
        01 & \ans[1in]{``by land''} \\
        \hline
        10 & \ans[1in]{``by land''} \\
        \hline
        11 & ``by sea'' \\
        \hline
      \end{tabular}
    \end{center}

  \Q The WWU CS department has four professors, Preston Carman, James Foster, John Foster, and Natalie Smith-Gray.
    Assign each a unique ID using just two bits.
    \vspace{10pt}
    \begin{center}
      \begin{tabular}{|c|l|}
        \hline
        \textbf{Code} & \textbf{Professor} \\
        \hline
        \ans[0.4in]{00} & Preston Carman \\
        \hline
        \ans[0.4in]{01} & James Foster \\
        \hline
        \ans[0.4in]{10} & John Foster \\
        \hline
        \ans[0.4in]{11} & Natalie Smith-Gray \\
        \hline
      \end{tabular}
    \end{center}

  \Q How many bits would be required to encode the following:\key\\[-2.5mm]
    \begin{enumerate}
      \item The letter grades A, B, C, D, F? (Hint: not 5!)
        \begin{answer}[0.5in]
          3 bits are needed to encode 5 values (2 bits give only 4 values)
        \end{answer}

      \item The 30 students enrolled in CPTR 280?
        \begin{answer}[0.5in]
          5 bits are needed to encode 30 values (4 bits give only 16 values)
        \end{answer}

      \item All 34 students if we admitted everyone on the wait list?
        \begin{answer}[0.5in]
          6 bits are needed to encode 34 values (5 bits give only 32 values)
        \end{answer}
    \end{enumerate}

  \vspace{-20pt}

  \Q If you were using four bits to represent a non-negative integer (to keep things simple), what is the range of numbers you could encode (that is, what would they convert to in decimal)?
    \begin{answer}[0.5in]
      4 bits can represent 16 values, from 0 to 15
    \end{answer}

\end{document}