% Source: CPTR 280 Computer Organization and Assembly Language Fall 2020
% File: "03 Number Systems (key).pdf"
% Author: James Foster, pogil@jgfoster.net

% comment out for student version
% \ifdefined\Student\relax\else\def\Teacher{}\fi

\documentclass[12pt]{article}

\title{Activity 3: Number Systems}

\author{James Foster}
\newcommand{\activityeditor}{James Foster}
\newcommand{\activitysource}{\url{pogil@jgfoster.net}}
\date{Fall 2020}

\input{../../cspogil.sty}
\usepackage{graphicx}
\usepackage{tabularx}

\begin{document}

  \begin{center}
    \maketitle
    \rolenames
  \end{center}

  \keyquestions{
  \item Model 1, Question \#6
  \item Model 2, Question \#14
  \item Model 3, Question \#23
  }

  \newpage
  \maketitle

  At the fundamental level digital computers work with high and low voltage, interpreted as ones and zeros and represented as binary numbers. Understanding how data is represented as binary numbers is important to understanding how computers work and how to writing programs.
  
  \guides{
    \item explain the role of exponents in positional notation; and,
    \item encode data in binary.
   }{
    \item convert between numbers in different bases.
   }{
    No additional notes
   }{
    \item Based in part on \href{https://drive.google.com/file/d/1twyZLJo5WvwJUgLYUG2FI21QRNWLwURL/view}{This activity}
    \item Based in part on \href{https://drive.google.com/file/d/1FR4omsdMQIDgPbypDl73fpQ4MzMTu9nn/view}{This activity}
   }

   \model{Representing Numbers}
  
  Whether we write cat, gato, or chat, we are referring to a \includegraphics[width=0.05\textwidth]{figures/cat.png}. Different words can be code for the same meaning. In a similar way, there are various ways of recording a number of ducks:

  \vspace{10pt}
  \begin{center}
    \includegraphics[width=0.3\textwidth]{figures/ducks.png} \quad VII\quad \includegraphics[width=0.05\textwidth]{figures/number.png}
  \end{center}

  {\it\large Refer to Model 1 above as your team develops consensus answers
  to the questions below.}

  \quest{10 min}

  \Q Write the number of ducks as a numeral and as a word.
    \ans{Seven (7)}

  \Q Recall that in Roman Numerals, X represents the number of digits (fingers and thumbs on two hands) for a typical person. Provide the representation for the number of items in a dozen using Roman Numerals and at least two other representations.
    \begin{answer}[0.5in]
      XII, 12, twelve
    \end{answer}

  \Q While Roman Numerals has some disadvantages, it works pretty well for some addition and subtraction problems. Calculate the following and record the answer in Roman Numerals. How does the answer compare to the problem?
    \vspace{5pt}
    \begin{center}
      XI + VII = \ans[1in]{XVIII}
    \end{center}
    \begin{answer}[0.5in]
      Uses the same symbols, just combining (and rearranging)
    \end{answer}

  \Q Can you think of another way Roman Numerals might be easy to use, especially for someone carving dates on stone?
    \begin{answer}[0.5in]
      Straight lines are easier to cut
    \end{answer}

  \vspace{10pt}
  Our number system is known as the Hindu-Arabic or Indo-Arabic, or just Arabic number system and was brought to Europe from India by Arab mathematicians.

  \newpage

  \Q Compare the numbers II (Roman) and 11 (Arabic).
    \begin{enumerate}
      \item What is the value/meaning of the second (right-hand) of each of the numerals?
        \begin{answer}[0.5in]
          One
        \end{answer}

      \item What is the value/meaning of the first (left-hand) of the two Roman Numerals?
        \begin{answer}[0.5in]
          Also one
        \end{answer}

      \item What is the value/meaning of the first (left-hand) of the two Arabic Numerals?
        \begin{answer}[0.5in]
          Because of its position, it represents ten
        \end{answer}

      \item How would you describe the difference?
        \begin{answer}[0.5in]
          In Arabic Numerals, position contains part of the meaning
        \end{answer}
    \end{enumerate}

  \vspace{-20pt}

  \Q Compare the numbers M (Roman) and 1000 (Arabic). What is the role of zero?\key\\[-2.5mm]
    \begin{answer}[1in]
      Zero acts as a placeholder to help the digits to the left find their right positional value.
    \end{answer}

  \Q Why do Arabic Numerals not have special symbols for ten, one hundred, and one thousand?
    \begin{answer}[1in]
      Because we communicate these values by the digit one with a position.
    \end{answer}
   \newpage
   \model{Positional Notation}

  The structure of the Hindu-Arabic number system is revealed in the way we pronounce numbers. The year one thousand, nine hundred, fifty-eight is represented as follows:

  \vspace{10pt}
  \begin{center}
    \begin{tabular}{|l|c|c|c|c|c|c|}
      \hline
      \textbf{Decimal Numbers} & ten-thousands & thousands & hundreds & tens & ones & one-tenths \\
      \hline
      Digit & 0 & 1 & 9 & 5 & 8 & 0 \\
      \hline
      Place (full) & 10,000 & 1,000 & 100 & 10 & 1 & 0.1 \\
      \hline
      Place (exponent notation) & $10^4$ & $10^3$ & $10^2$ & $10^1$ & $10^0$ & $10^{-1}$ \\
      \hline
      Digit * Place & 0 & 1,000 & 900 & 50 & 8 & 0 \\
      \hline
    \end{tabular}
  \end{center}

  {\it\large Refer to Model 2 above as your team develops consensus answers
  to the questions below.}

  \quest{10 min}

  \Q What is the value of the exponent for the hundreds place?
    \hfill\ans{two (2)}

  \Q Without changing the number represented, complete the column to the left of the thousands. What would be the digit for each additional column to the left for the given number?
    \begin{answer}[0.5in]
      0
    \end{answer}

  \Q Without changing the number represented, complete the column to the right of the ones. What would be the digit for each additional column to the right for the given number?
    \begin{answer}[0.5in]
      0
    \end{answer}

  \Q The use of ten as a base is presumably a consequence of most people having ten digits (fingers and thumbs on two hands) that are easy to use for counting. But there is nothing magical about base ten---other bases work just the same (and have some advantages, as we will see).
    Complete the following table to convert the octal number 1776$_{\text{eight}}$ to base ten (use of a calculator is fine).
    \vspace{10pt}
    \begin{center}
      \begin{tabular}{|l|c|c|c|c|c|}
        \hline
        \textbf{Octal Numbers} & five hundred twelves & sixty-fours & eights & ones & \textbf{Total} \\
        \hline
        Digit & 1 & 7 & 7 & 6 & \\
        \hline
        Place (full) & 512 & 64 & 8 & 1 & \\
        \hline
        Place (exponent notation) & $8^3$ & $8^2$ & $8^1$ & $8^0$ & \\
        \hline
        Digit * Place & \ans[0.5in]{512} & \ans[0.5in]{448} & \ans[0.5in]{56} & \ans[0.5in]{6} & \ans[0.5in]{1022} \\
        \hline
      \end{tabular}
    \end{center}

  \Q Complete the following table to convert the binary number 101010$_{\text{two}}$ to decimal.
    \vspace{10pt}
    \begin{center}
      \begin{tabular}{|l|c|c|c|c|c|c|c|}
        \hline
        \textbf{Binary Numbers} & 32s & sixteens & eights & fours & twos & ones & \textbf{Total} \\
        \hline
        Digit & 1 & 0 & 1 & 0 & 1 & 0 & \\
        \hline
        Place (full) & 32 & 16 & 8 & 4 & 2 & 1 & \\
        \hline
        Place (exponent notation) & $2^5$ & $2^4$ & $2^3$ & $2^2$ & $2^1$ & $2^0$ & \\
        \hline
        Digit * Place & \ans[0.4in]{32} & \ans[0.4in]{0} & \ans[0.4in]{8} & \ans[0.4in]{0} & \ans[0.4in]{2} & \ans[0.4in]{0} & \ans[0.5in]{= 42} \\
        \hline
      \end{tabular}
    \end{center}

  \Q Convert the number 10010000$_{\text{two}}$ to decimal.
    \hfill\ans{144}

  \vspace{-20pt}

  \Q What two symbols (digits) are used to represent quantities in a binary system?\key\\[-2.5mm]
    \begin{answer}[0.5in]
      0 and 1
    \end{answer}

  \Q In binary, what do all odd numbers have in common?
    \begin{answer}[0.5in]
      The one's place digit is always 1.
    \end{answer}
   \newpage
   \model{Convert Between Bases}

  There are (at least) two approaches to converting a decimal number to another base. Each approach involves divisions
  (by powers of the base) and special treatment of the (integer) quotient and remainder.

  \vspace{10pt}
  One approach to converting a decimal number to another base is to start building the number from the left (with the most significant digit) and working to the right (with the least significant digit).
  Observe how we use the following table to convert 1000$_{\text{ten}}$ to 1750$_{\text{eight}}$.

  \vspace{10pt}
  \begin{center}
    \begin{tabular}{|l|c|c|c|c|}
      \hline
      \textbf{Step} & 1 & 2 & 3 & 4 \\
      \hline
      Start and Remainder & 1,000 & 488 & 40 & 0 \\
      \hline
      Place & $8^3$ & $8^2$ & $8^1$ & $8^0$ \\
      \hline
      Divide by Place & $\div$ 512 & $\div$ 64 & $\div$ 8 & $\div$ 1 \\
      \hline
      Quotient & 1 & 7 & 5 & 0 \\
      \hline
    \end{tabular}
  \end{center}

  {\it\large Refer to Model 3 above as your team develops consensus answers
  to the questions below.}

  \quest{15 min}

  \Q Complete the following table to convert 10,000$_{\text{ten}}$ to octal.
    \vspace{10pt}
    \begin{center}
      \begin{tabular}{|l|c|c|c|c|c|}
        \hline
        Start and Remainder & \ans[0.5in]{10,000} & \ans[0.5in]{1808} & \ans[0.5in]{272} & \ans[0.5in]{16} & \ans[0.5in]{0} \\
        \hline
        Place & $8^4$ & $8^3$ & $8^2$ & $8^1$ & $8^0$ \\
        \hline
        Divide by Place & $\div$ 4096 & $\div$ 512 & $\div$ 64 & $\div$ 8 & $\div$ 1 \\
        \hline
        Quotient & \ans[0.5in]{2} & \ans[0.5in]{3} & \ans[0.5in]{4} & \ans[0.5in]{2} & \ans[0.5in]{0} \\
        \hline
      \end{tabular}
    \end{center}

  \Q Complete the following table to convert 100$_{\text{ten}}$ to binary.
    \vspace{10pt}
    \begin{center}
      \begin{tabular}{|l|c|c|c|c|c|c|c|}
        \hline
        Start and Remainder & \ans[0.4in]{100} & \ans[0.4in]{36} & \ans[0.4in]{4} & \ans[0.4in]{4} & \ans[0.4in]{4} & \ans[0.4in]{0} & \ans[0.4in]{0} \\
        \hline
        Place & $2^6$ & $2^5$ & $2^4$ & $2^3$ & $2^2$ & $2^1$ & $2^0$ \\
        \hline
        Divide by Place & $\div$ 64 & $\div$ 32 & $\div$ 16 & $\div$ 8 & $\div$ 4 & $\div$ 2 & $\div$ 1 \\
        \hline
        Quotient & \ans[0.4in]{1} & \ans[0.4in]{1} & \ans[0.4in]{0} & \ans[0.4in]{0} & \ans[0.4in]{1} & \ans[0.4in]{0} & \ans[0.4in]{0} \\
        \hline
      \end{tabular}
    \end{center}

  \newpage

  \Q Recall that one bit can have two values (0 and 1), and that two bits can have four values (00, 01, 10, and 11).
    \begin{enumerate}
      \item How many values can 3 bits represent? (hint: not 6!)
        \hfill\ans{8}

      \item How many values can 4 bits represent?
        \hfill\ans{16}

      \item How many values can 8 bits represent?
        \hfill\ans{256}

      \item How many values can 10 bits represent?
        \hfill\ans{1024}

      \item What is the formula for how many values N bits can represent?
        \hfill\ans[0.6in]{$2^N$}
    \end{enumerate}

  \vspace{5pt}
  \textit{These values are very useful; remember them!}\\

  Another approach to converting from decimal to a base is to start building the number from the right and work left. In this case we always divide by the base. Observe how we use the following table to convert 1000$_{\text{ten}}$ to 1750$_{\text{eight}}$.

  \vspace{10pt}
  \begin{center}
    \begin{tabular}{|l|c|c|c|c|}
      \hline
      \textbf{Step} & 4 & 3 & 2 & 1 \\
      \hline
      Start and Quotient & 1 & 15 & 125 & 1000 \\
      \hline
      Divide by Base & $\div$ 8 & $\div$ 8 & $\div$ 8 & $\div$ 8 \\
      \hline
      Remainder & 1 & 7 & 5 & 0 \\
      \hline
    \end{tabular}
  \end{center}

  \Q Complete the following table to convert 10,000$_{\text{ten}}$ to octal.
    \vspace{10pt}
    \begin{center}
      \begin{tabular}{|l|c|c|c|c|c|}
        \hline
        Start and Quotient & \ans[0.5in]{2} & \ans[0.5in]{19} & \ans[0.5in]{156} & \ans[0.5in]{1250} & \ans[0.5in]{10,000} \\
        \hline
        Divide by Base & $\div$ 8 & $\div$ 8 & $\div$ 8 & $\div$ 8 & $\div$ 8 \\
        \hline
        Remainder & \ans[0.5in]{2} & \ans[0.5in]{3} & \ans[0.5in]{4} & \ans[0.5in]{2} & \ans[0.5in]{0} \\
        \hline
      \end{tabular}
    \end{center}

  \Q Complete the following table to convert 100$_{\text{ten}}$ to binary.
    \vspace{10pt}
    \begin{center}
      \begin{tabular}{|l|c|c|c|c|c|c|c|}
        \hline
        Start and Quotient & \ans[0.4in]{1} & \ans[0.4in]{3} & \ans[0.4in]{6} & \ans[0.4in]{12} & \ans[0.4in]{25} & \ans[0.4in]{50} & \ans[0.4in]{100} \\
        \hline
        Divide by Base & $\div$ 2 & $\div$ 2 & $\div$ 2 & $\div$ 2 & $\div$ 2 & $\div$ 2 & $\div$ 2 \\
        \hline
        Remainder & \ans[0.4in]{1} & \ans[0.4in]{1} & \ans[0.4in]{0} & \ans[0.4in]{0} & \ans[0.4in]{1} & \ans[0.4in]{0} & \ans[0.4in]{0} \\
        \hline
      \end{tabular}
    \end{center}

  \newpage
  As a general rule, converting between non-decimal bases is done by converting to decimal and then the target base (unless you want to learn multiplication and division in other bases!). But shortcuts are possible when converting between bases where one is a power of the other. Specifically, converting between binary and octal (and later hexadecimal) is trivial. Note that each single octal digit (0-7) is always represented by exactly three binary digits, so you can build (or memorize) a simple conversion table.

  \Q Complete the following octal to binary conversion table:
    \vspace{10pt}
    \begin{center}
      \begin{tabular}{|c|c|c|c|c|c|c|c|c|}
        \hline
        \textbf{Octal} & 7 & 6 & 5 & 4 & 3 & 2 & 1 & 0 \\
        \hline
        \textbf{Binary} & \ans[0.4in]{111} & \ans[0.4in]{110} & \ans[0.4in]{101} & \ans[0.4in]{100} & \ans[0.4in]{011} & \ans[0.4in]{010} & \ans[0.4in]{001} & \ans[0.4in]{000} \\
        \hline
      \end{tabular}
    \end{center}

  \Q What is 4321$_{\text{eight}}$ in binary?
    \hfill\ans{100 011 010 001}

  \Q What is 11010101$_{\text{binary}}$ in octal?\key\\[-2.5mm]
    \begin{answer}[0.5in]
      325
    \end{answer}
   \newpage
   \model{Hexadecimal}

  \vspace{10pt}
  So far, we have looked at number systems with bases of ten or less. But if we had six fingers per hand, then we would likely grow up using base twelve
  (which is evenly divisible by 2, 3, 4, and 6, so has advantages!).

  As it happens, while octal was popular in the early days of computers, it is more common today to use hexadecimal (base 16) to represent internal values.
  By convention, we use A-F to represent ten through fifteen.\\

  {\it\large Refer to Model 4 above as your team develops consensus answers
  to the questions below.}

  \quest{15 min}

  \Q Complete the following table to convert the hexadecimal number 7E3$_{\text{sixteen}}$ to base ten.
    \vspace{10pt}
    \begin{center}
      \begin{tabular}{|l|c|c|c|c|}
        \hline
        \textbf{Hexadecimal Numbers} & two hundred fifty-sixes & sixteens & ones & \textbf{Total} \\
        \hline
        Digit & 7 & E & 3 & \\
        \hline
        Place (full) & 256 & 16 & 1 & \\
        \hline
        Place (exponent notation) & $16^2$ & $16^1$ & $16^0$ & \\
        \hline
        Digit * Place & \ans[0.5in]{1792} & \ans[0.5in]{224} & \ans[0.5in]{3} & \ans[0.5in]{2025} \\
        \hline
      \end{tabular}
    \end{center}

  \Q Complete the following table to convert 99,324$_{\text{ten}}$ to hexadecimal.
    \vspace{10pt}
    \begin{center}
      \begin{tabular}{|l|c|c|c|c|c|}
        \hline
        Start and Quotient & \ans[0.5in]{1} & \ans[0.5in]{24} & \ans[0.5in]{387} & \ans[0.5in]{6207} & \ans[0.5in]{99,324} \\
        \hline
        Divide by Base & $\div$ 16 & $\div$ 16 & $\div$ 16 & $\div$ 16 & $\div$ 16 \\
        \hline
        Remainder & \ans[0.5in]{1} & \ans[0.5in]{8} & \ans[0.5in]{3} & \ans[0.5in]{F} & \ans[0.5in]{C} \\
        \hline
      \end{tabular}
    \end{center}

  \Q As an extension to question 21, complete the following hexadecimal to binary table:
    \vspace{10pt}
    \begin{center}
      \begin{tabular}{|c|c|c|c|c|c|c|c|c|}
        \hline
        \textbf{Hexadecimal} & F & E & D & C & B & A & 9 & 8 \\
        \hline
        \textbf{Binary} & \ans[0.4in]{1111} & \ans[0.4in]{1110} & \ans[0.4in]{1101} & \ans[0.4in]{1100} & \ans[0.4in]{1011} & \ans[0.4in]{1010} & \ans[0.4in]{1001} & \ans[0.4in]{1000} \\
        \hline
      \end{tabular}
    \end{center}

  \Q What is 1101'0101$_{\text{binary}}$ in hexadecimal (the apostrophe character is used to group four binary digits, much like commas are used to group three decimal digits)?
    \begin{answer}[0.3in]
      D5
    \end{answer}

  \Q Instead of using a subscript with the base spelled out as a word, many languages use a prefix of ``0x'' to designate a hexadecimal number. What is 0x7C1 in binary?
    \begin{answer}[0.2in]
      0111'1100'0001
    \end{answer}

\end{document}