\model{Hexadecimal}

  \vspace{10pt}
  So far, we have looked at number systems with bases of ten or less. But if we had six fingers per hand, then we would likely grow up using base twelve
  (which is evenly divisible by 2, 3, 4, and 6, so has advantages!).

  As it happens, while octal was popular in the early days of computers, it is more common today to use hexadecimal (base 16) to represent internal values.
  By convention, we use A-F to represent ten through fifteen.\\

  {\it\large Refer to Model 4 above as your team develops consensus answers
  to the questions below.}

  \quest{15 min}

  \Q Complete the following table to convert the hexadecimal number 7E3$_{\text{sixteen}}$ to base ten.
    \vspace{10pt}
    \begin{center}
      \begin{tabular}{|l|c|c|c|c|}
        \hline
        \textbf{Hexadecimal Numbers} & two hundred fifty-sixes & sixteens & ones & \textbf{Total} \\
        \hline
        Digit & 7 & E & 3 & \\
        \hline
        Place (full) & 256 & 16 & 1 & \\
        \hline
        Place (exponent notation) & $16^2$ & $16^1$ & $16^0$ & \\
        \hline
        Digit * Place & \ans[0.5in]{1792} & \ans[0.5in]{224} & \ans[0.5in]{3} & \ans[0.5in]{2025} \\
        \hline
      \end{tabular}
    \end{center}

  \Q Complete the following table to convert 99,324$_{\text{ten}}$ to hexadecimal.
    \vspace{10pt}
    \begin{center}
      \begin{tabular}{|l|c|c|c|c|c|}
        \hline
        Start and Quotient & \ans[0.5in]{1} & \ans[0.5in]{24} & \ans[0.5in]{387} & \ans[0.5in]{6207} & \ans[0.5in]{99,324} \\
        \hline
        Divide by Base & $\div$ 16 & $\div$ 16 & $\div$ 16 & $\div$ 16 & $\div$ 16 \\
        \hline
        Remainder & \ans[0.5in]{1} & \ans[0.5in]{8} & \ans[0.5in]{3} & \ans[0.5in]{F} & \ans[0.5in]{C} \\
        \hline
      \end{tabular}
    \end{center}

  \Q As an extension to question 21, complete the following hexadecimal to binary table:
    \vspace{10pt}
    \begin{center}
      \begin{tabular}{|c|c|c|c|c|c|c|c|c|}
        \hline
        \textbf{Hexadecimal} & F & E & D & C & B & A & 9 & 8 \\
        \hline
        \textbf{Binary} & \ans[0.4in]{1111} & \ans[0.4in]{1110} & \ans[0.4in]{1101} & \ans[0.4in]{1100} & \ans[0.4in]{1011} & \ans[0.4in]{1010} & \ans[0.4in]{1001} & \ans[0.4in]{1000} \\
        \hline
      \end{tabular}
    \end{center}

  \Q What is 1101'0101$_{\text{binary}}$ in hexadecimal (the apostrophe character is used to group four binary digits, much like commas are used to group three decimal digits)?
    \begin{answer}[0.3in]
      D5
    \end{answer}

  \Q Instead of using a subscript with the base spelled out as a word, many languages use a prefix of ``0x'' to designate a hexadecimal number. What is 0x7C1 in binary?
    \begin{answer}[0.2in]
      0111'1100'0001
    \end{answer}