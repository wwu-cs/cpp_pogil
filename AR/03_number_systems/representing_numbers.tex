\model{Representing Numbers}
  
  Whether we write cat, gato, or chat, we are referring to a \includegraphics[width=0.05\textwidth]{figures/cat.png}. Different words can be code for the same meaning. In a similar way, there are various ways of recording a number of ducks:

  \vspace{10pt}
  \begin{center}
    \includegraphics[width=0.3\textwidth]{figures/ducks.png} \quad VII\quad \includegraphics[width=0.05\textwidth]{figures/number.png}
  \end{center}

  {\it\large Refer to Model 1 above as your team develops consensus answers
  to the questions below.}

  \quest{10 min}

  \Q Write the number of ducks as a numeral and as a word.
    \ans{Seven (7)}

  \Q Recall that in Roman Numerals, X represents the number of digits (fingers and thumbs on two hands) for a typical person. Provide the representation for the number of items in a dozen using Roman Numerals and at least two other representations.
    \begin{answer}[0.5in]
      XII, 12, twelve
    \end{answer}

  \Q While Roman Numerals has some disadvantages, it works pretty well for some addition and subtraction problems. Calculate the following and record the answer in Roman Numerals. How does the answer compare to the problem?
    \vspace{5pt}
    \begin{center}
      XI + VII = \ans[1in]{XVIII}
    \end{center}
    \begin{answer}[0.5in]
      Uses the same symbols, just combining (and rearranging)
    \end{answer}

  \Q Can you think of another way Roman Numerals might be easy to use, especially for someone carving dates on stone?
    \begin{answer}[0.5in]
      Straight lines are easier to cut
    \end{answer}

  \vspace{10pt}
  Our number system is known as the Hindu-Arabic or Indo-Arabic, or just Arabic number system and was brought to Europe from India by Arab mathematicians.

  \newpage

  \Q Compare the numbers II (Roman) and 11 (Arabic).
    \begin{enumerate}
      \item What is the value/meaning of the second (right-hand) of each of the numerals?
        \begin{answer}[0.5in]
          One
        \end{answer}

      \item What is the value/meaning of the first (left-hand) of the two Roman Numerals?
        \begin{answer}[0.5in]
          Also one
        \end{answer}

      \item What is the value/meaning of the first (left-hand) of the two Arabic Numerals?
        \begin{answer}[0.5in]
          Because of its position, it represents ten
        \end{answer}

      \item How would you describe the difference?
        \begin{answer}[0.5in]
          In Arabic Numerals, position contains part of the meaning
        \end{answer}
    \end{enumerate}

  \vspace{-20pt}

  \Q Compare the numbers M (Roman) and 1000 (Arabic). What is the role of zero?\key\\[-2.5mm]
    \begin{answer}[1in]
      Zero acts as a placeholder to help the digits to the left find their right positional value.
    \end{answer}

  \Q Why do Arabic Numerals not have special symbols for ten, one hundred, and one thousand?
    \begin{answer}[1in]
      Because we communicate these values by the digit one with a position.
    \end{answer}