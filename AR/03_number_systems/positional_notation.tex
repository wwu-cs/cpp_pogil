\model{Positional Notation}

  The structure of the Hindu-Arabic number system is revealed in the way we pronounce numbers. The year one thousand, nine hundred, fifty-eight is represented as follows:

  \vspace{10pt}
  \begin{center}
    \begin{tabular}{|l|c|c|c|c|c|c|}
      \hline
      \textbf{Decimal Numbers} & ten-thousands & thousands & hundreds & tens & ones & one-tenths \\
      \hline
      Digit & 0 & 1 & 9 & 5 & 8 & 0 \\
      \hline
      Place (full) & 10,000 & 1,000 & 100 & 10 & 1 & 0.1 \\
      \hline
      Place (exponent notation) & $10^4$ & $10^3$ & $10^2$ & $10^1$ & $10^0$ & $10^{-1}$ \\
      \hline
      Digit * Place & 0 & 1,000 & 900 & 50 & 8 & 0 \\
      \hline
    \end{tabular}
  \end{center}

  {\it\large Refer to Model 2 above as your team develops consensus answers
  to the questions below.}

  \quest{10 min}

  \Q What is the value of the exponent for the hundreds place?
    \hfill\ans{two (2)}

  \Q Without changing the number represented, complete the column to the left of the thousands. What would be the digit for each additional column to the left for the given number?
    \begin{answer}[0.5in]
      0
    \end{answer}

  \Q Without changing the number represented, complete the column to the right of the ones. What would be the digit for each additional column to the right for the given number?
    \begin{answer}[0.5in]
      0
    \end{answer}

  \Q The use of ten as a base is presumably a consequence of most people having ten digits (fingers and thumbs on two hands) that are easy to use for counting. But there is nothing magical about base ten---other bases work just the same (and have some advantages, as we will see).
    Complete the following table to convert the octal number 1776$_{\text{eight}}$ to base ten (use of a calculator is fine).
    \vspace{10pt}
    \begin{center}
      \begin{tabular}{|l|c|c|c|c|c|}
        \hline
        \textbf{Octal Numbers} & five hundred twelves & sixty-fours & eights & ones & \textbf{Total} \\
        \hline
        Digit & 1 & 7 & 7 & 6 & \\
        \hline
        Place (full) & 512 & 64 & 8 & 1 & \\
        \hline
        Place (exponent notation) & $8^3$ & $8^2$ & $8^1$ & $8^0$ & \\
        \hline
        Digit * Place & \ans[0.5in]{512} & \ans[0.5in]{448} & \ans[0.5in]{56} & \ans[0.5in]{6} & \ans[0.5in]{1022} \\
        \hline
      \end{tabular}
    \end{center}

  \Q Complete the following table to convert the binary number 101010$_{\text{two}}$ to decimal.
    \vspace{10pt}
    \begin{center}
      \begin{tabular}{|l|c|c|c|c|c|c|c|}
        \hline
        \textbf{Binary Numbers} & 32s & sixteens & eights & fours & twos & ones & \textbf{Total} \\
        \hline
        Digit & 1 & 0 & 1 & 0 & 1 & 0 & \\
        \hline
        Place (full) & 32 & 16 & 8 & 4 & 2 & 1 & \\
        \hline
        Place (exponent notation) & $2^5$ & $2^4$ & $2^3$ & $2^2$ & $2^1$ & $2^0$ & \\
        \hline
        Digit * Place & \ans[0.4in]{32} & \ans[0.4in]{0} & \ans[0.4in]{8} & \ans[0.4in]{0} & \ans[0.4in]{2} & \ans[0.4in]{0} & \ans[0.5in]{= 42} \\
        \hline
      \end{tabular}
    \end{center}

  \Q Convert the number 10010000$_{\text{two}}$ to decimal.
    \hfill\ans{144}

  \vspace{-20pt}

  \Q What two symbols (digits) are used to represent quantities in a binary system?\key\\[-2.5mm]
    \begin{answer}[0.5in]
      0 and 1
    \end{answer}

  \Q In binary, what do all odd numbers have in common?
    \begin{answer}[0.5in]
      The one's place digit is always 1.
    \end{answer}