  \model{Writing}

  One limitation in communication is when the recipient is distant in space or time. In that case you can write your message and deliver it to the recipient (or wait for them to find it). That writing needs to be encoded in a medium that the recipient can decode.

  \vspace{10pt}
  \begin{center}
    \includegraphics[width=0.4\textwidth]{figures/pictographs.png}
  \end{center}

  {\it\large Refer to Model 2 above as your team develops consensus answers
  to the questions below.}

  \quest{7 min}

  \Q What items are depicted in this picture?
    \begin{answer}[1in]
      Four-footed animal (deer?), two-footed animal (duck?), humans, an arch (rainbow? mountain?)
    \end{answer}

  \Q Pictograms are little schematic pictures of things, actions, or concepts, and form the basis of cuneiform and hieroglyphs. Ideograms are graphical symbols that represent ideas. Logogram is the name of the symbols that are used in writing Chinese, Japanese, and Korean.

  In what sense would it be easier for children to learn to read if dog were written as '\includegraphics[height=1em]{figures/dog_emoji.png}'?
    \begin{answer}[1.5in]
      The logogram is an immediate, pictorial representation of the meaning of the word. There is no need to learn a separate alphabet, phonetics, or spelling.
    \end{answer}

  \newpage

  \Q While there are over 100,000 Chinese characters, a literate person needs to know only\key\\[-2.5mm] 4,000--5,000 characters. Printing using movable type (each letter is a separate piece) was invented in China several hundred years before Gutenberg created his press in 15th century Europe. Why might it have caught on more quickly in Europe?
    \begin{answer}[1.5in]
      Creating, storing, and searching 100,000 unique pieces of movable type would be much more cumbersome than doing the same with up to 100 pieces.
    \end{answer}

  \Q Even when words are made up of letters, the letters themselves can have different presentations. Consider the following examples of the letters A, B, and C. What does that suggest about the difference between a letter and how it is written (a glyph)?
  \vspace{5pt}
  \begin{center}
    {\Large ABC} \quad {\large abc} \quad {\Large\textit{ABC}}
  \end{center}
  \begin{center}
    \includegraphics[width=3in]{figures/hand.png}
  \end{center}
  \begin{answer}[1.5in]
    The idea or meaning of a letter is different from how it is transmitted. The letter 'A' can be drawn or gestured in a variety of ways, but all have the same meaning.
  \end{answer}