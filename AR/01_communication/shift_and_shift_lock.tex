  \model{Shift and Shift Lock}

  One way to reduce the cost of a keyboard is to reduce the number of keys. Early typewriters often left out the digit 1 (see image). Even today's modern computer keyboards (with over 100 keys!) do not have separate keys for uppercase and lowercase letters.

  \vspace{10pt}
  \begin{center}
    \includegraphics[width=0.5\textwidth]{figures/typewriter.png}
  \end{center}

  {\it\large Refer to Model 5 above as your team develops consensus answers
  to the questions below.}

  \quest{5 min}

  \Q How would someone type (encode) the digit 1 (one) on a typewriter missing that key?
    \begin{answer}[1in]
      A lowercase L (l) or uppercase I.
    \end{answer}

  \Q How would someone read (decode) the typed page and know that the digit one was intended instead of letters or something else?
    \begin{answer}[1in]
      Context
    \end{answer}

  \Q If we think of the keys as codes that tell the typewriter what to print on the paper, how do we type (encode) the '\$' symbol (shared with the '4' key)? How many codes (keys) are used?
    \begin{answer}[1in]
      Two codes (keystrokes): <shift> + <4>
    \end{answer}

  \Q If we have a continuous sequence of alternate values (say, four uppercase in a row),\key\\[-2.5mm] what option do we have to reduce the number of codes (key presses)?
    \begin{answer}[1in]
      Shift Lock
    \end{answer}