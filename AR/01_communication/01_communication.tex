% Source: CPTR 280 Computer Organization and Assembly Language Fall 2020
% File: "01 Communication (key).pdf"
% Author: James Foster, pogil@jgfoster.net

% comment out for student version
% \ifdefined\Student\relax\else\def\Teacher{}\fi

\documentclass[12pt]{article}

\title{Activity 1: Communication: Codes and Combinations}
\author{James Foster}
\newcommand{\activityeditor}{James Foster}
\newcommand{\activitysource}{\url{pogil@jgfoster.net}}
\date{Fall 2020}

\input{../../cspogil.sty}
\usepackage{graphicx}
\usepackage{tabularx}

\begin{document}

  \begin{center}
    \maketitle
    \rolenames
  \end{center}

  \keyquestions{
  \item Model 1, Question \#5
  \item Model 3, Question \#12
  \item Model 4, Question \#19
  \item Model 5, Question \#25
  }

  \newpage
  \maketitle

  We communicate both data and instructions to computers. This activity will explore how we encode information and what constraints the media used places on the information conveyed.
  
  \guide{
    \item Calculate the number of values that can be encoded in a number of bits; and,
    \item Identify ways to increase the information encoded in a number of bits.
   }{
    \item Use a translation table to encode and decode Morse Code and Braille (information processing).
   }{
    No additional notes
   }

  \model{Communication}
  
  \begin{center}
    \includegraphics[width=0.5\textwidth]{figures/schramm_model.png}
    \par\vspace{5pt}
    {\small Figure 1: Schramm's Model of Communication}
  \end{center}

  \vspace{10pt}
  
  A model of communication by Wilbur Schramm includes the following elements:
  \begin{itemize}
    \item Messenger
    \item Encoding
    \item Channel
    \item Decoding
    \item Recipient
  \end{itemize}

  {\it\large Refer to Model 1 above as your team develops consensus answers
  to the questions below.}

  \quest{10 min}

  \Q In Schramm's model, what elements are between the messenger and the recipient?
    \begin{answer}[0.5in]
      Encoding, Channel, and Decoding
    \end{answer}

  \Q What channels are specified in the above image?
    \hfill\ans{ Air waves (sound), light waves}

  \Q According to the image, what is carried on the channel?
    \hfill\ans{Codes}

  \Q With verbal communication, what are the codes?
    \hfill\ans{Spoken words (syllables, phonemes)}

  \newpage

  \Q Based on the model (and your personal experience), with in-person, face-to-face\key\\[-2.5mm] communication, what channel(s) are used?
    \begin{answer}[1in]
      Sound, sight (body language), touch, smell
    \end{answer}

  \Q In communication, we often find that the message doesn't get through or is misunderstood. Using this model, what are possible causes for this situation?
    \begin{answer}[1in]
      Poor encoding, changes to codes on the channel, poor decoding
    \end{answer}
  \newpage
    \model{Writing}

  One limitation in communication is when the recipient is distant in space or time. In that case you can write your message and deliver it to the recipient (or wait for them to find it). That writing needs to be encoded in a medium that the recipient can decode.

  \vspace{10pt}
  \begin{center}
    \includegraphics[width=0.4\textwidth]{figures/pictographs.png}
  \end{center}

  {\it\large Refer to Model 2 above as your team develops consensus answers
  to the questions below.}

  \quest{7 min}

  \Q What items are depicted in this picture?
    \begin{answer}[1in]
      Four-footed animal (deer?), two-footed animal (duck?), humans, an arch (rainbow? mountain?)
    \end{answer}

  \Q Pictograms are little schematic pictures of things, actions, or concepts, and form the basis of cuneiform and hieroglyphs. Ideograms are graphical symbols that represent ideas. Logogram is the name of the symbols that are used in writing Chinese, Japanese, and Korean.

  In what sense would it be easier for children to learn to read if dog were written as '\includegraphics[height=1em]{figures/dog_emoji.png}'?
    \begin{answer}[1.5in]
      The logogram is an immediate, pictorial representation of the meaning of the word. There is no need to learn a separate alphabet, phonetics, or spelling.
    \end{answer}

  \newpage

  \Q While there are over 100,000 Chinese characters, a literate person needs to know only\key\\[-2.5mm] 4,000--5,000 characters. Printing using movable type (each letter is a separate piece) was invented in China several hundred years before Gutenberg created his press in 15th century Europe. Why might it have caught on more quickly in Europe?
    \begin{answer}[1.5in]
      Creating, storing, and searching 100,000 unique pieces of movable type would be much more cumbersome than doing the same with up to 100 pieces.
    \end{answer}

  \Q Even when words are made up of letters, the letters themselves can have different presentations. Consider the following examples of the letters A, B, and C. What does that suggest about the difference between a letter and how it is written (a glyph)?
  \vspace{5pt}
  \begin{center}
    {\Large ABC} \quad {\large abc} \quad {\Large\textit{ABC}}
  \end{center}
  \begin{center}
    \includegraphics[width=3in]{figures/hand.png}
  \end{center}
  \begin{answer}[1.5in]
    The idea or meaning of a letter is different from how it is transmitted. The letter 'A' can be drawn or gestured in a variety of ways, but all have the same meaning.
  \end{answer}
  \newpage
    \model{Morse Code}

  As suggested by above with American Sign Language, there are ways of presenting letters that don't involve placing ink on paper. Morse Code transfers letters using combinations of short (dot) or long (dash) bursts of light, electricity, radio, or sound.

  \vspace{10pt}
  \begin{center}
  \begin{tabular}{ll|ll|ll}
    A & $\bullet$ -- & J & $\bullet$ --- & S & $\bullet\bullet\bullet$ \\
    B & --$\bullet\bullet\bullet$ & K & --$\bullet$-- & T & -- \\
    C & --$\bullet$-- & L & $\bullet$--$\bullet\bullet$ & U & $\bullet\bullet$-- \\
    D & --$\bullet\bullet$ & M & ---- & V & $\bullet\bullet\bullet$-- \\
    E & $\bullet$ & N & --$\bullet$ & W & $\bullet$---- \\
    F & $\bullet\bullet$--$\bullet$ & O & ----- & X & --$\bullet\bullet$-- \\
    G & ----$\bullet$ & P & $\bullet$----$\bullet$ & Y & --$\bullet$---- \\
    H & $\bullet\bullet\bullet\bullet$ & Q & ----$\bullet$-- & Z & ----$\bullet\bullet$ \\
    I & $\bullet\bullet$ & R & $\bullet$--$\bullet$ & & \\
  \end{tabular}
  \end{center}

  {\it\large Refer to Model 3 above as your team develops consensus answers
  to the questions below.}

  \quest{7 min}

  \Q Using the chart to the left, encode 'COMPUTER'.
    \begin{answer}[1in]
      --$\bullet$--$\bullet$ \quad ----- \quad $\bullet$----$\bullet$ \quad --$\bullet\bullet$ \quad $\bullet\bullet$-- \quad -- \quad $\bullet$ \quad $\bullet$--$\bullet$
    \end{answer}

  \Q Decode the following:
  \vspace{5pt}
  \begin{center}
    $\bullet\bullet\bullet$ \quad --$\bullet$--$\bullet$ \quad $\bullet\bullet\bullet\bullet$ \quad ----- \quad ----- \quad $\bullet$--$\bullet\bullet$
  \end{center}

    \begin{answer}[1in]
      S \quad C \quad H \quad O \quad O \quad L
    \end{answer}

  \Q Using the chart above, how many tests or comparisons would you have to make to find each of the letters in question 12 (how many rows would you examine starting your search at the top)?
    \begin{answer}[1in]
      19 \quad 3 \quad 8 \quad 15 \quad 15 \quad 12
    \end{answer}

  \Q Using the tree below, how many tests or comparisons would you have to make\key\\[-2.5mm] for each of the letters in question 12 (how many lines would you follow starting your search at the top)?
    \begin{answer}[1in]
      3 \quad 4 \quad 4 \quad 3 \quad 3 \quad 4
    \end{answer}

  \vspace{10pt}
  \begin{center}
    {\small Decode Morse Code by following dots to the left and dashes to the right.}
    \par\vspace{5pt}
    \includegraphics[width=0.9\textwidth]{figures/morse_tree.png}
  \end{center}
  \newpage
  \model{Braille}

  As already mentioned, there are ways of presenting letters that don't involve placing ink on paper. Braille uses raised dots on heavy paper (represented here as black dots since raised dots is more difficult!).

  \vspace{10pt}
  \begin{center}
    \includegraphics[width=0.3\textwidth]{figures/braille_chart.png}
  \end{center}

  {\it\large Refer to Model 4 above as your team develops consensus answers
  to the questions below.}

  \quest{10 min}

  \Q Using the information above, give the Latin letters equivalent to the following:
    \vspace{10pt}
    \begin{center}
      \includegraphics[width=0.3\textwidth]{figures/braille_code.png}
    \end{center}
    \begin{answer}[0.5in]
      C \quad O \quad L \quad L \quad E \quad G \quad E
    \end{answer}

  \Q Looking at the top right dot for B and C, how many possible unique values are there for one dot?
    \begin{answer}[0.5in]
      two (2)
    \end{answer}

  \Q Looking at the middle row for A, E, F, and G, how many possible unique combinations can two dots take?
    \begin{answer}[0.5in]
      four (4)
    \end{answer}

  \Q How many dots are possible in one Braille block (or cell)?
    \hfill\ans{six (6)}

  \Q How many unique combinations of dots can exist in one Braille cell?\key\\[-2.5mm]
    \begin{answer}[0.5in]
      $2^6 = 64$
    \end{answer}

  \Q How did you calculate that value?
    \begin{answer}[0.5in]
      combinations = $2^n$
    \end{answer}

  \Q Imagine an alphabet with two letters, O and X. How many words could exist in this encoding if the word size
    were the following values (the first two have been done for you)? What is the formula for this calculation?
    \vspace{10pt}
    \begin{center}
    \begin{tabular}{|c|c|c|c|c|c|c|c|c|c|}
      \hline
      1 & 2 & 3 & 4 & 5 & 6 & 7 & 8 & 9 & 10 \\
      \hline
      2 & 4 & 8 & \ans[0.2in]{16} & \ans[0.2in]{32} & \ans[0.2in]{64} & \ans[0.2in]{128} & \ans[0.2in]{256} & \ans[0.2in]{512} & \ans[0.3in]{1024} \\
      \hline
    \end{tabular}
    \end{center}

    \begin{answer}[0.5in]
      The formula is $2^n$ (as given in question 20).
    \end{answer}

  \newpage
    \model{Shift and Shift Lock}

  One way to reduce the cost of a keyboard is to reduce the number of keys. Early typewriters often left out the digit 1 (see image). Even today's modern computer keyboards (with over 100 keys!) do not have separate keys for uppercase and lowercase letters.

  \vspace{10pt}
  \begin{center}
    \includegraphics[width=0.5\textwidth]{figures/typewriter.png}
  \end{center}

  {\it\large Refer to Model 5 above as your team develops consensus answers
  to the questions below.}

  \quest{5 min}

  \Q How would someone type (encode) the digit 1 (one) on a typewriter missing that key?
    \begin{answer}[1in]
      A lowercase L (l) or uppercase I.
    \end{answer}

  \Q How would someone read (decode) the typed page and know that the digit one was intended instead of letters or something else?
    \begin{answer}[1in]
      Context
    \end{answer}

  \Q If we think of the keys as codes that tell the typewriter what to print on the paper, how do we type (encode) the '\$' symbol (shared with the '4' key)? How many codes (keys) are used?
    \begin{answer}[1in]
      Two codes (keystrokes): <shift> + <4>
    \end{answer}

  \Q If we have a continuous sequence of alternate values (say, four uppercase in a row),\key\\[-2.5mm] what option do we have to reduce the number of codes (key presses)?
    \begin{answer}[1in]
      Shift Lock
    \end{answer}

\end{document}