\model{Addition with Carry}

  \quest{10 min}

  \Q Complete the decimal addition problem in the table below, then answer the following questions:
    \vspace{10pt}
    \begin{center}
      \begin{tabular}{|c|c|c|}
        \hline
        & \ans[0.3in]{1} & \\
        \hline
        & 2 & 6 \\
        \hline
        + & 4 & 5 \\
        \hline
        & \ans[0.3in]{7} & \ans[0.3in]{1} \\
        \hline
      \end{tabular}
    \end{center}

    \begin{enumerate}
      \item What is the sum of 6 + 5 (answer after completing the table)?
        \hfill\ans[1in]{11}

      \item How many digits are in that value?
        \hfill\ans{2}

      \item Where did you put the least significant (right-most) digit?
        \begin{answer}[0.5in]
          {The ``ones'' go below as part of the sum.}
        \end{answer}

      \item Where did you put the most significant (left-most) digit?
        \begin{answer}[0.5in]
          {The ``ten'' goes above the next column of digits to be included in the next set of additions.}
        \end{answer}

      \item When you have a one that ``carries'' to the next column, how is it used?
        \begin{answer}[0.5in]
          It is just another number that is added to the other values in the column.
        \end{answer}

      \item For the expression (9 + 8 + 7), do you add all three numbers at once or do you first add two of them and then add the third to the intermediate sum?
        \begin{answer}[0.5in]
          I only add two digits at a time; the third one is added to the intermediate sum.
        \end{answer}

      \item When adding two N-digit numbers, what is the maximum number of digits for the sum?
        \begin{answer}[0.5in]
          N + 1
        \end{answer}
    \end{enumerate}

  \newpage

  \Q The binary addition table (with four totals) is much simpler than the decimal addition table (with 100 totals).
    \begin{enumerate}
      \item Complete the left table below (labeled ``total'') with a two-digit sum in each of the empty cells.
      \item Complete the ``sum'' table with the least-significant digit (right-most) from the first table.
      \item Complete the ``carry'' table with the most-significant digit (left-most) from the first table.
    \end{enumerate}

  \vspace{10pt}
  \begin{center}
  \begin{minipage}{0.3\textwidth}
    \centering
    \begin{tabular}{|c|c|c|}
      \hline
      \textbf{total} & $A = 0$ & $A = 1$ \\
      \hline
      $B = 0$ & \ans[0.3in]{00} & \ans[0.3in]{01} \\
      \hline
      $B = 1$ & \ans[0.3in]{01} & \ans[0.3in]{10} \\
      \hline
    \end{tabular}
  \end{minipage}
  \hfill
  \begin{minipage}{0.3\textwidth}
    \centering
    \begin{tabular}{|c|c|c|}
      \hline
      \textbf{sum} & $A = 0$ & $A = 1$ \\
      \hline
      $B = 0$ & \ans[0.3in]{0} & \ans[0.3in]{1} \\
      \hline
      $B = 1$ & \ans[0.3in]{1} & \ans[0.3in]{0} \\
      \hline
    \end{tabular}
  \end{minipage}
  \hfill
  \begin{minipage}{0.3\textwidth}
    \centering
    \begin{tabular}{|c|c|c|}
      \hline
      \textbf{carry} & $A = 0$ & $A = 1$ \\
      \hline
      $B = 0$ & \ans[0.3in]{0} & \ans[0.3in]{0} \\
      \hline
      $B = 1$ & \ans[0.3in]{0} & \ans[0.3in]{1} \\
      \hline
    \end{tabular}
  \end{minipage}
  \end{center}

  \Q The truth tables below were developed in an earlier exercise.\key\\[-2.5mm]
    \begin{enumerate}
      \item Which one matches the sum table?
        \hfill\ans{XOR}

      \item Which one matches the carry table?
        \hfill\ans{AND}
    \end{enumerate}

  \vspace{10pt}
  \begin{center}
    \begin{tabular}{|c|c|c||c|c|c||c|c||c|c|c||c|c|c||c|c|c|}
      \hline
      \textbf{AND} & 0 & 1 & \textbf{OR} & 0 & 1 & \textbf{NOT} & & \textbf{NAND} & 0 & 1 & \textbf{NOR} & 0 & 1 & \textbf{XOR} & 0 & 1 \\
      \hline
      $B = 0$ & 0 & 0 & $B = 0$ & 0 & 1 & $B = 0$ & 1 & $B = 0$ & 1 & 1 & $B = 0$ & 1 & 0 & $B = 0$ & 0 & 1 \\
      \hline
      $B = 1$ & 0 & 1 & $B = 1$ & 1 & 1 & $B = 1$ & 0 & $B = 1$ & 1 & 0 & $B = 1$ & 0 & 0 & $B = 1$ & 1 & 0 \\
      \hline
    \end{tabular}
  \end{center}