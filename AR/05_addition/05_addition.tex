% Source: CPTR 280 Computer Organization and Assembly Language Fall 2020
% File: "05 Addition (key).pdf"
% Author: James Foster, pogil@jgfoster.net

% comment out for student version
% \ifdefined\Student\relax\else\def\Teacher{}\fi

\documentclass[12pt]{article}

\title{Activity 5: Addition}
\author{James Foster}
\newcommand{\activityeditor}{James Foster}
\newcommand{\activitysource}{\url{pogil@jgfoster.net}}
\date{Fall 2020}

\input{../../cspogil.sty}
\usepackage{graphicx}
\usepackage{tabularx}

\begin{document}

  \begin{center}
    \maketitle
    \rolenames
  \end{center}

  \keyquestions{
  \item Model 1, Question \#3
  \item Model 3, Question \#8.g
  \item Model 4, Question \#9
  }

  \newpage
  \maketitle

  Just as number systems can be simplified to binary, most math operations can be simplified to addition. In this lesson we will build on earlier exercises dealing with binary and with digital logic gates to understand how computers do addition and subtraction.
  
  \guide{
    \item Describe the logic circuits used in adding binary numbers;
    \item Use nine's complement to subtract without borrowing; and,
    \item Represent negative numbers in binary using two's complement.
   }{
    \item Reflect on how teams can ensure all members participate.
   }{
    No additional notes
   }

  \model{Addition with Carry}

  \quest{10 min}

  \Q Complete the decimal addition problem in the table below, then answer the following questions:
    \vspace{10pt}
    \begin{center}
      \begin{tabular}{|c|c|c|}
        \hline
        & \ans[0.3in]{1} & \\
        \hline
        & 2 & 6 \\
        \hline
        + & 4 & 5 \\
        \hline
        & \ans[0.3in]{7} & \ans[0.3in]{1} \\
        \hline
      \end{tabular}
    \end{center}

    \begin{enumerate}
      \item What is the sum of 6 + 5 (answer after completing the table)?
        \hfill\ans[1in]{11}

      \item How many digits are in that value?
        \hfill\ans{2}

      \item Where did you put the least significant (right-most) digit?
        \begin{answer}[0.5in]
          {The ``ones'' go below as part of the sum.}
        \end{answer}

      \item Where did you put the most significant (left-most) digit?
        \begin{answer}[0.5in]
          {The ``ten'' goes above the next column of digits to be included in the next set of additions.}
        \end{answer}

      \item When you have a one that ``carries'' to the next column, how is it used?
        \begin{answer}[0.5in]
          It is just another number that is added to the other values in the column.
        \end{answer}

      \item For the expression (9 + 8 + 7), do you add all three numbers at once or do you first add two of them and then add the third to the intermediate sum?
        \begin{answer}[0.5in]
          I only add two digits at a time; the third one is added to the intermediate sum.
        \end{answer}

      \item When adding two N-digit numbers, what is the maximum number of digits for the sum?
        \begin{answer}[0.5in]
          N + 1
        \end{answer}
    \end{enumerate}

  \newpage

  \Q The binary addition table (with four totals) is much simpler than the decimal addition table (with 100 totals).
    \begin{enumerate}
      \item Complete the left table below (labeled ``total'') with a two-digit sum in each of the empty cells.
      \item Complete the ``sum'' table with the least-significant digit (right-most) from the first table.
      \item Complete the ``carry'' table with the most-significant digit (left-most) from the first table.
    \end{enumerate}

  \vspace{10pt}
  \begin{center}
  \begin{minipage}{0.3\textwidth}
    \centering
    \begin{tabular}{|c|c|c|}
      \hline
      \textbf{total} & $A = 0$ & $A = 1$ \\
      \hline
      $B = 0$ & \ans[0.3in]{00} & \ans[0.3in]{01} \\
      \hline
      $B = 1$ & \ans[0.3in]{01} & \ans[0.3in]{10} \\
      \hline
    \end{tabular}
  \end{minipage}
  \hfill
  \begin{minipage}{0.3\textwidth}
    \centering
    \begin{tabular}{|c|c|c|}
      \hline
      \textbf{sum} & $A = 0$ & $A = 1$ \\
      \hline
      $B = 0$ & \ans[0.3in]{0} & \ans[0.3in]{1} \\
      \hline
      $B = 1$ & \ans[0.3in]{1} & \ans[0.3in]{0} \\
      \hline
    \end{tabular}
  \end{minipage}
  \hfill
  \begin{minipage}{0.3\textwidth}
    \centering
    \begin{tabular}{|c|c|c|}
      \hline
      \textbf{carry} & $A = 0$ & $A = 1$ \\
      \hline
      $B = 0$ & \ans[0.3in]{0} & \ans[0.3in]{0} \\
      \hline
      $B = 1$ & \ans[0.3in]{0} & \ans[0.3in]{1} \\
      \hline
    \end{tabular}
  \end{minipage}
  \end{center}

  \Q The truth tables below were developed in an earlier exercise.\key\\[-2.5mm]
    \begin{enumerate}
      \item Which one matches the sum table?
        \hfill\ans{XOR}

      \item Which one matches the carry table?
        \hfill\ans{AND}
    \end{enumerate}

  \vspace{10pt}
  \begin{center}
    \begin{tabular}{|c|c|c||c|c|c||c|c||c|c|c||c|c|c||c|c|c|}
      \hline
      \textbf{AND} & 0 & 1 & \textbf{OR} & 0 & 1 & \textbf{NOT} & & \textbf{NAND} & 0 & 1 & \textbf{NOR} & 0 & 1 & \textbf{XOR} & 0 & 1 \\
      \hline
      $B = 0$ & 0 & 0 & $B = 0$ & 0 & 1 & $B = 0$ & 1 & $B = 0$ & 1 & 1 & $B = 0$ & 1 & 0 & $B = 0$ & 0 & 1 \\
      \hline
      $B = 1$ & 0 & 1 & $B = 1$ & 1 & 1 & $B = 1$ & 0 & $B = 1$ & 1 & 0 & $B = 1$ & 0 & 0 & $B = 1$ & 1 & 0 \\
      \hline
    \end{tabular}
  \end{center}
  \newpage
    \model{Digital Circuit for a Half Adder}

  \quest{10 min}

  \Q The following symbols are used to represent digital logic circuits that produce the truth tables described above.
    Based on your answer to question 3, label one of them sum and another one of them carry.
    \vspace{10pt}
    \begin{center}
      \includegraphics[width=0.9\textwidth]{figures/logic_gate_symbols.png}
      \par\vspace{5pt}
      {\small Figure 1: Logic Gate Symbols}
    \end{center}
    \begin{answer}[0.5in]
      CARRY\hfill SUM
    \end{answer}

  \Q Complete Figure 2 by identifying the output value for each of the gates given the input values for A and B:
    \vspace{10pt}
    \begin{center}
      \ifdefined\Teacher
        \includegraphics[width=0.9\textwidth]{figures/logic_gate_outputs_solution.png}
      \else
        \includegraphics[width=0.9\textwidth]{figures/logic_gate_outputs.png}
      \fi
      \par\vspace{5pt}
      {\small Figure 2: Logic Gate Outputs}
    \end{center}

  \Q Follow the instructions in (a) and (b) below to complete the circuit diagram in Figure 3 so that it can add two binary digits A and B:
    \begin{enumerate}
      \item Connect A and B to the appropriate input lines on each of the two gates. (Hint: see question 5.)
      \item Label the outputs sum and carry. (Hint: see question 4.)
    \end{enumerate}

  \vspace{10pt}
  \begin{center}
    \ifdefined\Teacher
      \includegraphics[width=0.3\textwidth]{figures/adding_two_binary_digits_solution.png}
    \else
      \includegraphics[width=0.3\textwidth]{figures/adding_two_binary_digits.png}
    \fi
    \par\vspace{5pt}
    {\small Figure 3: Adding Two Binary Digits}
  \end{center}

  \vspace{10pt}
  A circuit that has two inputs and a sum and carry output is called a Half Adder and has a special symbol (Figure 4).

  \vspace{10pt}
  \begin{center}
    \includegraphics[width=0.2\textwidth]{figures/half_adder.png}
    \par\vspace{5pt}
    {\small Figure 4: Half Adder}
  \end{center}

  \newpage
  \model{Digital Circuit for a Full Adder}

  \quest{15 min}

  \Q Add the binary numbers 110 + 011 in the table to the right (fill each of the non-shaded cells). (Hint: see question 2.)
    \vspace{10pt}
    \begin{center}
      \begin{tabular}{|c|c|c|c|c|c|}
        \hline
        \rowcolor{gray!25}
        \cellcolor{gray!25} & digit & 3 & 2 & 1 & 0 \\
        \hline
        \cellcolor{gray!25} & \cellcolor{gray!25}carry & \ans[0.3in]{1} & \ans[0.3in]{1} & \ans[0.3in]{0} & \ans[0.3in]{0} \\
        \hline
        \cellcolor{gray!25} & \cellcolor{gray!25}A & \cellcolor{gray!25}& 1 & 1 & 0 \\
        \hline
        \cellcolor{gray!25}+ & \cellcolor{gray!25}B & \cellcolor{gray!25} & 0 & 1 & 1 \\
        \hline
        \cellcolor{gray!25} & \cellcolor{gray!25}sum & \ans[0.3in]{1} & \ans[0.3in]{0} & \ans[0.3in]{0} & \ans[0.3in]{1} \\
        \hline
      \end{tabular}
    \end{center}

    \begin{enumerate}
      \item What is the value for carry for digit 2 in this problem (the third column from the right)? Will it always be that even for other values of A \& B?
        \begin{answer}[0.5in]
          1, no
        \end{answer}

      \item What is the value for carry for digit 1 in this problem? Will it always be that even for other values of A \& B?
        \begin{answer}[0.5in]
          0, no
        \end{answer}

      \item What is the value for carry for digit 0 in this problem? Will it always be that even for other values of A \& B?
        \begin{answer}[0.5in]
          0, yes
        \end{answer}

      \item If we wanted the circuit for each column to be the same (simplicity/consistency), how many inputs (digits or bits) should we allow for each column?
        \begin{answer}[0.5in]
          three
        \end{answer}

      \item How many outputs (digits or bits) should we allow for each column?
        \hfill\ans[0.6in]{two}

      \item When adding digits, does the order matter?
        \begin{answer}[0.75in]
          Overall, we need to go right to left, but within a column order doesn't matter.
        \end{answer}
    \end{enumerate}

  \Q Follow the instructions below to create a Full Adder in Figure 5.
    \vspace{10pt}
    \begin{center}
      \ifdefined\Teacher
        \includegraphics[width=0.3\textwidth]{figures/adding_three_binary_digits_solution.png}
      \else
        \includegraphics[width=0.3\textwidth]{figures/adding_three_binary_digits.png}
      \fi
      \par\vspace{5pt}
      {\small Figure 5: Adding Three Binary Digits}
    \end{center}

    \begin{enumerate}
      \item Label the three inputs (on the left of the figure) for carry, A, and B. (Hint: see question 7f.)

      \item How many outputs should we have? (Hint: see question 7e.)
        \hfill\ans[1in]{two}

      \item What should the outputs be named?
        \hfill\ans{sum \& carry}

      \item When adding two, one-bit values, will sum and carry ever both be one? (Hint: see question 2.)
        \begin{answer}[0.5in]
          no
        \end{answer}

      \item If C = 1 on the left Half Adder, what is the value of S on the left one?
        \hfill\ans[1in]{always 0}

      \item If S = 0 on the left Half Adder, what is the value of C on the right one?
        \hfill\ans[1in]{always 0}

      \item Will both C values from the Half Adders in Figure 5 ever both be one?\key\\[-2.5mm]
        \hfill\ans[1in]{no}

      \item Add the appropriate gate from Figure 1 on the right of Figure 5 to complete the circuit.

      \item Label the outputs. (Hint: try various inputs and see what outputs you get.)

    \end{enumerate}
  \newpage
  \model{Multi-Bit Adder}

  A circuit like the one built in question 8, that has three inputs and two outputs, a sum and carry, is called a Full Adder.
  It can be represented with either of the images in Figure 6. The difference between them is whether then inputs and outputs are
  on the left or right. While we have had inputs on the left so far, we will soon see how it may be convenient to have the inputs on the right.

  \vspace{10pt}
  \begin{center}
    \includegraphics[width=0.4\textwidth]{figures/full_adders.png}
    \par\vspace{5pt}
    {\small Figure 6: Full Adders}
  \end{center}

  {\it\large Refer to Model 4 above as your team develops consensus answers
  to the questions below.}

  \quest{15 min}

  \Q To add a pair of N-digit binary numbers, how many Full Adders do we need? (Hint:\key\\[-2.5mm] how many columns are we adding in question 7?)
    \begin{answer}[0.5in]
      N Full Adders, one for each column.
    \end{answer}

  \Q Follow the instructions below to make a 3-Bit Adder in Figure 7 to add A + B.
    \vspace{10pt}
    \begin{center}
      \ifdefined\Teacher
        \includegraphics[width=0.7\textwidth]{figures/3bit_adder_solution.png}
      \else
        \includegraphics[width=0.7\textwidth]{figures/3bit_adder.png}
      \fi
      \par\vspace{5pt}
      {\small Figure 7: 3-Bit Adder}
    \end{center}

    \begin{enumerate}
      \item Given that A is a three-bit number with the least-significant bit on the right (a0) and the most significant bit on the left (a2),
      draw lines connecting each digit of A to the appropriate Full Adder.

      \item Draw lines connecting each digit of B to the appropriate Full Adder.

      \item Draw lines connecting the S output of each Full Adder to the appropriate output (s0-s2).

      \item Draw lines connecting the C output of each Full Adder to the appropriate place.

      \item Draw a component for the input for Cin on the least-significant Full Adder (at the right). (Hint: what is the value needed and what circuit element provides that value?)
    \end{enumerate}


\end{document}