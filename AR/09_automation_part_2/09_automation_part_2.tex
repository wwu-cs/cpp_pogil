% Source: CPTR 280 Computer Organization and Assembly Language Fall 2023
% File: "09 Automation (part 2) (key).pdf"
% Author: James Foster, pogil@jgfoster.net

% comment out for student version
% \ifdefined\Student\relax\else\def\Teacher{}\fi

\documentclass[12pt]{article}

\title{Activity 9: Addressable Memory}
\author{James Foster}
\newcommand{\activityeditor}{James Foster}
\newcommand{\activitysource}{\url{pogil@jgfoster.net}}
\date{Fall 2023}

\input{../../cspogil.sty}
\usepackage{graphicx}
\usepackage{tabularx}

\begin{document}

  \begin{center}
    \maketitle
    \rolenames
  \end{center}

  \keyquestions{
  \item Model 1, Question \#5
  \item Model 2, Question \#11
  \item Model 3, Question \#20
  }

  \newpage
  \maketitle

  We have looked a simple programmable computer. Now we add hardware and instructions for storing the code and the data in the same memory module.
  
  \guides{
    \item Do 16-bit math with 8-bit memory.
   }{
    \item No additional process skills.
   }{
    No additional notes
   }

  \model{Load Instruction}

  \vspace{10pt}
  \begin{center}
    \includegraphics[width=0.7\textwidth]{figures/programmable_computer_part2.png}
    \par\vspace{5pt}
    {\small Figure 1: Programmable Computer}
  \end{center}

  \vspace{10pt}
  \begin{center}
    \begin{tabular}{|l|c|c|}
      \hline
      \textbf{Instruction} & \textbf{Hexadecimal Code} & \textbf{Binary Code} \\
      \hline
      Add (data to latch) & 0x00 & 0000 0000 \\
      \hline
      Clear (latch) & 0x01 & 0000 0001 \\
      \hline
      Subtract (data from latch) & 0x02 & 0000 0010 \\
      \hline
      Stop (the clock) & 0x04 & 0000 0100 \\
      \hline
      Store (latch to data) & 0x08 & 0000 1000 \\
      \hline
    \end{tabular}
  \end{center}

  Figure 1 (copied from the previous lesson) shows a simple programmable computer with the following characteristics:
  \begin{itemize}
    \item A counter with 16-bit output that will increment on each clock tick (it can be reset to zero with the Start control line),
    \item Separate addressable memory for code and data that each output an 8-bit value based on the same 16-bit address (the value of the data memory can be set to the current data latch value by the Store control line),
    \item An 8-bit adder that calculates the sum (or difference based on the Subtract control line) of the latch and the value from memory,
    \item A latch that captures the result of the adder on each clock tick (or is cleared based on the Clear sum control line), and
    \item A decoder that takes each 8-bit instruction and sets the control lines appropriately.
  \end{itemize}

  \newpage

  {\it\large Refer to Model 1 above as your team develops consensus answers
  to the questions below.}

  \quest{15 min}

  \Q In Figure 1, the data path into the latch comes from what component?
    \hfill\ans[1in]{Adder}

  \Q In Figure 1, what two instructions are required to get the value of a single memory location into the latch?
    (Hint: what would you add to a value to get the same value?)
    \begin{answer}[0.5in]
      Clear, Add
    \end{answer}

  Figure 2 adds a new control line and a new component to our computer.

  \vspace{10pt}
  \begin{center}
    \includegraphics[width=0.5\textwidth]{figures/load_instruction.png}
    \par\vspace{5pt}
    {\small Figure 2: Load Instruction}
  \end{center}

  \Q What are the new elements?
    \hfill\ans{A 2-to-1 Selector and a Selection control line}

  \vspace{10pt}
  A 2-to-1 Selector will pass on one of two inputs based on a control line. In this configuration, if the control line is 0,
  then the value from the adder will be passed to the latch. If the control line is 1, then the value from memory will be passed to the latch.

  \Q How does this circuit simplify loading a value from memory into the latch?
    \begin{answer}[0.5in]
      Now it takes only one instruction.
    \end{answer}

  \Q Suggest a name and a code for this instruction. (Hint: see the name of Figure 2\key\\[-2.5mm] and the pattern of hexadecimal codes used for previous instructions.)
    \begin{answer}[0.5in]
      Load (0x10)
    \end{answer}
  \newpage
  \model{Addressable Memory}

  The latch from Model D has two inputs for each bit, a write line and a data line. If you were going to build many such latches
  (think millions or billions), it would be nice to have a way to let them share the write and data lines and do the selection with fewer lines.
  For example, a thousand separate locations could be identified with only 10 lines and a billion locations could be identified with 30 lines.
  Think of each location as having a unique integer address.

  {\it\large Refer to Model 5 above as your team develops consensus answers
  to the questions below.}

  \quest{10 min}

  \Q Consider the following 3-to-8 decoder. It has eight latches below (not shown), but only a single data in line and write line. A three-bit address is added on the left.
    \vspace{10pt}
    \begin{center}
      \includegraphics[width=0.6\textwidth]{figures/decoder_fig13.png}
      \par\vspace{5pt}
      {\small Figure 13}
    \end{center}
    \begin{enumerate}
      \item If the write line is 1 and each of the address lines (A0, A1, and A2) are 0, what are the output values for the 4-input AND gates labeled 7 to 0?
        \vspace{10pt}
        \begin{center}
          \begin{tabular}{|c|c|c|c|c|c|c|c|}
            \hline
            7 & 6 & 5 & 4 & 3 & 2 & 1 & 0 \\
            \hline
            \ans[0.2in]{0} & \ans[0.2in]{0} & \ans[0.2in]{0} & \ans[0.2in]{0} & \ans[0.2in]{0} & \ans[0.2in]{0} & \ans[0.2in]{0} & \ans[0.2in]{1} \\
            \hline
          \end{tabular}
        \end{center}

      \item If the write line is 1 and each of the address lines are 1, what are the values for the 4-input AND gates labeled 7 to 0?
        \vspace{10pt}
        \begin{center}
          \begin{tabular}{|c|c|c|c|c|c|c|c|}
            \hline
            7 & 6 & 5 & 4 & 3 & 2 & 1 & 0 \\
            \hline
            \ans[0.2in]{1} & \ans[0.2in]{0} & \ans[0.2in]{0} & \ans[0.2in]{0} & \ans[0.2in]{0} & \ans[0.2in]{0} & \ans[0.2in]{0} & \ans[0.2in]{0} \\
            \hline
          \end{tabular}
        \end{center}

      \item Generalize the relationship between the address lines and the AND gates.
        \begin{answer}[0.75in]
          The address lines make up a binary number equal to the AND gate number.
        \end{answer}
    \end{enumerate}

  \Q Just like it would be nice to have a single data-in line for many latches, it would be nice to have a single data-out line.
    An 8-input OR gate would have a 1 output if any of the latches had a 1 output. How could you modify Figure 13 to select which latch to read?
    (After thinking about it, see Figure 14 for a hint!)
    \begin{answer}[1.5in]
      Have each 4-input AND gate receive three inputs from the address lines (as before) but have the fourth input for each be the output of the respective flip-flop.
      So the output of the AND gate would be 1 if the address selected that gate and the flip-flop had a 1 as well.
    \end{answer}

  \vspace{10pt}
  \begin{center}
    \includegraphics[width=0.7\textwidth]{figures/addressable_memory_fig14.png}
    \par\vspace{5pt}
    {\small Figure 14}
  \end{center}

  \vspace{10pt}
  Figure 14 shows an addressable 8-bit array of memory. The address control lines specify which bit to read or write. There is a single data-in line that can be used to store to memory (if the write control line is enabled) and read from memory.
  Instead of storing eight bits these circuits could be stacked eight high to store eight bytes. The address and write control lines would be shared, and the data lines would be unique for each bit of the byte (so 20 lines total).
  Instead of having only eight locations (with a three-bit address), these circuits could be expanded to have (say) a 16-bit address and 65,536 locations (64 KiB of RAM).
  We now have addressable memory!

  \newpage
  \model{Multi-Byte Numbers}

  {\it\large Refer to Model 3 above as your team develops consensus answers
  to the questions below.}

  \quest{20 min}

  \Q Consider the expression with two unsigned integers: (1100 0001$_2$ + 0100 0011$_2$).
    \begin{itemize}
      \item What is the binary sum?
        \hfill\ans{1 0000 0100}
      \item How many bits are required for the answer?
        \hfill\ans{9}
    \end{itemize}

  \Q Consider the 8-Bit Adder shown in Figure 4.
    \vspace{10pt}
    \begin{center}
      \includegraphics[width=0.3\textwidth]{figures/8bit_adder.png}
      \par\vspace{5pt}
      {\small Figure 4: 8-Bit Adder}
    \end{center}
    \begin{itemize}
      \item What are the data path widths (number of bits) for A, B, and Sum?
        \hfill\ans[1.5in]{8, 8, and 8}

      \item What are the data path widths for Carry In and Carry Out?
        \hfill\ans[1in]{1 and 1}

      \item How is this 8-Bit Adder different from the one shown in Figure 1?
        \begin{answer}[0.75in]
          It has a carry in and carry out.
        \end{answer}

      \item If you were to use this 8-Bit Adder in the circuit shown in Figure 1, what value would you need for Carry In?
        What circuit component (from an earlier lesson) would you attach to the Carry In line to do proper math?
        \begin{answer}[1in]
          Carry in should always be zero; this can be achieved with ground.
        \end{answer}

      \item What component would you add to the Carry Out line to ignore the overflow bit?
        \begin{answer}[0.75in]
          Carry out could be attached to ground.
        \end{answer}

      \item If the carry lines are ignored for simple 8-bit addition, why do they exist?
        \begin{answer}[0.75in]
          So that we can do complex addition (say, 16-bit values).
        \end{answer}
    \end{itemize}
  
  \vspace{-40pt}

  \Q Consider the expression with two unsigned hexadecimal integers: (2345$_{16}$ + 3456$_{16}$)\key\\[-2.5mm]
    \begin{itemize}
      \item How many bits are required for each number?
        \begin{answer}[0.5in]
          4 bits for each digit, so 16-bits for each number.
        \end{answer}

      \item How many bits are required to represent the sum?
        \begin{answer}[0.5in]
          The value 579B$_{16}$ is also just 16 bits.
        \end{answer}

      \item Describe how you would use an 8-Bit Adder to perform this operation.
        \begin{answer}[1.5in]
          Add the low-order bytes: 45$_{16}$ + 56$_{16}$ = 9B$_{16}$ \\
          Add the high-order bytes: 23$_{16}$ + 34$_{16}$ = 57$_{16}$ \\
          Combine the high-order and low-order bytes to get the result of 579B$_{16}$
        \end{answer}
    \end{itemize}

  \Q Consider the expression with two unsigned hexadecimal integers: (2385$_{16}$ + 3496$_{16}$).
    \begin{itemize}
      \item How would you use the Carry In line to add the low-order 8 bits (85$_{16}$ + 96$_{16}$)?
        \begin{answer}[0.75in]
          The carry-in value should be zero.
        \end{answer}

      \item How would you use the Carry In line to add the high-order 8 bits?
        \begin{answer}[1in]
          The carry-in value for the high-order addition should be the (saved) carry out for the low-order addition.
        \end{answer}

      \item Suggest a name for this new kind of Add command.
        \begin{answer}[0.75in]
          Add with carry.
        \end{answer}
    \end{itemize}

  \Q What does the circuit in Figure 1 use to hold intermediate 8-bit sums?
    \begin{answer}[0.75in]
      An 8-bit latch.
    \end{answer}

  \Q What has been added in Figure 5 compared to Figure 4?
    \vspace{10pt}
    \begin{center}
      \includegraphics[width=0.35\textwidth]{figures/add_with_carry.png}
      \par\vspace{5pt}
      {\small Figure 5: Add with Carry}
    \end{center}

    \begin{answer}[0.75in]
      A Use Carry control line and a carry buffer or latch.
    \end{answer}

  \Q If we used 0x20 as the code for the Add with Carry instruction, how would it differ from the Add instruction code?
    \begin{answer}[1in]
      It has a 1 in bit 5 (the 6th bit from the right), while all other are 0.
    \end{answer}

  \Q If we attached bit 5 of the instruction decoder to the ``Use Carry'' control line, what logic gate(s) would provide the desired behavior?
    \begin{answer}[1in]
      An AND gate would have a 1 out if both the carry latch and the use carry control line are 1.
    \end{answer}

\end{document}