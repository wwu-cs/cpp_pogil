\model{Multi-Byte Numbers}

  {\it\large Refer to Model 3 above as your team develops consensus answers
  to the questions below.}

  \quest{20 min}

  \Q Consider the expression with two unsigned integers: (1100 0001$_2$ + 0100 0011$_2$).
    \begin{itemize}
      \item What is the binary sum?
        \hfill\ans{1 0000 0100}
      \item How many bits are required for the answer?
        \hfill\ans{9}
    \end{itemize}

  \Q Consider the 8-Bit Adder shown in Figure 4.
    \vspace{10pt}
    \begin{center}
      \includegraphics[width=0.3\textwidth]{figures/8bit_adder.png}
      \par\vspace{5pt}
      {\small Figure 4: 8-Bit Adder}
    \end{center}
    \begin{itemize}
      \item What are the data path widths (number of bits) for A, B, and Sum?
        \hfill\ans[1.5in]{8, 8, and 8}

      \item What are the data path widths for Carry In and Carry Out?
        \hfill\ans[1in]{1 and 1}

      \item How is this 8-Bit Adder different from the one shown in Figure 1?
        \begin{answer}[0.75in]
          It has a carry in and carry out.
        \end{answer}

      \item If you were to use this 8-Bit Adder in the circuit shown in Figure 1, what value would you need for Carry In?
        What circuit component (from an earlier lesson) would you attach to the Carry In line to do proper math?
        \begin{answer}[1in]
          Carry in should always be zero; this can be achieved with ground.
        \end{answer}

      \item What component would you add to the Carry Out line to ignore the overflow bit?
        \begin{answer}[0.75in]
          Carry out could be attached to ground.
        \end{answer}

      \item If the carry lines are ignored for simple 8-bit addition, why do they exist?
        \begin{answer}[0.75in]
          So that we can do complex addition (say, 16-bit values).
        \end{answer}
    \end{itemize}
  
  \vspace{-40pt}

  \Q Consider the expression with two unsigned hexadecimal integers: (2345$_{16}$ + 3456$_{16}$)\key\\[-2.5mm]
    \begin{itemize}
      \item How many bits are required for each number?
        \begin{answer}[0.5in]
          4 bits for each digit, so 16-bits for each number.
        \end{answer}

      \item How many bits are required to represent the sum?
        \begin{answer}[0.5in]
          The value 579B$_{16}$ is also just 16 bits.
        \end{answer}

      \item Describe how you would use an 8-Bit Adder to perform this operation.
        \begin{answer}[1.5in]
          Add the low-order bytes: 45$_{16}$ + 56$_{16}$ = 9B$_{16}$ \\
          Add the high-order bytes: 23$_{16}$ + 34$_{16}$ = 57$_{16}$ \\
          Combine the high-order and low-order bytes to get the result of 579B$_{16}$
        \end{answer}
    \end{itemize}

  \Q Consider the expression with two unsigned hexadecimal integers: (2385$_{16}$ + 3496$_{16}$).
    \begin{itemize}
      \item How would you use the Carry In line to add the low-order 8 bits (85$_{16}$ + 96$_{16}$)?
        \begin{answer}[0.75in]
          The carry-in value should be zero.
        \end{answer}

      \item How would you use the Carry In line to add the high-order 8 bits?
        \begin{answer}[1in]
          The carry-in value for the high-order addition should be the (saved) carry out for the low-order addition.
        \end{answer}

      \item Suggest a name for this new kind of Add command.
        \begin{answer}[0.75in]
          Add with carry.
        \end{answer}
    \end{itemize}

  \Q What does the circuit in Figure 1 use to hold intermediate 8-bit sums?
    \begin{answer}[0.75in]
      An 8-bit latch.
    \end{answer}

  \Q What has been added in Figure 5 compared to Figure 4?
    \vspace{10pt}
    \begin{center}
      \includegraphics[width=0.35\textwidth]{figures/add_with_carry.png}
      \par\vspace{5pt}
      {\small Figure 5: Add with Carry}
    \end{center}

    \begin{answer}[0.75in]
      A Use Carry control line and a carry buffer or latch.
    \end{answer}

  \Q If we used 0x20 as the code for the Add with Carry instruction, how would it differ from the Add instruction code?
    \begin{answer}[1in]
      It has a 1 in bit 5 (the 6th bit from the right), while all other are 0.
    \end{answer}

  \Q If we attached bit 5 of the instruction decoder to the ``Use Carry'' control line, what logic gate(s) would provide the desired behavior?
    \begin{answer}[1in]
      An AND gate would have a 1 out if both the carry latch and the use carry control line are 1.
    \end{answer}