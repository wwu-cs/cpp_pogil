% Source: CPTR 280 Computer Organization and Assembly Language Fall 2023
% File: "08 Automation (part 1) (key).pdf"
% Author: James Foster, pogil@jgfoster.net

% comment out for student version
% \ifdefined\Student\relax\else\def\Teacher{}\fi

\documentclass[12pt]{article}

\title{Activity 8: Automation}
\author{James Foster}
\newcommand{\activityeditor}{James Foster}
\newcommand{\activitysource}{\url{pogil@jgfoster.net}}
\date{Fall 2023}

\input{../../cspogil.sty}
\usepackage{graphicx}
\usepackage{tabularx}

\begin{document}

  \begin{center}
    \maketitle
    \rolenames
  \end{center}

  \keyquestions{
  \item Model 1, Question \#13
  \item Model 2, Question \#18
  }

  \newpage
  \maketitle

  We have looked at basic logic gates and combined them into a circuit that can do 8-bit addition. Now we look at a programmable computer.
  
  \guides{
    \item given a simple instruction set, read, and write a program using machine code.
   }{
    \item No additional process skills.
   }{
    No additional notes
   }{

   }

  \model{Infinite Addition}

  \vspace{10pt}
  \begin{center}
    \includegraphics[width=0.7\textwidth]{figures/simple_addition.png}
    \par\vspace{5pt}
    {\small Figure 1: Simple Addition}
  \end{center}

  With digital logic circuits we can build various components:
  \begin{itemize}
    \item An oscillator with output that switches rapidly between 0 and 1;
    \item A counter with 16-bit output that can be reset to all zeros and will increment on each ``clock tick'' (the oscillator has a change from 0 to 1);
    \item Addressable memory that outputs an 8-bit value based on a 16-bit address (a separate control panel with various switches is used to manipulate memory and is not shown);
    \item An adder that takes as input two 8-bit values and gives as output an 8-bit sum (carry in and out is ignored for now);
    \item A latch that captures an input value at a clock tick (and can be reset); and,
    \item Lights to show the current latch value.
  \end{itemize}

  {\it\large Refer to Model 1 above as your team develops consensus answers
  to the questions below.}

  \quest{25 min}

  \Q What is the size of the data path out of RAM?
    \hfill\ans{8 bits}

  \Q What are the sizes of the data paths into and out of the Adder and Latch?
    \hfill\ans[1in]{8 bits}

  \Q What is the size (number of bits) of the address path into RAM?
    \hfill\ans[1.5in]{16 bits}

  \Q The range of values for a decimal digit is [0, 9]. What is the range of values for a hexadecimal digit?
    \begin{answer}[0.5in]
      [0, F] (yes, you could say [0, 15] but when we get to question 14 you should think of `F')
    \end{answer}

  \Q How many bits are required to represent the full range of a single hexadecimal digit? (Hint: not 16!)
    \begin{answer}[0.5in]
      4 bits
    \end{answer}

  \Q How many hexadecimal digits can be represented on the address path into RAM?
    \begin{answer}[0.5in]
      4 digits
    \end{answer}

  \Q What happens to the Counter when you close the ``Clear'' switch?
    \begin{answer}[0.5in]
      The counter will be reset to all zeros.
    \end{answer}

  \Q What happens to the Latch when you close the ``Clear'' switch?
    \begin{answer}[0.5in]
      The latch will be reset to all zeros.
    \end{answer}

  \Q What happens to the Counter on a 0-to-1 clock tick?
    \begin{answer}[0.5in]
      The counter will increment by one.
    \end{answer}

  \Q What happens to the Latch on a 1-to-0 clock tick?
    \begin{answer}[0.5in]
      The latch will capture the output of the adder.
    \end{answer}

  \Q Describe the value on the data line coming out of RAM when the value on the address line into RAM is all zero bits (you might not be able to give an exact value).
    \begin{answer}[0.5in]
      The contents of location 0x0000 in RAM.
    \end{answer}

  \Q Describe the value of the Counter
    \begin{itemize}
      \item after a clear,
        \hfill\ans[1in]{0}
      \item after one clock tick,
        \hfill\ans[1in]{1}
      \item and after N clock ticks.
        \hfill\ans[1in]{N}
    \end{itemize}

  \Q Describe the value of the Latch\key\\[-2.5mm]
    \begin{itemize}
      \item after a clear,
        \hfill\ans[1in]{0}

      \item after one clock tick (hint: probably not 0 or 1),
        \hfill\ans{The value of memory location 0x0000 (the sum of that and zero).}
        
      \item And after N clock ticks (hint: probably not N).
        \hfill\ans{The sum of the first N memory locations.}
    \end{itemize}

  \Q What is the maximum value of the Counter in hexadecimal (the convention for addresses)? (Hint: see question 6.)
    \begin{answer}[0.5in]
      0xFFFF
    \end{answer}

  \Q If the Counter is at the maximum value, what is the value after a clock tick?
    \begin{answer}[0.5in]
      0
    \end{answer}

  \Q Does circuit ever stop? If, so, how? If not, why not?
    \begin{answer}[0.5in]
      No, the oscillator keeps running and incrementing the counter (which eventually wraps).
    \end{answer}
  \newpage
  \model{A Programmable Computer}

  \vspace{10pt}
  \begin{center}
    \includegraphics[width=0.85\textwidth]{figures/programmable_computer.png}
    \par\vspace{5pt}
    {\small Figure 2: Programmable Computer}
  \end{center}

  Early computers were hard-wired to perform certain tasks, and this worked okay for well-defined tasks such as calculating logarithms or trigonometry functions. But building a new circuit for each new task is cumbersome and computer engineers looked for a more flexible approach.

  The circuit shown above demonstrates some enhancements. It has:
  \begin{itemize}
    \item a new memory module for code (having code in a separate module is less common now),
    \item a decoder to interpret the instructions and set the appropriate control lines,
    \item a control line to specify whether to read or write (store) memory,
    \item a control line to stop the clock,
    \item a control line to subtract instead of add, and
    \item a separate control line to clear the latch (independent of the memory counter).
  \end{itemize}

  \newpage

  {\it\large Refer to Model 2 above as your team develops consensus answers
  to the questions below.}

  \quest{25 min}

  \Q The following table shows the five instructions supported by our computer. Convert the instruction codes from hexadecimal to binary and then identify what each bit means.

  \vspace{10pt}
  \begin{center}
    \begin{tabular}{|l|c|c|}
      \hline
      \textbf{Instruction} & \textbf{Hexadecimal Code} & \textbf{Binary Code} \\
      \hline
      Add (data to latch) & 0x00 & 0000 0000 \\
      \hline
      Clear (latch) & 0x01 & \ans[1in]{0000 0001} \\
      \hline
      Subtract (data from latch) & 0x02 & \ans[1in]{0000 0010} \\
      \hline
      Stop (the clock) & 0x04 & \ans[1in]{0000 0100} \\
      \hline
      Store (latch to data) & 0x08 & \ans[1in]{0000 1000} \\
      \hline
    \end{tabular}
  \end{center}

  \begin{center}
    \begin{tabular}{|c|c|c|c|c|}
      \hline
      \textbf{Bit} & 3 & 2 & 1 & 0 \\
      \hline
      \textbf{Meaning} & \ans[0.6in]{Store} & \ans[0.6in]{Stop} & \ans[0.6in]{Subtract} & Clear \\
      \hline
    \end{tabular}
  \end{center}

  \vspace{5pt}
  Note that bit 0 is the least significant (right-most) bit.

  \Q Describe the behavior of the decoder. How does it translate instruction codes\key\\[-2.5mm] to controls? (Hint: compare the binary code bits with the control lines coming from the decoder.)
    \begin{answer}[0.5in]
      It takes each bit from the instruction and puts it on a separate control line. Very simple!
    \end{answer}

  \Q What is the relationship between a code address and a data address?
    \begin{answer}[0.5in]
      Always the same.
    \end{answer}

  \Q If the instruction at memory address 0x100 in the memory module labeled Code is a Store, at what address in Data will it place a value?
    \begin{answer}[0.5in]
      Same: 0x100.
    \end{answer}

  \newpage

  \Q Using the appropriate control panels, a programmer loaded five bytes of code and two bytes of data into memory. The remaining portions of both memory are undefined. Complete the ``Meaning'' column with the name of the instruction then ``execute'' the code showing what value is in the appropriate memory location and the latch after each instruction. Unspecified values should be indicated with a question mark (?).
  \vspace{10pt}
  \begin{center}
    \begin{tabular}{|c|c|l|c|c|}
      \hline
      \textbf{Address} & \textbf{Code} & \textbf{Meaning} & \textbf{Data} & \textbf{Latch} \\
      \hline
      0x00 & 0x01 & \ans[1in]{Clear} & \ans[0.4in]{?} & \ans[0.4in]{0} \\
      \hline
      0x01 & 0x00 & \ans[1in]{Add} & 5 & \ans[0.4in]{5} \\
      \hline
      0x02 & 0x02 & \ans[1in]{Subtract} & 3 & \ans[0.4in]{2} \\
      \hline
      0x04 & 0x08 & \ans[1in]{Store} & \ans[0.4in]{2} & \ans[0.4in]{2} \\
      \hline
      0x08 & 0x04 & \ans[1in]{Stop} & \ans[0.4in]{?} & \ans[0.4in]{2} \\
      \hline
    \end{tabular}
  \end{center}

  \Q Write a program to calculate and store (11 + 6 -- 9) and then calculate and store (40 + 17). You need not use all memory.

  \vspace{10pt}
  \begin{center}
    \begin{tabular}{|c|c|l|c|c|}
      \hline
      \textbf{Address} & \textbf{Code} & \textbf{Meaning} & \textbf{Data} & \textbf{Latch} \\
      \hline
      0x00 & \ans[0.4in]{0x01} & \ans[1in]{Clear} & \ans[0.4in]{?} & \ans[0.4in]{0} \\
      \hline
      0x01 & \ans[0.4in]{0x00} & \ans[1in]{Add} & \ans[0.4in]{11} & \ans[0.4in]{11} \\
      \hline
      0x02 & \ans[0.4in]{0x00} & \ans[1in]{Add} & \ans[0.4in]{6} & \ans[0.4in]{17} \\
      \hline
      0x03 & \ans[0.4in]{0x02} & \ans[1in]{Subtract} & \ans[0.4in]{9} & \ans[0.4in]{8} \\
      \hline
      0x04 & \ans[0.4in]{0x08} & \ans[1in]{Store} & \ans[0.4in]{8} & \ans[0.4in]{8} \\
      \hline
      0x05 & \ans[0.4in]{0x01} & \ans[1in]{Clear} & \ans[0.4in]{?} & \ans[0.4in]{0} \\
      \hline
      0x06 & \ans[0.4in]{0x00} & \ans[1in]{Add} & \ans[0.4in]{40} & \ans[0.4in]{40} \\
      \hline
      0x07 & \ans[0.4in]{0x00} & \ans[1in]{Add} & \ans[0.4in]{17} & \ans[0.4in]{57} \\
      \hline
      0x08 & \ans[0.4in]{0x08} & \ans[1in]{Store} & \ans[0.4in]{57} & \ans[0.4in]{57} \\
      \hline
      0x09 & \ans[0.4in]{0x04} & \ans[1in]{Stop} & \ans[0.4in]{?} & \ans[0.4in]{57} \\
      \hline
      0x0A & \ans[0.4in]{?} & \ans[1in]{?} & \ans[0.4in]{?} & \\
      \hline
      0x0B & \ans[0.4in]{?} & \ans[1in]{?} & \ans[0.4in]{?} & \\
      \hline
    \end{tabular}
  \end{center}


\end{document}