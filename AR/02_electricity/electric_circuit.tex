\model{Electric Circuit}

  Water flowing through pipes is often used as an (imperfect) analogy for electricity (electrons flowing through wires). The circuit diagram in Figure 2 has the following components:
  \begin{itemize}
    \item Battery (left)
    \item Switch (top)
    \item Resistor (right)
    \item Wires (to connect each element)
  \end{itemize}

  In practice, the resistor may be a light or a heating element. Portions of the circuit with pressure are shown in green.

  \vspace{10pt}
  \begin{center}
    \includegraphics[width=0.3\textwidth]{figures/circuit_open_switch.png}
    \par\vspace{5pt}
    {\small Figure 2: Circuit with open switch}
  \end{center}

  {\it\large Refer to Model 2 above as your team develops consensus answers
  to the questions below.}

  \quest{14 min}

  \Q For each of the above components, what is the analogous item in a hydraulic\key\\[-2.5mm] system (you may write it below or next to the component above)?
    \begin{answer}[1in]
      Battery: Pump; Switch: Faucet; Resistor: Nozzle; Wires: Pipes
    \end{answer}

  \Q In the diagram shown above, the switch connecting the battery to the resistor is open and so electricity cannot flow. How does that compare with how you would describe a valve in a hydraulic system? That is, is the faucet open or closed when the water is moving?
    \begin{answer}[1in]
      When a faucet is open then water can flow.
    \end{answer}

  \Q Even with the switch open, are the electrons between the battery and the switch under pressure?
    \begin{answer}[0.75in]
      Yes, there is still pressure (shown in green) when the switch is open.
    \end{answer}

  \Q With the switch closed (see the diagram below), is there pressure between the switch and the resistor?
  \vspace{10pt}
  \begin{center}
    \includegraphics[width=0.3\textwidth]{figures/circuit_closed_switch.png}
    \par\vspace{5pt}
    {\small Figure 3: Circuit with closed switch}
  \end{center}
    \begin{answer}[1in]
      Yes, there is pressure between the switch and the resistor.
    \end{answer}

  \Q Is there pressure between the resistor and the battery along the bottom wire? Does it make a difference if the switch is open or closed?
    \begin{answer}[1in]
      No, there is no (significant) pressure between the resistor and the battery along the bottom wire. This is the case if the switch is open or closed.
    \end{answer}

  \Q An analogy for the circuit in Figure 3 might be a fish tank where water is taken out, run through a filter, and pumped back in. Would the size of the tank impact the amount of water flowing through the filter?
    \begin{answer}[1in]
      No, the size of the tank does not impact the amount of water flowing through the filter.
    \end{answer}

  The new symbol at the bottom of Figure 4 shows an infinite ``reservoir'' of electrons available (and not under pressure) to which the circuit can dump excess electrons or acquire needed electrons to push through the circuit. This new symbol is called ground (or earth).
  \vspace{10pt}
  \begin{center}
    \includegraphics[width=0.3\textwidth]{figures/circuit_one_ground.png}
    \par\vspace{5pt}
    {\small Figure 4: Circuit with one ground}
  \end{center}

  \Q The circuit in Figure 5 does not have a wire across the bottom completing the circuit. How would this change impact the number of electrons moving through the circuit?
  \vspace{10pt}
  \begin{center}
    \includegraphics[width=0.3\textwidth]{figures/circuit_two_grounds.png}
    \par\vspace{5pt}
    {\small Figure 5: Circuit with two grounds}
  \end{center}
    \begin{answer}[1in]
      Because the ``reservoir'' is infinite and not under pressure, it makes no difference when the bottom wire is replaced with two grounds.
    \end{answer}

  \Q The following table has observations of electrons/second (current is measured in amperes), pressure (voltage is measured in volts), and resistance (measured in ohms). From this data, derive Ohm's Law, a formula for I in terms of E and R.

  \vspace{10pt}
  \begin{center}
  \begin{tabular}{|c|c|c|}
    \hline
    Amperes (I) & Voltage (E) & Resistance (R) \\
    \hline
    5 & 100 & 20 \\
    \hline
    5 & 200 & 40 \\
    \hline
    10 & 200 & 20 \\
    \hline
    2 & 50 & 25 \\
    \hline
  \end{tabular}
  \end{center}

    \begin{answer}[1in]
      I = E / R
    \end{answer}

  \Q While a wire is designed to conduct electricity, it does not do so in a perfect fash-\key\\[-2.5mm] ion. That is, there is some resistance and over a long distance the pressure (voltage) will drop. If we need light at a distant location, what are two things that could be done to make sure that adequate amperage is available? Consider your answers to questions 8 and 18.
    \begin{answer}[1.5in]
      Increase the voltage (with a more powerful battery) and/or decrease the resistance (with a thicker wire).
    \end{answer}