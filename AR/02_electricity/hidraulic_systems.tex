\model{Hydraulic Systems}
  
  \begin{center}
    \includegraphics[width=0.3\textwidth]{figures/laura_washing_car.png}
    \par\vspace{5pt}
    {\small Figure 1: Laura washing the car}
  \end{center}

  \vspace{10pt}
  
  When washing your car, you might connect a hose to a faucet and spray the car with a nozzle. You might also put water in a bucket so you could use a sponge to wipe dirt off the car.\\

  {\it\large Refer to Model 1 above as your team develops consensus answers
  to the questions below.}

  \quest{13 min}

  \Q What path does the water take to get to the bucket? What is the source? And the source before that? What is the origin?
    \begin{answer}[2in]
      Water gets to the bucket from the hose; it gets to the hose from the faucet; it get to the faucet from the pipes; it gets to the pipes from the municipal water supply; it gets to the municipal water supply from a well or river; it gets to the well or river from rain or snow melt; rain and snow comes from clouds; clouds come from ocean evaporation; ocean are fed by rivers\ldots
    \end{answer}

  \Q Where does the water go when it runs off the car? What is the ultimate destination?
    \begin{answer}[1in]
      From the car to the driveway to the gutter to the storm drain to the river to the ocean.
    \end{answer}

  \Q What is the relationship between the ultimate source and the ultimate destination?
    \begin{answer}[1in]
      The water cycle can be seen to start/end with the oceans (or any other point in the cycle).
    \end{answer}

  \Q What moves water from a well into an above-ground tank?
    \begin{answer}[0.75in]
      A pump.
    \end{answer}

  \Q Is the water behind a faucet under pressure when the faucet valve is closed?
    \begin{answer}[0.75in]
      Yes, the pressure is there but it isn't moving.
    \end{answer}

  \Q Is the water under pressure when it sits in a bucket?
    \begin{answer}[0.75in]
      No significant pressure (other than gravity holding it in the bucket).
    \end{answer}

  \Q What adjustment to the nozzle causes the water pressure to increase?
    \begin{answer}[1in]
      Making the passage through the nozzle smaller increases the speed (and therefore pressure over the smaller surface area).
    \end{answer}

  \Q Does it take more pressure to move a smoothie through a fat straw or a thin straw?
    \begin{answer}[1in]
      More pressure is required to move the same volume through a smaller opening.
    \end{answer}

  \Q What two changes to the system would allow you to fill a bucket (or swimming pool) more quickly?
    \begin{answer}[1in]
      Increase the pressure (with the same diameter hose) or increase the hose diameter (with the same pressure).
    \end{answer}

  \Q Describe gallons per minute in terms of water pressure and the diameter of the hose.
    \begin{answer}[1in]
      As pressure increases, more fluid will pass; as diameter increases, more fluid will pass.
    \end{answer}