\model{Telegraphs and Relays}

  \quest{10 min}

  \Q An electric circuit could be used by a person controlling a switch to send a mess-\key\\[-2.5mm] age (say, with Morse Code) to a distant person observing a light or buzzer.
  But, as suggested by question 19, there is a limit to how long a distance the circuit can stretch.
  If a single circuit with a person at each end can transmit a message 10 miles, how would you send a message 20 miles (without modifying or augmenting the circuit)?
    \begin{answer}[1in]
      Use two circuits with a person in between who could ``repeat'' the code.
    \end{answer}

  \vspace{10pt}

  An electromagnet is a type of magnet in which the magnetic field is produced by an electric current. Electromagnets usually consist of wire wound into a coil.
  A current through the wire creates a magnetic field which is concentrated in the hole, denoting the center of the coil. The magnetic field disappears when the current is turned off.
  The wire turns are often wound around a magnetic core made from a ferromagnetic or ferrimagnetic material such as iron; the magnetic core concentrates the magnetic flux and makes a more powerful magnet.

  The main advantage of an electromagnet over a permanent magnet is that the magnetic field can be quickly changed by controlling the amount of electric current in the winding.
  However, unlike a permanent magnet that needs no power, an electromagnet requires a continuous supply of current to maintain the magnetic field.

  Electromagnets are widely used as components of other electrical devices, such as motors, generators, electromechanical solenoids, relays, loudspeakers, hard disks, MRI machines, scientific instruments, and magnetic separation equipment.
  Electromagnets are also employed in industry for picking up and moving heavy iron objects such as scrap iron and steel.

  \vspace{10pt}
  \begin{center}
    \includegraphics[width=0.25\textwidth]{figures/electromagnet.png}
    \par\vspace{5pt}
    {\small Figure 6: Electromagnet}
  \end{center}

  \newpage

  In an electric circuit diagram, an electromagnet is called an inductor and looks
  (and acts) very much like a resistor (Figure 7).

  \vspace{10pt}
  \begin{center}
    \includegraphics[width=0.3\textwidth]{figures/circuit_inductor.png}
    \par\vspace{5pt}
    {\small Figure 7: Circuit with inductor}
  \end{center}

  \Q Samuel Morse demonstrated that an inductor (magnet) could be used to control (turn on and off) a switch (see Figures 8 and 9). Use this insight to modify Figure 10 to extend the distance over which you could send a message using a single electric circuit.

\vspace{10pt}
\begin{center}
  \begin{minipage}{0.45\textwidth}
    \centering
    \includegraphics[width=0.9\textwidth]{figures/circuit_open_relay.png}
    \par\vspace{5pt}
    {\small Figure 8: Circuit with open relay}
  \end{minipage}
  \hfill
  \begin{minipage}{0.45\textwidth}
    \centering
    \includegraphics[width=0.9\textwidth]{figures/circuit_closed_relay.png}
    \par\vspace{5pt}
    {\small Figure 9: Circuit with closed relay}
  \end{minipage}
\end{center}

  \vspace{10pt}
  \begin{center}
    \includegraphics[width=0.8\textwidth]{figures/relay_extend_distance.png}
    \par\vspace{5pt}
    {\small Figure 10: Use of relay to extend distance}
  \end{center}