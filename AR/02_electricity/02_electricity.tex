% Source: CPTR 280 Computer Organization and Assembly Language Fall 2020
% File: "02 Electricity (key).pdf"
% Author: James Foster, pogil@jgfoster.net

% comment out for student version
\ifdefined\Student\relax\else\def\Teacher{}\fi

\documentclass[12pt]{article}

\title{Activity 2: Electric Circuits and Relays}
\author{James Foster}
\newcommand{\activityeditor}{James Foster}
\newcommand{\activitysource}{\url{pogil@jgfoster.net}}
\date{Fall 2020}

\input{../../cspogil.sty}
\usepackage{graphicx}
\usepackage{tabularx}

\begin{document}

  \begin{center}
    \maketitle
    \rolenames
  \end{center}

  \keyquestions{
  \item Model 2, Question \#11
  \item Model 2, Question \#19
  \item Model 3, Question \#20
  }

  \newpage
  \maketitle

  Electricity can be used to communicate information over long distances. This is done using relays, which forms a core foundation for building an electronic computer.
  
  \guides{
    \item Identify the components of a circuit;
    \item Use Ohm's Law to solve equations; and,
    \item Describe how a relay works.
   }{
    \item Draw basic circuit diagrams.
   }{
    No additional notes
   }{
    \item [1] \url{https://en.wikipedia.org/wiki/Electromagnet}
   }

  \model{Hydraulic Systems}
  
  \begin{center}
    \includegraphics[width=0.3\textwidth]{figures/laura_washing_car.png}
    \par\vspace{5pt}
    {\small Figure 1: Laura washing the car}
  \end{center}

  \vspace{10pt}
  
  When washing your car, you might connect a hose to a faucet and spray the car with a nozzle. You might also put water in a bucket so you could use a sponge to wipe dirt off the car.\\

  {\it\large Refer to Model 1 above as your team develops consensus answers
  to the questions below.}

  \quest{13 min}

  \Q What path does the water take to get to the bucket? What is the source? And the source before that? What is the origin?
    \begin{answer}[2in]
      Water gets to the bucket from the hose; it gets to the hose from the faucet; it get to the faucet from the pipes; it gets to the pipes from the municipal water supply; it gets to the municipal water supply from a well or river; it gets to the well or river from rain or snow melt; rain and snow comes from clouds; clouds come from ocean evaporation; ocean are fed by rivers\ldots
    \end{answer}

  \Q Where does the water go when it runs off the car? What is the ultimate destination?
    \begin{answer}[1in]
      From the car to the driveway to the gutter to the storm drain to the river to the ocean.
    \end{answer}

  \Q What is the relationship between the ultimate source and the ultimate destination?
    \begin{answer}[1in]
      The water cycle can be seen to start/end with the oceans (or any other point in the cycle).
    \end{answer}

  \Q What moves water from a well into an above-ground tank?
    \begin{answer}[0.75in]
      A pump.
    \end{answer}

  \Q Is the water behind a faucet under pressure when the faucet valve is closed?
    \begin{answer}[0.75in]
      Yes, the pressure is there but it isn't moving.
    \end{answer}

  \Q Is the water under pressure when it sits in a bucket?
    \begin{answer}[0.75in]
      No significant pressure (other than gravity holding it in the bucket).
    \end{answer}

  \Q What adjustment to the nozzle causes the water pressure to increase?
    \begin{answer}[1in]
      Making the passage through the nozzle smaller increases the speed (and therefore pressure over the smaller surface area).
    \end{answer}

  \Q Does it take more pressure to move a smoothie through a fat straw or a thin straw?
    \begin{answer}[1in]
      More pressure is required to move the same volume through a smaller opening.
    \end{answer}

  \Q What two changes to the system would allow you to fill a bucket (or swimming pool) more quickly?
    \begin{answer}[1in]
      Increase the pressure (with the same diameter hose) or increase the hose diameter (with the same pressure).
    \end{answer}

  \Q Describe gallons per minute in terms of water pressure and the diameter of the hose.
    \begin{answer}[1in]
      As pressure increases, more fluid will pass; as diameter increases, more fluid will pass.
    \end{answer}
  \newpage
  \model{Electric Circuit}

  Water flowing through pipes is often used as an (imperfect) analogy for electricity (electrons flowing through wires). The circuit diagram in Figure 2 has the following components:
  \begin{itemize}
    \item Battery (left)
    \item Switch (top)
    \item Resistor (right)
    \item Wires (to connect each element)
  \end{itemize}

  In practice, the resistor may be a light or a heating element. Portions of the circuit with pressure are shown in green.

  \vspace{10pt}
  \begin{center}
    \includegraphics[width=0.3\textwidth]{figures/circuit_open_switch.png}
    \par\vspace{5pt}
    {\small Figure 2: Circuit with open switch}
  \end{center}

  {\it\large Refer to Model 2 above as your team develops consensus answers
  to the questions below.}

  \quest{14 min}

  \Q For each of the above components, what is the analogous item in a hydraulic\key\\[-2.5mm] system (you may write it below or next to the component above)?
    \begin{answer}[1in]
      Battery: Pump; Switch: Faucet; Resistor: Nozzle; Wires: Pipes
    \end{answer}

  \Q In the diagram shown above, the switch connecting the battery to the resistor is open and so electricity cannot flow. How does that compare with how you would describe a valve in a hydraulic system? That is, is the faucet open or closed when the water is moving?
    \begin{answer}[1in]
      When a faucet is open then water can flow.
    \end{answer}

  \Q Even with the switch open, are the electrons between the battery and the switch under pressure?
    \begin{answer}[0.75in]
      Yes, there is still pressure (shown in green) when the switch is open.
    \end{answer}

  \Q With the switch closed (see the diagram below), is there pressure between the switch and the resistor?
  \vspace{10pt}
  \begin{center}
    \includegraphics[width=0.3\textwidth]{figures/circuit_closed_switch.png}
    \par\vspace{5pt}
    {\small Figure 3: Circuit with closed switch}
  \end{center}
    \begin{answer}[1in]
      Yes, there is pressure between the switch and the resistor.
    \end{answer}

  \Q Is there pressure between the resistor and the battery along the bottom wire? Does it make a difference if the switch is open or closed?
    \begin{answer}[1in]
      No, there is no (significant) pressure between the resistor and the battery along the bottom wire. This is the case if the switch is open or closed.
    \end{answer}

  \Q An analogy for the circuit in Figure 3 might be a fish tank where water is taken out, run through a filter, and pumped back in. Would the size of the tank impact the amount of water flowing through the filter?
    \begin{answer}[1in]
      No, the size of the tank does not impact the amount of water flowing through the filter.
    \end{answer}

  The new symbol at the bottom of Figure 4 shows an infinite ``reservoir'' of electrons available (and not under pressure) to which the circuit can dump excess electrons or acquire needed electrons to push through the circuit. This new symbol is called ground (or earth).
  \vspace{10pt}
  \begin{center}
    \includegraphics[width=0.3\textwidth]{figures/circuit_one_ground.png}
    \par\vspace{5pt}
    {\small Figure 4: Circuit with one ground}
  \end{center}

  \Q The circuit in Figure 5 does not have a wire across the bottom completing the circuit. How would this change impact the number of electrons moving through the circuit?
  \vspace{10pt}
  \begin{center}
    \includegraphics[width=0.3\textwidth]{figures/circuit_two_grounds.png}
    \par\vspace{5pt}
    {\small Figure 5: Circuit with two grounds}
  \end{center}
    \begin{answer}[1in]
      Because the ``reservoir'' is infinite and not under pressure, it makes no difference when the bottom wire is replaced with two grounds.
    \end{answer}

  \Q The following table has observations of electrons/second (current is measured in amperes), pressure (voltage is measured in volts), and resistance (measured in ohms). From this data, derive Ohm's Law, a formula for I in terms of E and R.

  \vspace{10pt}
  \begin{center}
  \begin{tabular}{|c|c|c|}
    \hline
    Amperes (I) & Voltage (E) & Resistance (R) \\
    \hline
    5 & 100 & 20 \\
    \hline
    5 & 200 & 40 \\
    \hline
    10 & 200 & 20 \\
    \hline
    2 & 50 & 25 \\
    \hline
  \end{tabular}
  \end{center}

    \begin{answer}[1in]
      I = E / R
    \end{answer}

  \Q While a wire is designed to conduct electricity, it does not do so in a perfect fash-\key\\[-2.5mm] ion. That is, there is some resistance and over a long distance the pressure (voltage) will drop. If we need light at a distant location, what are two things that could be done to make sure that adequate amperage is available? Consider your answers to questions 8 and 18.
    \begin{answer}[1.5in]
      Increase the voltage (with a more powerful battery) and/or decrease the resistance (with a thicker wire).
    \end{answer}
  \newpage
  \model{Telegraphs and Relays}

  \quest{10 min}

  \Q An electric circuit could be used by a person controlling a switch to send a mess-\key\\[-2.5mm] age (say, with Morse Code) to a distant person observing a light or buzzer.
  But, as suggested by question 19, there is a limit to how long a distance the circuit can stretch.
  If a single circuit with a person at each end can transmit a message 10 miles, how would you send a message 20 miles (without modifying or augmenting the circuit)?
    \begin{answer}[1in]
      Use two circuits with a person in between who could ``repeat'' the code.
    \end{answer}

  \vspace{10pt}

  An electromagnet is a type of magnet in which the magnetic field is produced by an electric current. Electromagnets usually consist of wire wound into a coil.
  A current through the wire creates a magnetic field which is concentrated in the hole, denoting the center of the coil. The magnetic field disappears when the current is turned off.
  The wire turns are often wound around a magnetic core made from a ferromagnetic or ferrimagnetic material such as iron; the magnetic core concentrates the magnetic flux and makes a more powerful magnet.

  The main advantage of an electromagnet over a permanent magnet is that the magnetic field can be quickly changed by controlling the amount of electric current in the winding.
  However, unlike a permanent magnet that needs no power, an electromagnet requires a continuous supply of current to maintain the magnetic field.

  Electromagnets are widely used as components of other electrical devices, such as motors, generators, electromechanical solenoids, relays, loudspeakers, hard disks, MRI machines, scientific instruments, and magnetic separation equipment.
  Electromagnets are also employed in industry for picking up and moving heavy iron objects such as scrap iron and steel.

  \vspace{10pt}
  \begin{center}
    \includegraphics[width=0.25\textwidth]{figures/electromagnet.png}
    \par\vspace{5pt}
    {\small Figure 6: Electromagnet}
  \end{center}

  \newpage

  In an electric circuit diagram, an electromagnet is called an inductor and looks
  (and acts) very much like a resistor (Figure 7).

  \vspace{10pt}
  \begin{center}
    \includegraphics[width=0.3\textwidth]{figures/circuit_inductor.png}
    \par\vspace{5pt}
    {\small Figure 7: Circuit with inductor}
  \end{center}

  \Q Samuel Morse demonstrated that an inductor (magnet) could be used to control (turn on and off) a switch (see Figures 8 and 9). Use this insight to modify Figure 10 to extend the distance over which you could send a message using a single electric circuit.

\vspace{10pt}
\begin{center}
  \begin{minipage}{0.45\textwidth}
    \centering
    \includegraphics[width=0.9\textwidth]{figures/circuit_open_relay.png}
    \par\vspace{5pt}
    {\small Figure 8: Circuit with open relay}
  \end{minipage}
  \hfill
  \begin{minipage}{0.45\textwidth}
    \centering
    \includegraphics[width=0.9\textwidth]{figures/circuit_closed_relay.png}
    \par\vspace{5pt}
    {\small Figure 9: Circuit with closed relay}
  \end{minipage}
\end{center}

  \vspace{10pt}
  \begin{center}
    \includegraphics[width=0.8\textwidth]{figures/relay_extend_distance.png}
    \par\vspace{5pt}
    {\small Figure 10: Use of relay to extend distance}
  \end{center}

\end{document}