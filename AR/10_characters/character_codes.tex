\model{Character Codes}

  Four our purposes, a character is a letter, number, punctuation, or other symbol used in writing. Each character is an abstraction that can be represented in print
  (ink on paper or pixels on a screen) using a glyph from a font. So `A' and `A' are the same character represented with a different font (Times New Roman and Arial respectively).
  The same character can also be represented as a gesture (using American Sign Language), as a combination of short and long tones (Morse Code), or as raised dots on paper (Braille).

  A number is written as a sequence of digits (characters) in a base such as decimal, octal, or hexadecimal (see lesson 3). A character such as `F' can be used as a letter
  (in the name ``Foster'') or as a digit (in the hexadecimal number ``0xFF'').\\

  {\it\large Refer to Model 1 above as your team develops consensus answers
  to the questions below.}

  \quest{5 min}

  \Q How many digits (characters) are in the number 99324?
    \hfill\ans{5 digits}

  \Q How many bits are required to specify codes for the decimal digits (0-9)? (Hint: not ten!)
    \begin{answer}[0.5in]
      4 bits
    \end{answer}

  \Q How many bits are required to specify codes for the uppercase English letters (A-Z)?
    \begin{answer}[0.5in]
      5 bits
    \end{answer}

  \Q How many bits are required to specify codes for the uppercase English letters (A-Z), the lowercase English letters (a-z), and the decimal digits (0-9)?
    \begin{answer}[0.5in]
      6 bits
    \end{answer}

  \vspace{-20pt}

  \Q How many bits are required to specify codes for the uppercase English letters (A-Z),\key\\[-2.5mm] the lowercase English letters (a-z), the decimal digits (0-9), and a dozen or so punctuation characters?
    \begin{answer}[0.5in]
      7 bits
    \end{answer}

  \Q How many bits are in a byte?
    \hfill\ans{8 bits}

  \Q How many unique codes (values) can be stored in 8 bits?
    \hfill\ans{256 values}