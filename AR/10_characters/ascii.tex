\model{ASCII}

  \vspace{10pt}
  \begin{center}
    \includegraphics[width=0.9\textwidth]{figures/ascii_table.png}
    \par\vspace{5pt}
    {\small Figure 1: ASCII Table}
  \end{center}

  The American Standard Code for Information Interchange (ASCII) was formalized in 1967 and is the standard set of character codes to represent
  English letters (uppercase and lowercase), Arabic numerals, basic punctuation, and a number of control codes.

  {\it\large Refer to Model 2 above as your team develops consensus answers
  to the questions below.}

  \quest{5 min}

  \Q How many codes are defined in ASCII? What is the range (start to end) of values?
    \begin{answer}[0.2in]
      128 codes in the range of 0 to 127 (0x00 to 0x7F).
    \end{answer}

  \Q How many bits are required to represent one ASCII code?
    \hfill\ans{7 bits}

  \Q What are the hexadecimal codes for ``ASCII''?
    \hfill\ans{41 53 43 49 49}

  \Q What characters are represented by the hexadecimal codes 54 61 62 6C 65?
    \begin{answer}[0.2in]
      T a b l e
    \end{answer}

  \Q Give an example of a control character (0x00 to 0x20 and 0x7F) that would commonly appear in regular text.
    \begin{answer}[0.2in]
      Line feed (0x0A), carriage return (0x0D), or space (0x20)
    \end{answer}
