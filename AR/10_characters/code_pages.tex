\model{Code Pages}

  While ASCII was fine in America (it is, after the American Standard \ldots), other countries had reason to need character encoding as well.
  Even English-speaking countries, like England, would want to use a currency symbol (`\pounds') that is not provided in ASCII.\\

  {\it\large Refer to Model 3 above as your team develops consensus answers
  to the questions below.}

  \quest{10 min}

  \Q Are there any codes are available in a byte that are not used by ASCII (hint: see questions 7 \& 8)? If so, how many? What is their range?
    \begin{answer}[0.5in]
      Yes, there are 128 codes in the range 128 to 255 (0x80 to 0xFF).
    \end{answer}

  \vspace{10pt}
  In the closing decades of the 20th century codes were defined for a variety of geographical regions. ISO/IEC 8859 defined fifteen code sets or ``Parts'' that each keep ASCII in the lower half of the 8-bit range and in the upper half add just under 100 region-specific characters.

  \vspace{10pt}
  \begin{center}
    \begin{tabular}{|c|c|}
      \hline
      Part 1 & Latin-1 Western European \\
      \hline
      Part 2 & Latin-2 Central European \\
      \hline
      Part 3 & Latin-3 South European \\
      \hline
      Part 4 & Latin-4 North European \\
      \hline
      Part 5 & Latin/Cyrillic \\
      \hline
      Part 6 & Latin/Arabic \\
      \hline
      Part 7 & Latin/Greek \\
      \hline
      Part 8 & Latin/Hebrew \\
      \hline
      Part 9 & Latin-5 Turkish \\
      \hline
      Part 10 & Latin-6 Nordic \\
      \hline
      Part 11 & Latin/Thai \\
      \hline
      Part 13 & Latin-7 Baltic Rim \\
      \hline
      Part 14 & Latin-8 Celtic \\
      \hline
      Part 15 & Latin-9 \\
      \hline
      Part 16 & Latin-10 South-Eastern European \\
      \hline
    \end{tabular}
  \end{center}

  \newpage
  Part 1 has characters for the following Western European languages: Danish, Dutch, English, Faeroese, Finnish, French, German,
  Icelandic, Irish, Italian, Norwegian, Portuguese, Rhaeto-Romantic, Scottish Gaelic, Spanish, Catalan, and Swedish and is shown in the Figure 2.

  \vspace{10pt}
  \begin{center}
    \includegraphics[width=0.85\textwidth]{figures/iso_8859_1.png}
    \par\vspace{5pt}
    {\small Figure 2: ISO 8859-1}
  \end{center}

  \Q What does the inclusion of Hebrew (Part 8) in this list say about the number of characters in the Hebrew alphabet?
    What is a rough estimate of the upper limit?
    \begin{answer}[0.5in]
      The total number of characters in the Hebrew alphabet is well under 100.
    \end{answer}

  \Q While the code page approach allows many languages to be represented, each\key\\[-2.5mm] Part can use the same code for a different character,
    so 0xAA is `\={E}' in Part 4 and `\v{S}' in Part 10. So, while documents written in Part 8 (Hebrew) can be read by other users in Israel,
    the document would be gibberish to someone in the United States (using Part 1), even if that person was fluent in Hebrew because the computer
    would present the wrong character for a particular code. When using Code Pages, what information needs to accompany every string of text (document, email, etc.)?
    \begin{answer}[0.5in]
      Every string of text must be accompanied by the Code Page used to interpret it.
    \end{answer}

  \Q Because each Part uses ASCII for the first 128 values, (American) English can be shared with another language (that is, Part 6 supports English and Arabic while Part 8
    supports English and Hebrew). But a computer can generally run in only one mode at a time (using a specific Part to describe its character set). What problem does this present?
    \begin{answer}[0.5in]
      A document cannot have text from different Code Pages, such as Arabic and Hebrew.
    \end{answer}

  \Q China, Japan, and Korea had as much or more need for computer support as many of the countries supported by ISO/IEC 8859 (Japan was---and is---more technologically advanced than, say, Turkey or Thailand),
    yet there was no eight-bit encoding for their alphabets. What characteristic of their alphabets would make 8-bit encoding impossible?
    \begin{answer}[0.5in]
      Instead of having less than 100 characters in their alphabets, they have several thousand characters.
    \end{answer}

  \Q Because memory is typically addressed in units of a byte, data is typically coded into multiples of a byte. How many bits are in two bytes? What is the range of values that can be encoded in two bytes?
    \begin{answer}[0.5in]
      Two bytes contain sixteen bits and can hold values from 0 to 65535 (0x0000 to 0xFFFF).
    \end{answer}