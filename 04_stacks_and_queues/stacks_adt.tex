\model{Stacks - Abstract Data Type}

Note that stacks are often described as \textbf{Last-In, First-Out (LIFO)} 
or \textbf{First-In, Last-Out (FILO)} (particularly if you are Greek :-) ).


\Q (3 min) Based on the key characteristics of stacks that you identified above,
\textbf{list at least 3 key operations} for a stack ADT, and \textbf{rank them} (last column)
by \textbf{importance} (1=high, 5=low).

\begin{center}
    \begin{tabular}{ |r|p{12cm}|l| }
        \hline
        \rowcolor{lightgray} & \textbf{action or operation} & \textbf{rank} \\
        \hline
        a.                   &                              &               \\
        \hline
        b.                   &                              &               \\
        \hline
        c.                   &                              &               \\
        \hline
        d.                   &                              &               \\
        \hline
    \end{tabular}
\end{center}

\Q (5 min) Start with the most important stack operation above, and define
a \textbf{method signature} including an appropriate name, input parameters, and return types.
You may write the method below, or create an interface file in your IDE (e.g. Code.CS).
Use "T" or "Object" as a generic placeholder for the type of object stored in the stack.
You may include constructors, although Java does not allow interfaces to specify constructors.


\begin{javabox}
public interface IStack<T> {

    public IStack<T>(int maxSize); // constructor

    ?

    ?

    ?

    ?

    ?

} // end interface IStack
\end{javabox}

Review progress with the facilitator before continuing.

\begin{answer}[1.5in]
see Facilitator - Sample Code
NOTE: if students are unfamiliar with generics, use Stack of Strings

REPORT OUT: operations and method signatures, so teams have common framework
Maybe have teams compare their interface to the Java API, Wikipedia,
or their textbook, and summarize any differences, insights, or questions.
\end{answer}
