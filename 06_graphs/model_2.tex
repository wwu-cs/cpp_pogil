\model{Modeling with Graphs}

Now that we have looked at graphs and the terminology, let's now focus on how graphs are useful for computation.
Graphs are the basis for modeling relationships among entities.
Here are some examples:


\begin{itemize}[itemsep=-0.5pt]

  \item The global internetwork of computers
        \begin{itemize}
          \item Vertices are routers and computers; Edges are links between routers/computers
        \end{itemize}
  \item Social networks
        \begin{itemize}
          \item Vertices are people; Edges represent the ``friend'' relationship
        \end{itemize}
  \item State diagrams
        \begin{itemize}[itemsep=-0.5pt]
          \item Vertices represent current state of computer; Edges represent transition to new states
          \item Often the model used by event-driven programming
        \end{itemize}
  \item Rubik's cube
        \begin{itemize}
          \item Vertices represent current state of cube; Edges represent one twist to new
                configuration of the Rubik's cube
        \end{itemize}
  \item Transportation networks
        \begin{itemize}
          \item Vertices are intersections; Edges are roads/highways
        \end{itemize}
  \item Airline flights
        \begin{itemize}
          \item Vertices represent cities/airports; Edges are flights between the cities
        \end{itemize}
\end{itemize}

\Q What real life scenario (not from above list) could you model with a directed graph?

\begin{answer}[1in]
\end{answer}

\Q What real life scenario (not from above list) could you model with an undirected graph?

\begin{answer}[1in]
\end{answer}

At this point, hopefully, your group realizes the importance of graphs in computing.
Note that a graph can be thought of as a tree that can have cycles.
A tree is also always a graph.
A graph is not always a tree.
