\model{Normalize Function Analysis}

  \texttt{Normalize} is a function that maps values in the range $[min .. max]$ to the range $[0 .. 1]$.

  \begin{cpplst}
void normalize(double array[], int size, double min, double max) {
    for (int i = 0; i < size; i++) {
        array[i] = (array[i] - min) / (max - min);
    }
}
\end{cpplst}

  \par\vskip 10pt

  {\it\large Refer to Model 2 above as your team develops consensus answers
    to the questions below.}

  \quest{15 min}

  \Q Suppose an array \texttt{a} contains the values \texttt{\{5, 15, 10\}} and the
    function is called with the following function call:
    \begin{verbatim}
      normalize(a, 3, 5, 15);
    \end{verbatim}
    What are the contents of the array after this function call?
    \begin{answer}[1.25in]
      The contents of \texttt{a} are \texttt{\{0.0, 1.0, 0.5\}}.
    \end{answer}

  \Q How many operations does the function execute when \texttt{normalize(a, 3, 5, 15)} is called?
    \\
    {\small Note: the initialization of the variable \texttt{i} executes before the first
    iteration of the loop. The iteration and comparison statements occur after
    each iteration of the loop.}
    \begin{answer}[1.25in]
      The function executes 13 operations.
    \end{answer}

  \Q Suppose the \texttt{normalize} function is called with an array of length 20 as
    an argument.  How many operations are executed by the function?
    \begin{answer}[1.25in]
			The function executes 83 operations.
    \end{answer}

  \Q Suppose the \texttt{normalize} function is called with an array of length $n$ as
    an argument.\key\\[-2.5mm]  How many operations are executed by the function?
    \begin{answer}[1.25in]
			The function executes $5n + 3$ operations.
    \end{answer}

  \Q We say that the \texttt{normalize} function runs in \emph{linear} time. Another
    way to say this is that the function is $\Theta(n)$.  Complete the following
    sentence: \\
    \ \\
    A function is $\Theta(n)$ (or executes in linear time) if...
    \begin{answer}[1.25in]
			...the number of operations it executes increases linearly with the size of its input.
    \end{answer}

  \Q We say that \emph{quadratic} time functions are $\Theta(n^{2})$. Complete the
    following sentence: \\
    \ \\
    A function is $\Theta(n^{2})$ (or executes in quadratic time) if...
    \begin{answer}[1.25in]
			...the number of operations it executes increases quadratically with the size of its input.
    \end{answer}