\model{Normalize Function Analysis}

Below is a Java method.  It maps values that are in the range $[min .. max]$ to the range $[0 .. 1]$.


\begin{javalst}
void normalize(double[] array, double min, double max) {
    for (int i = 0; i < array.length; i++) {
        array[i] = (array[i] - min) / (max - min);
    }
}
\end{javalst}

\Q
Suppose an array \texttt{a} contains the values \texttt{\{5, 15, 10\}} and the
method is called with the following method call:
\begin{verbatim}
normalize(a, 5, 15);
\end{verbatim}

What are the contents of the array after this method call?

\begin{answer}[1.25in]
\end{answer}

\Q
How many operations does the method execute when \texttt{normalize(a, 5, 15)} is called?
\\
{\small Note: the initialization of the variable \texttt{i} executes before the first
iteration of the loop. The iteration and comparison statements occur after
each iteration of the loop.}

\begin{answer}[1.25in]
\end{answer}

\Q
Suppose the \texttt{normalize} method is called with an array of length 20 as
an argument.  How many operations are executed by the method?

\begin{answer}[1.25in]
\end{answer}

\Q
Suppose the \texttt{normalize} method is called with an array of length $n$ as
an argument.  How many operations are executed by the method?

\begin{answer}[1.25in]
\end{answer}

\Q
We say that the \texttt{normalize} method runs in \emph{linear} time. Another
way to say this is that the method is $\Theta(n)$.  Complete the following
sentence: \\
\ \\
A method is $\Theta(n)$ (or executes in linear time) if...

\begin{answer}[1.25in]
\end{answer}

\Q
We say that \emph{quadratic} time methods are $\Theta(n^{2})$. Complete the
following sentence: \\
\ \\
A method is $\Theta(n^{2})$ (or executes in quadratic time) if...

\begin{answer}[1.25in]
\end{answer}
