\model{Analysis Functions}

  \quest{20 min}

  Label each of the following functions either $\Theta(1)$, $\Theta(n)$, or $\Theta(n^{2})$.

  \begin{cpplst}
int max(int a, int b) {
    if (a > b) {
        return a;
    } else {
        return b;
    }
}
  \end{cpplst}

  \Q The \texttt{max} function is $\Theta(\ans[0.2in]{1})$. Justify your answer.
    \begin{answer}[.25in]
      Because the number of operations executed does not depend on the size of the input.
    \end{answer}


  \begin{cpplst}
int maxElement(int array[], int size) {
    int max = array[0];
    for (int i = 0; i < size; i++) {
        if (array[i] > max) {
            max = array[i];
        }
    }
    return max;
}
  \end{cpplst}

  \Q The \texttt{maxElement} function is $\Theta(\ans[0.2in]{n})$. Justify your answer.
    \begin{answer}[.25in]
      Because the number of operations executed increases linearly with the size of the input.
    \end{answer}
  
  \begin{cpplst}
int maxSubseqSum(int array[], int size) {
    int max = array[0];
    for (int i = 0; i < size; i++) {
        int sum = 0;
        for (int j = i; j < size; j++) {
            sum += j;
            if (sum > max) {
                max = sum;
            }
        }
    }
    return max;
}
  \end{cpplst}

  \Q The \texttt{maxSubseqSum} function is $\Theta(\ans[0.2in]{n^2})$. Justify your answer.
    \begin{answer}[.25in]
      Because the number of operations executed increases quadratically with the size of the input.
    \end{answer}

  \Q We are using the number of operations a function executes as a measure
    of its run\key\\[-2.5mm] time.  In a few complete sentences, explain why we are using
    this measure of time rather than a wall-clock measure of time (\emph{i.e.},
    minutes, seconds, \emph{etc}.).
    \begin{answer}[1.25in]
      Because wall-clock time can vary based on the hardware and software environment,
      while counting operations provides a more consistent and comparable measure of
      algorithm efficiency.
    \end{answer}

  \Q Why is knowing that a function is $\Theta(n)$ more valuable than knowing that
    it takes fifteen seconds to execute on a 2.7GHz i7? In the space below,
    list the pros and cons for each statement.\\
    \begin{itemize}
        \item ``The function is $\Theta(n)$.''
              \begin{answer}[1.25in]
                Because it provides a general understanding of how the function's performance
                scales with input size, independent of specific hardware or conditions.
              \end{answer}

        \item ``The function took 15s on my i7.''
              \begin{answer}[1.25in]
                Because it gives a specific performance metric for a particular instance,
                but does not inform about scalability or performance on different hardware.
              \end{answer}
    \end{itemize}

  \par\vskip -30pt

  \Q Is it possible that there are inputs for which a $\Theta(1)$ function
    executes more\key\\[-2.5mm] operations than a $\Theta(n)$ function that has the same
    specification? Why or why not?
    \begin{answer}[0.7in]
      Yes. Because $\Theta(1)$ functions have a constant upper bound on operations,
      but that bound could be higher than the number of operations executed by
      a specific instance of a $\Theta(n)$ function with a small input size.
    \end{answer}
