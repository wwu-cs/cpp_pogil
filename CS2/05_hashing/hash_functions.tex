
\model{Hash Functions}

  \quest{10 min}

  Above, we have considered the situation where the items are numbers to insert into the dictionary.
  Suppose now we want to insert strings into the dictionary (like an actual dictionary!!).
  We first need to map the string to a number.
  In this case, the string is our key and the number is our hash value.
  Once we have the value, we need to map the value to a location in the array.

  \par\vskip 10pt

  The mapping from a key to an integer is called the hash function.
  There are many implementations of hash functions.
  A good hash function spreads keys across the range of integers.
  A good hash function is fast to compute, given the length of the key.
  A really bad hash function is one that maps all keys to the value 6.
  You will learn more about creating good hash functions in CS 324, Algorithms, if you take that course.
  Another property of a good hash function is that it maps the strings ``abc'' and ``cba'' to different integers, so permutations of the same set of letters have different hash values.

  \par\vskip 10pt

  Another property of a good hash function is that if two keys are considered equal, they will map to the same hash value.
  For example, if uppercase ``SAM'' and lowercase ``sam'' are considered equivalent in your application, then they should map to the same hash value.

  \par\vskip 10pt

  Here is an example of a hash function:

  \begin{cpplst}
unsigned int hash(std::string key) {
  unsigned int rtnVal = 3253;
  for (int i = 0; i < key.size(); i++) {
    rtnVal *= 28277;
    rtnVal += key[i] * 2749;
  }
  return rtnVal;
}
    \end{cpplst}

  So, hash("a") is:

  \begin{cpplst}
rtnVal = 3253
rtnVal = 3253 * 28277 // 91985081
rtnVal = 91985081 + 97*2749 // note: 'a' in ASCII is 97
rtnVal = 92251734
  \end{cpplst}

  \vskip -10pt

  \Q What is hash(``ab'')? \hfill\ans{It's 2611387551}

  \vskip -10pt

  \Q What is hash(``ba'')? (Note: 'b' in ASCII is value 98)\key\\[-2.5mm] \hfill\ans{It's 2611452044}

  \vskip -15pt

  \Q Explain one reason a Dictionary ADT might use a hash function?\key\\[-2.5mm]
    \begin{answer}[0.2in]
      A Dictionary ADT might use a hash function to map keys (like strings) to integer values
      that can then be used to determine the location in an array where the corresponding value should be stored.
    \end{answer}