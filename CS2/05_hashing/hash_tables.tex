\model{Hash Tables}

  \quest{10 min}

  The array that was mentioned earlier is called a hash table. A hash table stores dictionary items. The
  storage location is based on the hash value and using \% of the hash table size.

  \texttt{Insertion}

  Assume we have a hash table of size 10. We want to insert the following items:
    \begin{itemize}
      \item "coconut''
      \item ``milk''
      \item ``apple''
    \end{itemize}

  STEP 1: We'll use the hash function listed above to calculate the hash values for these keys:
    \begin{itemize}
      \item ``coconut'' (hash value = 2104178476)
      \item ``milk'' (hash value = 461110994)
      \item ``apple'' (hash value = 3515030035)
    \end{itemize}

  STEP 2: We'll map the hash values to the size for this hash table (\% 10):
    \begin{itemize}
      \item  ``coconut'' (location = 6)
      \item ``milk'' (location = 4)
      \item ``apple'' (location = 5)
    \end{itemize}

  \includegraphics[width=.9\textwidth]{figures/hashing1.png}

  \Q Now, suppose we want to insert ``orange'' into this hash table.
    Its hash value is 3410197053.
    Insert this key into the table above.
    \begin{answer}[1in]
      It goes into position 3.
    \end{answer}

  \Q Now, let's insert ``corn''. Its hash value is 1347376851. Insert this key into the table above.
    \begin{answer}[1in]
      It goes into position 1.
    \end{answer}

  \Q Now, let's insert ``eggs''. Its hash value is 881505635. Insert this key into the table above.
    \begin{answer}[1in]
      It goes into position 5, but that position is already taken by ``apple''.
      This is called a collision.
    \end{answer}

  What happens now?
  ``apple'' is already stored at position 5.
  There are several techniques (see textbook) to address collisions.

  \par\vspace{10pt}

  We'll first resolve collisions using open address linear probing.