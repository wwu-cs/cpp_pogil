\model{Dictionary ADT}

  \quest{10 min}

  \Q Think about how you use a dictionary. What do you "search for" in a dictionary?
    \begin{answer}[1in]
      I search for a word (the key).
    \end{answer}

  \Q  What information to you get back from a dictionary?
    \begin{answer}[1in]
      I get back the definition of the word (the value).
    \end{answer}

  A dictionary ADT stores a collection of items and supports the following operations:
    
  \begin{cpplst}
create()                           // create new Dictionary
insert(Dictionary d, key k, value v) // insert a new value with given key
find(Dictionary d, key k)      // search for a given key and return the value
delete(Dictionary d, key k)    // delete key and item that goes with the key
print(Dictionary d)            // print all items in dictionary
  \end{cpplst}

  \par\vskip 10pt

  One common tradeoff in computer science is the time/space tradeoff.
  In some (maybe many) cases, one can increase runtime speed at the expense of storage space.
  On the flip side, one can often gain efficiency in storage at the expense of running time.

  \par\vskip 10pt

  We'll look at a way to implement the Dictionary ADT using hash tables.
  As you might guess, we will speed up the dictionary operations at the expense of using more storage space.

  \par\vskip 10pt

  Consider this scenario.
  Your dictionary should store (student ID, student record) information.
  Student IDs are non-negative numbers.
  Your dictionary stores only currently enrolled WWU students.
  Suppose the range of possible student IDs goes from 0 to 999,999,999.


  \Q  How might you build a Dictionary ADT for this scenario so that insert, find, and delete are O(1) operations?
    \begin{answer}[1in]
      Create an array of size 1,000,000,000.
      Use the student ID as the index into the array.
      Store the student record at that index.
    \end{answer}

  \Q  How much memory does your solution in \#3 take?
    \begin{answer}[1in]
      It takes memory for an array of size 1,000,000,000.
      Each entry in the array stores a student record (or null if no student with that ID is enrolled).
    \end{answer}

  \par\vskip 10pt

  Consider that there are about 4000 currently enrolled WWU students.
  If you implement an array of size 1,000,000,000 to store (student ID, student record) pairs, the array is sparse.
  Certainly, the insert, delete, and find operations are O(1) with this implementation, but the wasted space is quite large given the number of items in the dictionary.

  \par\vskip 10pt

  Here is one way to reduce the size of the array.
  If we know there are 4000 items in the dictionary, let's instead create an array of size 100,000 to store these items.
  Note that there are still 96,000 empty cells with this approach, but that is quite a bit smaller than 1 billion.

  \par\vskip 10pt

  So, in order to insert a (student ID, student record) pair into the dictionary of size 100,000, we need to map student IDs to the range [0...99,999].
  One common way to do this is to use the mod operator:

  \begin{cpplst}
location = studentID % 100000;
  \end{cpplst}

  The mod (remainder) operator gives us values in the appropriate range.

  \Q What is your student ID? \hfill\ans{Answers will vary}

  \Q In what array location would your ID be inserted, assuming the array has 100,000 items?
    \begin{answer}[0.3in]
      Answers will vary
    \end{answer}

  \Q Consider student ID 1152436. In what array location would this ID be inserted?
    \begin{answer}[0.3in]
      52436
    \end{answer}

  \Q  What might be a downside to this approach?
    \begin{answer}[0.3in]
      Different student IDs might map to the same array location.
      For example, student ID 2152436 would also map to location 52436.
      This is called a collision.
    \end{answer}

  Because we have limited the array size, it is now possible to have collisions.
  A collision happens when two dictionary items are mapped to the same location in the array.

  \Q What is another student ID number that would be mapped to the same array location as 1152436?
    \begin{answer}[0.3in]
      152436, 2152436, 3152436, ...
    \end{answer}