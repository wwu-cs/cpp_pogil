% Source: https://www.dropbox.com/sh/rl0yyth9g06psva/AADB0Cj4isIX5DAyrspqj8mFa
% File: "CS305 activity 18 hashing.pdf"
% Access: 04-28-2022

% comment out for student version
%\ifdefined\Student\relax\else\def\Teacher{}\fi

\documentclass[12pt]{article}

\title{Activity \#5: Hashing}
\author{Tammy VanDeGrift}
\newcommand{\activityeditor}{Preston Carman}
\newcommand{\activitysource}{\url{https://www.dropbox.com/sh/rl0yyth9g06psva/AADB0Cj4isIX5DAyrspqj8mFa}}
\date{Spring 2022}

\input{../../cspogil.sty}

\begin{document}

  \begin{center}
    \maketitle
    \rolenames
  \end{center}

  \keyquestions{
    \item Model 1, Question \#3
    \item Model 2, Question \#11
    \item Model 2, Question \#12
    \item Model 4, Question \#20
    \item Model 4, Question \#21
  }

  \newpage
  \maketitle

  Recall the List ADT.
  Its operations include insert, find, and delete.
  Earlier, we looked at how find can be implemented using binary search over a sorted list.
  One advantage of binary search is that it only cost $O(log_2N)$ time to execute.
  \par

  One common tradeoff in computer science is the time/space tradeoff.
  In some (maybe many) cases, one can increase runtime speed at the expense of storage space.
  On the flip side, one can often gain efficiency in storage at the expense of running time.
  \par

  We'll look at a way to implement the Dictionary ADT using hash tables.
  As you might guess, we will speed up the dictionary operations at the expense of using more storage space.

  %\rolenames

  \guide{
    \item Explain a hash function
    \item Explain the structure of hash table
    \item Explain how collisions are managed
  }{
    \item Write code that adds, removes, and accesses a hash table
  }{
    No additional notes.
  }

  \model{Dictionary ADT}

  \quest{10 min}

  \Q Think about how you use a dictionary. What do you "search for" in a dictionary?
    \begin{answer}[1in]
      I search for a word (the key).
    \end{answer}

  \Q  What information to you get back from a dictionary?
    \begin{answer}[1in]
      I get back the definition of the word (the value).
    \end{answer}

  A dictionary ADT stores a collection of items and supports the following operations:
    
  \begin{cpplst}
create()                           // create new Dictionary
insert(Dictionary d, key k, value v) // insert a new value with given key
find(Dictionary d, key k)      // search for a given key and return the value
delete(Dictionary d, key k)    // delete key and item that goes with the key
print(Dictionary d)            // print all items in dictionary
  \end{cpplst}

  \par\vskip 10pt

  One common tradeoff in computer science is the time/space tradeoff.
  In some (maybe many) cases, one can increase runtime speed at the expense of storage space.
  On the flip side, one can often gain efficiency in storage at the expense of running time.

  \par\vskip 10pt

  We'll look at a way to implement the Dictionary ADT using hash tables.
  As you might guess, we will speed up the dictionary operations at the expense of using more storage space.

  \par\vskip 10pt

  Consider this scenario.
  Your dictionary should store (student ID, student record) information.
  Student IDs are non-negative numbers.
  Your dictionary stores only currently enrolled WWU students.
  Suppose the range of possible student IDs goes from 0 to 999,999,999.


  \Q  How might you build a Dictionary ADT for this scenario so that insert, find, and delete are O(1) operations?
    \begin{answer}[1in]
      Create an array of size 1,000,000,000.
      Use the student ID as the index into the array.
      Store the student record at that index.
    \end{answer}

  \Q  How much memory does your solution in \#3 take?
    \begin{answer}[1in]
      It takes memory for an array of size 1,000,000,000.
      Each entry in the array stores a student record (or null if no student with that ID is enrolled).
    \end{answer}

  \par\vskip 10pt

  Consider that there are about 4000 currently enrolled WWU students.
  If you implement an array of size 1,000,000,000 to store (student ID, student record) pairs, the array is sparse.
  Certainly, the insert, delete, and find operations are O(1) with this implementation, but the wasted space is quite large given the number of items in the dictionary.

  \par\vskip 10pt

  Here is one way to reduce the size of the array.
  If we know there are 4000 items in the dictionary, let's instead create an array of size 100,000 to store these items.
  Note that there are still 96,000 empty cells with this approach, but that is quite a bit smaller than 1 billion.

  \par\vskip 10pt

  So, in order to insert a (student ID, student record) pair into the dictionary of size 100,000, we need to map student IDs to the range [0...99,999].
  One common way to do this is to use the mod operator:

  \begin{cpplst}
location = studentID % 100000;
  \end{cpplst}

  The mod (remainder) operator gives us values in the appropriate range.

  \Q What is your student ID? \hfill\ans{Answers will vary}

  \Q In what array location would your ID be inserted, assuming the array has 100,000 items?
    \begin{answer}[0.3in]
      Answers will vary
    \end{answer}

  \Q Consider student ID 1152436. In what array location would this ID be inserted?
    \begin{answer}[0.3in]
      52436
    \end{answer}

  \Q  What might be a downside to this approach?
    \begin{answer}[0.3in]
      Different student IDs might map to the same array location.
      For example, student ID 2152436 would also map to location 52436.
      This is called a collision.
    \end{answer}

  Because we have limited the array size, it is now possible to have collisions.
  A collision happens when two dictionary items are mapped to the same location in the array.

  \Q What is another student ID number that would be mapped to the same array location as 1152436?
    \begin{answer}[0.3in]
      152436, 2152436, 3152436, ...
    \end{answer}
  \newpage
  
\model{Hash Functions}

  \quest{10 min}

  Above, we have considered the situation where the items are numbers to insert into the dictionary.
  Suppose now we want to insert strings into the dictionary (like an actual dictionary!!).
  We first need to map the string to a number.
  In this case, the string is our key and the number is our hash value.
  Once we have the value, we need to map the value to a location in the array.

  \par\vskip 10pt

  The mapping from a key to an integer is called the hash function.
  There are many implementations of hash functions.
  A good hash function spreads keys across the range of integers.
  A good hash function is fast to compute, given the length of the key.
  A really bad hash function is one that maps all keys to the value 6.
  You will learn more about creating good hash functions in CS 324, Algorithms, if you take that course.
  Another property of a good hash function is that it maps the strings ``abc'' and ``cba'' to different integers, so permutations of the same set of letters have different hash values.

  \par\vskip 10pt

  Another property of a good hash function is that if two keys are considered equal, they will map to the same hash value.
  For example, if uppercase ``SAM'' and lowercase ``sam'' are considered equivalent in your application, then they should map to the same hash value.

  \par\vskip 10pt

  Here is an example of a hash function:

  \begin{cpplst}
unsigned int hash(std::string key) {
  unsigned int rtnVal = 3253;
  for (int i = 0; i < key.size(); i++) {
    rtnVal *= 28277;
    rtnVal += key[i] * 2749;
  }
  return rtnVal;
}
    \end{cpplst}

  So, hash("a") is:

  \begin{cpplst}
rtnVal = 3253
rtnVal = 3253 * 28277 // 91985081
rtnVal = 91985081 + 97*2749 // note: 'a' in ASCII is 97
rtnVal = 92251734
  \end{cpplst}

  \vskip -10pt

  \Q What is hash(``ab'')? \hfill\ans{It's 2611387551}

  \vskip -10pt

  \Q What is hash(``ba'')? (Note: 'b' in ASCII is value 98)\key\\[-2.5mm] \hfill\ans{It's 2611452044}

  \vskip -15pt

  \Q Explain one reason a Dictionary ADT might use a hash function?\key\\[-2.5mm]
    \begin{answer}[0.2in]
      A Dictionary ADT might use a hash function to map keys (like strings) to integer values
      that can then be used to determine the location in an array where the corresponding value should be stored.
    \end{answer}
  \newpage
  \model{Hash Tables}

  \quest{10 min}

  The array that was mentioned earlier is called a hash table. A hash table stores dictionary items. The
  storage location is based on the hash value and using \% of the hash table size.

  \texttt{Insertion}

  Assume we have a hash table of size 10. We want to insert the following items:
    \begin{itemize}
      \item "coconut''
      \item ``milk''
      \item ``apple''
    \end{itemize}

  STEP 1: We'll use the hash function listed above to calculate the hash values for these keys:
    \begin{itemize}
      \item ``coconut'' (hash value = 2104178476)
      \item ``milk'' (hash value = 461110994)
      \item ``apple'' (hash value = 3515030035)
    \end{itemize}

  STEP 2: We'll map the hash values to the size for this hash table (\% 10):
    \begin{itemize}
      \item  ``coconut'' (location = 6)
      \item ``milk'' (location = 4)
      \item ``apple'' (location = 5)
    \end{itemize}

  \includegraphics[width=.9\textwidth]{figures/hashing1.png}

  \Q Now, suppose we want to insert ``orange'' into this hash table.
    Its hash value is 3410197053.
    Insert this key into the table above.
    \begin{answer}[1in]
      It goes into position 3.
    \end{answer}

  \Q Now, let's insert ``corn''. Its hash value is 1347376851. Insert this key into the table above.
    \begin{answer}[1in]
      It goes into position 1.
    \end{answer}

  \Q Now, let's insert ``eggs''. Its hash value is 881505635. Insert this key into the table above.
    \begin{answer}[1in]
      It goes into position 5, but that position is already taken by ``apple''.
      This is called a collision.
    \end{answer}

  What happens now?
  ``apple'' is already stored at position 5.
  There are several techniques (see textbook) to address collisions.

  \par\vspace{10pt}

  We'll first resolve collisions using open address linear probing.
  \newpage
  \input{open_address_linear_probing.tex}

\end{document}