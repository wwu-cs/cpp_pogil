\model{Comparing}

In computing, we often need to \textbf{sort} a set of items. As computer scientists,
we study ways to sort very large sets, with thousands or millions of values,
since searching and other operations are often easier when the set of values is sorted.

For example, the Harvard University Library has roughly 16,000,000 volumes, and the
US Library of Congress has roughly 22,000,000 books, and over 100,000,000 total items.
In 2010, a team at UC San Diego sorted one trillion ($10^{12}$) data records in 172 minutes.
Simple $O(N^2)$ sorting algorithms work well for small lists, but are too slow for larger lists.
Most software libraries (APIs) include excellent sorting algorithms,  but exploring better
sorting algorithms also demonstrates more general concepts in algorithm design and analysis.

\begin{cpplst}
public    int[] sort(   int[] d) { ... } 
public String[] sort(String[] d) { ... } 
public   City[] sort(  City[] d) { ... } 
 
public void demo() { 
  int   [] iu = { 6,3,8,2,9 }; 
  int   [] is = sort(iu); 
  String[] su = { "banana", "grape", "apple", "mango" }; 
  String[] ss = sort(su); 
  City  [] cu = { City.NYC, City.LAX, City.PHL, City.CHI }; 
  City  [] cs = sort(cu); 
}
\end{cpplst}



      \Q (1 min) Each \textbf{sort()} method above takes an array input and returns a sorted array.
            What is \textbf{different} about the first 3 methods?

            \begin{answer}[0.6in]
                  The 1st works on ints,  the 2nd on Strings; the 3rd on Cities.
            \end{answer}


      \Q (2 min) What sequence of values should appear in each variable below:

            \begin{center}
                  \renewcommand{\arraystretch}{2}
                  \begin{tabular}{ |c|m{3cm}|m{11cm}| }
                        \hline
                        a. & is & \ifprintanswers is = { 2,3,6,8,9 } \fi                           \\
                        \hline
                        b. & ss & \ifprintanswers ss = { "apple", "banana", "grape", "mango" } \fi \\
                        \hline
                  \end{tabular}
            \end{center}

            \newpage

      \Q   (1 min) In the table below, specify tests for \textbf{sort()},

            \begin{center}
                  \renewcommand{\arraystretch}{2}
                  \begin{tabular}{ |c|l|m{11cm}| }
                        \hline
                        \rowcolor{lightgray} & \textbf{Input: array to sort} & \textbf{Expected Result}                \\
                        \hline
                        a.                   & [ ] (empty list)              & empty list                              \\
                        \hline
                        b.                   & [ 5 ]                         & 5                                       \\
                        \hline
                        c.                   & [ 2, 4, 8 ]                   & \ifprintanswers { 2,4,8 } \fi           \\
                        \hline
                        d.                   & [ 9, 7, 3, 5 ]                & \ifprintanswers { 3,5,7,9 } \fi         \\
                        \hline
                        e.                   & [ 27, 42, 35 ]                & \ifprintanswers { 27,35,42 } \fi        \\
                        \hline
                        f.                   & [ "a" ]                       & \ifprintanswers { a } \fi               \\
                        \hline
                        g.                   & [ "a", "b", "c" ]             & \ifprintanswers { a, b, c } \fi         \\
                        \hline
                        h.                   & [ "c", "b", "a" ]             & \ifprintanswers { a, b, c } \fi         \\
                        \hline
                        i.                   & [ "a3", "a1", "a2" ]          & \ifprintanswers { a1, a2, a3 } \fi      \\
                        \hline
                        j.                   & [ "bar", "ball", "back" ]     & \ifprintanswers { back, ball, bar } \fi \\
                        \hline
                  \end{tabular}
            \end{center}

      \Q  (2 min) It is easier to compare integers than Strings. Explain why.

            \begin{solution}[0.6in]
                  Integers have a single value, but Strings have sets of character values.
                  Integers have one clear ordering, but Strings have upper/lower case, unusual characters, etc.
            \end{solution}

            Suppose we had a list (e.g. a database or spreadsheet) of detailed information about cities:

            \begin{center}
                  \begin{tabular}{ |c|c|c|c|c|c| }
                        \hline
                        \rowcolor{lightgray} \textbf{Name} & \textbf{Area (sq mi)} & \textbf{Population} & \textbf{Altitude} & \textbf{Latitude} & \textbf{Longitude} \\
                        \hline
                        Chicago                            & 227.6                 & 2,700,000           & ...               & ...               & ...                \\
                        \hline
                        Los Angeles                        & 468.7                 & 3,800,000           & ...               & ...               & ...                \\
                        \hline
                        New York                           & 302.6                 & 8,200,000           & ...               & ...               & ...                \\
                        \hline
                        ...                                & ...                   & ...                 & ...               & ...               & ...                \\
                        \hline
                  \end{tabular}
            \end{center}


      \Q (2 min) Explain why it could be harder to compare Cities than Strings.


            \begin{solution}[0.6in]
                  Cities could be compared in many different ways - population, area, location, etc.
            \end{solution}

      \Q (3 min) It can be useful to \textbf{abstract} the comparisons used in searching & sorting.

            \begin{enumerate}
                  \item In the \textbf{Sample Code} handout, what \textbf{method} is declared in interface Comparable?

                        \begin{solution}[0.6in]
                              \texttt{int compareTo(T that);}
                        \end{solution}


                  \item The Java API method \texttt{Math.signum()} returns 0 if its input is 0,
                        +1 if its input is positive, and -1 if its input is negative.
                        In the Sample Code handout, how does class City compare two cities?


                        \begin{solution}[0.6in]
                              It subtracts their areas, and takes the signum of the result (-1,0,+1).
                        \end{solution}



                  \item The Java API class String implements interface Comparable.
                        Given this, what Java expression would compare two Strings: s1 and s2?

                        \begin{solution}[0.6in]
                              \texttt{s1.compareTo(s2)}
                        \end{solution}


            \end{enumerate}
      \Q (4 min) In class City, modify the return statement to compare Cities using:
            \begin{center}
                  \renewcommand{\arraystretch}{2}
                  \begin{tabular}{ |c|m{2.3cm}|m{11cm}| }
                        \hline
                        a. & population                  & \ifprintanswers return (int) Math.signum(this.getPop () - that.getPop () ); \fi \\
                        \hline
                        b. & name (alphabetical)         & \ifprintanswers return this.getName().compareTo( that.getName() ); \fi          \\
                        \hline
                        c. & name (reverse alphabetical) & \ifprintanswers return that.getName().compareTo( this.getName() ); \fi          \\
                        \hline
                        d. & population density          & \ifprintanswers return (int) Math.signum(
                              this.getPop () / this.getArea()
                              - that.getPop () / that.getArea() );
                              return (int) Math.signum(
                              this.getDens() - that.getDens() ); \fi                                                                       \\
                        \hline
                  \end{tabular}
            \end{center}

      \Q (1 min) Comparison can be abstracted in more than one way.
            In the Sample Code handout, what methods are declared in interface Comparator?

            \begin{solution}[0.6in]
                  \texttt{int     compare(T o1, T o2);}
                  \texttt{boolean equals (Object obj);}
            \end{solution}


      \Q (2 min) For each class listed below, describe how it compares 2 objects.
            \begin{center}
                  \renewcommand{\arraystretch}{2}
                  \begin{tabular}{ |c|m{2.3cm}|m{11cm}| }
                        \hline
                        a. & \texttt{class StringCompI} & \ifprintanswers compares 2 Strings alphabetically, ignoring uppercase/lowercase differences. \fi \\
                        \hline
                        b. & \texttt{class StringCompL} & \ifprintanswers compares 2 Strings using their length (not their contents) \fi                   \\
                        \hline
                        c. & \texttt{class CityCompA}   & \ifprintanswers compares 2 Cities using their area \fi                                           \\
                        \hline
                        d. & \texttt{class CityCompP}   & \ifprintanswers compares 2 Cities using their population \fi                                     \\
                        \hline
                  \end{tabular}
            \end{center}


      \Q (3 min) Change the return statement in \texttt{CityCompA.compare()} to compare:
            \begin{center}
                  \renewcommand{\arraystretch}{2}
                  \begin{tabular}{ |c|m{2.3cm}|m{11cm}| }
                        \hline
                        a. & name (alphabetical)         & \ifprintanswers return c1.getName().compareTo( c2.getName() ); \fi \\
                        \hline
                        b. & name (reverse alphabetical) & \ifprintanswers return c2.getName().compareTo( c1.getName() ); \fi \\
                        \hline
                        c. & population density          & \ifprintanswers return (int)Math.signum(
                              c1.getPop () / c1.getArea()
                              - c2.getPop () / c2.getArea() );
                              return (int)Math.signum(
                              c1.getDens() - c2.getDens() );  \fi                                                             \\
                        \hline
                  \end{tabular}
            \end{center}
