\model{Stacks - Array Implementation}

  A stack ADT can be implemented in multiple ways, as you will see.
  For now, we will consider how to implement a stack using an array, e.g.:

  \begin{cpplst}
template<typename T>
class Stack {
private:
    T* data;      // array to store stack elements
    int capacity; // maximum size of stack
    int size;     // current number of elements

public:
    Stack(int maxSize) {
        this->capacity = maxSize;
        this->size = 0;
        this->data = new T[maxSize];
    }

    // methods
};
  \end{cpplst}

  {\it\large Refer to Model 3 above as your team develops consensus answers
    to the questions below.}

  \quest{10 min}

  \Q \textbf{Show the contents of the data array after each operation.}
    Assume we start with an empty stack. If you use additional fields (e.g. to keep track
    of positions in the array),  add a column for each one and show how its value changes.
    \begin{center}
      \begin{tabular}{ |r|l|c|c|c|c|c|c|c| }
        \hline
        \rowcolor{lightgray} & \textbf{Operation}         & \textbf{[0]}  & \textbf{[1]}  & \textbf{[2]} & \textbf{[3]} & \ans[.2in]{[4]} & \ans[.2in]{[5]} & \ans[.2in]{size} \\
        \hline
        a.                   & Create new data structure. & \ans[.2in]{}  & \ans[.2in]{}  & \ans[.2in]{} & \ans[.2in]{} & \ans[.2in]{} & \ans[.2in]{} & \ans[.2in]{0}    \\
        \hline
        b.                   & Add value 'A'.             & \ans[.2in]{A} & \ans[.2in]{}  & \ans[.2in]{} & \ans[.2in]{} & \ans[.2in]{} & \ans[.2in]{} & \ans[.2in]{1}    \\
        \hline
        c.                   & Add value 'B'.             & \ans[.2in]{A} & \ans[.2in]{B} & \ans[.2in]{} & \ans[.2in]{} & \ans[.2in]{} & \ans[.2in]{} & \ans[.2in]{2}    \\
        \hline
        d.                   & Remove value.              & \ans[.2in]{A} & \ans[.2in]{}  & \ans[.2in]{} & \ans[.2in]{} & \ans[.2in]{} & \ans[.2in]{} & \ans[.2in]{1}    \\
        \hline
        e.                   & Add value 'C'.             & \ans[.2in]{A} & \ans[.2in]{C} & \ans[.2in]{} & \ans[.2in]{} & \ans[.2in]{} & \ans[.2in]{} & \ans[.2in]{2}    \\
        \hline
        f.                   & Get current size.          & \ans[.2in]{A} & \ans[.2in]{C} & \ans[.2in]{} & \ans[.2in]{} & \ans[.2in]{} & \ans[.2in]{} & \ans[.2in]{2}    \\
        \hline
        g.                   & Remove value.              & \ans[.2in]{A} & \ans[.2in]{}  & \ans[.2in]{} & \ans[.2in]{} & \ans[.2in]{} & \ans[.2in]{} & \ans[.2in]{1}    \\
        \hline
        h.                   & Add value 'D'.             & \ans[.2in]{A} & \ans[.2in]{D} & \ans[.2in]{} & \ans[.2in]{} & \ans[.2in]{} & \ans[.2in]{} & \ans[.2in]{2}    \\
        \hline
      \end{tabular}
    \end{center}

  \newpage

  \Q For each stack method, write a complete English sentence
    to describe \textbf{how it could\key\\[-2.5mm] be implemented} using an array, and its \textbf{O() performance}.

    Hint: Ideally, the methods should be O(1), or O(N) at worst.

    \begin{center}
      \renewcommand{\arraystretch}{2}
      \begin{tabular}{ |r|l|l|l| }
        \hline
        \rowcolor{lightgray} & \textbf{method}         & \textbf{implementation}                                  & \textbf{O()}  \\
        \hline
        a.                   & \ans[1.5in]{add or push}   & \ans[4in]{this.data[ this.size ] = value; this.size++;}     & \ans[.2in]{1} \\
        \hline
        b.                   & \ans[1.5in]{get or peek}   & \ans[4in]{return this.data[ this.size - 1 ];  }           & \ans[.2in]{1} \\
        \hline
        c.                   & \ans[1.5in]{remove or pop} & \ans[4in]{this.start--; return this.data[ this.size ]; } & \ans[.2in]{1} \\
        \hline
        d.                   & \ans[1.5in]{size}       & \ans[4in]{return this.size;}                             & \ans[.2in]{1}  \\
        \hline
        e.                   & \ans[1.5in]{}           & \ans[4in]{}                                              & \ans[.2in]{}  \\
        \hline
      \end{tabular}
    \end{center}

  Review progress with the facilitator before continuing.
    \begin{answer}[0.5in]
      see Facilitator - Sample Code
    
      REPORT OUT: implementation descriptions and O() performance, so teams have common framework
    \end{answer}