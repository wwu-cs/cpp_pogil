\model{Stacks - Abstract Data Type}

  Note that stacks are often described as \textbf{Last-In, First-Out (LIFO)}
  or \textbf{First-In, Last-Out (FILO)} (particularly if you are Greek :-) ).

  \quest{5 min}

  \Q Based on the key characteristics of stacks that you identified above,
    \textbf{list at least 3 key operations} for a stack ADT, and \textbf{rank them} (last column)
    by \textbf{importance} (1=high, 5=low).
    \begin{center}
      \begin{tabular}{ |r|p{12cm}|l| }
        \hline
        \rowcolor{lightgray} & \textbf{action or operation} & \textbf{rank} \\
        \hline
        a.                   &                              &               \\
        \hline
        b.                   &                              &               \\
        \hline
        c.                   &                              &               \\
        \hline
        d.                   &                              &               \\
        \hline
      \end{tabular}
    \end{center}

  \par\vskip -20pt

  \Q Start with the most important stack operation above, and define
    a \textbf{method signature}\key\\[-2.5mm] including an appropriate name, input parameters, and return types.
    You may write the method below, or create a class file in your IDE.
    Use "T" as a generic placeholder for the type of object stored in the stack.
    \begin{cpplst}
template<typename T>
class Stack {
public:
    Stack(int maxSize);  // constructor

    ?

    ?

    ?

    ?

    ?

};  // end class Stack
    \end{cpplst}

    Review progress with the facilitator before continuing.

    \begin{answer}[0.5in]
      see Facilitator - Sample Code

      REPORT OUT: operations and method signatures, so teams have common framework
      Maybe have teams compare their class to the C++ STL stack, Wikipedia,
      or their textbook, and summarize any differences, insights, or questions.
    \end{answer}