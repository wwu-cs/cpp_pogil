\model{Queues - Real World}

  Consider the following situations:

  \begin{enumerate}
    \item People standing in a checkout line at a store or fast-food restaurant.
    \item Suitcases or packages on a conveyor belt.
    \item Phone callers on hold waiting for the "next available customer service representative".
    \item Documents waiting to be printed on a printer.
  \end{enumerate}

  {\it\large Refer to Model 4 above as your team develops consensus answers
    to the questions below.}

  \quest{10 min}

  \Q These situations are all examples of \textbf{queues}.
    Summarize their key common characteristics in 2-4 complete English sentences.
    \begin{answer}[1in]
      Queues are a first-in, first-out sequence - items are removed in the order they were added.
    \end{answer}

  \Q List any variations or exceptions to these characteristics.
    \begin{answer}[1in]
      Variations include one queue with multiple outputs (e.g. line at a bank),
      or queues with priorities (higher priorities are processed before lower priorities).
      TODO: skip this Q or rephrase to be more directed

      REPORT OUT: key characteristics and variations, so teams have a common understanding.
    \end{answer}

  \newpage

  \Q (3 min) Trace the following actions in a queue (steps a through e are provided as examples).

    \begin{center}
      \newcolumntype{s}{>{\columncolor{lightgray}} p{.25cm}}
      \begin{tabular}{ |r|l|p{.25cm}|l|l|l|l|p{.25cm}|l| }
        \hline
        \rowcolor{lightgray} & \textbf{action}   &  & \multicolumn{4}{|c|}{\textbf{contents}} &              & \textbf{outcome}                                                              \\
        \hline
        a.                   & empty queue       &  &                                         &              &                  &              &  &                                          \\
        \hline
        b.                   & get first item    &  &                                         &              &                  &              &  & ERROR - empty queue                      \\
        \hline
        c.                   & add A             &  & A                                       &              &                  &              &  &                                          \\
        \hline
        d.                   & add B             &  & A                                       & B            &                  &              &  &                                          \\
        \hline
        e.                   & remove first item &  & B                                       &              &                  &              &  & A is removed                             \\
        \hline
        f.                   & get first item    &  & \ans[1em]{B}                            & \ans[1em]{}  & \ans[1em]{}      & \ans[1em]{}  &  & \ans[2.5in]{B is returned (not removed)} \\
        \hline
        g.                   & add C             &  & \ans[1em]{B}                            & \ans[1em]{C} & \ans[1em]{}      & \ans[1em]{}  &  & \ans[2.5in]{}                            \\
        \hline
        h.                   & add D             &  & \ans[1em]{B}                            & \ans[1em]{C} & \ans[1em]{D}     & \ans[1em]{}  &  & \ans[2.5in]{}                            \\
        \hline
        i.                   & get first item    &  & \ans[1em]{B}                            & \ans[1em]{C} & \ans[1em]{D}     & \ans[1em]{}  &  & \ans[2.5in]{B is returned (not removed)} \\
        \hline
        j.                   & add E             &  & \ans[1em]{B}                            & \ans[1em]{C} & \ans[1em]{D}     & \ans[1em]{E} &  & \ans[2.5in]{} \\
        \hline
        k.                   & add F             &  & \ans[1em]{B}                            & \ans[1em]{C} & \ans[1em]{D}     & \ans[1em]{E} &  & \ans[2.5in]{possible stack full, depends on implementation} \\
        \hline
      \end{tabular}
    \end{center}

  \begin{center}
    \includegraphics[width=2in]{figures/alice.jpeg}
  \end{center}

  \begin{quote}
    The Queen had only one way of settling all difficulties, great or small.

    'Off with his head!' she said, without even looking round.

    --- Lewis Carroll, Alice in Wonderland
  \end{quote}