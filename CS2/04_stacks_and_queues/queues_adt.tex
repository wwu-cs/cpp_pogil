\model{Queues - Abstract Data Type}

  Note that queues are often described as \textbf{First-In, First-Out (FIFO)}
  or \textbf{Last-In, Last-Out (LILO)} (particularly if you are lazy :-) ).

  \quest{5 min}

  \Q Based on the key characteristics of queues that you identified above,
    \textbf{list at least 3 key operations} for a queue ADT, and \textbf{rank them} (last column)
    by \textbf{importance} (1=high, 5=low).
    Review progress with the facilitator before continuing.
    \begin{center}
      \begin{tabular}{ |r|p{12cm}|l| }
        \hline
        \rowcolor{lightgray} & \textbf{action or operation} & \textbf{rank} \\
        \hline
        a.                   &                              &               \\
        \hline
        b.                   &                              &               \\
        \hline
        c.                   &                              &               \\
        \hline
        d.                   &                              &               \\
        \hline
      \end{tabular}
    \end{center}

  \par\vskip -20pt

  \Q Start with the most important queue operation above, and define
    a method\key\\[-2.5mm] signature including an appropriate name, input parameters, and return types.
    You may write the method below, or create a class file in your IDE.
    Use "T" as a generic placeholder for the type of object stored in the queue.
    \begin{cpplst}
template<typename T>
class Queue {
public:
    Queue(int maxSize);  // constructor

    ?

    ?

    ?

    ?

    ?

};  // end class Queue
    \end{cpplst}

  Review progress with the facilitator before continuing.
  \begin{answer}
    see Facilitator - Sample Code
    REPORT OUT: operations and method signatures, so teams have common framework
    Maybe have teams compare their class to the C++ STL queue, Wikipedia,
    or their textbook, and summarize any differences, insights, or questions.
  \end{answer}