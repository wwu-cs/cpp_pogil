\model{Stacks - Real World}

  Consider the following situations:

  \begin{itemize}
    \item A pile of papers or books on a table, where the top item is usually
      the most recently added.
    \item In arithmetic expressions, each ")" matches the most recent unmatched "(".
      In Java programming, each "\}" matches the most recent unmatched "\{".
      In HTML, each </DIV> matches the most recent unmatched <DIV>.
      In each case, a parsing program must keep track of all unmatched symbols.
    \item If method A calls B, and B calls C, then when C ends control returns to B,
      and when B ends control returns to A.
      (Control returns to the most recently called method).
    \item As people, if we have too many tasks or questions of similar importance,
      we tend to focus on the most recent ones.
  \end{itemize}

  {\it\large Refer to Model 1 above as your team develops consensus answers
    to the questions below.}

  \quest{10 min}

  \Q These situations are all examples of \textbf{stacks}.
    Summarize their key common characteristics in 2-4 complete English sentences.
    \begin{answer}[1in]
      Stacks are first-in, last-out (or last-in, first-out) - the most recently added item is the next removed.
    \end{answer}

  \Q List any variations or exceptions to these characteristics.
    \begin{answer}[1in]
      Variations include stacks with multiple ways to access other element
      TODO: skip this Q or rephrase to be more directed

      REPORT OUT: key characteristics and variations, so teams have a common understanding.
    \end{answer}

  \newpage

  \Q Trace the following actions in a stack (steps a through e are provided as examples).
    \begin{center}
      \newcolumntype{s}{>{\columncolor{lightgray}} p{.25cm}}
      \begin{tabular}{ |r|l|p{.25cm}|l|l|l|l|p{.25cm}|l| }
        \hline
        \rowcolor{lightgray} & \textbf{action} &  & \multicolumn{4}{|c|}{\textbf{contents}} &              & \textbf{outcome}                                                                                 \\
        \hline
        a.                   & empty stack     &  &                                         &              &                  &              &  &                                                             \\
        \hline
        b.                   & get top item    &  &                                         &              &                  &              &  & ERROR - empty stack                                         \\
        \hline
        c.                   & add A           &  & A                                       &              &                  &              &  &                                                             \\
        \hline
        d.                   & add B           &  & A                                       & B            &                  &              &  &                                                             \\
        \hline
        e.                   & remove top item &  & A                                       &              &                  &              &  & B is removed                                                \\
        \hline
        f.                   & get top item    &  & \ans[1em]{A}                            & \ans[1em]{}  & \ans[1em]{}      & \ans[1em]{}  &  & \ans[2.5in]{A is returned (not removed)}                                               \\
        \hline
        g.                   & add C           &  & \ans[1em]{A}                            & \ans[1em]{C} & \ans[1em]{}      & \ans[1em]{}  &  & \ans[2.5in]{}                                               \\
        \hline
        h.                   & add D           &  & \ans[1em]{A}                            & \ans[1em]{C} & \ans[1em]{D}     & \ans[1em]{}  &  & \ans[2.5in]{}                                               \\
        \hline
        i.                   & get top item    &  & \ans[1em]{A}                            & \ans[1em]{C} & \ans[1em]{D}     & \ans[1em]{}  &  & \ans[2.5in]{D is returned (not removed)}                    \\
        \hline
        j.                   & add E           &  & \ans[1em]{A}                            & \ans[1em]{C} & \ans[1em]{D}     & \ans[1em]{E} &  & \ans[2.5in]{}                                               \\
        \hline
        k.                   & add F           &  & \ans[1em]{A}                            & \ans[1em]{C} & \ans[1em]{D}     & \ans[1em]{E} &  & \ans[2.5in]{possible stack full, depends on implementation} \\
        \hline
      \end{tabular}
    \end{center}

  \vspace{20pt}

  \begin{quote}
    Otherwise, my whole career has just been flinging myself

    at whatever is most overdue first and letting everything else stack up.

    -- Cathy Guisewite, American cartoonist, 1950-
  \end{quote}