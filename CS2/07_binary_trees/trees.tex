\model{Trees}

Below is an hierarchical filesystem storing a collection of files. Files marked by arrows are directories.
Files marked by bullets are data files.

\vspace{0.2in}

\begin{figure}[h]
    \begin{center}
        \begin{minipage}{3in}
            \begin{compactitem}
                \item /home
                \begin{compactitem}
                    \item /Documents
                    \begin{compactitem}
                        \item report.txt
                        \item source.txt
                        \item lab\_3.txt
                    \end{compactitem}
                    \item /Desktop
                    \begin{compactitem}
                        \item midterm.doc
                        \item contacts
                        \item model4
                    \end{compactitem}
                    \item /Pictures
                    \begin{compactitem}
                        \item 332450.png
                        \item profile.jpg
                    \end{compactitem}
                    \setdefaultitem{$\rightarrow$}{$\rightarrow$}{$\rightarrow$}{}
                    \begin{compactitem}
                        \item /vacation2012
                    \end{compactitem}
                \end{compactitem}
            \end{compactitem}
        \end{minipage}
    \end{center}
    \caption{Filesystem}
    \label{fig:filesystem}
\end{figure}


\Q How many directories are there? \ans{}

\Q How many data files are there? \ans{}

\Q Imagine this filesystem being represented visually as it would be on your computer: as a series of folders, windows, and files.
   How would you navigate from the {\tt /home} directory to {\tt model4}?


\begin{answer}[0.5in]
\end{answer}

\Q Are there any files which can not be reached from the {\tt /home} directory ?


\begin{answer}[0.5in]
\end{answer}

\Q Using only Figure \ref{fig:filesystem}, is the {\tt /home} directory located inside of any other directories?


\begin{answer}[0.5in]
\end{answer}

\newpage
\Q Illustrate the filesystem beginning with the {\tt /home} directory provided below.
    For files, draw a node (\includegraphics[scale=0.3]{figures/circle.pdf}).
    To illustrate that a file is located within another, draw a directed edge (\includegraphics[scale=0.5]{figures/directedExample.pdf}).


\centerline{\includegraphics[scale=0.55]{figures/initial_tree.pdf}}

\vspace{4in}

\Q
    Compare your illustration to the structure below. Does the orientation of the information affect how you interpret
    the information being conveyed?

\begin{figure}[h]
    \begin{center}
        \includegraphics[scale=0.5]{figures/tree_graph.pdf}
    \end{center}
    \caption{Alternative structure}
    \label{fig:graph}
\end{figure}



\begin{answer}[.5in]
\end{answer}

\Q
    Is it clear in Figure \ref{fig:graph} that the {\tt /home} node is in any way distinguished?


\begin{answer}[1in]
\end{answer}

\Q
    In your illustration, the {\tt /home} directory node is the \emph{root} of the structure.
    Define what it means for a node to be the root.


\begin{answer}[1in]
\end{answer}

\Q
    The graph you drew is an illustration of a \emph{tree data structure}.
    Using the terms "node," "edge," and "root," define tree data structure.


\begin{answer}[1in]
\end{answer}

\Q
    {\tt report.txt} is a \emph{child} node of the {\tt /Documents} node. The {\tt /Documents} node, in turn, is the
    \emph{parent} of the {\tt report.txt} node. What does it mean for a node to be a child? A parent?


\begin{answer}[1in]
\end{answer}

\Q
    Are there any nodes in the tree that do not have a parent? How many?


\begin{answer}[.5in]
\end{answer}

\Q
    Are there any nodes in the tree that do not have any children? How many?


\begin{answer}[.5in]
\end{answer}

\Q
    {\tt report.txt}, {\tt model4}, and {\tt /vacation2012} are examples of \emph{leaf} nodes.
        {\tt /home} and {\tt /Pictures} are not.
    Define what it means for a node to be a leaf.


\begin{answer}[1in]
\end{answer}

\Q
    Starting from the {\tt /home} node how many edges are traversed navigating to the {\tt contacts} node?


\begin{answer}[1in]
\end{answer}

\Q
    The {\tt contacts} node is at a \emph{depth} of $2$. Define the depth of a node.


\begin{answer}[1in]
\end{answer}

\Q
    How deep is the tree? \ans{}

\Q
    If a tree has $16$ nodes, what is the maximum possible depth of the tree? The minimum?


\begin{answer}[1in]
\end{answer}

\Q
    How many children does {\tt /home} have? {\tt /Documents}? {\tt /vacation2012}?


\begin{answer}[1in]
\end{answer}

\Q
    The tree that you drew has a \emph{branching factor} of $3$. Define branching factor.


\begin{answer}[1in]
\end{answer}

\Q
    What is the branching factor of the tree below?

    \begin{figure}[h]
        \centering
        \includegraphics[scale=0.5]{figures/binary.pdf}
        \caption{}
        \label{fig:binary}
    \end{figure}


\begin{answer}[.5in]
\end{answer}

\Q
    The tree in Figure \ref{fig:binary} is a \emph{binary} tree. Define binary tree.


    \begin{answer}[1in]
    \end{answer}
    

\Q
    If a binary tree has $16$ nodes, what is the maximum possible depth of the tree? The minimum?


\begin{answer}[1in]
\end{answer}
\newpage

\Q
    Illustrate the management structure within an organization with the following employees:

    \begin{center}
        \begin{tabular}[]{l}
            President                         \\
            Director of IT                    \\
            Director of Customer Service (CS) \\
            Director of Sales                 \\
            IT Manager                        \\
            CS Supervisor                     \\
            Sales Coordinator                 \\
            Telephone Support                 \\
            IT Specialist                     \\
            Salesman
        \end{tabular}
    \end{center}

    \begin{answer}[3in]
    \end{answer}

\Q  What data structure did you choose to use?
    What properties of the data structure motivated that choice?


% \newpage

% \Q
%     Illustrate the connections that exist between webpages in a website that are connected by hyperlinks.
%     At minimum, assume that all of the pages are accessible from the homepage. For the sake of brevity,
%     do not include more than $8$ webpages. What data structure did you choose to use?
%     What properties of the data structure motivated that choice?


% \begin{answer}[2in]
% \end{answer}

% \begin{figure}
%     \begin{center}
%         \begin{lstlisting}
%     class TreeNode:
%         private array<TreeNode> children
%         private String data
        
%         public TreeNode(String data):
%             children = new array()
%             this.data = data
            
%         public String getData():
%             return data
            
%         public void addChild(TreeNode child):
%             children.append(child)
            
%         public String listChildren():
%             String output
        
%             for TreeNode n in children:
%                 output + n.getData + "\n"
            
%             return output
            
%         public array<TreeNode> getChildren():
%             return children
            
%     end class TreeNode
% \end{lstlisting}
%     \end{center}
%     \caption{A simplified Tree implementation}
%     \label{fig:implementation}
% \end{figure}

% \Q Building upon the initial pseudocode provided below, construct the tree in Figure \ref{fig:impTree} using the {\tt TreeNode} class in Figure \ref{fig:implementation}


% \begin{figure}
%     \begin{center}
%         \includegraphics[scale=0.5]{figures/tree_final.pdf}
%     \end{center}
%     \caption{}
%     \label{fig:impTree}
% \end{figure}


% \begin{lstlisting}
%     TreeNode root = new TreeNode("John")
% \end{lstlisting}

% \begin{answer}[1in]
% \end{answer}