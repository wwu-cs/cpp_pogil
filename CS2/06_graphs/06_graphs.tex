% Source: https://www.dropbox.com/sh/rl0yyth9g06psva/AADB0Cj4isIX5DAyrspqj8mFa
% File: "CS305 activity 15 graphs.pdf"
% Access: 05-05-2022

% comment out for student version
% \ifdefined\Student\relax\else\def\Teacher{}\fi

\documentclass[12pt]{article}

\title{Activity \#6: Graphs}
\author{Tammy VanDeGrift}
\newcommand{\activityeditor}{Preston Carman}
\newcommand{\activitysource}{\url{https://www.dropbox.com/sh/rl0yyth9g06psva/AADB0Cj4isIX5DAyrspqj8mFa}}
\date{Spring 2022}

\input{../../cspogil.sty}
\usepackage{amssymb}

\begin{document}

  \begin{center}
    \maketitle
    \rolenames
  \end{center}

  \keyquestions{
    \item Model 1, Question \#20
    \item Model 2, Question \#21
    \item Model 2, Question \#22
    \item Model 3, Question \#23
    \item Model 3, Question \#32
  }

  \newpage
  \maketitle

  Many items in the world are connected, such as computers on a network connected by wires, cities connected by roads, or people connected by friendship.
  A graph is a data structure for representing connections among items, and consists of vertices connected by edges. 

  %\rolenames

  \guide{
    \item Explain the components of a graph (vetices and edges)
    \item Describe the difference between undirected and directed graphs
    \item Describe the cycle in a graph
    \item Explain how to represent a graph as a data structure
  }{
    \item Find cycles in a graph
    \item Translate a graph into an adjacency list or adjacency matrix
  }{
    No additional notes.
  }

  
\model{Model 1: Graphs}

A graph in computer science consists of vertices (also called nodes) and edges (also called links).
Nodes are generally depicted as circles and labeled with a unique identifier.
Edges are generally depicted as lines for undirected graphs and arrows for directed graphs.
See how much you can glean from the graph below.

\includegraphics[width=.9\textwidth]{figures/graph1.png}

\Q What are the vertices?  \ans{}

\Q Is this graph directed or undirected? \ans[.5in]{X} directed \ans[.5in]{} undirected

\Q Is there an edge from B to D? \ans[.5in]{X} yes \ans[.5in]{} no

\Q Is there an edge from D to C? \ans[.5in]{X} yes \ans[.5in]{} no

\Q What is the cost of the edge from E to A?  \ans{}

\par\vskip 10pt
Graphs with edges that have costs are called \textbf{weighted} graphs.
\par\vskip 10pt

\Q What is the lowest cost path from D to A?   \ans{}

\newpage
The \textbf{degree} of a vertex is the number of edges that are incident (touch) the vertex.
For directed graphs, each node has an \textbf{``in degree''} and an \textbf{``out degree''}.
The in degree is the number of edges that point to the vertex.
The out degree is the number of edges that point away from the vertex.
The in degree and out degree values do not need to match for a vertex.
However, the total degree for a vertex of a directed graph is the in degree plus the out degree.

\Q What is the out degree for E?   \ans{}

\Q What is the in degree for E?   \ans{}

\Q What is the out degree for B?   \ans{}

\Q What is the in degree for B?   \ans{}

\par\vskip 10pt
A \textbf{path} through a graph is a sequence of vertices where each successive pair of vertices has an edge in the graph.
\par\vskip 10pt

\Q  Name two different paths, as a sequence of vertices, to get from B to C:

\begin{itemize}
    \item path 1: \ans{}
    \item path 2: \ans{}
\end{itemize}

\par\vskip 10pt
A \textbf{connected} graph has the property that every vertex is connected via a path to every other vertex in the graph.
\par\vskip 10pt

\Q  Is the graph above connected?  \ans[.5in]{X} yes \ans[.5in]{} no

\par\vskip 10pt
A \textbf{cyclic} graph has the property that it contains one or more cycles.
A \textbf{cycle} is a path that starts and ends with the same vertex.
An \textbf{acyclic} graph has no cycles.
A directed graph that is acyclic is often called a \textbf{DAG} (directed acyclic graph).
\par\vskip 10pt


\Q Is the graph above cyclic?  \ans[.5in]{X} yes \ans[.5in]{} no
\par\vskip 10pt

We can think of a graph as having $|V|$ vertices and $|E|$ edges where V is the set of vertices and E is the set of edges.
The vertical bars mean ``size of'' in mathematical notation.
So, $|V|$ is the size of the vertex set.
\par\vskip 10pt


\Q What is $|V|$ for the graph above? \ans{}

\Q What is $|E|$ for the graph above? \ans{}

\newpage
Two vertices are \textbf{adjacent} if they are connected by an edge.
In an undirected graph, you can think of adjacent vertices as neighbors.
In an undirected graph, if x is adjacent to y, then y is adjacent to x.
In a directed graph, a vertex x is adjacent to vertex y if there is an edge from y to x.
\par\vskip 10pt

\Q In the graph above, is e adjacent to b?  \ans[.5in]{X} yes \ans[.5in]{} no

\Q In the graph above, is d adjacent to b?  \ans[.5in]{X} yes \ans[.5in]{} no

\par\vskip 10pt
We have names for vertices in relationship to other vertices in a directed graph.
A \textbf{successor} of a vertex v is any node n where there is a path from v to n.
A \textbf{predecessor} of vertex v is any node n where there is a path from n to v.
\par\vskip 10pt

\Q Is a a predecessor of d?  \ans[.5in]{X} yes \ans[.5in]{} no

\Q Is a a successor of d?  \ans[.5in]{X} yes \ans[.5in]{} no

\newpage
\textbf{Undirected Graphs}
\par\vskip 10pt

Now, let's look at an undirected graph. An undirected graph may or may not be weighted (costs for edges).
Recall that an undirected graph has links for edges instead of arrows.

\includegraphics[width=.7\textwidth]{figures/graph2.png}

\Q What is different about this undirected graph versus a directed graph?

\begin{answer}[1in]
\end{answer}

Note that the edges are links, so the cost between node b and c is the same as the cost between c and b.
Here, the cost between b and c (and c and b) is 3.
A path can be traversed in either direction of the link.
So, in an undirected graph, if there is a path from vertex x to vertex y, there must also be a path from vertex y to vertex x (following the same set of vertices in reverse order).
Because there are no more directed arrows, it no longer makes sense to think about cycles, successors, and predecessors in an undirected graph. An undirected graph could be connected, as follows:

\includegraphics[width=.7\textwidth]{figures/graph3.png}

  \newpage
  \model{Representing Graphs}

  \quest{15 min}

  Now, to the implementation part of graphs.
  There are two standard ways to represent a graph in code.
  One representation uses an adjacency matrix.
  The other representation uses adjacency lists.

  \par\vskip 10pt

  \textbf{Adjacency Matrix}

  \par\vskip 10pt

  An adjacency matrix M is a $|V| x |V|$ (2D array) of integers.
  You can number the vertices 0 to $|V|-1$ (or have a mapping of vertex names to numbers 0 to $|V|-1$).
  If there is no edge from vertex x and vertex y, then $M[x][y]=0$ (note that book uses infinity; either value works as long as the implementation knows how to treat the value representing ``no edge'').
  If there is an edge from vertex x to y with cost C, then $M[x][y] = C$.

  \par\vskip 10pt

  If the graph is directed, $M[x][y]$ does not need to equal $M[y][x]$.
  If the graph is undirected, $M[x][y] = M[y][x]$.
  If the graph edges have no costs, then M contains just 0's and 1's (and can store bits instead of ints).

  \par\vskip 10pt

  \textbf{Adjacency List}

  \par\vskip 10pt

  An adjacency list L is a list (linked list or array) of vertices.
  Each vertex stores a list of the vertices that are adjacent to it.
  If the edges have costs, then the adjacent vertices are stored as (V, C) pairs where V is the vertex and C is the cost.

  \par\vskip 10pt

  Here is the adjacency matrix for the first graph in this handout:

  \begin{tabular}{ c | c c c c c }
      & A & B & C & D & E \\ \hline
    a & 0 & 0 & 0 & 0 & 0 \\
    b & 0 & 0 & 1 & 0 & 2 \\
    c & 0 & 3 & 2 & 0 & 0 \\
    d & 0 & 4 & 5 & 0 & 0 \\
    e & 6 & 0 & 0 & 7 & 1 \\
  \end{tabular}

  \par\vskip 10pt

  Here is the adjacency list for the first graph in this handout:

  \begin{tabular}{ l }
    a $-> \{\}$                       \\
    b $-> \{(c, 1), (e, 2)\}$         \\
    c $-> \{(b, 3), (c, 2)\}$         \\
    d $-> \{(b, 4), (c, 5)\}$         \\
    e $-> \{(a, 6), (d, 7), (e, 1)\}$ \\
    \end{tabular}

  \newpage

  \Q Here is the adjacency matrix for an undirected graph.
    Draw the circles and edges\key\\[-2.5mm] (links) below.

    \par\vskip 10pt

    \begin{tabular}{ c | c c c c c }
        & a & b & c & d \\
      a & 0 & 2 & 1 & 0 \\
      b & 2 & 0 & 3 & 1 \\
      c & 1 & 3 & 0 & 4 \\
      d & 0 & 1 & 4 & 0 \\
    \end{tabular}

  \vskip -20pt

  \Q Here is the adjacency list for an undirected unweighted graph.
    Draw the circles and\key\\[-2.5mm] edges (links) below.
    \par\vskip 10pt
    \begin{tabular}{ l }
      $a -> \{c, e\} $      \\
      $b -> \{c, d\} $      \\
      $c -> \{a, b, d, e\}$ \\
      $d -> \{b, c, e\} $   \\
      $e -> \{a, c, d\} $   \\
    \end{tabular}

  \vskip -20pt

  \Q What is the maximum number of edges an undirected graph with 5 vertices can\key\\[-2.5mm] have?
  \begin{answer}[0.3in]
    It can have 10 edges
  \end{answer}

  \Q What is the maximum number of edges a directed graph with 5 vertices can have?
  \begin{answer}[0.3in]
    It can have 20 edges
  \end{answer}

  A \textit{sparse} graph is a graph that has few edges compared to the number of vertices in the graph.

  \Q Draw a sparse graph here:
    \begin{answer}[1in]
      Drawings will vary.
    \end{answer}

  A \textit{dense} graph is a graph that has lots of edges compared to the number of vertices in the graph.

  \Q Draw a dense graph here:
    \begin{answer}[1in]
      Drawings will vary.
    \end{answer}

  \Q Suppose you know your graph is sparse. Would you use the adjacency matrix or adjacency list representation?
    \ans[.25in]{} matrix \ans[.25in]{\checkmark} list

  \Q Suppose you know your graph is dense. Would you use the adjacency matrix or adjacency list
    representation?
    \ans[.25in]{\checkmark} matrix \ans[.25in]{} list

  \Q Suppose you want to determine if node a is connected to node c in a graph.
    Node a is mapped to position 0 in the matrix and list representations.
    Node c is mapped to position 2 in the matrix and list representations.

    \par\vskip 10pt

    How long (worst-case) does it take to determine if a is connected to c using the matrix representation?
    O(\ans[.25in]{1})

    \par\vskip 10pt

    How long (worst-case) does it take to determine if a is connected to c using the adjacency list representation?
    Assume the list of adjacent nodes is not in order. O(\ans[.25in]{V})

    \par\vskip 10pt

  \Q What questions does your group have about graphs?
    \begin{answer}[1in]
      Answers will vary.
    \end{answer}
  \newpage
  \input{modeling_with_graphs.tex}

\end{document}