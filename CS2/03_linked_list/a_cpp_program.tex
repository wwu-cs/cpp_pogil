\model{A C++ Program}
  \begin{center}
    \centering {\bf C++ Code}\vskip -15pt\null
    \footnotesize
    \begin{cpplst}
#include <iostream>

class Node {
public:
  int num;    // value stored in node
  Node *next; // pointer to next node in list
};

/* function prototypes on linked lists */
Node *makeNode(int n, Node *nextItem);
int length(Node *list);
void print(Node *list);
void insertTail(int n, Node **list);
Node *find(int n, Node *list);
int remove(Node *toRemove, Node **listPtr);

/* main function */
int main(void) {
  // create linked list
  Node *top = NULL;
  top = makeNode(6, top);
  top = makeNode(10, top);
  top = makeNode(-3, top);
  top = makeNode(2, top);
  print(top);

  // q1: What does the picture of top look like now?
  int len = length(top);
  std::cout << "Length of list: " << len << std::endl;

  // q2: What is the value of len?
  insertTail(25, &top);
  print(top);

  // q3: What does the picture of top look like now?
  Node *ten = find(10, top);
  if (ten == NULL) {
    std::cout << "Not found: 10\n";
  } else {
    std::cout << "Found: 10, memory location: " << ten << std::endl;
  }
  std::cout << "removing 10:\n";
  int ret = remove(ten, &top);
  print(top);

  // q4: What does the picture of top look like now?
  std::cout << "removing 2:\n";
  Node *two = find(2, top);
  ret = remove(two, &top);
  print(top);

  // q5: What does the picture of top look like now?
  std::cout << "removing 15:\n";
  Node *fifteen = find(15, top);
  ret = remove(fifteen, &top);
  print(top);
  return EXIT_SUCCESS;
}
    \end{cpplst}
  \end{center}

  {\it\large Refer to Model 2 above as your team develops consensus answers
    to the questions below.}

  \quest{20 min}

  \Q Without seeing the code implementation, answer the questions in the comments in the main function above.
    \begin{enumerate}
      \itemsep 10pt
      \item q1: What does the picture of top look like now?
        \begin{answer}[1.2in]
          The linked list would look like this:
          2 -> -3 -> 10 -> 6 -> NULL
        \end{answer}

      \item q2: What is the value of len?
        \begin{answer}[1.2in]
          The value of len would be 4, since there are 4 nodes in the linked list.
        \end{answer}

      \item q3: What does the picture of top look like now?
        \begin{answer}[1.2in]
          The linked list would look like this:
          2 -> -3 -> 10 -> 6 -> 25 -> NULL
        \end{answer}

      \item q4: What does the picture of top look like now?
        \begin{answer}[1.2in]
          The linked list would look like this:
          2 -> -3 -> 6 -> 25 -> NULL
        \end{answer}

      \item q5: What does the picture of top look like now?
        \begin{answer}[1.2in]
          The linked list would look like this
          -3 -> 6 -> 25 -> NULL
        \end{answer}
    \end{enumerate}

  \vskip -20pt

  \Q Write the code snippet to insert at the bottom of the main function (before the return) to insert a node with a value of 11 at the front of {\tt top}:
    \begin{answer}[0.7in]
      \cpp{top = makeNode(11, top);}
    \end{answer}

  \Q Write the code snippet to insert at the bottom of the main function (before the return) to insert a node with value -8 at the back of {\tt top}:
    \begin{answer}[0.7in]
      \cpp{insertTail(-8, &top);}
    \end{answer}

  \Q This code can be found in {\tt activity3a.cpp}.
    Run it.
    Review the implementations of \cpp{makeNode}, \cpp{length}, \cpp{print}, \cpp{insertTail}, \cpp{find}, and \cpp{remove}.

    Now that you can see the function definitions, do you need to edit any answers in question 10?
    If so, write your new answers to the right of your original answers.

  \Q Why must the functions \cpp{insertTail} and \cpp{remove} take \cpp{Node ** listPtr} as parameters instead of \cpp{Node *list}?
    \begin{answer}[0.7in]
      Because these functions may need to modify the head pointer of the linked list, they need a pointer to a pointer (Node **).
      This allows the function to change where the head of the list points to if necessary.
    \end{answer}

  \newpage

  \Q Complete the function definition for \texttt{insertHead}.
    This function should insert a\key\\[-2.5mm] new node with value n at the head of the list pointed to by listPtr.
    \begin{answer}[1in]
      \begin{cpplst}
void insertHead(int n, Node ** listPtr) {
  Node *newNode = makeNode(n, *listPtr);
  *listPtr = newNode;
}
      \end{cpplst}
    \end{answer}